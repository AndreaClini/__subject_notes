%-----------------------------------------------------------
%=======================================================
\section{Tetrad formalism \& tensor-valued forms}\todotag{Reformat whole section}
%=======================================================
%------------------------------------------------------------

The tetrad formalism amounts to allowing arbitrary frames for the tangent $\{e_i(x)\}_i$ and cotangent  $\{\theta^i(x)\}_i$ bundles, not necessarily induced by a coordinate system.
It is the natural generalization to the framework necessary for an arbitrary fiber bundle $E_F\to M$.
The only typical requirement is that the two frame are dual to each other, i.e. $\theta^i (e_j) = \delta^i_j$.
In the literature, tetrads are also called \emph{vierbeins} (4-legs) in 4 dimensions or more generally \emph{vielbeins} (n-legs) in n dimensions.


\begin{mytheorem}[Avantages \& disadvantages of the tetrad formalism]
The main advantage of the tetrad formalism is that coefficients of tensors in arbitrary frames transform as scalars under coordinate transformations.
That is, their coordinates with respect to the tetrad bases do not change under change of coordinates, simply because we change the cohordinates $x^\mu\to \tilde{x}^\mu$ but the tetrad bases $\{e_i(x)\}_i$, $\{\theta^i(x)\}_i$ remain the same.
The main disadvantage is that expressions might get longer or more cumbersome, since we cannot use simplifications coming from coordinate bases e.g. the fact that $d (dx^\mu) = 0$ or $[\partial_\mu, \partial_\nu] = 0$.

In any case, this has several advtanges.
For example, we can choose a tetrad that is orthonormal with respect to a given metric $g$ i.e. we have Minkowski (or Euclidean) metric at every point
\begin{equation}
    g = \eta_{ij} \theta^i \otimes \theta^j,\quad \eta_{ij}=\text{diag}(\pm1\dots \pm1)\,.
\end{equation}
This is particularly useful for spinors.
By definition, in any chosen frame, the $\gamma$ matrices are required to respect the Clifford algebra
\begin{equation}
    \{\gamma^\mu, \gamma^\nu\} = 2 g^{\mu\nu} \mathbb{I}.
\end{equation}
In the such an orthonormal frame, this reduces to $\{\gamma^\mu, \gamma^\nu\} = 2 \eta^{\mu\nu} \mathbb{I}$ and we can take the $\gamma^\mu$ as in the usual Clifford algebra in flat space all over the manifold.
\end{mytheorem}


\begin{mytheorem}[Fiber valued differential forms]
Fiber valued differential forms are just usual $p$-forms taking values in a fiber space $F$.
That is, given a fibre bundle $E_F\to M$ with fibre $F$, a fiber valued $p$-form is a section of the product bundle
\begin{equation}
     E_F \otimes \Lambda^p (M) \to M.
\end{equation}
Given a local frame $\{f_a\}_a$ for the fiber bundle $E_F \to M$, any fiber valued $p$-form $\alpha \in \Lambda^p (M) \otimes E_F$ can be locally written as
\begin{equation}
    \alpha =  f_a \otimes \alpha^a,
\end{equation}
where $\alpha^a \in \Lambda^p (M)$ are ordinary differential $p$-forms.
\smallskip

We can still define a wedge product between any two fiber valued forms $\alpha \in \Lambda^p (M) \otimes E_F$, $\beta \in \Lambda^q (M) \otimes E_G$ as
\begin{align}
    &E_F \otimes \Lambda^p (M) \times E_G \otimes \Lambda^q (M) \to (E_F \otimes E_G) \otimes \Lambda^{p+q} (M), \\[4pt]
    &\alpha,\,\beta \mapsto \alpha \wedge \beta := (f_a \otimes \alpha^a) \wedge (g_b \otimes \beta^b) 
    := (f_a \otimes g_b) \otimes (\alpha^a \wedge \beta^b) \in (E_F \otimes E_G) \otimes \Lambda^{p+q} (M).
\end{align}
In fact, the wedge prodcut is well-defined already at the level of linear spaces at each point $p\in M$.

For example, when $E_F = T^h_kM$ is the bundle of $(h,k)$-tensors we have tensor-valued differential forms.
This notion generalizes all the structures we have seen so far.
\begin{itemize}
    \item Ordinary differential $p$-form are just fiber valued $p$-form with trivial fiber $F = \mathbb{R}$.
    \item An $(h,k)$-tensor field is just a fiber valued $0$-form with fiber $F = T^h_k M$.
\end{itemize}
%
In fact, since $\Lambda*(M)\subset T^*_*(M)$ the splitting in `differntial form part' and `tensor valued image' is somewhat arbitrary.
The most natural convention is to take all the antisymmetric part as the differential form part and the rest as the tensor valued image.
\begin{itemize}
    \item The torsion $T_{\mu\nu}^{\, \rho}$, antisymmetric in the $\mu\nu$ indices, is a $(1,0)$-tensor valued 2-form i.e. a fiber valued 2-form with fiber $F = T^1_0 M$.
    \item The Riemann curvature $R_{\,\rho, \mu \nu}^{\sigma}$, antisymmetric in the $\mu\nu$ indices, is a $(1,1)$-tensor valued 2-form i.e. a fiber valued 2-form with fiber $F = T^1_1 M$.
\end{itemize}
\end{mytheorem}



\begin{mytheorem}[Exterior covariant derivative]
Given a fiber bundle $E_F \to M$ with fiber $F$ and a connection $\nabla$ on this bundle with connection 1-forms $\omega^b_a$, we define the exterior covariant derivative $d_\nabla$ acting on fiber valued $p$-forms as the unique map
\begin{equation}
    d_\nabla: E_F \otimes \Lambda^\bullet (M)  \to E_F \otimes \Lambda^{\bullet +1} (M)
\end{equation}
that is linear, satisfies the Leibniz rule
\begin{equation}
    d_\nabla(\alpha \wedge \beta) = (d_\nabla \alpha) \wedge \beta + (-1)^p \alpha \wedge (d_\nabla \beta),\qquad \alpha \in E_F \otimes \Lambda^p (M),\, \beta \in E_G \otimes \Lambda^\bullet (M),
\end{equation}
and reduces to the usual covariant derivative $\nabla$ when acting on fiber valued $0$-forms and the usual exterior derivative $d$ when acting on ordinary differential forms i.e. fiber valued forms with trivial fiber $F = \mathbb{R}$.

Concretely, given a local frame $\{f_a\}_a$ for the fiber bundle $E_F \to M$, any fiber valued $p$-form $\alpha \in \Lambda^p (M) \otimes E_F$ can be locally written as
\begin{equation}
    \alpha = f_a\otimes \alpha^a,
\end{equation}
The condition that $d_\nabla$ reduces to $\nabla$ on fiber valued $0$-forms implies and to the usual exterior derivative on forms with trivial fiber $F = \mathbb{R}$ then implies that the exterior covariant derivative acts as
\begin{equation}
    d_\nabla \alpha = \nabla f_a \wedge \alpha^a + f_a \otimes d \alpha^a  
    = f_a\otimes\left( \omega^a_{b} \wedge \alpha^b + d \alpha^a \right),
\end{equation}
where note that $f_a$ is a fiber valued $0$-form and $\nabla f_a$ is a fiber valued $1$-form via $T^1_0M\ni X \mapsto \nabla_X f_a \in F$ and $\omega^a_{b}$ are the connection 1-forms associated to the connection $\nabla$ on $E_F \to M$.
Beware that the \textbf{exterior covariant derivative does not square to zero} in general, i.e. $d_\nabla^2 \neq 0$ due to the nontriivial connection 1-forms $\omega^a_b$.
\smallskip

The Leibniz rule is then obeyed since
\begin{align}
    d_\nabla(\alpha \wedge \beta) 
    &= \nabla (f_a\otimes g_b) \wedge (\alpha^a \wedge \beta^b) + (f_a\otimes g_b) \otimes d(\alpha^a \wedge \beta^b) \\
    &= \left( \nabla f_a \otimes g_b + f_a \otimes \nabla g_b \right) \wedge (\alpha^a \wedge \beta^b) +
    (f_a\otimes g_b) \otimes \left( d \alpha^a \wedge \beta^b + (-1)^p \alpha^a \wedge d \beta^b \right)
    \\
    &= f_a \otimes g_b \otimes \omega^a_{(F)\,c} \wedge \alpha^c \wedge \beta^b + f_a \otimes g_b \otimes \omega^b_{(G)\,d} \wedge \alpha^a \wedge \beta^d \\
    &\quad + f_a \otimes g_b \otimes d \alpha^a \wedge \beta^b + (-1)^p f_a \otimes g_b \otimes \alpha^a \wedge d \beta^b
    \\
    &= f_a\otimes\left( \omega^a_{(F)\,c} \wedge \alpha^c + d \alpha^a \right) \wedge (g_b \otimes \beta^b) + (-1)^p (f_a \otimes \alpha^a) \wedge \left( g_d \otimes \left( \omega^b_{(G)\,d} \wedge \beta^d + d \beta^b \right) \right)\\
    &=(d_\nabla \alpha) \wedge \beta + (-1)^p \alpha \wedge (d_\nabla \beta)\,.
\end{align}
\end{mytheorem}



\begin{mytheorem}[Structure constants of a tetrad frame]
Given a frame $\{\theta^i\}_i$ for the cotangent bundle, define its structure constants $C^i_{\, jk}$ as the coefficients, antisymmetric in the two lower indices, of the expansions 
\begin{equation}
    d \theta^i = - \frac{1}{2} C^i_{jk} \theta^j \wedge \theta^k = - C^i_{j k} \,\, \theta^j \otimes \theta^k \quad \text{(sum over repeated indices)}.
\end{equation}
The structure constants measure the nonclosure of infinitesimal parallelograms constructed using the tetrad basis vector fields $\{e_i\}_i$ dual to $\{\theta^i\}_i$.
Indeed, one can equivalently define $C^i_{jk}$ as the commutator coefficients
\begin{equation}
    [e_j, e_k] = C^i_{jk} e_i.
\end{equation}
By construction, the structure constants are antisymmetric in the lower indices $C^i_{jk} = - C^i_{kj}$.
If the tetrad is induced by a coordinate system $\theta^\mu=dx^\mu,\,e_\nu=\frac{\partial}{\partial x^\nu}$, then the structure constants vanish identically.
Note that structure constants make sense only for the tangent bundle $T^\bullet_\bullet M$ since only there we have a natural notion of commutator of vector fields, and in fact arbitrary fiber-valued forms are expressed as linear combinations of $\R$-valued forms times fiber basis elements.
\end{mytheorem}
\begin{proof}
Fix any coordinate system $\{x^\mu\}_\mu$ and let $\theta^i = \theta^i_\mu dx^\mu$, $e_j = e_j^\nu \partial_\nu$.
By duality of the tetrad we have
\begin{align}
    0= \partial_\mu \delta^i_j &= \partial_\mu \big(\theta^i (e_j) \big) = \partial_\mu\big(\theta^i_\nu e_j^\nu \big) = (\partial_\mu \theta^i_\nu) e_j^\nu + \theta^i_\nu (\partial_\mu e_j^\nu)
    \quad \Rightarrow\quad (\partial_\mu \theta^i_\nu) e_j^\nu =- \theta^i_\nu (\partial_\mu e_j^\nu)\,.
\end{align}
By definition of exterior derivative we have
\begin{align}
    d \theta^i &= (\partial_\mu \theta^i_\nu) dx^\mu \wedge dx^\nu = \partial_\mu \theta^i_\nu \Big(dx^\mu\otimes dx^\nu- dx^\nu \otimes dx^\mu \Big)\,.
\end{align}
Evaluating on the tetrad and using the previous relation we get
\begin{align}
    d\theta^i (e_j, e_k) 
    = \partial_\mu \theta^i_\nu \Big( e_j^\mu e_k^\nu - e_j^\nu e_k^\mu \Big) 
    = - \theta^i_\nu \Big( (\partial_\mu e_j^\nu) e_k^\mu - (\partial_\mu e_k^\nu) e_j^\mu \Big) 
     =: - \theta^i \big( [e_j, e_k]\big) = - C^i_{jk}\,.
\end{align}
That is to say
\begin{align}
    d\theta^i = - \frac{1}{2} C^i_{jk} \theta^j \wedge \theta^k\, \Leftrightarrow [e_j, e_k] = C^i_{jk} e_i\,.
\end{align}
%
\end{proof}


\begin{mytheorem}[Connection 1-form\todotag{Redo accounting for general $F$}]
%
Given an affine connection $\nabla$ on a fibre bundle $E_F \to M$ with fiber $F$ and structure group $G$, define the $\mathfrak{g}$-valued connection 1-forms $\omega^{(F)}$ as the 1-form that in a given local frame $\{e_i\}_i$ for the fiber bundle $E_F \to M$ satisfy
\begin{equation}
     \nabla\cdot e_j=:\omega(\cdot) e_j =: e_i\otimes \omega^i_{\,j}(\cdot)\,.
\end{equation}
In particular, the covariant derivative of a vector field $X = X^i e_i$ reads
\begin{equation}
    \nabla X = e_i \otimes (dX^i + \omega^i_{\, j}\,\, X^j)\,.
\end{equation}
Under a change of tetrad frame $\{\theta^i,\,e_i\}_i \to \{\tilde{\theta}^i = A^i_{\, j} \theta^j,\,\tilde{e}_i = (A^{-1})^k_{\, i} e_k\}_i$, the connection 1-forms transform as
\begin{equation}
    \tilde{\omega}^i_{\, j} = (A^{-1})^i_{\, k}\,\, \omega^k_{\, l} \,\,A^l_{\, j} + (A^{-1})^i_{\, k} \,\,d A^k_{\, j}.
\end{equation}
This is nothing else that the abstract tetrad form of the well-known transformation of Christoffel symbols under change of coordinates.

To confirm the connection 1-form does indeed take values in the Lie algebra $\mathfrak{g}$ of the structure group $G$, note that $\nabla$-parallel transport in $E_F\to M$ along a given curve $\gamma: [0,1] \to M$ yields linear isomorphisms (also metric preserving if $\nabla$ is a metric connection)
\begin{align}
    P_\gamma(t): E_{F,\,\gamma(0)} &\to E_{F,\,\gamma(t)} \mid e_a \to M(t)_a^b e_b\quad \text{for some }M(t)\in G \subseteq GL(F).
\end{align}  
The covariant derivative then reurns the infinitesimal change, that is 
\begin{align}
    P_\gamma(\epsilon) f = f + \epsilon \frac{d}{d t} \Big|_{t=0} P_\gamma(t) f + O(\epsilon^2) = f + \epsilon \nabla_{\dot{\gamma}(0)} f + O(\epsilon^2)
\end{align}
On the other hand, since $P_\gamma(t)$ is a linear isomorphism we have $M(t) = \exp(Lt)$ for some $L \in \mathfrak{g}$ so that
\begin{align}
    P_\gamma(\epsilon) f = M(\epsilon)^a_b f^b e_a = \left( \delta^a_b + \epsilon \frac{d}{d t} \Big|_{t=0} M(t)^a_b + O(\epsilon^2) \right) f^b e_a = f + \epsilon L f + O(\epsilon^2).
\end{align}
Comparing the two expressions we get
\begin{align}
    \nabla_{\dot{\gamma}(0)}  = \left( \frac{d}{d t} \Big|_{t=0} M(t) \right) = L \in \mathfrak{g}.
\end{align}


Just like the other forms, the connection 1-form can be seen as a fiber valued differential form.
The fiber is the Lie algebra of the structure group of the frame bundle, i.e. $\mathfrak{gl}(n,\R)$ for a general connectin, and $\mathfrak{so}(p,q)$ for an orthonormal frame of metric connection with signature $(p,q)$.
Indeed, for an orthonormal tetrad $\{e_\mu\}_\mu$, metric compatibility $\nabla g=0$ implies
\begin{align}
    0 = \nabla_\bullet \eta_{\mu\nu} &= \nabla_\bullet g(e_\mu, e_\nu) = g(\nabla_\bullet e_\mu, e_\nu) + g(e_\mu, \nabla_\bullet e_\nu) = g(\omega_\mu^\rho e_\rho, e_\nu) + g(e_\mu, \omega_\nu^\sigma e_\sigma) = \omega_{\mu \nu} + \omega_{\nu \mu}.
\end{align}
Lowering the indices of the connection form with the metric we thus get an antisymmetric form $\omega_{\mu\nu} = - \omega_{\nu\mu}$, which is the hallmark of the Lie algebra $\mathfrak{so}(p,q)$.
If $G\subseteq GL(T_pM) \equiv GL(n,\R)$ is the subgroup preserving the metric $g_p$ at point $p\in M$, the connection 1-form then takes values in the Lie algebra $\mathfrak{g}$ of $G$.
Namely, for any curve $t\mapsto A(t)\sim\mathbb{I}+t\omega \in G\subseteq GL(T_pM)$
\begin{align}
    g_{\rho\sigma}A^\rho_\mu A^\sigma_\nu = g_{\mu\nu}
    &\Rightarrow 0 = \frac{d}{dt} \Big|_{t=0} \left( g_{\rho\sigma}A^\rho_\mu(t) A^\sigma_\nu(t) \right) = g_{\rho\sigma} \left( \frac{d}{dt} \Big|_{t=0} A^\rho_\mu(t) \right) \delta^\sigma_\nu + g_{\rho\sigma} \delta^\rho_\mu \left( \frac{d}{dt} \Big|_{t=0} A^\sigma_\nu(t) \right) \\
    &\Rightarrow g_{\rho\nu} \omega^\rho_\mu + g_{\mu\sigma} \omega^\sigma_\nu = 0.
\end{align}
%
This was indeed expected: the connection ultimately yields the infinitesimal change of a vector under parallel transport, which preserves angles and lengths by definition.
The change of a vector can thus only be an infinitesimal `generalized rotation' in the tangent space, i.e. an element of the Lie algebra of the orthogonal group of $g_p$, which is $SO(p,q)$ in a suitable basis.
That is, for $\epsilon \mapsto X(\epsilon)$ a parallelly transported vector along a curve $\epsilon \mapsto \gamma(\epsilon)$,
\begin{align}
    X(\epsilon) &= X(0) + \epsilon\, \nabla_{\dot{\gamma}(0)}\, X + O(\epsilon^2) \\
    &= X(0) + \epsilon\,\, \omega(\dot{\gamma}(0))\, X(0) + O(\epsilon^2) \\
    & = \left( \mathbb{I} + \epsilon \omega(\dot{\gamma}(0)) \right) X(0) + O(\epsilon^2).
\end{align}

This tells us how the connection $\nabla$ should act on a general fiber bundle $E_F\to M$ with structure group $G$: via the representation of the Lie algebra $\mathfrak{g}$ on the fiber $F$.
Namely, if $E_F \to M$ is associated to the frame bundle via a representation $\rho: G \to GL(F)$, the connection 1-form on $E_F \to M$ will be given by
\begin{equation}
    \omega^a_{\, b} = \rho_*(\omega^i_{\, j}),
\end{equation}
where $\rho_*: \mathfrak{g} \to \mathfrak{gl}(F)$ is the induced Lie algebra representation.
For example, for the tensor bundle $T^h_kM \to M$ the representation is the tensor product of $h$ copies of the given representation on $T_pM$ and $k$ copies of the dual representation of $G$.
%
For spinors, it is the spin representation of the spin group $\text{Spin}(p,q)$, that is the double cover of $SO(p,q)$.
If $\omega_{\mu\nu}$ is are the connection 1-forms of the connection on $TM$ in a given tetrad frame, and 
$\{\gamma^a\}_a$ are the gamma matrices in a given representation with basis $s_1\dots s_m$ of the spinor space, the spin representation $\rho_{spin}: \mathfrak{so}(p,q) \to \mathfrak{gl}(\C^m)$ is given by
\begin{equation}
    \omega^{\text{sp.}\, b}_{~a}:=\rho_{spin}(\omega_{\mu\nu})^b_{\,a} := \frac{1}{4} \omega_{\mu\nu} \,\{\gamma^\mu, \gamma^\nu\}^b_{\,a}.
\end{equation}
The connection 1-form act on basis spinors as 
\begin{equation}
    \nabla s_a = \omega^{\text{sp.}\, b}_{~ a}{} s_b = \tfrac{1}{4}\omega_{\mu\nu}\,[\gamma^\mu, \gamma^\nu]^b_{\,a} s_b,
\end{equation}
and on general spinor fields $\psi = \psi^a s_a$ as
\begin{equation}
    \nabla \psi = d \psi^a \otimes s_a + \psi^a \, \nabla s_a = \left( d \psi^b + \tfrac{1}{4} \omega_{\mu\nu}\, [\gamma^\mu, \gamma^\nu]^b_{\,a} \, \psi^a \right) \otimes s_b,\,\,\quad \text{i.e.} \quad
    \nabla_\mu \psi^b = \partial_\mu \psi^b + \tfrac{1}{4} \omega_{\mu\nu}\, [\gamma^\mu, \gamma^\nu]^b_{\,a} \, \psi^a.
\end{equation}
%
\end{mytheorem}


\begin{mytheorem}[Curvature 2-form]
Given an affine connection $\nabla$ on a vector bundle $E_F\to M$ and a tetrad frame $\{\theta^i,\,e_i\}_i$, we define the $T^1_0E_F$-valued $\Omega_{\, j}$ and $T^1_1E_F$-valued $\Omega^i_{\,j}$  curvature 2-forms as the set of 1-forms satisfying
\begin{equation}
    R(e_j) :=\Omega_j=: e_i \otimes \Omega^i_{j} 
\end{equation}
where $R$ is the Riemann curvature operator.
In particular, the Riemann curvature operator acting on a vector field $X = X^j e_j$ reads
\begin{equation}
    R(X) =  X^j\,\, \Omega^i_{\,j}\otimes e_i\,.
\end{equation}
Explicitating all the cohordinates\todotag{redo}
\begin{align}
    R^i_{j,\,\mu\nu}= \Omega^i_j(e_\mu,e_\nu) \theta^\mu \wedge \theta^\nu.
\end{align}
\end{mytheorem}


\begin{mytheorem}[Cartan structure equations\questiontag{Check}]
Given a tetrad frame $\{\theta^i,\, e_i\}_i$, consider the $(1,0)$ tensor valued 1-form $\theta :=  e_i\otimes \theta^i$
The connection 1-forms, and the torsion and curvature 2-forms satisfy the Cartan structure equations, expressed concisely in terms of the exterior covariant derivative $d_\nabla$ as
\begin{align}
    T &= d_\nabla \theta = e_i \otimes \left(d \theta^i + \omega^i_{\, j} \wedge \theta^j \right) 
    \quad \text{i.e.} \quad T^i = d \theta^i + \omega^i_{\, j} \wedge \theta^j,
    \\
    \Omega_{\, j} &= d_\nabla \omega_{\, j} = e_i \otimes \left(d \omega^i_{\, j} + \omega^i_{\,k} \wedge \omega^k_{\, j}\right) 
    \quad\text{i.e.} \quad \Omega^i_{\, j} = d \omega^i_{\, j} + \omega^i_{\, k} \wedge \omega^k_{\, j}.
\end{align}
The \textbf{Cartan equations are essentially definitions} of the torsion and curvature 2-forms in terms of the connection 1-forms and the tetrad basis.
In particular, in absence of torsion the first Cartan structure equation reduces to
\begin{equation}
    d \theta^i + \omega^i_{\, j} \wedge \theta^j = 0.
\end{equation}
\end{mytheorem}
\begin{proof}
Since the identity are tensorial, we can proove it in a coordinate frame $\{\theta^\mu=dx^\mu,\, e_\nu=\partial_{x^\nu}\}_\nu$.
The second equalities follow since the exterior covariant derivative indeed acts as
\begin{align}
    d_\nabla \theta &= d_\nabla (e_i \otimes \theta^i) = e_i\otimes d \theta^i  +  \nabla e_i \wedge \theta^i = e_i\otimes (d \theta^i + \omega^i_{\, j} \wedge \theta^j) ,
    \\
    d_\nabla \omega_{\, j} &= d_\nabla (e_i\otimes \omega^i_{\, j} ) = e_i\otimes d \omega^i_{\, j}  + \nabla e_i \wedge \omega^i_{\, j} = e_i\otimes (d \omega^i_{\, j} + \omega^i_{\, k} \wedge \omega^k_{\, j}).
\end{align}
In turn, the first equalities are easily verified by direct computation using a coordinate frame $\{\theta^\mu=dx^\mu,\, e_\nu=\partial_{x^\nu}\}_\nu$, using the known expressions $T\sim \Gamma - \Gamma^T$ and $R\sim \partial \Gamma + \Gamma \Gamma$ for the torsion and Riemann tensor and the relations
\begin{align}
    \theta^\rho = dx^\rho,\quad d\theta^\rho = 0,\quad
    \omega^\rho_{\, \nu} = \Gamma^\rho_{\mu \nu} dx^\mu,
    \quad
    d \omega^\rho_{\, \nu} = \partial_\sigma \Gamma^\rho_{\mu \nu} dx^\sigma \wedge dx^\mu,
    \quad
    \omega^\rho_{\, \sigma} \wedge \omega^\sigma_{\, \nu} = \Gamma^\rho_{\mu \sigma} \Gamma^\sigma_{\lambda \nu} dx^\mu \wedge dx^\lambda.
\end{align}
%
\end{proof}


\begin{mytheorem}[Bianchi identities in the tetrad formalism\questiontag{Check}]
The Bianchi identities can be concisely expressed in terms of the exterior covariant derivative $d_\nabla$ as
\begin{align}
    d_\nabla T &= \Omega_{\, j} \wedge \theta^j\qquad \text{i.e.}\quad d_\nabla T^i = \Omega^i_{\, j} \wedge \theta^j,
    \\
    d \Omega^i_{\, j} &= 0\qquad \text{i.e.}\quad d_\nabla \Omega_{\, j} = de_i \wedge \Omega^i_{\, j} = e_i \otimes \Big(\omega_k^i\wedge \Omega^k_j + \underbrace{d \Omega_{\, j}^i}_{=0}\Big) = e_i \otimes\omega_k^i\wedge \Omega^k_j\,.
\end{align}
NBeware the 2nd identity says the exterior derivative of the $\R$-valued 2-form $\Omega^i_{\, j}$ vanished, \emph{not} that the exterior covariant derivative of the $(1,0)$-tensor valued 2-form $\Omega_{\, j}$ vanishes!
The \textbf{Bianchi identities are} essentially express \textbf{symmetries that follow from the} very \textbf{definition} of torsion and curvature.
In particular, in absence of torsion the first Bianchi identity reduces to
\begin{equation}
    \Omega^i_{\, j} \wedge \theta^j = 0.
\end{equation}
\end{mytheorem}
\begin{proof}
The first identity follows from direct computation of the exterior covariant derivative of the torsion 2-form and the second Cartan equation above
\begin{align}
    d_\nabla T &= d_\nabla (d_\nabla \theta) = d_\nabla \left[e_i\otimes \left(d \theta^i + \omega^i_{\, j} \wedge \theta^j \right)\right]
    \\
    &=
    e_i\otimes \left(\graycancel{d^2 \theta^i} + d\omega^i_{\, j} \wedge \theta^j \bluecancel{- \omega^i_{\, j} \wedge d\theta^j}  \right)
    + e_i \omega^i_k\wedge \left(\bluecancel{d \theta^k} + \omega^k_{\, j} \wedge \theta^j \right) 
    \\
    &=
    e_i \otimes \left( d\omega^i_{\, j} + \omega^i_{\, k} \wedge \omega^k_{\, j} \right) \wedge \theta^j
    \equiv \Omega_{\, j} \wedge \theta^j.
\end{align}
For the second equation, renaming dummy indices and using that $\Omega^i_{\, j}$ is a 2-form thereby commuting under the wedge product, we have
\begin{align}
    d \Omega_{\,j}^i &= d \left( d \omega^i_{\, j} + \omega^i_{\, k} \wedge \omega^k_{\, j} \right) = \graycancel{d^2 \omega^i_{\, j}} + d\omega^i_{\, k} \wedge \omega^k_{\, j} - \omega^i_{\, k} \wedge d\omega^k_{\, j}
    \\
    &= \big(\Omega^i_k-\omega^i_\ell\wedge \omega^\ell_k\big)\wedge \omega^k_{\, j} - \omega^i_{\, k} \wedge \big(\Omega_j^k-\omega^k_\ell\wedge \omega^\ell_j\big)
    \\
    & = \underbrace{\Omega^i_k \wedge \omega^k_{\, j} - \omega^i_{\, \ell}\wedge \Omega^\ell_j}_{=0}
    \,\,\,\,\underbrace{-\omega^i_\ell\wedge \omega^\ell_k\wedge \omega^k_{\, j}
    + \omega^i_{\, k} \wedge \omega^k_\ell\wedge \omega^\ell_j}_{=0}
    = 0.
\end{align}
The rest follows from the definition of the exterior covariant derivative.
\end{proof}



\begin{mytheorem}[Metric compatibility in arbitrary fiber bundle\todotag{Check and finish}]
%
Given a fiber bundle $E_F \to M$ with fiber $F$ and structure group $G$, endowed with a metric $g: F \times F \to \R$ and a connection $\nabla$ on $E_F$ with $\mathfrak{g}$-valued connection form $\omega$, we say that the connection is metric compatible if 
\begin{align}
    \nabla g \equiv 0\quad \text{i.e.}\quad d \left( g(s,t) \right) = g(\nabla s, t) + g(s, \nabla t)\quad \text{$\forall$ sections }s,t \in \Gamma(E_F).
\end{align}
In a tetrad frame $\{f_a\}_a$ for $E_F \to M$ with connection forms $\omega_a^{~b}$ and metric $g_{ab} := g(f_a, f_b)$, the condition reads
\begin{equation}
    d g_{ab} = \omega_{a}^{~c}\, g_{cb} + \omega_{b}^{~c} \,g_{ac}.
\end{equation}
If the tetrad is orthonormal, i.e. $g_{ab} = (\pm1\dots\pm1)$ constant, the condition reduces to \emph{antisymmetry} of the connection 1-forms in this tetrad
\begin{equation}
    \omega_{ab} = - \omega_{ba}\quad \text{ where }\omega_{ab} := g_{ac} \, \omega^c_{~b}.
\end{equation}
%
\end{mytheorem}



\begin{mytheorem}[Christoffel symbols in a tetrad frame of the tangent bundle\todotag{Check and finish}]
Given a tetrad $\{\theta^i,\, e_i\}_i$ on the tangent bundle of a manifold \(M\) with affine connection \(\nabla\), define the Christoffel symbols \(\Gamma^i_{\, jk}\) as the coefficients in the expansion 
\begin{equation}
    \nabla_{e_j} e_k =: \Gamma^i_{\, jk} e_i.
\end{equation}
That is, the connection 1-forms is expressed in terms of the Christoffel symbols as
\begin{equation}
    \omega^i_{\, j} = \Gamma^i_{ jk} \theta^k.
\end{equation}

In case $\nabla$ is a metric connection, we can express the Christoffel symbols in terms of the metric $g_{ij} := g(e_i, e_j)$ and the structure constants \(C^i_{\, jk}:=\theta^k([e_i,e_j])\) of the tetrad as
\begin{equation}
    \Gamma^i_{jk} = \tfrac{1}{2} \left( C^i_{jk} - g^{i\ell}g_{j m}C^m_{k\ell} + g_{km}g^{i\ell} C^m_{\ell j} \right) + \tfrac{1}{2}g^{i\ell} \left( \partial_j g_{\ell k} + \partial_k g_{\ell j} - \partial_\ell g_{jk} \right)\,.
\end{equation}
If $C_{jk}^i = 0$ (coordinate tetrad) this reduces to the usual expression for the Christoffel symbols in a coordinate basis.
If $g_{ij} = \eta_{ij}$ (orthonormal tetrad) this reduces to
\begin{equation}
    \Gamma^i_{jk} = \tfrac{1}{2} \left( C^i_{jk} - \eta^{i\ell}\eta_{j m}C^m_{k\ell} + \eta_{km}\eta^{i\ell} C^m_{\ell j} \right).
\end{equation}
\end{mytheorem}
\begin{proof}
By metric compatibility $\nabla g=0$, that is $0=X(g(Y,Z)) - g(\nabla_X Y,Z) - g(Y, \nabla_X Z)$.
Permutating the vectors $X,Y,z$ and assuming torsion vanishes $\nabla_XY-\nabla_YX = [X,Y]$, the Koszul formula gives
\begin{align}
\begin{aligned}
    2\,g(\nabla_X Y,Z)
    &=X\,g(Y,Z) + Y\,g(X,Z) - Z\,g(X,Y) 
    + g([X,Y],Z) - g([X,Z],Y) - g([Y,Z],X).
\end{aligned}
\end{align}
Setting \(X=e_j,\, Y=e_k,\, Z=e_i\) and using the definition of the Christoffel symbols and structure constants yields
\begin{align}
    2\, g_{i\ell} \Gamma^\ell_{jk}
    &=
    e_j\,g_{k i} + e_k\,g_{j i} - e_i\,g_{j k} \\
    &\quad
    + g_{i\ell} C^\ell_{jk} - g_{j\ell} C^\ell_{ik} - g_{k\ell} C^\ell_{ij}.
\end{align}
\end{proof}


\begin{mytheorem}[Riemann curvature tensor in a tetrad frame\todotag{Check and finish}]
Given a tetrad frame \(\{\theta^i,\, e_i\}_i\) on a manifold \(M\) with affine connection \(\nabla\), we define the Riemann curvature tensor components \(R^i_{\, jkl}\) as the coefficients in the expansion 
\begin{equation}
    R(e_k, e_l) e_j =: R^i_{\, jkl} e_i.
\end{equation}
In particular, the curvature 2-forms can be expressed in terms of the Riemann tensor components as
\begin{equation}
    \Omega^i_{\, j} = \tfrac{1}{2} R^i_{\, jkl} \theta^k \wedge \theta^l.
\end{equation}
\end{mytheorem}


\begin{mytheorem}[Einstein tensor in a tetrad frame\todotag{Check and finish}]
    
\end{mytheorem}

\begin{mytheorem}[EM tensor varying wrt tetrad fields\todotag{Check and finish}]
    
\end{mytheorem}

{
\color{red}
From Kempf GR L10 missing:
\begin{itemize}
    \item connection $\mathfrak{g}$-valued 1-form [ANY FIBER]
    \item change of cohordinates $dA A^{-1}+ A\omega A^{-1}$ for connection 1-form [ANY FIBER]
    \item curvature $T^1_1F$-valued 2-form [ANY FIBER]
    \item torsion $T^1_0M$-valued 2-form [ONLY TM]
    \item Cartan structure equations
    \item Bianchi identities
    \item metric compatibility in arbitrary fiber bundle [ANY FIBER WITH METRIC]
    \item Christoffel symbols for metric connection on tangent bundle via structure constants [ONLY TM]
    \item Riemann curvature tensor components in tetrad frame [ONLY TM]
\end{itemize}
}