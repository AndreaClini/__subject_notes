% !TeX root = ../cosmology_main.tex
%=========================================================
%=========================================================
\chapter{Observations \& State of the Research}\label{ch:observations}
%========================================================
%=========================================================


%------------------------------------------------------------
%=======================================================================
\section{Observational tensions}
%=================================================================
%---------------------------------------------------------------------


Despite its many predictions and successes, $\Lambda$ CDM cosmology still exhibits tensions across different datasets; the most notable is the so-called $H_{0}$ tension [13, 14]. It appears when comparing local measurements of the expansion rate of the universe, such as those obtained by the SH0ES collaboration [15], with the value calculated by CMB experiments [5]. Another tension, albeit slightly milder, is the discrepancy that appears $[16,17]$ when measuring the clustering of small scale structure, often quantified with the $S_{8}$ parameter: in this case, weak lensing experiments such as DES [11, 18, 19], KiDS [12, 20-24], and CFHTLens [10] all find lower values of $S_{8}$ than those calculated by CMB experiments assuming $\Lambda$ CDM cosmology. A recent analysis of [12] has obtained $S_{8}=0.762_{-0.024}^{+0.025}$, which is in $2.3 \sigma$ tension with the Planck 2018 [5] value of $S_{8}=0.825 \pm 0.011$. We will adopt this measurement as our baseline $S_{8}$ data. We note, however, that by using slightly different assumptions and data, reference [22] gets a stronger ( $3.2 \sigma$ ) tension; while [24] also gets a $\sim 3 \sigma$ tension when combining KiDS1000 weak lensing data with BOSS galaxy clustering data. These tensions have motivated the exploration of dark matter models beyond the standard CDM paradigm.


%=================================================================
\subsection{The $H_0$ tensions}
%=================================================================
