% !TeX root = ../structure_matter_main.tex
%==========================================================
%=========================================================
\chapter{Theory of Many Body Systems}
\label{ch:many_body_systems}
%=========================================================
%=========================================================


\section{Second quantization (Fock space and many-body operators).}
\subsection{Fock space and creation/annihilation operators.}
Let $\mathcal{H}$ be the one-particle Hilbert space. 
The (unsymmetrized) Fock space is
\begin{align}
\begin{aligned}
\mathcal{F}=\bigoplus_{N=0}^{\infty}\mathcal{H}^{\otimes N}.
\end{aligned}
\end{align}
The symmetrized/antisymmetrized Fock space is
\begin{align}
\begin{aligned}
\mathcal{F}_{\pm}=\bigoplus_{N=0}^{\infty}\mathcal{H}^{\otimes_{\pm} N}.
\end{aligned}
\end{align}
We denote by $S(N)$ the (anti)symmetrization operator on $\mathcal{H}^{\otimes N}$. 
For $u,u_1,\dots,u_N\in\mathcal{H}$, the creation operator $c_u^\dagger$ acts as
\begin{align}
\begin{aligned}
c_u^\dagger\,S(N)\ket{u_1,\dots,u_N}=\sqrt{N+1}\,S(N+1)\ket{u,u_1,\dots,u_N}.
\end{aligned}
\end{align}
The annihilation operator $c_u$ is defined as the adjoint of $c_u^\dagger$. 
Its action on $N$-particle (anti)symmetrized states is
\begin{align}
\begin{aligned}
c_u\,S(N)\ket{u_1,\dots,u_N}=\frac{1}{\sqrt{N}}\sum_{j=1}^{N}(\pm 1)^{j-1}\braket{u|u_j}\,S(N-1)\ket{u_1,\dots,\widehat{u_j},\dots,u_N}.
\end{aligned}
\end{align}
In particular,
\begin{align}
\begin{aligned}
c_u\ket{0}=0.
\end{aligned}
\end{align}

\subsection{Action on normalized occupation-number states.}
Let $\{\ket{r}\}_{r=1}^{M}$ be an orthonormal basis of $\mathcal{H}$, and define mode operators $b_r^\dagger,b_r$ (bosons) and $a_r^\dagger,a_r$ (fermions). 
For bosons,
\begin{align}
\begin{aligned}
b_r^\dagger\ket{n_1,\dots,n_r,\dots,n_M}=\sqrt{n_r+1}\,\ket{n_1,\dots,n_r+1,\dots,n_M},
\end{aligned}
\end{align}
\begin{align}
\begin{aligned}
b_r\ket{n_1,\dots,n_r,\dots,n_M}=\sqrt{n_r}\,\ket{n_1,\dots,n_r-1,\dots,n_M}.
\end{aligned}
\end{align}
For fermions (with $n_r\in\{0,1\}$),
\begin{align}
\begin{aligned}
a_r^\dagger\ket{n_1,\dots,n_r,\dots,n_M}=
\begin{cases}
0, & n_r=1,\\
(-1)^{n_1+\cdots+n_{r-1}}\ket{n_1,\dots,1,\dots,n_M}, & n_r=0,
\end{cases}
\end{aligned}
\end{align}
\begin{align}
\begin{aligned}
a_r\ket{n_1,\dots,n_r,\dots,n_M}=
\begin{cases}
0, & n_r=0,\\
(-1)^{n_1+\cdots+n_{r-1}}\ket{n_1,\dots,0,\dots,n_M}, & n_r=1.
\end{cases}
\end{aligned}
\end{align}
The orthonormal occupation-number basis can be generated from the vacuum as
\begin{align}
\begin{aligned}
\ket{n_1,\dots,n_M}=\frac{1}{\sqrt{n_1!\cdots n_M!}}\,(b_1^\dagger)^{n_1}\cdots(b_M^\dagger)^{n_M}\ket{0}\qquad\text{(bosons)},
\end{aligned}
\end{align}
\begin{align}
\begin{aligned}
\ket{n_1,\dots,n_M}=(a_1^\dagger)^{n_1}\cdots(a_M^\dagger)^{n_M}\ket{0}\qquad\text{(fermions)}.
\end{aligned}
\end{align}

\subsection{Change of basis and canonical transformations.}
Let $\{\ket{r}\}$ and $\{\ket{s}\}$ be two orthonormal bases of $\mathcal{H}$, and define
\begin{align}
\begin{aligned}
M_{rs}=\braket{r|s}.
\end{aligned}
\end{align}
Then the corresponding creation/annihilation operators transform as
\begin{align}
\begin{aligned}
a_r^\dagger=\sum_{s}M_{rs}\,a_s^\dagger,\qquad a_r=\sum_{s}M_{rs}^*\,a_s,
\end{aligned}
\end{align}
and similarly for bosons $(b_r^\dagger,b_r)$. 
If both bases are orthonormal then $M$ is unitary, and the transformation is canonical. 

\section{Commutation rules and useful composition formulas.}
\subsection{(Anti)commutation relations.}
For any $u,v\in\mathcal{H}$ (not necessarily from an orthonormal basis), the bosonic operators satisfy
\begin{align}
\begin{aligned}
[b_u,b_v]=0,\qquad [b_u^\dagger,b_v^\dagger]=0,\qquad [b_u,b_v^\dagger]=\braket{u|v}.
\end{aligned}
\end{align}
The fermionic operators satisfy
\begin{align}
\begin{aligned}
\{a_u,a_v\}=0,\qquad \{a_u^\dagger,a_v^\dagger\}=0,\qquad \{a_u,a_v^\dagger\}=\braket{u|v}.
\end{aligned}
\end{align}
For fermions there is a hole-particle symmetry: swapping creation and annihilation operators leaves the CAR form-invariant. 
A standard example of canonical transformation is the Bogoliubov transformation. 

\subsection{Useful rewriting formulas (moving operators inside symmetrized states).}
From the action of $c_u$ on $S(N)\ket{u_1,\dots,u_N}$, one gets the composition identity
\begin{align}
\begin{aligned}
c_u^\dagger c_v\,S(N)\ket{u_1,\dots,u_N}=\sum_{j=1}^{N}(\pm 1)^{j-1}\braket{v|u_j}\,S(N)\ket{u,u_1,\dots,\widehat{u_j},\dots,u_N}.
\end{aligned}
\end{align}
Equivalently, the created vector $u$ can be moved into the position of the annihilated one (useful e.g.\ when rewriting one-body operators). 
Similarly, for two-body structures one can iterate the same idea to obtain identities with double sums over the removed slots (useful when rewriting two-body operators). 

\subsection{Occupation number operators.}
Define number operators
\begin{align}
\begin{aligned}
\hat n_r=b_r^\dagger b_r\qquad\text{(bosons)},\qquad \hat n_r=a_r^\dagger a_r\qquad\text{(fermions)}.
\end{aligned}
\end{align}
They commute among themselves, and their eigenvalues on $\ket{n_1,\dots,n_M}$ are $n_r$. 
