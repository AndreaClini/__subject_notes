% !TeX root = ../cosmology_main.tex
%=========================================================
%=========================================================
\chapter{Perturbation theory in Cosmology}\label{ch:perturbation_theory}
%========================================================
%=========================================================


\begin{mytheorem}[Metric perturbations remain small]
%
Treating metric perturbations to linear order is justified since they remain very small, even when energy-density perturbations become extremely non-linear.
For example, compare the (dimensionless) gravitational potential\footnote{Identified with the Newtonian-gauge potential $\Phi$} $\phi\sim \frac{GM}{R}$ and the density contrast $\delta\rho/\bar\rho$, relative to the cosmic average matter density today, for various astrophysical objects:
\begin{itemize}
    \item Earth: $\phi\sim 10^{-9}, \quad \tfrac{\delta\rho}{\rho}\sim 10^{30}$
    \item Sun: $\phi\sim 10^{-6}, \quad \tfrac{\delta\rho}{\rho}\sim 10^{28}$
    \item Milky Way: $\phi\sim 10^{-6}, \quad \tfrac{\delta\rho}{\rho}\sim 10^{6}$
    \item Galaxy cluster: $\phi\sim 10^{-5}, \quad \tfrac{\delta\rho}{\rho}\sim 10^{3}$
    \item Black hole at Schwarzschild radius $R_S=2GM$: $\phi\sim \tfrac{1}{2}, \quad \tfrac{\delta\rho}{\rho}\sim \infty$.
\end{itemize}
\end{mytheorem}




%--------------------------------------------------------
%=========================================================
\section{Gauge fixing}
%=========================================================
%------------------------------------------------------------


We will mainly follow the standard notation from \cite{MaBertschinger_CosmologicalPerturbationTheorySynchronousConformalNewtonianGauges_1995}.
The metric has signature $-+++$.
The conformal time is denoted by $\ct$ and differentiation with respect to it is denoted by $\dot{f}$.
The proper time (of a comoving observer) is denoted by $t$, is related to conformal time by $dt=a \,d\tau$ and differentiation with respect to it is denoted by $f'$.
We use natural units $c=1$, $\hbar=1$ and $\kappa_B=1$.

The perturbation to a homogeneous isotropic FLRW metric is written in full generality as
\begin{equation} \label{eq:metric_general_form}
-\mathrm{d} s^{2}:=g_{\mu\nu}dx^\mu dx^\nu=a^{2}(\tau)\left[-(1+2 A) \,\mathrm{d} \tau^{2}+2 B_{i} \mathrm{~d} x^{i} \mathrm{~d} \tau+\left((1+2C)\delta_{ij}+2E_{ij}\right) \mathrm{d} x^{i} \,\mathrm{~d} x^{j}\right],
\end{equation}
with SVT decomposition
\begin{equation*}
\underbrace{A,\, C}_{\text {scalar }}, \quad
B_{i}=\underbrace{\partial_{i} \tilde{B}}_{\text {scalar }}+\underbrace{\hat{B}_{i}}_{\text {vector }},\quad E_{i j}=\underbrace{\partial_{\langle i} \partial_{j\rangle} \tilde{E}}_{\text {scalar }}+\underbrace{\partial_{(i} \hat{E}_{j)}}_{\text {vector }}+\underbrace{\bar{E}_{i j}}_{\text {tensor }},
\end{equation*}
where we use the shorthand
\begin{align*}
\partial_{\langle i} \partial_{j\rangle} \tilde{E} \equiv\left(\partial_{i} \partial_{j}-\frac{1}{3} \delta_{i j} \nabla^{2}\right) \tilde{E},\qquad
\partial_{(i} \hat{E}_{j)}\equiv \frac{1}{2}\left(\partial_{i} \hat{E}_{j}+\partial_{j} \hat{E}_{i}\right).
\end{align*}



%==============================================================================
\subsubsection*{Newtonian Gauge}
%===============================================================================

We consider the \emph{conformal newtonian} gauge.
In this gauge, vector and tensor modes are excluded from the beginning $\hat{B}=\hat{E}=\bar{E}=0$, and the residual gauge freedom is used to kill the non-diagonal scalar terms $\tilde{B}=\tilde{E}=0$.
The metric is conventionally\footnote{Beware of different conventions in the literature, swapping $\phi \leftrightarrow \psi$ or their signs. Here we adopt the convention of Baumann \cite{Baumann_Cosmology_2022}, \texttt{CLASS} \cite{Lesgourgues_CosmicLinearAnisotropySolvingSystemCLASSIOverview_2011} and \cite{MaBertschinger_CosmologicalPerturbationTheorySynchronousConformalNewtonianGauges_1995} so that $\psi\approx \phi$ in the absence of anisotropic stress.
Dodelson \cite{DodelsonSchmidt_ModernCosmology_2025} uses $\phi\mapsto -\phi$.} written by calling $A=\psi$, $C=-\phi$
\begin{equation} \label{eq:metric_newtonian_gauge}
    -\mathrm{d} s^{2}:=g_{\mu\nu}dx^\mu dx^\nu = a^2(\tau)\left[
        -(1 + 2\psi)\, d\ct^2
        +(1 - 2\phi)\, \delta_{ij}\, dx^i dx^j
    \right].
\end{equation}
This gauge has a simple physical interpretation: in the Newtonian limit $\psi$ plays the role of the gravitational potential, while $\phi$ is interpreted as a local perturbation to the scale factor.
Upon ignoring anisotropic stress, Einstein equations will imply that $\psi\simeq \phi$.

The Christoffel symbols in this gauge (to linear order)
\begin{align}\label{eq:christoffel_newtonian_gauge}
    \begin{aligned}[t]
    \Gamma_{00}^{0} & =\mathcal{H}+\dot\psi,  \\
    \Gamma_{0 i}^{0} & =\partial_{i} \psi,  \\
    \Gamma_{00}^{i} & = \partial_{i} \psi,
    \end{aligned}
    \qquad
    \begin{aligned}[t]
    \Gamma_{i j}^{0} & =\mathcal{H} \delta_{i j}-\left[\dot\phi+2 \mathcal{H}(\phi+\psi)\right] \delta_{i j}, \\
    \Gamma_{j 0}^{i} & =\mathcal{H} \delta_{ij}-\dot\phi \delta_{ij}, \\
    \Gamma_{j k}^{i} & = -\partial_j \phi\, \delta_{ki} - \partial_k \phi\,\delta_{ji} + \partial_i \phi\,\delta_{jk}.
    \end{aligned}
\end{align}





%---------------------------------------------------------
%========================================================================================
\section{Relativistic kinematics and the Boltzmann equation}
\label{eq:relativistic_kinemtics_Boltzmann_equations}
%======================================================================================
%---------------------------------------------------------------------------------

%---------------------invariant volumes------------------------------------
We set the notation for relativistic kinematics and the Boltzmann equations.
The covariant $4$-momentum is $P^\mu=\frac{dx^\mu}{d\lambda}$ where $d\lambda$ is an affine parameter with $[\lambda]=E^{-2}$ \emph{defined} so that the geodesic equation $\frac{d P^\mu}{d\lambda}=\Gamma_{\alpha\beta}^\mu P^\alpha P^\beta$ holds\footnote{For massive particles it is simply $d\lambda=\frac{ds}{m}$.}.
We have the invariant volumes in position and/or momentum space
\begin{equation}
    \sqrt{-g}\,d^4x^\mu,\quad \sqrt{-g}\,d^4P^\mu,\quad \frac{d^4P_\mu}{\sqrt{-g}}, \quad d^4\,x^\mu d^4P_\mu.
\end{equation}
Enforcing the on-shell condition, we have the invariant volume in momentum space
\begin{equation}
    \frac{d^4 P_\mu}{\sqrt{-g}}\,\delta_+(P^\alpha P_\alpha+m^2)=\frac{d^3 P_i}{\sqrt{-g}}\frac{1}{2P^0},
\end{equation}
and in turn the invariant volume in phase space
\begin{equation}
    \sqrt{-g}\,d^4x^\mu \frac{d^3 P_i}{\sqrt{-g}}\frac{1}{2P^0}= d^4\,x^\mu \frac{d^3 P_i}{2P^0}.
\end{equation}

%----------------physical comoving quantities-------------------------------
Define the physical $3$-momentum $\vec{p}$ measured by a comoving observer, by enforcing $p^{2} := g_{i j} P^i P^j$ with direction $\hat{p}^{i} \,\propto\, P^{i}$ normalized such that $\hat{p}^{i} \hat{p}^{j} \delta_{i j}=1$.
We have $g_{\mu\nu}P^\mu P^\nu=-m^2$ and enforce the relativistic energy relation $E=\sqrt{p^{2}+m^{2}}$, so that
\begin{equation}\label{eq:relativistic_energy_relation}
    \quad E^2=-g_{00}(P^0)^2-2g_{0i}P^0P^i, \quad P^0_{\pm}=\frac{g_{0i}P^i\pm\sqrt{(-g_{00})E^2+(g_{0i}P^i)^2}}{(-g_{00})},
\end{equation}
where $P^0=P^0_+$ always denotes the positive sign solution.
Solving to linear order in the metric perturbations \eqref{eq:metric_general_form} we relate covariant and physical quantities
\begin{equation}\label{eq:comoving_energy_momentum}
    P^i=\frac{p}{a}\left(1-C-E_{mn}\,\hat p^m \hat p^n\right)\hat{p}^i,\quad P^0_{\pm}=\pm\frac{E}{a}(1-A)+\frac{p}{a}B_i\,\hat p^i.
\end{equation}
Finally, for any metric parametrized as in \eqref{eq:metric_general_form}, using \eqref{eq:relativistic_energy_relation} we recover the usual invariant momentum volume element
\begin{equation}
    \frac{d^3 P_i}{\sqrt{-g}}\frac{1}{2P^0}=\frac{d^3\vec{p}}{2E}.
\end{equation}

%--------------------geodesic euations in newtonian gauge-----------------------
%-------------------------------------------------------
\subsubsection*{Geodesic equation in Newtonian gauge}
%-------------------------------------------------------------------
In the newtonian gauge \eqref{eq:metric_newtonian_gauge} we have $A=\psi$, $B=E=0$, $C=-\phi$ and thus
\begin{equation}
\label{eq:comoving_energy_momentum_newtonian_gauge}
P^i=\frac{p}{a}(1+\phi)\hat{p},\quad P^0=\frac{E}{a}(1-\psi), \quad P_0=-a(1+\psi)E,\quad P_i=a(1-\phi)p\hat{p}_i.
\end{equation}
The metric potentials then obey (to first order in metric perturbations)
\begin{align}
\begin{aligned}
    \frac{d \psi}{d\lambda}=\frac{E}{a}\dot\psi +\frac{p}{a}\hat{p}^j\partial_j\psi,
    \qquad
    \frac{d\phi}{d\lambda}=\frac{E}{a}\dot\phi +\frac{p}{a}\hat{p}^j\partial_j\phi.
\end{aligned}
\end{align}
Using the Christoffel symbols \eqref{eq:christoffel_newtonian_gauge} we have the geodesic equations:
\begin{align}\label{eq:geodesic_equation_covariant_momentum_newtonian}
\begin{aligned}
    \frac{d P^0}{d\lambda}
    &= -\frac{E^2}{a^2}\left(\mathcal{H}(1-2\psi)+\dot\psi\right)-2\frac{Ep}{a^2}\hat{p}^j\partial_j\psi-\frac{p^2}{a^2}\left(\mathcal{H}(1-2\psi)+\dot\phi\right),
    \\
    \frac{dP^k}{d\lambda}&=-\frac{E^2}{a^2}\partial_k\psi-\frac{Ep}{a^2}\hat{p}^k\,2\left(\mathcal{H}(1-\psi+\phi)-\dot\phi\right)-\frac{p^2}{a^2}\left(\partial_k\phi-2\hat{p}^k\hat{p}^j\partial_j\phi\right).
\end{aligned}
\end{align}
Using the above relations, we obtain corresponding evolution equations for the physical quantities, already expressed in conformal time $\tau$ \todotag{Check}
\begin{align}\label{eq:geodesic_equation_physical_momentum_newtonian}
\begin{aligned}
    \frac{d E}{d\ct}&=-\frac{p^2}{E}(\mathcal{H}-\dot\phi)-p\hat{p}^j\partial_j\psi,
    \\
    \frac{d p}{d\ct}&=-p(\mathcal{H}-\dot\phi)-E\hat{p}^j\partial_j\psi,
    \\
    \frac{d\hat{p}^i}{d\ct}&=\big(\hat{p}^i\hat{p}^j-\delta^{ij}\big)\left(\frac{E}{p}\partial_j\psi+\frac{p}{E}\partial_j\phi\right).
\end{aligned}
\end{align}


%====================================================================================
\subsection{Boltzmann equation}
%=====================================================================================

%------------------Bolztmann distribution function --------------------------------------
The phase space distribution $f_{s}(\tau,\vec{x},P^{\mu})$ describes the probability of finding a particle, in a big collection of a given species $s$, with (on-shell) covariant momentum $P^\mu$ at spacetime location $(\tau,\bar{x})$.
The relativistic Boltzmann equation (BE) describes the evolution of this distribution
\begin{equation}\label{eq:boltzmann_eq_covariant_momentum}
   \frac{d f}{d\lambda} = \frac{\partial f}{\partial \tau} P^0
+ \frac{\partial f}{\partial x^i} P^i
+ \frac{\partial f}{\partial P^\mu} \frac{dP^\mu}{d\lambda}
 = C[f],
\end{equation}
where $\lambda$ is the affine parameter for the geodesic equation and the collision term $C[f]$ accounts for the interactions of the species $s$.
%Since we will not be interested in quantitities depending on internal degrees of freedom, we use the convation of absorbing any possible degeneracy $g_s$ in the distribution itself, that is $f_s:=g_s f_s$.

For a general interaction between species $a_i$, $b_j$ of the schematic form
\begin{equation}
    \label{eq:general_interaction}
    a_1 + a_2 + \dots + a_m \leftrightarrow b_1 + b_2 + \dots + b_n,
\end{equation}
the corresponding contribution to the collision term $C[f]$ is given by
\begin{align*}
C[f_{a_1}](\vec{p}_1) &= \frac{1}{2}
\int \left[ \prod_{i=2}^{m} \frac{d^3 p_i}{(2\pi)^3 2E_i} \right]
\left[ \prod_{j=1}^{n} \frac{d^3 p_j'}{(2\pi)^3 2E_j'} \right] \ \times (2\pi)^4 \,\delta^{(4)}_D\!\!\left( \sum_{i=1}^{m} p_i-\sum_{j=1}^{n} p_j' \right)
\, |\mathcal{M}|^2  \\
&\,\,\,\,\qquad \times \left[
f_{b_1} \dots f_{b_n} \prod_{i=1}^{m} (1 \pm f_{a_i})
-
f_{a_1} f_{a_2} \dots f_{a_m} \prod_{j=1}^{n} (1 \pm f_{b_j})
\right],
\end{align*}
where $\mathcal{M}$ is the Feynman amplitude of the process, and we assumed crossing symmetry.


%-----------------------------------------------------------------------------
\subsubsection{Boltzmann equation in Newtonian gauge}
%------------------------------------------------------------------------------

It is convenient to rewrite everything in terms of physical comoving variables $f=f(\ct,x^i,E,\hat p^i)$ in Newtonian gauge.
Using formulas \eqref{eq:geodesic_equation_physical_momentum_newtonian}, the left-hand side of the Boltzmann equation \eqref{eq:boltzmann_eq_covariant_momentum} becomes (to first order in metric perturbation, but fully exact otherwise)\todotag{check}
{\small
\begin{align}\label{eq:Boltzmann_LHS_exact_newton_gauge}
\begin{aligned}
    \frac{d f}{d \lambda}
    %
    &=
    %
    P^{0}\left[\frac{\partial f}{\partial \tau}+\frac{\partial f}{\partial x^{i}} \frac{P^{i}}{P^{0}}+\frac{\partial f}{\partial E} \frac{1}{P^{0}} \frac{d E}{d \lambda}+\frac{\partial f}{\partial \hat{p}^{i}} \frac{1}{P^{0}} \frac{d \hat{p}^{i}}{d \lambda}\right]
    %
    \\&=
    %
    \frac{E}{a}\!\left[\frac{\partial f}{\partial \ct}(1-\psi)
    \!+\!\left(\hat{p}^j\partial_jf\right) \frac{p}{E}(1\!+\!\phi)
    -\frac{\partial f}{\partial E}\frac{p^2}{E}\left(\mathcal{H}(1\!-\!\psi)-\dot\phi\right)
    -\frac{\partial f}{\partial \bar{p}^j}\partial_j\psi \,E
    \!+\!\left(\frac{\partial f}{\partial p} \hat{p}^j-\frac{\partial f}{\partial \bar{p}^j}\right)\partial_j\phi\,\frac{p^2}{E}\right].
\end{aligned}
\end{align}
}
In the formula above we included derivatives both versus $\vec{p}_j$ and the sole direction $\hat{p}_j$, since this writing will turn useful when taking moments.




%=======================================================================
\subsection{Moments of the Boltzmann equation}
%===================================================================


The complete behavior of the species $s$ in phase space is entirely described by the distribution $f_s$, which in turn is completely specified by the Boltzmann equation.
In principle, we should solve the full system of Boltzmann equations, one for each species $s$, coupled through the mutual collision terms and the metric potentials, and Einstein equations, sourced by the energy-momentum of the species themselves.
This task is extremely difficult, even numerically.

%------------------------ perturbation theory----------------------------------------
In perturbation theory, we can still make great progress by considering moments of the distribution function and the corresponding evolution equations obtained taking the same moments of the Boltzmann equation.
In general, we define the bulk average of a quantity of interest $Q$ for the fluid\footnote{To simplify the notation, we do not indicate internal degrees of freedom. The quantities considered here are almost always independent of these. If this is the case, we implicitly include a suitable degeneracy factor $g_s$ in the definition of $f_s$.} by
\begin{equation}\label{eq:bulk_averages}
    \langle Q\rangle:=\left(\int\!\!\frac{d^3p}{(2\pi)^3}{f_s}\right)^{-1}\int\!\!\frac{d^3p}{(2\pi)^3}\,Q\,f_s.
\end{equation}
In particular, the covariant energy-momentum tensor is given by
\begin{equation}\label{eq:energy_momentum_tensor_from_distribution}
    T^{\mu}_{(s)\,\nu}=\int\!\!\frac{d^3P_i}{\sqrt{-g}(2\pi)^3}\,g_s\,\frac{P^\mu P_\nu}{2P^0}\, f_s=\int\!\!\frac{d^3\vec{p}}{(2\pi)^3}\,g_s\,\frac{P^\mu P_\nu}{2E} f_s.
\end{equation}
In turn, number density, energy density and pressure are obtained as 
{\small
\begin{equation}\label{eq:number_energy_density}
    n:=\int\!\!\frac{d^3p}{(2\pi)^3}\,f_s,\quad \rho_s= T^0_0=\int\!\!\frac{d^3p}{(2\pi)^3} E\,f_s,\quad P=\tfrac13 T^i_i=\int\!\!\frac{d^3\vec{p}}{(2\pi)^3}\,\frac{p^2}{3E} f_s.
\end{equation}}
Similarly the bulk velocity of the fluid, already introduced below equation \eqref{eq:em_tensor_perturbed_single_species}, is obtained as
\begin{equation}\label{eq:bulk_velocity}
    \vec{u}:=\left\langle \frac{\vec{p}}{E}\right\rangle:=\left(\int\!\!\frac{d^3p}{(2\pi)^3}{f_s}\right)^{-1}\int\!\!\frac{d^3p}{(2\pi)^3}\,\frac{\vec{p}}{E}\,f_s,
\end{equation}
and the stress-tensor of the fluid is obtained as 
\begin{equation}\label{eq:shear_stress_tensor}
    n\,\sigma^{ij}:=\left\langle \frac{p^ip^j}{E^2}\right\rangle-\left\langle \frac{p^i}{E}\right\rangle\left\langle \frac{p^j}{E}\right\rangle.
\end{equation}

%----------------- Boltzmann hyerarchy-------------------------------------------
Taking moments of the Boltzmann equation generates the well-known Boltzmann hierarchy, where the evolution of a moment $\sim\left(\frac{\vec{p}}{E}\right)^n$ depends on higher moments $\sim\left(\frac{\vec{p}}{E}\right)^{n+1}$.
In the next section we will consider \emph{cold} matter.
In this case $p\ll E \sim m$ and it will be sufficient to truncate this hierarchy at some order in $\frac{p}{E}$, typically $n=2$.
The treatment for radiation is far more delicate: in this case $p\sim E$, the above truncation argument is completely inadequate\footnote{In some cases, like for the photon-baryon fluid in the tight coupling regime this is actually still good.} and one has to account for the full hierarchy.
This is a well-studied (and well-resolved) problem and we refer to pedagogical treatments in \cite{Baumann_Cosmology_2022,MaBertschinger_CosmologicalPerturbationTheorySynchronousConformalNewtonianGauges_1995,DodelsonSchmidt_ModernCosmology_2025}.






%===================================================================================
\subsubsection{Velocity averages for cold matter}
\label{sec:velocity_averages_cold_matter}
%=====================================================================================

%----------- splitting of the distribution function and moments------------------------------
Finally we clear a subtlety regarding statistical averages of powers $\frac{p_{i_1}}{E}\dots\frac{p_{i_n}}{E}$ for cold matter, in order to correctly identify perturbations of the same order in the next section.
We will use the shorthand $v_i=\frac{p_i}{E}$ with modulus $v=\frac{p}{E}$. These quantities are not observable, they are just `free' variables that need to be integrated against $f$, while we reserve the notation $\vec{u}$ for the (bulk) velocity of the fluid \eqref{eq:bulk_velocity}.

Following the notation in \cite{MaBertschinger_CosmologicalPerturbationTheorySynchronousConformalNewtonianGauges_1995}, the distribution function is split as 
\begin{equation}\label{eq:distrbution_function_splitting}
    f(\tau,x,\vec{p})=:\bar{f}(\tau,p)\left(1+\Psi(\tau,x,\vec{p})\right)=:\bar{f}(\tau,p)+\delta f(\tau,x,\vec{p}),
\end{equation}
where the zeroth-order term is independent of position $\vec{x}$ and direction $\hat{p}$ of momentum, due to isotropy and homogeneity of the background.
The function $\Psi$, or equivalently $\delta f$, describes the small perturbations about the unperturbed solution. 

From isotropy of the background, to order zero in field perturbations we have 
{\small
\begin{equation}\label{eq:bulk_velocities_order_0}
    \big\langle v_{i_1}\dots v_{i_n}\big\rangle:=\left(\int\!\!\frac{d^3p}{(2\pi)^3}f\right)^{-1}\!\!\int\!\!\frac{d^3p}{(2\pi)^3}\frac{p}{E}\hat{p}_{i_1}\dots\frac{p}{E}\hat{p}_{i_n}\,f \overset{\text{ order 0}}{\equiv}\begin{cases}
        \langle v^{2j}\rangle_0\left(\delta^{\text{Kr.}}_{i_1i_2}\dots \delta^{\text{Kr.}}_{i_{2j-1}i_{2j}}\!\!+\text{perm.}\right) &n=2j,
        \\
        0 &n=2j+1,
    \end{cases}
\end{equation}
}
where $\langle\dots \rangle_0$ denotes averages taken with respect to the \emph{background} distribution \begin{equation}\label{eq:background_averages}
    \langle Q\rangle_0:=\left(\int\!\!\frac{d^3p}{(2\pi)^3}\bar{f}\right)^{-1}\int\!\!\frac{d^3p}{(2\pi)^3}\,Q\,\bar{f}.
\end{equation}

\begin{remark}
The background distribution is centered around $\vec{v}=0$ with some thermal spread, say a Boltzmann distribution centered around zero with finite temperature.
In the cold matter approximation $T_{\text{CDM}}\to 0$ we have a Dirac delta in momentum space
\[
\bar{f}=\bar{n}(t)\,(2\pi)^3\delta^{(3)}(\vec{q}),
\]
so that even all the norm averages $\langle v^{\ell}\rangle_0$ vanish.
In any case, these background averages would be negligibly small, since they only arise from thermal spread around zero and redshift as $\langle v^{n}\rangle_0\propto \, a^{-n}$.
\end{remark}
%
Perturbations to the distribution $\delta f(\tau,x,p,\hat{p})$ allow for nonisotropic averages $\big\langle v_{i_1}\dots v_{i_n}\big\rangle\neq0$.
Similarly to above, we could approximate $f_s$ as the full space density $n_s(x,\tau)$ times some thermal spread around a \emph{nonzero} peculiar velocity $\vec{u}_s(x,\tau)$, namely the bulk velocity of the fluid s.
In the cold limit $T_{\text{CDM}}\to 0$ we now have 
\[
    f_s = n_s(x,\tau)(2\pi)^3 \delta^{(3)}(\vec{q}-m_s \vec{u}_s(x,\tau)),
\]
so that all reduced moments vanish and we have 
\begin{equation}\label{eq:moments_peculiar_velocity_cold_limit}
    \big\langle v_{i_1}\dots v_{i_n}\big\rangle\simeq u_{i_1}\dots u_{i_n}.
\end{equation}
\begin{remark}
In linear theory the peculiar velocity $u(x,t)\,\propto\, \dot \delta \propto \dot D_+=\mathcal{H}\, \textbf{f}\, D_+$ grows in time (e.g. in EdS approximation $u\,\propto\, a^{\nicefrac{1}{2}}$) and the full distribution $f$ is centered around it in momentum space, regardless of the thermal spread being zero or not.
Therefore, regardless of integrating over directions $\hat{v}$ and anisotropy, averages of $\frac{p}{E}$ are dominated by the peculiar velocity, in the sense that 
\begin{equation}\label{eq:peculiar_velocity_dominates_spread_background_velocity}
    \fint \frac{d^3q}{(2\pi)^3}\bar{f}\, v^\ell \propto\, T_{CDM}^{\frac{\ell}{2}}\propto \,a^{-\ell}\ll \fint\frac{d^3q}{(2\pi)^3}f\, v^{\ell} \sim \,|u(x,t)|^\ell\propto \left(\mathcal{H}\textbf{f}D_+\right)^\ell.
\end{equation}
\end{remark}

In summary, we have two small parameters to expand with:
\begin{enumerate}
    \item the perturbations $\delta f$, or better to the derive fields like $\delta_s$ and $\vec{u}_s$;
    \item  the thermal spread of matter characterized e.g. by the non-vanishing of \emph{reduced} moments like $\langle v_i v_j\rangle -\langle v_i\rangle \langle v_j\rangle $, or equivalently, since the thermal spread of $\bar f$ and of $f$ should be of the same order, by the magnitude of \emph{background} averages  $\langle\left(\frac{p}{E}\right)^\ell\rangle_0=\langle v^\ell\rangle_0$.
\end{enumerate} 
Following the standard EFT approach \cite{baumann2012cosmologicalEffectiveFluid}, we first expand in field perturbations and only then add back also the thermal spread with \emph{effective} terms accounting for the microphysics we could not describe exactly.

Retaining powers $\langle \frac{p^2}{E^2}\rangle \sim u^2$ in our perturbative expansion is necessary because the contributions of a decay interaction to the RHS of the continuity and Euler equations beyond linear order are of the form $ \Gamma\,\psi\, \delta_m$ and $\langle v^2\rangle \Gamma$ and these terms are essentially of the same order (cf. next section).
Indeed, from the Poisson equation in the subhorizon limit $\delta_m\sim \frac{k^2}{\mathcal{H}^2}\psi$ and from the Euler equation \eqref{eq:euler_dcdm_second_order} to leading order $\vec{u} \sim \frac{\vec{k}}{\mathcal{H}} \psi$, so that we have
\begin{align}\label{eq:order_magnitude_comparison}
\begin{aligned}
    \langle \frac{p^2}{E^2}\rangle\sim|u(x,t)|^2 \sim\frac{k^2}{\mathcal{H}^2} \psi^2 \sim \psi\, \delta_m.
\end{aligned}
\end{align}

In conclusion, we remark again that retaining powers $\langle \frac{p^2}{E^2}\rangle$ is not directly related to the order at which we truncate the Boltzmann hierarchy.
Indeed, we will still close the continuity--Euler system by assuming no velocity dispersion 
\begin{equation}\label{eq:no_velocity_dispersion}
    \frac{1}{n}\sigma_{ij}=\langle v_i v_j\rangle- \langle v_i\rangle \langle v_j\rangle\,\propto\, \,\text{thermal spread}\,\simeq 0.
\end{equation}
Only afterwards, during renormalization, will we introduce back effective terms like
\begin{equation}\label{eq:closed_expression_velocity_dispersion}
    \langle v_i v_j\rangle- \langle v_i\rangle \langle v_j\rangle=\delta_{ij}\, c_s^2\,\delta_{m}+\dots
\end{equation}
In fact, in the same spirit, we might be forced to add back the corrections to $\Gamma\langle v^2\rangle=\Gamma u^2(1+\dots)$.




%================================================================================
\subsection{The continuity and Euler equations}
\label{sec:continuity_euler_equations}
%=========================================================================
%
We now take the zeroth and first moments of the of the Boltzmann equation \eqref{eq:Boltzmann_LHS_exact_newton_gauge} to obtain the continuity and Euler equations.
After some tedious but elementary passages, the continuity equation reads (to first order in metric perturbations, fully exact otherwise)
{\small
\begin{align}\label{eq:continuity_exact_newton_gauge}
\begin{aligned}
    \int\frac{d^3\vec{p}}{(2\pi)^3}\frac{1}{E}\frac{d f}{d\lambda}
    &=\frac{1}{a}\left[(1-\psi)\frac{\partial n}{\partial \ct}
    +(1+\phi)\,\partial_j\!\left(n\,u_j\right) 
    +3 \,n \left(\mathcal{H}(1-\psi)-\dot\phi\right)
    +n\,u_j\left(\partial_j\psi-2\partial_j\phi\right)\right]\\ &= \int\frac{d^3\vec{p}}{(2\pi)^3}\frac{1}{E}C[f;f_{\text{other}}].
\end{aligned}
\end{align}}
In the absence of number-changing interactions, the RHS of the continuity equation vanishes.
At order zero we find the usual Hubble dilution 
\begin{equation}\label{eq:conitnuity_equation_zero_order}
    \frac{\partial \bar{n}_\chi}{\partial \ct}
    +3 \,\bar{n}_\chi \mathcal{H}
    =
    a\,\bigg[\int\frac{d^3\vec{p}}{(2\pi)^3}\frac{1}{E}C[f]\bigg]_{0}
\end{equation}
Ww then split the density as $n=\bar{n}(1+\delta)$ and get the perturbed continuity equation (to first order in metric perturbations, fully exact otherwise)\todotag{Do it}
{\small
\begin{align}\label{eq:perturbed_continuity_exact_newton_gauge}
\begin{aligned}
    ...
\end{aligned}
\end{align}}
%
Similarly, integrating by parts and using \eqref{eq:continuity_exact_newton_gauge} to simplify the $\partial_{\tau}n$ term, we obtain the Euler equation (to first order in metric, fully exact otherwise)
\begin{align}\label{eq:Euler_LHS_exact_newton_gauge}
\begin{aligned}
    \int\frac{d^3\bar{p}}{(2\pi)^3}\frac{1}{E}\frac{d f}{d\lambda}\frac{p\hat{p}_i}{E}
    =\frac{1}{a}&\bigg\{
    (1-\psi)n\frac{\partial u_i}{\partial \ct}
    +(1+\phi)\,n\,u_j \,\partial_j u_i
    \\
    &+(1+\phi)\partial_j\Big[n\big(\langle v_iv_j\rangle-\langle v_i\rangle \langle v_j\rangle\big)\Big]
    \\
    &+\Big[\mathcal{H}(1-\psi)-\dot\phi\Big]\, n\,\left(\langle v_i\rangle-\langle v_iv^2\rangle\right)
    \\
    &+n \big[\delta^{ij}-\langle v_i\rangle\langle v_j\rangle\big]\,\partial_j\psi+ n\big[\delta^{ij}\langle v^2\rangle-3\langle v_i v_j\rangle+2\langle v_i\rangle\langle v_j\rangle\big]\partial_j\phi
    \\
    &+a\,\langle v_i\rangle\, \int\frac{d^3\vec{p}}{(2\pi)^3}\frac{1}{E}C[f,f_{\text{other}}]\bigg\} = \int\frac{d^3\vec{p}}{(2\pi)^3}\frac{1}{E}C[f,f_{\text{other}}]\frac{p\hat{p}_i}{E}.
\end{aligned}
\end{align}
where we used the shorthand $v_i=\frac{p_i}{E}$.
Note that the last term $a\,\langle v_i\rangle\, \int\frac{d^3\vec{p}}{(2\pi)^3}\frac{1}{E}C[f,f_{\text{other}}]$ at the LHS appeared since we used \eqref{eq:continuity_exact_newton_gauge} to simplify the $\partial_{\tau}n$ term.
It will cancel up to some order in $\langle(\frac{p}{E})^\ell\rangle$ with corresponding terms at the RHS when expanding in powers $\langle\frac{p}{E}\rangle$ for cold matter.
%
We remark that $\langle v_i\rangle \equiv u_i$ exactly by definition of bulk velocity \eqref{eq:bulk_velocity}, while the identification $\langle v_iv_j\rangle\simeq u_i u_j$ holds with the caveats of Section \ref{sec:velocity_averages_cold_matter}.
Multiplying by $1+\psi$, dropping second orders in metric perturbations and third orders $\sim \langle v\rangle ^3$ and $\sim \phi \,\langle v\rangle ^2$, and dividing by the density $n$ yield the familiar Euler equation for cold matter (now correct to second order in $\langle v\rangle$, and to first order in metric perturbations)
\begin{align}\label{eq:euler_equation_perturbed_up_to_second_order}
    \frac{\partial u_i}{\partial \ct}
    +\,u_j \,\partial_j u_i
    %
    +\frac{1}{n}\partial_j\Big[n\big(\langle v_iv_j\rangle-\langle v_i\rangle \langle v_j\rangle\big)\Big]
    %
    +\Big[\mathcal{H}-\dot\phi\Big]\, u_i
    %
    +\partial_i\psi
    %
    =\text{(interactions)}+ O\big(\langle v^3\rangle\big).
\end{align}
%
----------------------------------------------------------------------------------------------------------------------------

In conclusion, we drop leftover order $\geq3$ terms and separate linear and quadratic terms in the fields to get the final versions of the \todotag{Fix from here below} continuity and Euler equation (correct to second order in perturbations, upon ignoring terms $\sim \phi^2$):
\begin{align}\label{eq:continuity_dcdm_second_order_final_version}
    \dot{\delta}_{\chi}
    \!+\!\partial_j u_{\chi}^j
    {\color{orange}-3 \dot\phi}
    %
    \,{\color{lightblue} +\,\psi\, a\, \Gamma}
    %
    &=
    -\partial_j\big(\delta_\chi\,u_{\chi}^j\big)
    %
    {\color{red}+3 \dot\phi \delta_{\chi}}
    %
    {\color{forestgreen}-(\phi\!+\!\psi)\partial_j u_{\chi}^j
    -u_{\chi}^j\left(\partial_j\psi\!-\!2\partial_j\phi\right)}
    %
    {\color{blue}
    -\psi\,\delta_{\chi}\,a\,\Gamma+\frac{1}{2}\langle v_{\chi}^2\rangle\, a\,\Gamma},
    %
    \\
    %
    \label{eq:euler_dcdm_second_order_final_version}
    %
    \dot{u}_{\chi}^i+\mathcal{H}u_{\chi}^i +\partial_i\psi
    &=
    -\,u_{\chi}^j \,\partial_j u_{\chi}^i
    %
    {\color{myyellow}-\frac{1}{n_{\chi}}\partial_j\Big[n_{\chi}\big(\langle v_{\chi}^iv_{\chi}^j\rangle-\langle v_{\chi}^i\rangle \langle v_{\chi}^j\rangle\big)\Big]}
    %
    {\color{red}+\dot\phi\, u_{\chi}^i}.
\end{align}
Taking the divergence of the Euler equation,  with the usual notation $\nabla\cdot u=:\theta$, finally yields
\begin{align}\label{eq:divergence_euler_dcdm_final_version}
\begin{aligned}
\dot{\theta}_{\chi} + \mathcal{H}\,\theta_{\chi} + \nabla^2 \psi
    &=
    -\,\partial_i\!\left(u_{\chi}^j \,\partial_j u_{\chi}^i\right)
    -\frac{1}{n_{\chi}}\,\partial_i \partial_j\Big[n_{\chi}\big(\langle v_{\chi}^i v_{\chi}^j\rangle-\langle v_{\chi}^i\rangle \langle v_{\chi}^j\rangle\big)\Big]
    + \partial_i\left(\dot\phi\, u_{\chi}^i\right) .
\end{aligned}
\end{align}
%
\begin{remark}\label{remark:color_coding}
We comment on the color coding in the above equations, a full analysis is postoponed to Section \ref{sec:order_magnitude_analysis}.
The black parts are the usual equations for higher order perturbation theory in a $\Lambda$CDM universe.
The term {\color{orange}$3\dot{\phi}$} is a linear term already present in a $\Lambda$CDM universe, but typically dropped in the subhorizon limit.
The decay will somewhat enhance the growth over time of this term and reduce its subhorizon suppression.
The term ${\color{lightblue}\,\psi\, a\, \Gamma}$ is the linear correction from dcdm.
We will argue that it is of the same order of {\color{orange}$3\dot{\phi}$}.
Thus, if we want to see explicit linear corrections from dcdm beyond the pure background effect, we need to keep both terms and cannot invoke the subhorizon limit completely.

The situation is analogous at second order.
The green terms  {\color{forestgreen}$\sim \phi\,\partial_j u$} are quadratic terms already present in a $\Lambda$CDM universe, again typically dropped since metric perturbations are much smaller than matter ones.
These are genuinely suppressed in the subhorizon limit, regardless of the presence of decay.
The red terms  {\color{red} $\sim \dot\phi\,\delta\,,\,\,\dot\phi\,u$} are other quadratic terms already present in a $\Lambda$CDM universe and again typically dropped in the subhorizon limit.
As for the orange term above, the decay will enhance the growth over time of these terms and reduce the suppression.
Finally the blue terms are the quadratic corrections from dcdm.
Similarly to above, we will argue that {\color{blue}these quadratic term from dcdm} are of the same order as the {\color{red} red ones}.
Thus if we want to see genuine quadratic corrections from dcdm, we must to keep both types of terms.

Finally we have the {\color{myyellow} velocity dispersion term.}
Neglecting this term does not depend on the subhorizon limit, rather it is small due to the thermal spread of cold matter.
We already discussed that we will initially take this term to be zero, and only then add back effective corrections accounting for the microphysics we could not capture exactly.
\end{remark}

%---------------- stable matter----------------------------------
Finally we discuss baryons and `standard' non-decaying CDM.
If we consider times well after recombination so that baryon perturbations have already caught up with CDM ones, we can treat baryons and stable CDM as a single fluid (stable matter, denoted s).
The only difference from the above analysis for dcdm is the absence of the decay term.
In this case, the perturbed continuity and Euler equations are simply (correct to second order in perturbations, upon ignoring terms $\sim \phi^2$)
\begin{align}\label{eq:continuity_baryons_cdm_stable_second_order_final_version}
    \dot{\delta}_{s}
    +\partial_j u_{s}^j
    {\color{orange}-3 \dot\phi}
    %
    &=
    -\partial_j\big(\delta_s\,u_{s}^j\big)
    %
    {\color{red}+3 \dot\phi \delta_{s}}
    %
    {\color{forestgreen}-(\phi+\psi)\partial_j u_{s}^j
    -u_{s}^j\left(\partial_j\psi-2\partial_j\phi\right)},
    %
    \\
    \label{eq:euler_baryons_cdm_stable_second_order_final_version}
    %
    \dot{u}_{s}^i+\mathcal{H}u_{s}^i +\partial_i\psi
    &=
    -\,u_{s}^j \,\partial_j u_{s}^i
    %
    {\color{myyellow}-\frac{1}{n_{s}}\partial_j\Big[n_{s}\big(\langle v_{s}^iv_{s}^j\rangle-\langle v_{s}^i\rangle \langle v_{s}^j\rangle\big)\Big]}
    %
    {\color{red}+\dot\phi\, u_{s}^i},
\end{align}
with the same considerations on the colored terms as above.



%-----------------------------------------------------------------------------
%==============================================================================
\section{Collision term and RHS of Boltzmann equation}
%==============================================================================
%------------------------------------------------------------------------------

An excellent treatment of how to deal with the collision term when taking moments of the Boltzmann equation is given in \cite{Buen-AbadEmamiSchmaltz_CannibalDarkMatterLargeScaleStructure_2018}.








%-----------------------------------------------------------------------------
%==============================================================================
\section{Energy-momentum tensor \& Bianchi identities}
%==============================================================================
%------------------------------------------------------------------------------


%------- em tensor and solving the full Einstein-Boltzmann system-------------------------------------
The complete behavior of the species $s$ is described by the distribution $f_s$.
In particular, the energy-momentum tensor is given by
\begin{equation}\label{eq:energy_momentum_tensor_from_distribution}
    T^{\mu}_{(s)\,\nu}=\int\!\!\frac{d^3P_i}{\sqrt{-g}(2\pi)^3}\,\frac{P^\mu P_\nu}{P^0}\, f_s=\int\!\!\frac{d^3\vec{p}}{(2\pi)^3}\,\frac{P^\mu P_\nu}{E} f_s.
\end{equation}
The form of the energy-momentum tensor is justified since \todotag{Finish or move to GR}

Using formulas \eqref{eq:comoving_energy_momentum_newtonian_gauge} we compute
{\small\begin{align}\label{eq:energy_momentum_component_newt_gauge}
\begin{aligned}
    T^0_0=\int\!\!\frac{d^3\vec{p}}{(2\pi)^3} E\,f_s,
    \quad
    T^0_i=\int\!\!\frac{d^3\vec{p}}{(2\pi)^3} (1-\phi-\psi)\,p\hat{p}^i\,f_s,
    \quad
    T^i_j=\int\!\!\frac{d^3\vec{p}}{(2\pi)^3}\,\frac{p^2\hat{p}^i\hat{p}^j}{E} f_s.
\end{aligned}
\end{align}}
%
We now insert an equilibrium distribution with local temperature\todotag{Write the general case (cf. decaying)}
{\small 
\begin{equation}\label{eq:local_temp_photons}
    f(\ct, x^i, E, \hat{p}^i) = f(\tfrac{E}{T}), \quad T=\bar{T}(\tau)(1+\Theta(\tau, x^i, p, \hat{p}^i))=\bar{T}+\delta T\quad \Rightarrow \quad f\simeq \bar{f}-\frac{\partial \bar{f}}{\partial E}E\Theta +O(2).
\end{equation}}


In the case of photons, taking $E=p$ and assuming $\Theta$ independent of $E$ for simplicity we get
Then we compute \todotag{Finish}
\begin{align}\label{eq:energy_momentum_tensor_perturbed_newtonian_gauge}
\begin{aligned}
    T^0_0&=-\int\!\!\frac{d^3\vec{p}}{(2\pi)^3} E\,\bar{f}+\int\!\!\frac{d^3\vec{p}}{(2\pi)^3} E^2\frac{\partial \bar{f}}{\partial E}\Theta = -\bar{\rho}-\bar{\rho}\Theta_0,
    \\
    T^0_i&=\int\!\!\frac{d^3\vec{p}}{(2\pi)^3} (1-\phi-\psi)\,p\hat{p}^i\,\bar{f}-\int\!\!\frac{d^3\vec{p}}{(2\pi)^3} (1-\phi-\psi)\,p\hat{p}^i\,E\frac{\partial \bar{f}}{\partial E}\Theta = -4\bar{\rho} i\hat{k}^i\Theta_1 = -\tfrac13 (\bar{\rho}+\bar{p}) i\hat{k}^i \Theta_1,
    \\
    T^i_j&=\int\!\!\frac{d^3\vec{p}}{(2\pi)^3}\,\frac{p^2\hat{p}^i\hat{p}^j}{E} \bar{f}-\int\!\!\frac{d^3\vec{p}}{(2\pi)^3}\,\frac{p^2\hat{p}^i\hat{p}^j}{E} E\frac{\partial \bar{f}}{\partial E}\Theta.
\end{aligned}
\end{align}
Comparing with the perturbed perfect fluid energy-momentum tensor
\begin{align}
    &T^\mu_\nu = u^\mu u_\nu (\rho+P) + P \delta^\mu_\nu + \Sigma^\mu_\nu\\[3pt]
    &T^0_0 = -\bar{\rho} - \delta \rho, \quad T^0_i = -(\bar{\rho} + \bar{P}) v_i, \quad T^i_j = (\bar{P} + \delta P) \delta^i_j + \Sigma^i_j
\end{align}
we find
\begin{align}
    \delta_\gamma = \frac{\delta \rho}{\bar{\rho}} = 4 \Theta_0, \quad v_i = -\frac{3}{4} \int \frac{d^3\vec{p}}{(2\pi)^3} p \hat{p}_i E \frac{\partial \bar{f}}{\partial E} \Theta, \quad \Sigma^i_j = -\int\!\!\frac{d^3\vec{p}}{(2\pi)^3}\,\frac{p^2\hat{p}^i\hat{p}^j}{E} E\frac{\partial \bar{f}}{\partial E}\Theta.
\end{align}


It is convenient to split the \emph{scalar} part of the (i,j)-component of the stress tensor into the trace part and a traceless part, so as to separate the 2 scalar degrees of freedom.
Namely using 
\begin{align}
    P=\tfrac{k^ik^j}{k^2}-\tfrac13 \delta^{ij}, \quad (T^i_j)_{\text{scalar}}= \tfrac13 \delta^i_j T^k_k + \left(\tfrac{k^ik^j}{k^2}-\tfrac13 \delta^{ij}\right) \left(T^i_j - \tfrac13 \delta^i_j T^k_k\right).
\end{align}
We find\todotag{finish}
\begin{align}
    \Big(\tfrac{k^jk_i}{k^2}-\tfrac13 \delta^{j}_{i}\Big)T^i_j = \Big(\tfrac{k^jk_i}{k^2}-\tfrac13 \delta^{j}_{i}\Big)\int\!\!\frac{d^3\vec{p}}{(2\pi)^3}\,\frac{p^2\hat{p}^i\hat{p}^j}{E} f 
    = \int\!\!\frac{d^3\vec{p}}{(2\pi)^3}\,\frac{p^2}{E} f \big(\mu^2-\tfrac13\big)
\end{align}





%-----------------------------------------------------------------------------
%==============================================================================
\section{Einstein--Boltzmann system}
%==============================================================================
%------------------------------------------------------------------------------


The Einstein equation for $\psi-\phi\propto\, k^{-2}\Theta_2$ seems to suggest anisotropic stress is relevant for making $\psi\neq \phi$ only on large scales since it is suppressed by $k^{-2}$ and that it is also relevant at ealry times since then radiation dies off as $a^{-4}$ and matter grows as $a^{-3}$, so that the ratio of radiation to matter energy density is larger at early times.
In particular anisotropic stress may lead $\psi\neq\phi$ only due to anisotropic stress of neutrinos and then around recombination due to anisotropic stress of photons that develop a nonnegligible quadrupole moment $\Theta_2$.\todotag{do it}



%-----------------------------------------------------------------------------
%==============================================================================
\section{Higher-order perturbation theory}
%==============================================================================
%------------------------------------------------------------------------------