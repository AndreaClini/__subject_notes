% !TeX root = ../QFT_main.tex
%=========================================================
%=========================================================
\chapter{Canonical Quantization}
%========================================================
%=========================================================


%=========================================================
\section{Canonical Quantization in flat spacetime}
%=========================================================



\begin{mytheorem}[Review of classical \& quantum harmonic oscillator]\todotag{Finish}
The classical harmonic oscillator with mass $m$ and frequency $\omega$ is described by the Hamiltonian
\begin{equation}
    H = \frac{p^2}{2m} + \frac{1}{2} m \omega^2 q^2,
\end{equation}
where $q$ and $p$ are the canonical position and momentum variables, satisfying the Poisson bracket relation
\begin{equation}
    \{q, p\} = 1.
\end{equation}
Upon quantization, the canonical variables are promoted to operators $\hat{q}$ and $\hat{p}$ acting on a Hilbert space $\mathcal{H}$, satisfying the commutation relation
\begin{equation}
    [\hat{q}, \hat{p}] = i \hbar.
\end{equation}
The Hamiltonian operator is given by
\begin{equation}
    \hat{H} = \frac{\hat{p}^2}{2m} + \frac{1}{2} m \omega^2 \hat{q}^2.
\end{equation}
The energy eigenvalues of the quantum harmonic oscillator are
\begin{equation}
    E_n = \hbar \omega \left( n + \frac{1}{2} \right), \quad n = 0, 1, 2, \ldots
\end{equation}
It is crucial to understand how the frequency of the oscillators determines the likelyhood of finding the system at a give elongation--i.e. amplitude--from equilibrium.
Indeed, all wave functions are proportional to a Gaussian factor, multiplied by suitable Hermite polynomials,
\begin{align}
    \psi_n(q) &\propto e^{-\frac{m \omega}{2 \hbar} q^2}.
\end{align}
The crucial point is that the smaller the frequency $\omega$, the wider the Gaussian, and thus the larger the likelyhood of finding the oscillator at large elongations from equilibrium.
This is something already familiar from Electrodynamics: it is very unlikely to find high-frequency electromagnetic waves with large amplitudes, whereas low-frequency waves can easily have large amplitudes.
\end{mytheorem}


\begin{mytheorem}[Free Klein-Gordon field as waves on a lake \& Fourier modes as decoupled harmonic oscillators]
\todotag{insert image of lake \& fourier modes}
Under any point of view, we should see the classical Klein-Gordon field as a collection of decoupled harmonic oscillators, one per each Fourier mode.
In this sense, it is completely analogous to the waves on a lake, which can be decomposed into independent Fourier modes, each oscillating with its own frequency.
\end{mytheorem}

\begin{mytheorem}[Probing the Klein-Gordon field with a cork] \todotag{write}
\todotag{Insert image}
How would you probe the waves on a lake? A good idea is to use a cork floating on the water surface.
The cork moves up and down following the local height of the water surface, thus providing a direct measurement of the wave amplitude at that point.
Analogously, to probe the Klein-Gordon field, we can use a point-like detector that measures the field value at a specific spacetime point.
This detector interacts with the field, allowing us to extract information about the field's behavior and dynamics.
The cork would now be an atom or another suitable quantum system that can be acted upon by the field.
\end{mytheorem}

















\begin{mytheorem}[Canonical quantization of the Klein-Gordon field]
A real scalar field $\phi(x)$ obeys the Klein-Gordon equation
\begin{equation}
    (\partial_{\mu} \partial^{\mu} + m^2) \phi(x) = 0.
\end{equation}
The canonical quantization procedure promotes the classical field $\phi(x)$ and its conjugate momentum $\pi(x) = \frac{\delta \mathcal{L}}{\delta \partial_0 \phi(x)}$ to operators $\hat{\phi}(x)$ and $\hat{\pi}(x)$ acting on a Hilbert space $\mathcal{H}$, satisfying the equal-time commutation relations
\begin{align}
    [\hat{\phi}(t, \mathbf{x}), \hat{\pi}(t, \mathbf{y})] &= i \delta^{(3)}(\mathbf{x} - \mathbf{y}), \\
    [\hat{\phi}(t, \mathbf{x}), \hat{\phi}(t, \mathbf{y})] &= 0, \\
    [\hat{\pi}(t, \mathbf{x}), \hat{\pi}(t, \mathbf{y})] &= 0.
\end{align}

The Fourier transform of the fields are defined as
\begin{equation}
    \phi_k(t) :=\int d^3 x\, e^{-i \mathbf{k} \cdot \mathbf{x}} \phi(t, \mathbf{x}).
\end{equation}
The commutation relations in momentum space become
\begin{align}
    [\hat{\phi}_k(t), \hat{\pi}_{k'}(t)] &= i (2\pi)^3 \delta^{(3)}(\mathbf{k} + \mathbf{k}'), \\
    [\hat{\phi}_k(t), \hat{\phi}_{k'}(t)] &= 0, \\
    [\hat{\pi}_k(t), \hat{\pi}_{k'}(t)] &= 0.
\end{align}
The harmonic oscillator are now decoupled in momentum space, but something seems to be off track.
Indeed
\begin{itemize}
    \item the field $\phi_k$ is no more self-adjoint, since $\phi_k^{\dagger} = \phi_{-k}$; whereas we would like our DoFs to be observables;
    \item the delta functions have argument $\mathbf{k} + \mathbf{k}'$ instead of $\mathbf{k} - \mathbf{k}'$.
\end{itemize}
In fact, the two issues are essentially related.
We can fix them by explicitly rewriting $\phi_k$ as linear combinations of \emph{physical} self-adjoint harmonic oscillators.
Alternatively, we can directly pass to creation and annihilation operators.
\todotag{Write the two approaches in detail.}
    
\end{mytheorem}