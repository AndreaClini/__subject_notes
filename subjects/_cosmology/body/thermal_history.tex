% !TeX root = ../cosmology_main.tex
%=========================================================
%=========================================================
\chapter{Thermal History}\label{ch:thermal_history}
%========================================================
%=========================================================



%--------------------------------------------------------
%=========================================================
\section{The Boltzmann Equation}
%=========================================================
%------------------------------------------------------------

In Newtonian mechanics the Boltzmann equation reads
\begin{align}\label{eq:newtonian_boltzmann_eq}
\frac{df}{dt}=\frac{\partial f}{\partial t} + \frac{\vec{p}}{m} \cdot \partial_{\vec{x}} f + \vec{F} \cdot \partial_{\vec{p}} f = C[f]\,.
\end{align}
Making this covariant we have
\begin{align}\label{eq:covariant_boltzmann_eq}
    \frac{df}{d\lambda} = \frac{dx^\mu}{d\lambda} \frac{\partial f}{\partial x^\mu} + \frac{dP^\mu}{d\lambda} \frac{\partial f}{\partial P^\mu} = C[f]\,,
\end{align}
where $\lambda$ is an affine parameter along the particle's worldline with energy dimension $[\lambda]=[E]^{-2}$ \emph{defined} to enforce the geodesic equation holds
\begin{align}
    P^\mu=\frac{dx^\mu}{d\lambda},\quad \frac{dP^\mu}{d\lambda} + \Gamma^\mu_{\alpha\beta} P^\alpha P^\beta =0\,.
\end{align}






%--------------------------------------------------------
%=========================================================
\section{Recombination}\todotag{Finish}
%=========================================================
%------------------------------------------------------------

The main process affecting recombination
\begin{align}\label{eq:interactions_relevant_recombination}
    &e^- + p \leftrightarrow H + \gamma \quad \text{ionization/recombination}\\
    & e^- + \gamma \leftrightarrow e^- + \gamma \quad \text{electron Compton-Thomson scattering}\\
    & {\color{gray} p + \gamma \leftrightarrow p + \gamma\quad \text{proton Compton-Thomson scattering}}\\
    & e^-+p \leftrightarrow e^- + p (+  \gamma )\quad \text{Coulomb scattering (+Bremsstrahlung)}
\end{align}
Note that at temperatures relevant for recombination both proton and electron are highly nonrelativistic, the Compton cross-section $\sigma_X \propto \, m_X^{-2}$ for proton-photon scattering is thus suppressed by a factor $(m_e/m_p)^2 \sim 10^{-6}$ wrt electron-photon scattering and thus nonnegligible.
It is the electron-photon scattering that keeps photons and electrons tightly coupled before recombination. 
On the other hand, Coulomb scattering is highly effeicient to keep electrons and protons tightly coupled, forming a single photon-baryon fluid.

Other process should be considered, such as the two-step recombination of hydrogen via excited states 
\begin{align}{eq:interactions_relevant_recombination_excited_states}
    &e^-+p \Leftrightarrow H^* + \gamma \leftrightarrow H + \gamma + \gamma\\
    &H^* \leftrightarrow H + \gamma \quad \text{(Lyman-$\alpha$ transition)}\,,
\end{align}
and processes involving the nonnegligible fraction of helium, including intermediate steps with excited states,
\begin{align}{eq:interactions_relevant_recombination_helium}
    & e^- + He^{++} \leftrightarrow He^{+} + \gamma\\
    & e^- + He^{+} \leftrightarrow He + \gamma\,.
\end{align}

The bakcground level Bolztamann equation for the $e^- + p \leftrightarrow H + \gamma$ process reads
\begin{align}\label{eq:boltzmann_eq_recombination}
    \frac{dn_e}{dt} + 3 H n_e = - \langle \sigma v \rangle \left( n_e n_p - n_H n_\gamma \frac{n_e^{\text{eq}} n_p^{\text{eq}}}{n_H^{\text{eq}} n_\gamma^{\text{eq}}} \right)\,,
\end{align}
where $n_e$, $n_p$ and $n_H$ are the number densities of free electrons, protons and neutral hydrogen atoms respectively, and $\langle \sigma v \rangle$ is the thermally averaged recombination cross-section times velocity.

In the Saha approximation we assume that the reaction \eqref{eq:interactions_relevant_recombination} is strong enough $\Gamma \gtrsim H$ to maintain equilibrium, so that the rhs of \eqref{eq:boltzmann_eq_recombination} vanishes and we have
\begin{align}\label{eq:saha_equation}
    \frac{n_en_p}{n_H n_\gamma} \simeq \frac{n_e^{\text{eq}} n_p^{\text{eq}}}{n_H^{\text{eq}} n_\gamma^{\text{eq}}}.
\end{align}
To proceed further we assume
\begin{align}
    &\text{charge neutrality}\quad n_e = n_p\,,\\
    &\text{baryon number conservation}\quad n_b = n_p + n_H\,,\\
    &\text{photons follow the equilibrium distribution}\quad n_\gamma = n_\gamma^{\text{eq}}\,.
\end{align}
The latter assumption is justified since photons are still tightly coupled to electrons via Compton scattering and the ratio of phtons to baryyons is very huge $\eta_b = \frac{n_\gamma}{n_b} \sim 10^{9}$, so that the photon distribution is not significantly affected by the much less numerous baryons.
At recombination temperatures $T\sim 1eV$ everything is nonrelativistic, so that we can use nonrelativistic equilibrium number densities with Boltzmann sppression factor.
Finally we use 
\begin{align}
    &\text{photon number is not conserved}\quad \mu_\gamma=0\\
    &\text{chemical equilibrium}\quad \mu_H =\mu_H+\mu_\gamma= \mu_p + \mu_e\,,\\
    &\text{binding energy}\quad \epsilon_0 := m_p + m_e - m_H = 13.6 eV\,.
\end{align}
Defining the ionization fraction $X_e = n_e/n_b$, from \eqref{eq:saha_equation} we then obtain the Saha equation for the ionization fraction
\begin{align}\label{eq:saha_equation_ionization_fraction}
    \frac{1 - X_e}{X_e^2} = \frac{4 \sqrt{2} \zeta(3)}{\sqrt{\pi}} \left( \frac{m_e T}{2 \pi} \right)^{3/2} \frac{e^{- \epsilon_0/T}}{n_b}\,.
\end{align}

We can understand qualitatively recombination from \eqref{eq:saha_equation_ionization_fraction}.
At high temperatures $T \gg \epsilon_0$ the exponential suppression is negligible and the rhs is very small due to the huge photon-to-baryon ratio, so that $X_e \simeq 1$ and the universe is fully ionized.
As the temperature drops below the hydrogen binding energy $T \lesssim \epsilon_0$ the exponential suppression becomes important and the rhs grows large, so that $X_e \ll 1$ and the universe recombines.
However, due to the huge photon-to-baryon ratio, recombination is delayed down to temperatures $T_{\text{rec}} \sim 0.3 eV  \sim \tfrac{1}{40}\epsilon_0$ since the high-energy tail of the photon distribution is still able to ionize hydrogen atoms.