% !TeX root = ../algebra_main.tex
%====================================
%====================================
\chapter{Representations}\label{ch:representations}
%====================================
%====================================

\begin{mytheorem}[Symmetries \& Representations in Physics]
%
Symmetries have become a key idea in modern physics, and they are an indispensable tool for the construction of new physical theories. They play an important role in the formulation of practically all established physical theories, from Classical/Relativistic Mechanics, Electrodynamics to General Relativity and Particle Physics.

The word "symmetry" used in a physics context (usually) refers to the mathematical structure of a group.
For example, besides the standard Poincaré/Lorentz invariance of all such theories, one encounters internal (continuous) groups such as $S U(3)$ in QCD, $S U(5)$ and $S O(10)$ in grand unified theories (GUTs), and $E_{6}$ and $E_{8}$ in string theory. Discrete groups also play an important role in particle physics model building, for example in the context of models for fermion masses.

In physics, the typical problem is to construct a theory which "behaves" in a certain defined way under the action of a symmetry, for example, which is invariant. To tackle such a problem we need to know how symmetries act on the basic building blocks of physical theories, and these building blocks are often elements of vector spaces. (Think, for example, of the trajectory $\mathbf{r}(t)$ of a particle in Classical Mechanics which, for every time $t$, is an element of $\mathbb{R}^{3}$, a four-vector $x^{\mu}(t)$ which is an element of $\mathbb{R}^{4}$ or the electric and magnetic fields which, at each point in space-time, are elements of $\mathbb{R}^{3}$.) Hence, we need to study the action of groups on vector spaces and the mathematical theory dealing with this problem is called (linear) representation theory of groups. The translation between physical and mathematical terminology is summarised in the diagram below.
\vspace{-5mm}

\begin{center}
\begin{tikzpicture}[font=\normalsize]
% --- light-blue background box ---
\node[draw=none, inner sep=4pt] (box) {
  \begin{tikzpicture}[baseline]
    \matrix (m) [
      matrix of nodes,
      nodes={align=center, text depth=0.25ex},
      column sep=22mm,
      row sep=2.5mm
    ]{
      \textbf{physics} & \textbf{mathematics} \\
      symmetry         & group               \\
      $\downarrow$     & $\downarrow$        \\
      building blocks  & vector spaces       \\
    };

    % --- middle "cong" symbols ---
    \node at ($(m-2-1)!0.5!(m-2-2)$) {$\cong$};
    \node at ($(m-4-1)!0.5!(m-4-2)$) {$\cong$};
  \end{tikzpicture}
};
% --- side labels (outside the box) ---
\node[left=6mm of box.west]  (L) {action on};
\node[right=6mm of box.east] (R) {representation};
\end{tikzpicture}
\end{center}
%
\end{mytheorem}


%--------------------------------------------
%==========================================
\section{Basics of Representation Theory}
%=============================================
%----------------------------------------------

\begin{definition}[Representation of a group]
A representation $R$ of a group $G$ is a group homomorphism $R: G \rightarrow \mathrm{GL}(V)$ where $V$ is a vector space over $\mathbb{F}$, called \emph{base space}.
The dimension of the representation $R$ is defined by $\operatorname{dim}(R)=\operatorname{dim}_{\mathbb{F}}(V)$.
The trivial representation $R_{\text{triv}}$ of a group $G$ is defined by $R_{\text{triv}}(g)=\mathbb{1}$ for all $g \in G$.
\end{definition}


\begin{definition}[Equivalent representations]
Two representations $R_{1}: G \rightarrow \mathrm{GL}\left(V_{1}\right)$ and $R_{2}: G \rightarrow \mathrm{GL}\left(V_{2}\right)$ are called equivalent iff these exists an isomorphism $\phi: V_{1} \rightarrow V_{2}$ such that
\begin{equation*}
R_{1}(g)=\phi^{-1} \circ R_{2}(g) \circ \phi \quad \forall g \in G\,.
\end{equation*}
In this case, we write $R_{1} \cong R_{2}$. Otherwise, the representations are called inequivalent and we write $R_{1} \neq R_{2}$.

(i) Note that the linear map $\phi$ in the above definition is the same for all $g \in G$. Its existence (or otherwise) is, therefore, a non-trivial matter.

(ii) Since $\phi: V_{1} \rightarrow V_{2}$ is an isomorphism (so $\operatorname{dim}_{\mathbb{F}}\left(V_{1}\right)=\operatorname{dim}_{\mathbb{F}}\left(V_{2}\right)$ in particular) two representations $R_{1}$ and $R_{2}$ can only be equivalent if they have the same dimension, $\operatorname{dim}\left(R_{1}\right)=\operatorname{dim}\left(R_{2}\right)$.
\end{definition}


\subsection*{Basic properties of representations}\todotag{Fix}

Often vector spaces $V$ have additional structure, the most prominent one being a s scalar product, $\langle\cdot, \cdot\rangle$. A vector space with a scalar product is also called an inner product vector space. On such an inner product space we can consider the invertible linear maps $f \in \mathrm{GL}(V)$ which leave the scalar product invariant, that is, the maps $f \in \mathrm{GL}(V)$ which satisfy $\langle f(v), f(w)\rangle=\langle v, w\rangle$ for all $v, w \in V$. Such linear maps are called unitary and the set of all unitary linear maps (relative to the given scalar product), denoted $\mathrm{U}(V)$, forms a subgroup of $\operatorname{GL}(V)$. If $V=\mathbb{R}^{n}$ with the dot product, then the unitary maps are the orthogonal $n \times n$ matrices, denoted $\mathrm{O}(n)=\mathrm{U}\left(\mathbb{R}^{n}\right)$. For $V=\mathbb{C}^{n}$ with the standard hermitian scalar product the unitary maps are precisely the unitary $n \times n$ matrices $\mathrm{U}(n)=\mathrm{U}\left(\mathbb{C}^{n}\right)$.

Exercise 1.8. Let $V$ be an inner product vector space. Show that the set $\mathrm{U}(V)$ of unitary maps forms a subgroup of $\mathrm{GL}(V)$.

For an inner product vector space $V$ we can consider special representations where all representation maps are contained in the subgroup $\mathrm{U}(V) \subset \mathrm{GL}(V)$.

Definition 1.10. (Unitary representations) $A$ representation $R: V \rightarrow \mathrm{GL}(V)$ on an inner product vector space $V$ is called unitary if $R(g)$ is a unitary linear map for all $g \in G$ (or, equivalently, iff $\operatorname{Im}(R) \subset \mathrm{U}(V)$ ).

The following is just a somewhat redundant but often-used piece of terminology.
Definition 1.11. (Faithful representation) A representation $R: G \rightarrow \mathrm{GL}(V)$ is called faithful iff it is one-to-one, or, equivalently, iff $\operatorname{Ker}(R)=\{e\}$.


%-----------------------------------------------
%==================================================
\section{Finite groups}
%==================================================
%--------------------------------------------

\begin{theorem}
%
Any representation of a finite group $G$ is equivalent to a unitary representation wrt a suitable inner product on the base space.
For example, ig $G\to GL(V)$ and $\langle\cdot,\cdot\rangle$ is any inner product on $V$, then the inner product
\begin{equation*}
\langle v, w\rangle_{G}:=\frac{1}{|G|} \sum_{g \in G}\langle R(g) v, R(g) w\rangle
\end{equation*}
is invariant under the action of $G$, i.e. $\langle R(g) v, R(g) w\rangle_{G}=\langle v, w\rangle_{G}$ for all $g \in G$ and $v, w \in V$.
%
In particular we have that the dual representation $R^{*}: G \rightarrow \mathrm{GL}\left(V^{*}\right)$ defined by
\begin{equation*}
V^*\ni f\mapsto R^{dual}(g)f:=f\circ R\left(g^{-1}\right)
\end{equation*}
obeys that $\chi_{R^{dual}}(g)=\overline{\chi_{R}(g)}$ for all $g \in G$ since...
\end{theorem}



%======================================================
\subsection{Examples}
%======================================================

%----------------------------------------------
\subsubsection*{Quaternion group $Q_{8}$}
%------------------------------------------------

The conjugacy classes and characters of quaternions are

\noindent
\begin{minipage}[t]{0.49\textwidth}
\centering
\begin{tabular}{|r||c|c|c|c|}
\hline
$g$ & $\pm 1$ & $\pm i$ & $\pm j$ & $\pm k$ \\
\hline\hline
$R_{1}$ & 1 & 1 & 1 & 1 \\
\hline
$R_{i}$ & 1 & 1 & -1 & -1 \\
\hline
$R_{j}$ & 1 & -1 & 1 & -1 \\
\hline
$R_{k}$ & 1 & -1 & -1 & 1 \\
\hline
$R_{2}$ & $\pm \mathbb{I}_{2}$ & $\pm \sigma_i$ & $\pm \sigma_j$ & $\pm \sigma_k$ \\
\hline
\end{tabular}
\end{minipage}\hfill
\begin{minipage}[t]{0.49\textwidth}
\centering
\begin{tabular}{|c|c|c|c|c|c|}
\hline
 & $C_{1}$ & $C_{-1}$ & $C_{i}$ & $C_{j}$ & $C_{k}$ \\
\hline
\# elements & 1 & 1 & 2 & 2 & 2 \\
\hline\hline
$\chi_{1}$ & 1 & 1 & 1 & 1 & 1 \\
\hline
$\chi_{i}$ & 1 & 1 & 1 & -1 & -1 \\
\hline
$\chi_{j}$ & 1 & 1 & -1 & 1 & -1 \\
\hline
$\chi_{k}$ & 1 & 1 & -1 & -1 & 1 \\
\hline
$\chi_{2}$ & 2 & -2 & 0 & 0 & 0 \\
\hline
\end{tabular}
\end{minipage}
for the (rearranged) choice of Pauli matrices 
\begin{align}
\begin{aligned}
  \sigma_{i}=\begin{pmatrix}
        0 & i \\
        i & 0
    \end{pmatrix} , \quad
    \sigma_{j}=\begin{pmatrix}
        0 & 1 \\
        -1 & 0
    \end{pmatrix} , \quad
    \sigma_k=\begin{pmatrix}
        -i & 0 \\
        0 & i
    \end{pmatrix},\quad \sigma_\alpha\sigma_\beta =-\delta_{\alpha\beta}\mathbb{1}+ \epsilon_{\alpha\beta\gamma}\sigma_\gamma , \quad \epsilon_{ijk}=1 .
\end{aligned}
\end{align}




\begin{mytheorem}[Example:Yukawa model building for fermion masses with finite groups]
As a model building application, consider a Yukawa term of the form.
\begin{align}\label{eq:yukawa_coupling_fermion_masses}
\begin{aligned}
    \lambda_{i j} H \bar{\psi}_{L}^{i} \psi_{R}^{j} \subset \mathcal{L}_{\text{Yukawa}} \, ,
\end{aligned}
\end{align}
where $H$ is a complex scalar field and $\psi_{L}^{i}, \psi_{R}^{i}$ are left- and right-handed fermions and $i=1,2,3$ is a family index.
(Details of the structure of these fields will be discussed later, in the section on the Lorentz group, but this will not be important for the present discussion.)
The fields might be the quarks, $\left(\psi_{L}^{i}, \psi_{R}^{i}\right)=\left(u_{L}^{i}, u_{R}^{i}\right)$ or $\left(d_{L}^{i}, d_{R}^{i}\right)$ or the leptons $\left(\psi_{L}^{i}, \psi_{R}^{i}\right)=\left(e_{L}^{i}, e_{R}^{i}\right)$.
The Yukawa couplings times the vacuum expectation value of the field $H$ (the Higgs field) lead to a mass matrix, $M_{i j}=\lambda_{i j}\langle H\rangle$, which determines the masses (and mixings) of the particles.

One avenue of model building, aiming at an explanation of fermion masses and mixings, is to constrain the couplings \eqref{eq:yukawa_coupling_fermion_masses} by imposing a discrete symmetry under which $H, \psi_{L}^{i}$ and $\psi_{R}^{i}$ transform.
Without any ambition of producing a realistic model, we would like to illustrate how this works in principle, using the quaternion group $\mathbb{H}$.

From its character table it is easy to extract the Clebsch--Gordan decompositions.
\begin{align}
\begin{aligned}
    R_{x} \otimes R_{x}=R_{1}
    \qquad
    R_{i} \otimes R_{j}=R_{k} \text{ and cyclic}
    \qquad
    R_{x} \otimes R_{2}=R_{2}
    \qquad
    R_{2} \otimes R_{2}=R_{1} \oplus R_{i} \oplus R_{j} \oplus R_{k} ,
\end{aligned}
\end{align}
where $x=i,j,k$, and this information is the basis for building a model.
To be specific, assume the following assignment of representations.
\begin{align}
\begin{aligned}
    H \sim R_{i}, \quad \bar{\psi}_{L}^{3} \sim R_{j}, \quad \psi_{R}^{3} \sim R_{k}, \quad \bar{\chi}_{L}:=\left(\psi_{L}^{1}, \psi_{L}^{2}\right) \sim R_{2}, \quad \chi_{R}:=\left(\psi_{R}^{1}, \psi_{R}^{2}\right) \sim R_{2} .
\end{aligned}
\end{align}

In other words, we arrange the first two families, for both left- and right-handed fields, into the two-dimensional representations and all other fields into one-dimensional ones.
(There are of course many other ways to do this.)
With this assignment we should only keep the terms in Eq.~\eqref{eq:yukawa_coupling_fermion_masses} which are $\mathbb{H}$-invariant.
Since $H \bar{\psi}_{L}^{3} \chi_{R} \sim R_{i} \otimes R_{j} \otimes R_{2}=R_{2}$ we learn immediately that $\lambda_{31}=\lambda_{32}=0$.
For similar reasons, $\lambda_{13}=\lambda_{23}=0$.
So right away, the Yukawa matrix is restricted to.
\begin{align}
\begin{aligned}
    \lambda=\begin{pmatrix}
        \star & \star & 0 \\
        \star & \star & 0 \\
        0 & 0 & \star
    \end{pmatrix} ,
\end{aligned}
\end{align}
where the $\star$ indicates a potentially non-zero entry.
In fact, the $\lambda_{33}$ entry is allowed since $R_{i} \otimes R_{j} \otimes R_{k}=R_{1}$.
What about the $2 \times 2$ block?
First note that the 2x2 identity and Pauli matrices (or better the rearranged form used for the $R_2$ representation) plus  form a basis of the space of $2 \times 2$ complex matrices.
The 2x2 Yukawa block can then be written as $\lambda_{2 \times 2}=\sum_{\alpha=0, i, j, k} c_{\alpha} \sigma_{\alpha}$ with complex coefficients.
Since $H \sim R_{i}$ only the representation $R_{i} \in R_{2} \otimes R_{2}= R_0\oplus R_i \oplus R_j \oplus R_k$ is allowed, and we should choose $\lambda$ so as to extract this subspace.
Namely, the matrix $\lambda$ make the 2x2 block transform as
\begin{align}
  \bar{\psi}_L^i\lambda_{ij}\psi_R^j\mapsto \bar{\psi}_L^a (\sigma_g^*)^i_a\lambda_{ij}(\sigma_g)^j_b \psi_R^b = \bar{\psi}_L^a (\sigma_g^\dagger \lambda \sigma_g)_{ij} \psi_R^j,
\end{align}
and we should impose this matches the $R_i$ representation
\begin{align}
  \sigma_g^\dagger \lambda \sigma_g = \begin{cases}
  \lambda & \text{if } g=\pm1, \pm i \\[2pt]
  -\lambda & \text{if } g=\pm j,\pm k
  \end{cases}
\end{align}
Using that $\sigma^dagger_g=-sigma_g$ for $g=\pm i, \pm j, \pm k$ and the algebra of the $\sigma$ matrices, we find that only $\lambda \propto\, \sigma_j$ obey the condition.
Hence, the final form of the Yukawa matrix is.
\begin{align}
\begin{aligned}
    \lambda=\begin{pmatrix}
        0 & \lambda_{1} & 0 \\
        -\lambda_{1} & 0 & 0 \\
        0 & 0 & \lambda_{3}
    \end{pmatrix} .
\end{aligned}
\end{align}
\end{mytheorem}





%--------------------------------------------
%==========================================
\section{Lie groups \& Lie algebras}
%=============================================
%----------------------------------------------



(1) Is it always true that the tangent map at $e$ of a group homomorphism (a representation)

(2) Are all Lie algebra homomorphisms tangent maps at $e$ of group homomorphisms?

(3) Can the Lie group be "recovered" from the Lie algebra?

This essentially follows from the Frobenius theorem.
Indeed, if $f \colon T_{e}G \to T_{e}H$ is a Lie algebra homomorphism, then $\operatorname{im} f \le T_{e}H$ is closed under the Lie bracket, because $[f(x),f(y)] = f([x,y])$.
For each $h \in H$ define the subspace
\begin{align}
\begin{aligned}
    \mathcal{D}_{h} := d(l_{h})_{e}\big(\operatorname{im} f\big) \le T_{h}H .
\end{aligned}
\end{align}
Then $\mathcal{D} := \bigcup_{h\in H} \mathcal{D}_{h} \le TH$ is a left-invariant distribution, and it is involutive (closed under brackets of local sections).
By Frobenius, through $e \in H$ there exists an immersed connected submanifold $N \le H$ such that $T_{e}N = \operatorname{im} f$ and $TN = \mathcal{D}\!\mid_{N}$.

Now define a map $\phi \colon G \to H$ locally near $e$ by
\begin{align}
\begin{aligned}
    \phi\big(\exp_{G}(x)\big) := \exp_{H}\big(f(x)\big) .
\end{aligned}
\end{align}
The Lie algebra properties ensure that $\phi$ is (locally) a group homomorphism.
One then extends $\phi$ to the whole identity component of $G$ using the group property.
