% !TeX root = ../cosmology_main.tex
%=========================================================
%=========================================================
\chapter{Dark Matter Production}\label{ch:dm_production}
%========================================================
%=========================================================


There are various mechanisms to produce dark matter in the early universe. We can broadly classify them into thermal and non-thermal mechanisms as follows:\todotag{Redo}
\begin{itemize}
    \item \textbf{Thermal mechanisms}: In these scenarios, dark matter particles were once in thermal equilibrium with the Standard Model particles in the early universe. As the universe expanded and cooled, the interaction rates dropped below the Hubble expansion rate, leading to a "freeze-out" of dark matter particles. The relic abundance of dark matter is determined by the annihilation cross-section and mass of the dark matter particles. A classic example of this mechanism is the Weakly Interacting Massive Particle (WIMP) paradigm.
    \item \textbf{Non-thermal mechanisms}: In these scenarios, dark matter particles were never in thermal equilibrium with the Standard Model particles. Instead, they were produced through processes such as the decay of heavier particles, phase transitions, or other non-equilibrium processes. The relic abundance in these cases depends on the specifics of the production mechanism rather than thermal freeze-out. An example of this mechanism is the production of axions through the misalignment mechanism.
\end{itemize}


\begin{mytheorem}[Back of the envelope estimates for DM production]
%
Even without the full machinery of Boltzmann equations, we can get rough estimates for the relic abundance of dark matter produced via thermal and non-thermal mechanisms by considering the relevant interaction rates and the expansion rate of the universe.
\end{mytheorem}


%--------------------------------------------------------
%=========================================================
\section{Thermal mechanisms}
%=========================================================
%------------------------------------------------------------


%===========================================================
\subsection{Freeze-out}
%===========================================================



%--------------------------------------------------------
%=========================================================
\section{Non-thermal mechanisms}
%=========================================================
%------------------------------------------------------------


%===========================================================
\subsection{Freeze-in}
%===========================================================


