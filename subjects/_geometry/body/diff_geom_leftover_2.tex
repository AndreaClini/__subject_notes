% !TeX root = ../geometry_main.tex
%==========================================================
%=========================================================
\chapter{Towards General Relativity}
%=========================================================
%=========================================================



%-----------------------------------------------------------
%=======================================================
\section{Conservation laws}\todotag{To be written}
%=======================================================
%------------------------------------------------------------


\begin{mytheorem}[What even are energy and mometum?]
In flat spacetime physics we are used to notions like energy and momentum.
To make sense of similar notions in curved spacetime, we first need to investigate what these quantities really are.
Energy and momentum are the Noether charges corresponding symmetries of spacetime, namely time-translation and space-translation invariance of the laws of physics.
Indeed the equations we write for flat spacetimes  typically respect these symmetries, and hence we have conservation of energy and momentum.

In curved spacetime, even if we write down equations that do not explicitly depend on spacetime position, the curvature of spacetime itself can break these symmetries i.e. in practice the metric $g_{\mu\nu}$ may depend on spacetime position.
This means that in general curved spacetimes we may not have global conservation of energy and momentum, and in fact we may not even be able to define them properly.

In any case, we will \emph{identify} energy and momentum with the Noether charges associated to time-translation and space-translation invariance, whenever these symmetries exist.
In practice, whenever we have a timelike Killing vector field, we can define a conserved energy, and whenever we have spacelike Killing vector fields, we can define a conserved momentum in that direction.
\end{mytheorem}


\begin{mytheorem}[Killing vector field]\todotag{finish}
A Killing vector field $K$ is a vector field whose flow generates an isometry of the metric manifold i.e. the Lie derivative of the metric along $K$ vanishes
\begin{align}
    \mathcal{L}_K g = 0.
\end{align}
In components, this is equivalent to
\begin{align}
    \nabla_\mu K_\nu + \nabla_\nu K_\mu = 0.
\end{align}

The number of Killing vector fields of a spacetime is a measure of its symmetries.
In $n$-dimensional spacetimes, the maximum number of linearly independent Killing vector fields is $n(n+1)/2$.
This number is intuitively clear: first, we have $n$ local--i.e. in the tangent space--directions of translation symmetry and $n(n-1)/2$ directions of rotation symmetry; second, $n(n+1)/2$ is precisely the number of independent components of the metric tensor $g_{\mu\nu}$, so that exceeding this number would overconstrain the metric.
Spacetimes achieving this maximum number, like Minkowski, de Sitter and anti-de Sitter, are called \emph{maximally symmetric}.
\end{mytheorem}



\begin{mytheorem}[Energy-momentum tensor of continuum distributions \& conservation laws] \todotag{Finish}
The energy-momentum tensor $T^{\mu\nu}$ is a symmetric rank-2 tensor that encodes the density and flux of energy and momentum in spacetime.
In flat spacetime we define it directly as the Noether current associated to invariance of the action under spacetime translations 
\begin{align}
    x^\mu &\to x^\mu + \epsilon^\mu,\qquad \phi_a(x) \to \phi_a(x) + S^a_b \phi_b(x)- \epsilon^\mu \partial_\mu \phi_a(x).
\end{align}
%
In general spacetimes we do not have such symmetries precisely because the nontrivial curvature $g_{\mu\nu}(x)$ might obstruct them.
It is then intuitive that \emph{whatever we identify with energy and momentum} should be very \emph{sensitive to variations of the metric} $g_{\mu\nu}$, since indeed a trivial metric $g_{\mu\nu}=\eta_{\mu\nu}$ would immediately yield such conserved quantities.
We thus define the energy-momentum as the tensorial quantity identified in the variation of the matter action under $g_{\mu\nu}\mapsto g_{\mu\nu} + \delta g_{\mu\nu}$\todotag{Fix}
\begin{align}
    \delta S_\text{mat} = \delta \left(\int \mathrm{d}^dx\,\sqrt{|g|}\,\,\mathcal{L}_\text{mat}\right)
    =: \frac{1}{2} \int \mathrm{d}^d x \sqrt{-g}\, T^{\mu\nu}_\text{mat}\,\,\delta g_{\mu\nu}.
\end{align}
This quantity is indeed a tensor since $\sqrt{|g|}\, \mathrm{d}^d x $ is the volume form, $\delta g^{\mu\nu}$ is a tensor and the action is a scalar.
The quantity $T_{\mu\nu}$ must then transform as tensor in order to keep $T_{\mu\nu} \delta g^{\mu\nu}$ a scalar.

Using $\delta g_{\mu\nu} = - g_{\mu\alpha} g_{\nu\beta} \delta g^{\alpha\beta}$, we reconcile the different sign conventions in the literature
{\small
\begin{align}
    T^{\mu\nu} = \frac{2}{\sqrt{|g|}} \frac{\delta S_\text{mat}}{\delta g_{\mu\nu}}
    = \frac{2}{\sqrt{|g|}} \frac{\delta S_\text{mat}}{\delta g^{\alpha\beta}} \frac{\delta g^{\alpha\beta}}{\delta g_{\mu\nu}}
    = - \frac{2}{\sqrt{|g|}} g^{\mu\alpha} g^{\nu\beta} \frac{\delta S_\text{mat}}{\delta g^{\alpha\beta}}
    ,\qquad
    T_{\alpha\beta} = g_{\alpha\mu} g_{\beta\nu} T^{\mu\nu}
    = -\frac{2}{\sqrt{|g|}} \frac{\delta S_\text{mat}}{\delta g^{\mu\nu}}.
\end{align}
}
In terms of the matter lagrangian 
{\small
\begin{align}
    T_{\mu\nu} = -\frac{2}{\sqrt{|g|}} \frac{\delta}{\delta g^{\mu\nu}} \left(\sqrt{|g|} \mathcal{L}_\text{mat}\right)
    = -\frac{2}{\sqrt{|g|}} \left( \sqrt{|g|} \frac{\delta \mathcal{L}_\text{mat}}{\delta g^{\mu\nu}} -\frac{1}{2}\sqrt{|g|}g_{\mu\nu}\mathcal{L}_\text{mat} \right)
    = -2 \frac{\delta \mathcal{L}_\text{mat}}{\delta g^{\mu\nu}} + g_{\mu\nu} \mathcal{L}_\text{mat}.
\end{align}
}
\smallskip

The energy-momentum tensor obey the conservation equation
\begin{align}
    \nabla_\mu T^{\mu\nu} = 0,\qquad \nu=0,\ldots, n-1,
\end{align}
which is a consequence of diffeomorphism invariance of the matter action.
It is crucial to stress this should \emph{not} be interpreted as global conservation law, since in curved spacetimes we generally do not even have global energy and momentum definitions.
These $n$ equations are rather just \emph{unavoidable geometric constraints} that follows from demanding the matter action be invariant under diffeomorphisms $x\mapsto \tilde{x}(x)$.
Note that we have precisely $n$ consraints, as expected since we have $n$ DoFs for the general diffeomorphism in $n$-dimensions.
\end{mytheorem}


%---------------------------------------------------------------------
{\color{red}

\begin{mytheorem}[Action for point-particles]\todotag{fix}
The action for a massive particle moving along a geodesics wordline $\lambda \mapsto x^\mu(\lambda)$ is simply given by
\begin{align}
    S_{pp}=-m\int \mathrm{d}\tau = -\int d\lambda,
\end{align}
where $\tau$ is the proper time along the worldline defined by $\mathrm{d}\tau^2 = - g_{\mu\nu} \mathrm{d}x^\mu \mathrm{d}x^\nu$ and $\lambda =\tau/m$, so that the momentum is $p^\mu = \frac{dx^\mu}{d\lambda}$.

For massless particles, we have to be more careful in writing down the action using a Lagrange multiplier to enforce the null condition.
In full generality $m\geq 0$ we write
\begin{align}
    S_{pp} = \int \mathrm{d}\lambda \frac{1}{2}\left( \frac{1}{e(\lambda)} g_{\mu\nu} \frac{\mathrm{d}x^\mu}{\mathrm{d}\lambda} \frac{\mathrm{d}x^\nu}{\mathrm{d}\lambda} - e(\lambda) m^2 \right),
\end{align}
where $e(\lambda)$ is an einbein chosen to enforce the mass-shell condition along the worldline.
Varying with respect to $e(\lambda)$ gives indeed
{\small
\begin{align}
    \delta S_{pp} = -\int \mathrm{d}\lambda \, \frac{1}{2} \left( \frac{1}{e(\lambda)^2} g_{\mu\nu} \frac{\mathrm{d}x^\mu}{\mathrm{d}\lambda} \frac{\mathrm{d}x^\nu}{\mathrm{d}\lambda} + m^2 \right) \delta e(\lambda).
\end{align}
}
Varying with respect to the trajectory $x^\mu(\lambda)$ instead gives the geodesic equation
{\small
\begin{align}
    \delta S_{pp} &= \int \mathrm{d}\lambda  \left( - \frac{\mathrm{d}}{\mathrm{d}\lambda} \left( \frac{1}{e(\lambda)} g_{\mu\nu} \frac{\mathrm{d}x^\nu}{\mathrm{d}\lambda} \right) + \frac{1}{2 e(\lambda)} \partial_\mu g_{\alpha\beta} \frac{\mathrm{d}x^\alpha}{\mathrm{d}\lambda} \frac{\mathrm{d}x^\beta}{\mathrm{d}\lambda} \right)\, \delta x^\mu(\lambda)
    \\
    &=-\int \mathrm{d}\lambda \,\frac{1}{e(\lambda)} g_{\mu\nu}\Bigg[ \frac{\mathrm{d}^2 x^\nu}{\mathrm{d}\lambda^2}  + \underbrace{g^{\nu\rho}\left(\partial_\alpha g_{\beta\rho}-\frac{1}{2} \partial_\nu g_{\alpha\beta} \right)}_{=\Gamma_{\alpha\beta}^\nu} \frac{\mathrm{d}x^\alpha}{\mathrm{d}\lambda} \frac{\mathrm{d}x^\beta}{\mathrm{d}\lambda}-\frac{1}{e(\lambda)}\frac{de}{d\lambda}\frac{d x^\nu}{d\lambda}\Bigg]\, \delta x^\mu(\lambda),
\end{align}
}
up to a term $\propto \frac{\mathrm{d}e(\lambda)}{\mathrm{d}\lambda}$ that can be absorbed in a redefinition of the parameter $\lambda$ to keep $e$ constant along the worldline.
Namely, we have that $p^\mu \propto \frac{\mathrm{d}x^\mu}{\mathrm{d}\lambda}$ obeys the geodesic equation and this identifies $\lambda$ up to affine transformations $\lambda \mapsto a \lambda + b$.
We then choose $\lambda$ so that indeed ...
Substituting back the solution for $e(\lambda)$

yields the mass-shell constraint $g_{\mu\nu} \frac{\mathrm{d}x^\mu}{\mathrm{d}\lambda} \frac{\mathrm{d}x^\nu}{\mathrm{d}\lambda} + m^2 e(\lambda)^2 =0$.
In the massless case $m=0$ this simply enforces the null condition $g_{\mu\nu} \frac{\mathrm{d}x^\mu}{\mathrm{d}\lambda} \frac{\mathrm{d}x^\nu}{\mathrm{d}\lambda} =0$. 
\end{mytheorem}

\begin{mytheorem}[Energy-momentum tensor of point particles]\todotag{fix}
Note that we can equally well consider point particles instead of continuum distributions of matter.
For massive particles--time-like geodesics--we simply take $\lambda=\tau/m$ where $\tau$ is the proper time along the worldline  $\partial_\tau x^\mu \partial_\tau x_\mu=-1$, while for massless particles--i.e. null geodesics $\partial_\lambda x^\mu \partial_\lambda x_\mu=0$--the parameter $\lambda$ is \emph{defined} so that the geodesic equation holds.
The energy-momentum tensor of a point particle of mass $m$ moving along a worldline $x^\mu(\lambda)$ is given by
\begin{align}
    T^{\mu\nu}(x) = m \int \mathrm{d}\lambda \, \frac{1}{\sqrt{|g|}} \delta^{(n)}(x - x(\lambda)) \, \frac{\mathrm{d}x^\mu}{\mathrm{d}\lambda} \frac{\mathrm{d}x^\nu}{\mathrm{d}\lambda}.
\end{align}
\end{mytheorem}
}

%-----------------------------------------------------------



\begin{mytheorem}[Conserved quantities along Killing vector fields: point \& continuum distributions]\todotag{Finish}
Consider a particle moving along a geodesic $x^\mu(\lambda)$ with $4$-momentum $p^\mu = m\,\mathrm{d}x^\mu/\mathrm{d}\lambda$.
Note that $\lambda$ has energy-dimension $[E]^{-2}$.
For massive particles--timelike geodesics--we simply take $\lambda=\tau/m$ where $\tau$ is the proper time along the worldline  $\partial_\tau x^\mu \partial_\tau x_\mu=-1$.
For massless particles--i.e. null geodesics $\partial_\lambda x^\mu \partial_\lambda x_\mu=0$--the parameter $\lambda$ is \emph{defined} so that the geodesic equation holds, up to affine transformations $\lambda \mapsto a \lambda + b$; the rescaling $\lambda \mapsto a \lambda$ changes the momentum $p^\mu \mapsto p^\mu/a$ and is fixed demanding a specific energy normalization $E=p^\mu U_\mu$ with respect to an observer with $4$-velocity $U^\mu$.
If $K^\mu$ is a Killing vector field, the following quantity is conserved along the geodesic
\begin{align}
    Q := K_\mu \,p^\mu, \qquad \frac{\mathrm{d}Q}{\mathrm{d}\lambda} = 0.
\end{align}
For a continuum distribution of matter with stress-energy tensor $T^{\mu\nu}$, the following current is conserved
\begin{align}
    J^\mu := T^{\mu\nu} K_\nu,\qquad \nabla_\mu J^\mu = 0.
\end{align}
\end{mytheorem}
\begin{proof}
For the point like particle
{\small
\begin{align}
    \frac{\mathrm{d}}{\mathrm{d}\lambda}(K_\mu p^\mu) &= p^\nu \nabla_\nu (K_\mu p^\mu) = p^\nu (\nabla_\nu K_\mu) p^\mu + K_\mu\, p^\nu (\nabla_\nu p^\mu)
     =\tfrac{1}{2}p^\mu p^\nu \underbrace{(\nabla_\nu K_\mu + \nabla_\mu K_\nu)}_{=0} + K_\mu\, \underbrace{p^\nu \nabla_\nu p^\mu}_{=0}= 0.
\end{align}
}
For the continuum distribution
{\small
\begin{align}
    \nabla_\mu J^\mu &= \nabla_\mu (T^{\mu\nu} K_\nu) = (\nabla_\mu T^{\mu\nu}) K_\nu + T^{\mu\nu} (\nabla_\mu K_\nu) = \underbrace{(\nabla_\mu T^{\mu\nu})}_{=0} K_\nu + \tfrac{1}{2} T^{\mu\nu} \underbrace{(\nabla_\mu K_\nu + \nabla_\nu K_\mu)}_{=0} = 0.
\end{align}
}
\end{proof}





%-----------------------------------------------------------
%=======================================================
\section{Equations of motion: Einstein \& Euler--Lagrange equations}\todotag{To be written}
%=======================================================
%------------------------------------------------------------


{\color{red}
\begin{mytheorem}[Euler-Lagrange equations in curved spacetime]\todotag{Finish}
Consider a field theory in curved spacetime with action, where $\phi_a$ denotes any multiplet of arbitrary-spin fields,
\begin{align}
    S[\phi] = \int \mathrm{d}^d x \sqrt{|g|} \, \mathcal{L}(\phi_a, \partial_\mu \phi_a, g_{\mu\nu}).
\end{align}
The equations of motion for the field $\phi$ are given by the Euler-Lagrange equations
\begin{align}
    \frac{\partial (\sqrt{|g|} \mathcal{L})}{\partial \phi_a} - \partial_\mu \left( \frac{\partial (\sqrt{|g|} \mathcal{L})}{\partial (\partial_\mu \phi_a)} \right) = 0.
\end{align}
Indeed, varyin with respect to $\phi_a$ gives
{\small
\begin{align}
    \delta S &= \int \mathrm{d}^d x \sqrt{|g|} \left( \frac{\partial \mathcal{L}}{\partial \phi_a} \delta \phi_a + \frac{\partial \mathcal{L}}{\partial (\partial_\mu \phi_a)} \delta (\partial_\mu \phi_a) \right)
    \\
    &= \int \mathrm{d}^d x \sqrt{|g|} \left( \frac{\partial \mathcal{L}}{\partial \phi_a} \delta \phi_a - \partial_\mu \left( \frac{\partial \mathcal{L}}{\partial (\partial_\mu \phi_a)} \right) \delta \phi_a \right) + \text{bdy. terms}
    \\
    &= \int \mathrm{d}^d x \left( \frac{\partial (\sqrt{|g|} \mathcal{L})}{\partial \phi_a} - \partial_\mu \left( \frac{\partial (\sqrt{|g|} \mathcal{L})}{\partial (\partial_\mu \phi_a)} \right) \right) \delta \phi_a + \text{bdy. terms}.
\end{align}
}
Using 
\begin{equation}
    \delta (\partial_\mu \phi_a) = \partial_\mu (\delta \phi_a),
\end{equation}
and the integration by parts formula in curved spacetime
\begin{equation}
    \int \mathrm{d}^d x \sqrt{|g|} \nabla_\mu V^\mu = \text{bdy. terms},
\end{equation}
we arrive at the desired result.
%
\end{mytheorem}
}