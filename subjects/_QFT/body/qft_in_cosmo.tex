% !TeX root = ../QFT_main.tex
%=========================================================
%=========================================================
\chapter{Quantum Field Theory in Curved Spacetime}\label{ch:qft_in_curved_spacetime}
%========================================================
%=========================================================


\begin{mytheorem}[How to formulate QFT in curved spacetime?\todotag{polish}]
%
Canonical quantization is performed in Hamiltonian formalism, which requires a global time coordinate to define equal-time commutation relations, and is thus coordinate-dependent by definition.
We should find a way to identify conjugate variables and commutation relations in a coordinate-independent way.
Alternatively, we can formulate QFT in Lagrangian formalism via path integrals, which is manifestly covariant and does not require singling out a time coordinate.
This is even more powrful since it extends beyond the perturbative regime.
The diagram below summarize the two procedure.

\begin{tikzpicture}[
  font=\small,
  node distance=18mm and 38mm,
  >={Stealth[length=3mm]},
  box/.style={align=left, inner sep=1pt},
  arr/.style={->, line width=1.1pt},
  darr/.style={->, line width=1.1pt, dashed},
  equiv/.style={line width=1.1pt, <->},
  purpletxt/.style={text=purple!70!black},
  greentxt/.style={text=green!45!black},
  orangearr/.style={draw=orange!85!black}
]

% --- Top row (SR) ---
\node[box, purpletxt] (srH) {SR, classical field theory\\Hamiltonian formalism};
\node[box, purpletxt, right=of srH] (srL) {SR, classical field theory\\Lagrangian formalism};

\draw[arr, orangearr] (srH) -- node[above, greentxt, align=center]
  {Legendre transform\\equivalence} (srL);

% --- Middle row (GR, classical) ---
\node[box, purpletxt, below=of srH] (grHc) {GR, classical field theory\\Hamiltonian formalism};
\node[box, purpletxt, below=of srL] (grLc) {GR, classical field theory\\Lagrangian formalism};

\draw[arr, orangearr] (srL) -- node[right, greentxt, align=center]
  {allow curvature \\ \& make covariant} (grLc);

\draw[arr, orangearr] (grLc) -- node[above, greentxt, align=center]
  {Legendre transform\\equivalence} (grHc);

% --- Bottom row (GR, QFT) ---
\node[box, purpletxt, below=of grHc] (grHq) {GR, QFT\\Hamiltonian formalism\\ (Canonical quantization)};
\node[box, purpletxt, below=of grLc] (grLq) {GR, QFT\\Lagrangian formalism\\(Path integral QFT)};

\draw[arr, orangearr] (grHc) -- node[left, greentxt, align=center]
  {as outlined\\already} (grHq);

\draw[arr, orangearr] (grLc) -- node[above, greentxt, align=center]
  {path\\integral} (grLq);

\draw[equiv] (grHq) -- node[below, align=center]
  {\scriptsize Dyson--Schwinger eqs are the same\\[-1pt]\scriptsize equivalence} (grLq);

\end{tikzpicture}
%  
\end{mytheorem}



%--------------------------------------------------------
%=========================================================
\section{Green functions}
%=========================================================
%------------------------------------------------------------