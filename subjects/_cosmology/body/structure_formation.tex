% !TeX root = ../cosmology_main.tex
%=========================================================
%=========================================================
\chapter{Structure Formation}\label{ch:structure_formation}
%========================================================
%=========================================================


\begin{mytheorem}[Metric perturbations remain small]
%
Treating metric perturbations to linear order is justified since they remain very small, even when energy-density perturbations become extremely non-linear.
For example, compare the (dimensionless) gravitational potential\footnote{Identified with the Newtonian-gauge potential $\Phi$} $\phi\sim \frac{GM}{R}$ and the density contrast $\delta\rho/\bar\rho$, relative to the cosmic average matter density today, for various astrophysical objects:
\begin{itemize}
    \item Earth: $\phi\sim 10^{-9}, \quad \tfrac{\delta\rho}{\rho}\sim 10^{30}$
    \item Sun: $\phi\sim 10^{-6}, \quad \tfrac{\delta\rho}{\rho}\sim 10^{28}$
    \item Milky Way: $\phi\sim 10^{-6}, \quad \tfrac{\delta\rho}{\rho}\sim 10^{6}$
    \item Galaxy cluster: $\phi\sim 10^{-5}, \quad \tfrac{\delta\rho}{\rho}\sim 10^{3}$
    \item Black hole at Schwarzschild radius $R_S=2GM$: $\phi\sim \tfrac{1}{2}, \quad \tfrac{\delta\rho}{\rho}\sim \infty$.
\end{itemize}
\end{mytheorem}



%--------------------------------------------------------
%=========================================================
\section{Newtonian theory}
%=========================================================
%------------------------------------------------------------






%--------------------------------------------------------
%=========================================================
\section{Structure formation across epochs}
%=========================================================
%------------------------------------------------------------



%--------------------------------------------------------
%=========================================================
\section{Correlation functions, power spectrum, and BAO}
%=========================================================
%------------------------------------------------------------

The fit of the matter correlation function $\xi(r)$ to the observed galaxy correlation function $\xi^{\text{obs}}(r)$ is one of the most important cosmological observables, and it is used to constrain cosmological parameters.
The best fit is
\begin{align}
    \xi(r)\simeq \left(\frac{r}{r_0}\right)^{-\gamma}+\text{BAO feature at }r\sim 150\text{Mpc},
\end{align}
where $r_0\sim 5.4\text{Mpc}$ and $\gamma\sim 1.8$.
Suitable cutoffs are inserted to remove the unphysical divergence at $r=0$ and to account for the finite size of the universe at large $r$.



