% !TeX root = ../cosmology_main.tex
%=========================================================
%=========================================================
\chapter{Matheatical tools}\label{ch:math_tools}
%========================================================
%=========================================================


%------------------------------------------------------------
%=======================================================================
\section{Spherical harmonics \& Legendre polynomials}
%=================================================================
%---------------------------------------------------------------------



\subsubsection*{Spherical harmonics}

Start from the decomposition
\begin{align}
    L=x\wedge p = -i\hbar (x \wedge \nabla)
\end{align}
Then
\begin{align}
    -\frac{L^2}{\hbar^2}&= (x\wedge\nabla)\cdot(x\wedge\nabla)= x_a \partial_b x_a\partial_b - x_a \partial_b x_b \partial_a
    = r^2\partial^2 - x_a x_b \partial_a \partial_b - 2 x_a \partial_a \\
    & \Rightarrow \quad \nabla^2 = -\frac{L^2}{\hbar^2 r^2} + \frac{1}{r^2}\frac{\partial}{\partial r}\left(r^2 \frac{\partial}{\partial r}\right).
\end{align}

The angular momentum components are
\begin{align}
    L_x&= -i\hbar (y \partial_z - z \partial_y) =
    L_y&= -i\hbar (z \partial_x - x \partial_z) 
    L_z&= -i\hbar (x \partial_y - y \partial_x) = -i\hbar \frac{\partial}{\partial \phi}.
\end{align}
The raising and lowering operators are defined as
\begin{align}
    L_{\pm} = L_x \pm i L_y = \pm e^{\pm i\phi} \left( \frac{\partial}{\partial \theta} \pm i \cot \theta \frac{\partial}{\partial \phi} \right), \quad \text{i.e.}\quad L_{-}=L_{+}^{\dagger}.
\end{align}


We originally introduced angular momentum via the orbital angular momentum operators $\mathbf{J}= \mathbf{L}=\mathbf{X} \wedge \mathbf{P}$ in \eqref{}. 
On the other hand, we know that $\mathbf{X}$ and $\mathbf{P}$ act on wave functions $\psi(\mathbf{x})$ as multiplication by $\mathbf{x}$ and $-\mathrm{i} \hbar \nabla$, respectively, which makes the $\mathbf{L}$ differential operators acting on such wave functions. 
For example,
\begin{align}
\begin{aligned}
	L_{3}=X_{1} P_{2}-X_{2} P_{1}=-\mathrm{i} \hbar\left(x \frac{\partial}{\partial y}-y \frac{\partial}{\partial x}\right)
\end{aligned}
\end{align}
writing $\mathbf{x}=(x, y, z)$. 
As for the harmonic oscillator it must then be possible to find explicit eigenfunctions for these operators, representing the eigenstates of angular momentum we found abstractly in the previous subsection. 
It is convenient to first write the operators $L_{i}$ in terms of spherical polar coordinates $(x, y, z)=( r \sin \theta \cos \phi, r \sin \theta \sin \phi, r \cos \theta )$, and then using the chain rule
\begin{align}
\begin{aligned}
	\frac{\partial}{\partial \phi}=\frac{\partial x}{\partial \phi} \frac{\partial}{\partial x}+\frac{\partial y}{\partial \phi} \frac{\partial}{\partial y}+\frac{\partial z}{\partial \phi} \frac{\partial}{\partial z}=-y \frac{\partial}{\partial x}+x \frac{\partial}{\partial y}
\end{aligned}
\end{align}
Thus
\begin{align}
\begin{aligned}
	L_{3}=-\mathrm{i} \hbar \frac{\partial}{\partial \phi}
\end{aligned}
\end{align}
Recall that the eigenstates $\psi_{m}$ defined in the previous subsection had eigenvalue $m \hbar$ under $J_{3}=L_{3}$, so that $L_{3} \psi_{m}=m \hbar \psi_{m}$ reads
\begin{align}
\begin{aligned}
	\frac{\partial \psi_{m}}{\partial \phi}=\mathrm{i} m \psi_{m}
\end{aligned}
\end{align}
which integrates to
\begin{align}
\begin{aligned}
	\psi_{m}(r, \theta, \phi)=F(r, \theta) \mathrm{e}^{\mathrm{i} m \phi}
\end{aligned}
\end{align}
The coordinate $\phi$ has period $2 \pi$, so in order for the function \eqref{} to be single-valued (well defined) we must have $m \in \mathbb{Z}$ being an integer. 
Correspondingly then also $j \in \mathbb{Z}_{\geq 0}$ must be integer in Theorem 8.6. 
This proves
\textbf{Proposition 8.8} For orbital angular momentum the parameters $j, m$ in Theorem 8.6 must both be integers. 
In this context we instead label $j=\ell \in \mathbb{Z}_{\geq 0}$. 
To pursue this further, let us also work out the raising and lowering operators in terms of spherical polars. 
\textbf{Proposition 8.9} In spherical polar coordinates the raising and lowering operators are
\begin{align}
\begin{aligned}
	L_{+}=L_{1}+\mathrm{i} L_{2}=\hbar \mathrm{e}^{\mathrm{i} \phi}\left(\frac{\partial}{\partial \theta}+\mathrm{i} \cot \theta \frac{\partial}{\partial \phi}\right) \\
	L_{-}=L_{1}-\mathrm{i} L_{2}=-\hbar \mathrm{e}^{-\mathrm{i} \phi}\left(\frac{\partial}{\partial \theta}-\mathrm{i} \cot \theta \frac{\partial}{\partial \phi}\right)
\end{aligned}
\end{align}

\includegraphics[width=\textwidth]{images/spherical_harmonics.png}



\subsection{Legendre polynomials}


Legendre polynomials are simply defined as spherical harmonics with $m=0$ up to a normalization factor:
\begin{align}
    P_{\ell}(\cos \theta) = \sqrt{\frac{4\pi}{2\ell + 1}} Y_{\ell 0}(\theta, \phi).
\end{align}
Since spherical harmonics are eigenfunctions of the angular momentum operators, Legendre polynomials are also eigenfunctions of the $L^2$ operator.
Since $\{Y_{\ell m}\}$ form a CONS of $L^2(S^2)$, the set $\{P_{\ell}\}$ also form a CONS of the subspace of $L^2(S^2)$ consisting of functions independent of $\phi$ or equivalently of $L^2([-1, 1], d\cos \theta)$.
Since $Y_{\ell,m}^*=(-1)^m Y_{\ell,-m}$, Legendre polynomials are real $P_{\ell}^* = P_{\ell}$.
From the orthonormality of spherical harmonics $\int_{S^2} Y_{\ell m}^* Y_{\ell' m'} d\phi d\cos\theta = \delta_{\ell \ell'} \delta_{m m'}$ we derive the orthogonality relation for Legendre polynomials:
\begin{align}
    \int_{-1}^{1} P_{\ell}(\mu) P_{\ell'}(\mu) d\mu = \frac{2}{2\ell + 1} \delta_{\ell \ell'}.
\end{align}

We then expand functions $f(\mu)$ in Legendre polynomials, where it is ofetn convenient to absorb powers of $i$ in the expansion coefficients,
\begin{align}
    f(\mu) = \sum_{\ell=0}^{\infty} \frac{2\ell + 1}{i^\ell} f_{\ell} P_{\ell}(\mu), \quad \text{where}\quad f_{\ell} = \frac{i^\ell}{2} \int_{-1}^{1} f(\mu) P_{\ell}(\mu) d\mu.
\end{align}

They satisfy the following recursion relations:
\begin{align}
    \mu P_\ell(\mu) = \frac{\ell+1}{2\ell + 1} P_{\ell+1}(\mu) + \frac{\ell}{2\ell + 1} P_{\ell-1}(\mu), \quad \text{and}\quad \frac{dP_\ell}{d\mu} = \frac{\ell}{1-\mu^2} (P_{\ell-1}(\mu) - \mu P_\ell(\mu)).
\end{align}

The first 3 polynomials are
\begin{align}
    P_0(\mu) = 1, \quad P_1(\mu) = \mu, \quad P_2(\mu) = \frac{1}{2}(3\mu^2 - 1).
\end{align}