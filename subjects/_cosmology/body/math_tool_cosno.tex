% !TeX root = ../cosmology_main.tex
%=========================================================
%=========================================================
\chapter{Matheatical tools}\label{ch:math_tools}
%========================================================
%=========================================================



%------------------------------------------------------------
%=======================================================================
\section{Gaussian random fields}\todotag{Check \& finish}
%=================================================================
%---------------------------------------------------------------------

A Gaussian random field is a stochastic process
\begin{align}
    X:\Omega\times U \to F\,\mid \omega,u\mapsto X(\omega,u),
\end{align}
such that for any finite collection of parameters $u_1,\dots,u_n\in U$ the random vector $(X(\cdot,u_1),\dots,X(\cdot,u_n))$ is Gaussian distributed in $F^n$.
Crucially, the elements of a gaussian random vectors are independent if and only if they are uncorrelated, i.e. if the off-diagonal elements of the covariance matrix vanish.

Quantum mechanical processes naturally induce nearly Gaussian random fields, since free quantum fields are a collection of independent harmonic oscillators and interactions are small perturbations thereof.
The wave function of each oscillator is a Gaussian times Hermite polynomials, or simply a Gaussian for the ground state, so that the field wave function is simply the product of these wave functions, which is again a Gaussian up to polynomial factors.

Another key feature is that a Gaussian random field is always ergodic, since the covariance function decays to zero at large distances, so that the law of large numbers applies and ensemble averages can be replaced with parameter averages.

A key feature of Gaussian random fields is that they are completely determined by their mean and covariance functions.
In particular, all reduced n-point functions vanish and all n-point functions can be expressed in terms of the 2-point function via Wick's theorem.
\begin{align}
    \langle X(\cdot,u_1)\cdots X(\cdot,u_n)\rangle_\Omega = \sum_{\text{pairings}} \prod_{\text{pairs }(i,j)} \langle X(\cdot,u_i)X(\cdot,u_j)\rangle_\Omega.
\end{align}
In particular homogeneity requires the mean function is constant $\langle X(\cdot,u)\rangle_\Omega = \mu$, and isotropy further requires the covariance function depends only on the distance $|u-u'|$ between parameters $\langle X(\cdot,u)X(\cdot,u')\rangle_\Omega = \xi(|u-u'|)$.
For simplicity we subtract off the constant mean $\mu$ and consider real field $\delta\in\mathbb{R}$ in the discussion below.

A Gaussian random field in real space is a Gaussian random field in Fourier space and viceversa.
Notably, if the correlation function in real space is homogeneous $\xi(x,y)=\xi(x-y)$, then the Fourier modes are independent, i.e. the covariance function in Fourier space is always diagonal 
{\small\begin{align}
    \langle \delta_{\vec{k}} \delta_{\vec{k}'}\rangle_\Omega = (2\pi)^3 \delta^{(3)}(\vec{k}+\vec{k}') P(\vec{k}), \quad \text{where}\quad P(\vec{k}) := \int d^3\vec{x}e^{-i\vec{k}\cdot\vec{x}} \xi(\vec{x}).
\end{align}}
This is simply expressing the general fact that spatial homogeneity implies total momentum conservation.
The probability distibution of a single mode is a \emph{complex} Gaussian, since linear combination of Gaussians remain Gaussian, with variance given by the power spectrum $P(\vec{k})$ and \emph{uniformly distributed} random phase
\begin{align}
    \mathbb{P}(\delta_{\vec{k}})\, d^2_{\mathbb{C}}\!\delta_{\vec{k}}
    &= \frac{1}{\sqrt{2\pi P(\vec{k})}} \exp\left(-\frac{|\delta_{\vec{k}}|^2}{2P(\vec{k})}\right) d^2_{\mathbb{C}}\!\delta_{\vec{k}} \\
    &= \frac{|\delta_{\vec{k}}|}{\sqrt{2\pi P(\vec{k})}} \exp\left(-\frac{|\delta_{\vec{k}}|^2}{2P(\vec{k})}\right) d|\delta_{\vec{k}}|\, d\!\arg(\delta_{\vec{k}}).
\end{align}
Since the field $\delta(\vec{x})$ is real then $\delta_{\vec{k}}^* = \delta_{-\vec{k}}$.
The probability distribution of a given set of Fourier modes is just the product of the single mode distributions
{\small\begin{align}\begin{aligned}
    \mathbb{P}(\delta_{\vec{k}_1}\dots \vec{k}_n) \prod_{j=1}^n d^2_\mathbb{C}\!\delta_{\vec{k}_j} &= \prod_{j=1}^n \mathbb{P}(\delta_{\vec{k}_j})\, d^2_\mathbb{C}\!\delta_{\vec{k}_j} 
    \\
    &= \prod_{i=1}^n \frac{1}{\sqrt{2\pi P(k_i)}} \exp\left(-\frac{|\delta_{\vec{k}_i}|^2}{2P(k_i)}\right) d^2_\mathbb{C}\!\delta_{\vec{k}_i} = \prod_{i=1}^n \frac{1}{\sqrt{2\pi P(k_i)}} \exp\left(-\frac{|\delta_{\vec{k}_i}|^2}{2P(k_i)}\right) k_i^2 dk_i d\Omega_i.
\end{aligned}\end{align}}
If isotropy also holds i.e. $\xi(x,y)=\xi(|x-y|)$, the joint law of Fourier modes $\delta_{\vec{k}}$ only depend on the magnitude of momenta $k=|\vec k|$ and $P(\vec{k})\equiv P(k)$ is independent of the direction $\hat{k}$. 
This is the case in Cosmology.




%------------------------------------------------------------
%=======================================================================
\section{Spherical harmonics \& Legendre polynomials}
%=================================================================
%---------------------------------------------------------------------



\begin{mytheorem}[Angular scale of variation of spherical harmonics\todotag{Finish}]
%
Tha spherical harmonic $Y_{\ell m}(\theta, \phi)$ has an angular scale of variation $\Delta \theta \sim \pi/\ell$ in the $\theta$ direction and $\Delta \phi \sim 2\pi/m$ in the $\phi$ direction.
The latter is immediately understood since the $\phi$ dependence is of the form $Y_{\ell m}(\theta, \phi) = F_{\ell m}(\theta) e^{i m \phi}$, so that the function is periodic in $\phi$ with period $2\pi/m$.
The former is less trivial, it can be understood by looking at the explicit form of $Y_{\ell m}(\theta, \phi)$ in terms of associated Legendre polynomials $P_{\ell}^m(\cos \theta)$, which are polynomials of degree $\ell$ in $\cos \theta$ and thus have $\ell$ roots in the interval $[-1,1]$, corresponding to $\ell$ zeros of $Y_{\ell m}(\theta, \phi)$ in the interval $\theta\in[0,\pi]$, so that the typical distance between zeros is $\pi/\ell$.
Alternatively focus on the higher harmonics $Y_{\ell \ell}=(\sin\theta)^\ell e^{i\ell\phi}$, so that
\begin{align}
    \frac{1}{Y_{\ell\ell}}\frac{\partial Y_{\ell\ell}}{\partial \theta}\Delta\theta = \ell \cot\theta \Delta\theta,
\end{align}
so that the angular scale of variation in the $\theta$ direction is $\Delta \theta \sim 1/(\ell \cot\theta)$, which is of order $\pi/\ell$ for $\theta$ not too close to $0$ or $\pi$.
\end{mytheorem}

\subsubsection*{Spherical harmonics}

Start from the decomposition
\begin{align}
    L=x\wedge p = -i\hbar (x \wedge \nabla)
\end{align}
Then
\begin{align}
    -\frac{L^2}{\hbar^2}&= (x\wedge\nabla)\cdot(x\wedge\nabla)= x_a \partial_b x_a\partial_b - x_a \partial_b x_b \partial_a
    = r^2\partial^2 - x_a x_b \partial_a \partial_b - 2 x_a \partial_a \\
    & \Rightarrow \quad \nabla^2 = -\frac{L^2}{\hbar^2 r^2} + \frac{1}{r^2}\frac{\partial}{\partial r}\left(r^2 \frac{\partial}{\partial r}\right).
\end{align}

The angular momentum components are
\begin{align}
    L_x&= -i\hbar (y \partial_z - z \partial_y) =
    L_y&= -i\hbar (z \partial_x - x \partial_z) 
    L_z&= -i\hbar (x \partial_y - y \partial_x) = -i\hbar \frac{\partial}{\partial \phi}.
\end{align}
The raising and lowering operators are defined as
\begin{align}
    L_{\pm} = L_x \pm i L_y = \pm e^{\pm i\phi} \left( \frac{\partial}{\partial \theta} \pm i \cot \theta \frac{\partial}{\partial \phi} \right), \quad \text{i.e.}\quad L_{-}=L_{+}^{\dagger}.
\end{align}


We originally introduced angular momentum via the orbital angular momentum operators $\mathbf{J}= \mathbf{L}=\mathbf{X} \wedge \mathbf{P}$ in \eqref{}. 
On the other hand, we know that $\mathbf{X}$ and $\mathbf{P}$ act on wave functions $\psi(\mathbf{x})$ as multiplication by $\mathbf{x}$ and $-\mathrm{i} \hbar \nabla$, respectively, which makes the $\mathbf{L}$ differential operators acting on such wave functions. 
For example,
\begin{align}
\begin{aligned}
	L_{3}=X_{1} P_{2}-X_{2} P_{1}=-\mathrm{i} \hbar\left(x \frac{\partial}{\partial y}-y \frac{\partial}{\partial x}\right)
\end{aligned}
\end{align}
writing $\mathbf{x}=(x, y, z)$. 
As for the harmonic oscillator it must then be possible to find explicit eigenfunctions for these operators, representing the eigenstates of angular momentum we found abstractly in the previous subsection. 
It is convenient to first write the operators $L_{i}$ in terms of spherical polar coordinates $(x, y, z)=( r \sin \theta \cos \phi, r \sin \theta \sin \phi, r \cos \theta )$, and then using the chain rule
\begin{align}
\begin{aligned}
	\frac{\partial}{\partial \phi}=\frac{\partial x}{\partial \phi} \frac{\partial}{\partial x}+\frac{\partial y}{\partial \phi} \frac{\partial}{\partial y}+\frac{\partial z}{\partial \phi} \frac{\partial}{\partial z}=-y \frac{\partial}{\partial x}+x \frac{\partial}{\partial y}
\end{aligned}
\end{align}
Thus
\begin{align}
\begin{aligned}
	L_{3}=-\mathrm{i} \hbar \frac{\partial}{\partial \phi}
\end{aligned}
\end{align}
Recall that the eigenstates $\psi_{m}$ defined in the previous subsection had eigenvalue $m \hbar$ under $J_{3}=L_{3}$, so that $L_{3} \psi_{m}=m \hbar \psi_{m}$ reads
\begin{align}
\begin{aligned}
	\frac{\partial \psi_{m}}{\partial \phi}=\mathrm{i} m \psi_{m}
\end{aligned}
\end{align}
which integrates to
\begin{align}
\begin{aligned}
	\psi_{m}(r, \theta, \phi)=F(r, \theta) \mathrm{e}^{\mathrm{i} m \phi}
\end{aligned}
\end{align}
The coordinate $\phi$ has period $2 \pi$, so in order for the function \eqref{} to be single-valued (well defined) we must have $m \in \mathbb{Z}$ being an integer. 
Correspondingly then also $j \in \mathbb{Z}_{\geq 0}$ must be integer in Theorem 8.6. 
This proves
\textbf{Proposition 8.8} For orbital angular momentum the parameters $j, m$ in Theorem 8.6 must both be integers. 
In this context we instead label $j=\ell \in \mathbb{Z}_{\geq 0}$. 
To pursue this further, let us also work out the raising and lowering operators in terms of spherical polars. 
\textbf{Proposition 8.9} In spherical polar coordinates the raising and lowering operators are
\begin{align}
\begin{aligned}
	L_{+}=L_{1}+\mathrm{i} L_{2}=\hbar \mathrm{e}^{\mathrm{i} \phi}\left(\frac{\partial}{\partial \theta}+\mathrm{i} \cot \theta \frac{\partial}{\partial \phi}\right) \\
	L_{-}=L_{1}-\mathrm{i} L_{2}=-\hbar \mathrm{e}^{-\mathrm{i} \phi}\left(\frac{\partial}{\partial \theta}-\mathrm{i} \cot \theta \frac{\partial}{\partial \phi}\right)
\end{aligned}
\end{align}

\includegraphics[width=\textwidth]{images/spherical_harmonics.png}



\subsection{Legendre polynomials}


Legendre polynomials are simply defined as spherical harmonics with $m=0$ up to a normalization factor:
\begin{align}
    P_{\ell}(\cos \theta) = \sqrt{\frac{4\pi}{2\ell + 1}} Y_{\ell 0}(\theta, \phi).
\end{align}
Since spherical harmonics are eigenfunctions of the angular momentum operators, Legendre polynomials are also eigenfunctions of the $L^2$ operator.
Since $\{Y_{\ell m}\}$ form a CONS of $L^2(S^2)$, the set $\{P_{\ell}\}$ also form a CONS of the subspace of $L^2(S^2)$ consisting of functions independent of $\phi$ or equivalently of $L^2([-1, 1], d\cos \theta)$.
Since $Y_{\ell,m}^*=(-1)^m Y_{\ell,-m}$, Legendre polynomials are real $P_{\ell}^* = P_{\ell}$.
From the orthonormality of spherical harmonics $\int_{S^2} Y_{\ell m}^* Y_{\ell' m'} d\phi d\cos\theta = \delta_{\ell \ell'} \delta_{m m'}$ we derive the orthogonality relation for Legendre polynomials:
\begin{align}
    \int_{-1}^{1} P_{\ell}(\mu) P_{\ell'}(\mu) d\mu = \frac{2}{2\ell + 1} \delta_{\ell \ell'}.
\end{align}

We then expand functions $f(\mu)$ in Legendre polynomials, where it is ofetn convenient to absorb powers of $i$ in the expansion coefficients,
\begin{align}
    f(\mu) = \sum_{\ell=0}^{\infty} \frac{2\ell + 1}{i^\ell} f_{\ell} P_{\ell}(\mu), \quad \text{where}\quad f_{\ell} = \frac{i^\ell}{2} \int_{-1}^{1} f(\mu) P_{\ell}(\mu) d\mu.
\end{align}

They satisfy the following recursion relations:
\begin{align}
    \mu P_\ell(\mu) = \frac{\ell+1}{2\ell + 1} P_{\ell+1}(\mu) + \frac{\ell}{2\ell + 1} P_{\ell-1}(\mu), \quad \text{and}\quad \frac{dP_\ell}{d\mu} = \frac{\ell}{1-\mu^2} (P_{\ell-1}(\mu) - \mu P_\ell(\mu)).
\end{align}

The first 3 polynomials are
\begin{align}
    P_0(\mu) = 1, \quad P_1(\mu) = \mu, \quad P_2(\mu) = \frac{1}{2}(3\mu^2 - 1).
\end{align}