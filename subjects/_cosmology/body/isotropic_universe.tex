% !TeX root = ../cosmology_main.tex
%=========================================================
%=========================================================
\chapter{Isotropic and Homogeneous Universe}
%========================================================
%=========================================================


%---------------------------------------------------
%==================================================
\section{Cosmological Principle \& FLRW Metric}
%==================================================
%--------------------------------------------------

%------------------------------------------------------------
\subsubsection{A simple derivation of constant curvature spaces}\todotag{See baumann}
%-------------------------------------------------------------


The \textbf{Cosmological Principle} states that the Universe is homogeneous and isotropic on large scales. The properties of the Universe are thus same everywhere and in all directions on sufficiently large scales (greater than about 100 Mpc).
The CMB is the main experimental evidence for isotropy and LSS surveys support homogeneity.


Isotropy and homogeneity imply the sectional curvature of each 2d tangent plane at each point is the same.
The spatial sections are thus constant-curvature 3d spaces, classified by the curvature $K$.
The most general metric consistent with the Cosmological Principle is the \textbf{Friedmann-Lemaître-Robertson-Walker (FLRW) metric}
\begin{equation}
    ds^2 = -dt^2 + a^2(t) \left[ \frac{dr^2}{1-Kr^2} + r^2 \underbrace{(d\theta^2 + \sin^2\theta d\phi^2)}_{d\Omega^2} \right]
    = -dt^2 + a^2(t) \left[ d\chi^2 + S_K^2(\chi) d\Omega^2 \right],
\end{equation}
The relation between the comoving distance $\chi$ and the metric distance $r$ is given by
\begin{equation}
    r(\chi) = \begin{cases}
        \frac{1}{\sqrt{K}} \sin(\sqrt{K}\chi) & K > 0 \\
        \chi & K = 0 \\
        \frac{1}{\sqrt{|K|}} \sinh(\sqrt{|K|}\chi) & K < 0
    \end{cases},
\end{equation}


The Christoffel symbols for the FLRW metric are
\begin{equation}
    \Gamma^0_{ij} = a\dot{a} \gamma_{ij}, \quad
    \Gamma^i_{0j} = \frac{\dot{a}}{a} \delta^i_j, \quad
    \Gamma^i_{jk} = \tilde{\Gamma}^i_{jk},
\end{equation}
where $\tilde{\Gamma}^i_{jk}$ are the Christoffel symbols of the spatial metric $\gamma_{ij}$.


The Friedmann equations governing the dynamics of the scale factor $a(t)$ are
\begin{equation}
    \left( \frac{\dot{a}}{a} \right)^2 = \frac{8\pi G}{3} \rho - \frac{K}{a^2} + \frac{\Lambda}{3},
\end{equation}
\begin{equation}
    \frac{\ddot{a}}{a} = -\frac{4\pi G}{3} (\rho + 3p) + \frac{\Lambda}{3}.
\end{equation}

The Hublle constant today is
\begin{align}
    H_0 = 100 h \; \mathrm{km \, s^{-1} \, Mpc^{-1}} \approx\, h \,3.3 \times 10^{-3} \, \mathrm{Mpc^{-1}}
\end{align}
The Friedmann equation are written in terms of the density parameters as
\begin{equation}
    H^2 = H_0^2 \left[ \sum_x \Omega_x a^{-3(1+w_x)} + \Omega_K a^{-2} + \Omega_\Lambda \right],
\end{equation}
where
\begin{equation}
    \Omega_r = \frac{8\pi G \rho_{r0}}{3H_0^2}, \quad
    \Omega_m = \frac{8\pi G \rho_{m0}}{3H_0^2}, \quad
    \Omega_\Lambda = \frac{\Lambda}{3H_0^2}, \quad
    \Omega_K = 1 - \sum_x \Omega_x - \Omega_\Lambda.
\end{equation}



%===========================================================
\subsection{Distances in Cosmology}
%===========================================================




%=====================================================
\subsection{Back of the envelope calculations}
%======================================================


\begin{mytheorem}[The age of the Universe\todotag{Finsh}]
Constraining the cosmological parameters $\Omega_x$ we can get an estimate for the age of the universe from Friedman equation
\begin{align} 
    \frac{(da/dt)}{a} = H_0 \Big[\sum_x \Omega_x a^{-3(1+w_x)} \Big]^{1/2} \quad \Rightarrow \quad T_{\text{uni}}=\int dt = \int_0^1\frac{da}{a H_0 \Big[\sum_x \Omega_x a^{-3(1+w_x)} \Big]^{1/2}}.
\end{align}
Inserting approximate values we ger $T_{\text{uni}} \approx 13.8$ Gyr.
%
\end{mytheorem}


\begin{mytheorem}[Size of the ovserbale universe: particle horizon vs Hubble radius\todotag{Finsh}]
Similarly we can estimate the size of the observable universe as the comoving distance traveled by a photon since the Big Bang, i.e. the particle horizon
\begin{align}
    \dots
\end{align}
For polynomial expansion $a(t) \sim t^\beta$ we find this is similar to the Hubble radius $d_H = c/H(t)$ up to a factor $\beta/(1-\beta)$.
    
\end{mytheorem}





%=====================================================
\subsection{Fine tuning problems}
%======================================================



%-------------------------------------------------
\subsubsection*{The cosmological constant problem}\todotag{Polish}
%------------------------------------------------------


The effect of a cosmological constant in the EH action $\int \sqrt{-g} (R-2\Lambda^\mathrm{bare})$ can be reabsorbed as a vacuum energy contribution to the stress-energy tensor
\begin{align}
    G_{\mu\nu}+\Lambda^\mathrm{bare} g_{\mu\nu} = 8\pi G T_{\mu\nu}  \quad \Rightarrow \quad G_{\mu\nu} = 8\pi G \left(T_{\mu\nu} + T^\Lambda_{\mu\nu}\right), \quad T^\Lambda_{\mu\nu} = -\frac{\Lambda^\mathrm{bare}}{8\pi G} g_{\mu\nu}\,.
\end{align}
Quantum vacuum energy also contributes to the stress tensor as a cosmological constant $\Lambda_{\mathrm{vac}}$ i.e. $\rho_{\mathrm{vac}}= -\frac{\Lambda_{\mathrm{vac}}}{8\pi G} = -p_{\mathrm{vac}}$, the total effective cosmological constant is then
\begin{align}
    \Lambda_{\mathrm{eff}} = \Lambda_\mathrm{bare} + \Lambda_{\mathrm{vac}}\,.
\end{align}
The theoretical prediction for vacuum energy density from QFT is $\rho_{\mathrm{vac}} \sim M_{\mathrm{Pl}}^4 \sim 10^{76} \, \mathrm{GeV}^4$, while the experimentally observed value from cosmology is $\rho_{\Lambda} = \Omega_\Lambda \rho_{\mathrm{crit} 0} \approx (0.7)\tfrac{3H_0^2}{8\pi G}\sim 10^{-47} \, \mathrm{GeV}^4$.
This discrepancy of about 123 orders of magnitude is known as the cosmological constant problem, representing a significant fine-tuning issue since the \emph{bare} vacuum energy entering Einstein equations must cancel the \emph{quantum} vacuum energy to an extraordinary degree of precision to yield the observed value of $\Lambda_{\mathrm{eff}}$
\begin{align}
    \frac{\Lambda_{\mathrm{bare}}}{\Lambda_{\mathrm{vac}}} = -1 + \frac{\Lambda_{\mathrm{eff}}}{\Lambda_{\mathrm{vac}}} \simeq -1 + 10^{123}.
\end{align}
The equation of state of vacuum energy is also reason of debate.
The interpretation as vacuum energy implies $w=-1$, but dynamical DE models with $w\neq -1$ are also possible.
The current experimental value from Planck 2018 is $w=-1.03\pm 0.03$.


%-------------------------------------------------
\subsubsection*{The flatness problem}
%------------------------------------------------------


%-------------------------------------------------
\subsubsection*{The horizon problem}
%------------------------------------------------------



%-------------------------------------------------------------
%===============================================================
\section{Homogeneous Boltzmann Equation \& Thermal History}\todotag{Finish}
%===============================================================
%-------------------------------------------------------------------


In Newtonian mechanics the Boltzmann equation reads
\begin{align}\label{eq:newtonian_boltzmann_eq}
\frac{df}{dt}=\frac{\partial f}{\partial t} + \frac{\vec{p}}{m} \cdot \partial_{\vec{x}} f + \vec{F} \cdot \partial_{\vec{p}} f = C[f]\,.
\end{align}
Making this covariant we have
\begin{align}\label{eq:covariant_boltzmann_eq}
    \frac{df}{d\lambda} = \frac{dx^\mu}{d\lambda} \frac{\partial f}{\partial x^\mu} + \frac{dP^\mu}{d\lambda} \frac{\partial f}{\partial P^\mu} = C[f]\,,
\end{align}
where $\lambda$ is an affine parameter along the particle's worldline with energy dimension $[\lambda]=[E]^{-2}$ \emph{defined} to enforce the geodesic equation holds
\begin{align}
    P^\mu=\frac{dx^\mu}{d\lambda},\quad \frac{dP^\mu}{d\lambda} + \Gamma^\mu_{\alpha\beta} P^\alpha P^\beta =0\,.
\end{align}








%----------------------------------------------------------
\subsubsection{Overview of interactions \& their relevance}
%-------------------------------------------------------


\begin{align}
    e^++e^- \leftrightarrow \gamma + \gamma \quad \text{pair production/annihilation}
\end{align}
This process is relevant at temperatures $T \gtrsim m_e \sim 0.5$ MeV, and thus for the thermal history of the universe before BBN.
After electron-positron annihilation there is not enough energy to produce $e^+e^-$ pairs, blocking the $\leftarrow$ direction, and soonafter there won't be enough positrons for the $\rightarrow$ direction.
This interaction would indeed change number density and momentum bwteen the two components of the photon-baryon fluid but is blocked at later times.

\begin{align}
    \nu + \bar{\nu} \leftrightarrow e^+ + e^- \quad \text{neutrino decoupling}
\end{align}
This process is relevant at temperatures $T \gtrsim 1$ MeV, and thus for the thermal history of the universe before BBN, but negligible at later times.

\begin{align}
    e^-+p \leftrightarrow e^-+ p (+\gamma)\quad \text{Coulomb scattering (+Bremsstrahlung)}
\end{align}
This process is relevant at all times, but especially at late times when it keeps electrons and protons tightly coupled, forming a single photon-baryon fluid.

\begin{align}
    n\to p + e^- + \bar{\nu}_e \quad \text{neutron decay}
\end{align}
Free neutrons are unstable and decay with a lifetime of about 15 minutes.
This process would thus have plenty of time to occur from the end of inflation down to recombination $\sim 380.000$ years after the Big Bang.
However, shortly aftern BBN all neutrons have either decayed, or are bound in nuclei and are thus stable, making this process negligible at later times.












%-------------------------------------------------------------
%===============================================================
\section{Big Bang Nucleosynthesis}\todotag{Finish}
%===============================================================
%-------------------------------------------------------------------




%--------------------------------------------------------
%=========================================================
\section{Recombination}\todotag{Finish}
%=========================================================
%------------------------------------------------------------

The main process affecting recombination
\begin{align}\label{eq:interactions_relevant_recombination}
    &e^- + p \leftrightarrow H + \gamma \quad \text{ionization/recombination}\\
    & e^- + \gamma \leftrightarrow e^- + \gamma \quad \text{electron Compton-Thomson scattering}\\
    & {\color{gray} p + \gamma \leftrightarrow p + \gamma\quad \text{proton Compton-Thomson scattering}}\\
    & e^-+p \leftrightarrow e^- + p (+  \gamma )\quad \text{Coulomb scattering (+Bremsstrahlung)}
\end{align}
Note that at temperatures relevant for recombination both proton and electron are highly nonrelativistic, the Compton cross-section $\sigma_X \propto \, m_X^{-2}$ for proton-photon scattering is thus suppressed by a factor $(m_e/m_p)^2 \sim 10^{-6}$ wrt electron-photon scattering and thus nonnegligible.
It is the electron-photon scattering that keeps photons and electrons tightly coupled before recombination. 
On the other hand, Coulomb scattering is highly effeicient to keep electrons and protons tightly coupled, forming a single photon-baryon fluid.

Beware there is no vanilla ``recombination'' process, and other should be considered.
In fact the naive recombination directly into the ground state is highly inefficient since the high-energy photon emitted would immediately ionize another hydrogen atom.
One solution is to emit two photons in the recombination process $e^- + p \to H + \gamma + \gamma$, but this is a higher order process, and thus suppressed with much smaller rate.
Alternatively, hydrogen could recombine into excited states $H^*$ and then emit another Lyman-$\alpha$ photon to reach the ground state.
In formulas
\begin{align}\label{eq:interactions_relevant_recombination_excited_states}
    &e^- + p \to H + \gamma + \gamma \text{(two photon emission)}\\
    &e^-+p \Leftrightarrow H^* + \gamma \leftrightarrow H + \gamma + \gamma\\
    &H^* \leftrightarrow H + \gamma \quad \text{(Lyman-$\alpha$ transition)}\,,
\end{align}
Moreover we should consider processes involving the nonnegligible fraction of helium, also including intermediate steps with excited states,
\begin{align}\label{eq:interactions_relevant_recombination_helium}
    & e^- + He^{++} \leftrightarrow He^{+} + \gamma\\
    & e^- + He^{+} \leftrightarrow He + \gamma\,.
\end{align}

The famous Peebles equations \cite{} take into account all these processes to accurately describe recombination, but ultimately give the same qualitative picture
\begin{align}
    ....
\end{align}

The background level Boltzmann equation for the $e^- + p \leftrightarrow H + \gamma$ process reads
\begin{align}\label{eq:boltzmann_eq_recombination}
    \frac{dn_e}{dt} + 3 H n_e = - \langle \sigma v \rangle \left( n_e n_p - n_H n_\gamma \frac{n_e^{\text{eq}} n_p^{\text{eq}}}{n_H^{\text{eq}} n_\gamma^{\text{eq}}} \right)\,,
\end{align}
where $n_e$, $n_p$ and $n_H$ are the number densities of free electrons, protons and neutral hydrogen atoms respectively, and $\langle \sigma v \rangle$ is the thermally averaged recombination cross-section times velocity.

In the Saha approximation we assume that the reaction \eqref{eq:interactions_relevant_recombination} is strong enough $\Gamma \gtrsim H$ to maintain equilibrium, so that the rhs of \eqref{eq:boltzmann_eq_recombination} vanishes and we have
\begin{align}\label{eq:saha_equation}
    \frac{n_en_p}{n_H n_\gamma} \simeq \frac{n_e^{\text{eq}} n_p^{\text{eq}}}{n_H^{\text{eq}} n_\gamma^{\text{eq}}}.
\end{align}
To proceed further we assume
\begin{align}
    &\text{charge neutrality}\quad n_e = n_p\,,\\
    &\text{baryon number conservation}\quad n_b = n_p + n_H\,,\\
    &\text{photons follow the equilibrium distribution}\quad n_\gamma = n_\gamma^{\text{eq}}\,.
\end{align}
The latter assumption is justified since photons are still tightly coupled to electrons via Compton scattering and the ratio of phtons to baryyons is very huge $\eta_b = \frac{n_\gamma}{n_b} \sim 10^{9}$, so that the photon distribution is not significantly affected by the much less numerous baryons.
At recombination temperatures $T\sim 1eV$ everything is nonrelativistic, so that we can use nonrelativistic equilibrium number densities with Boltzmann sppression factor.
Finally we use 
\begin{align}
    &\text{photon number is not conserved}\quad \mu_\gamma=0\\
    &\text{chemical equilibrium}\quad \mu_H =\mu_H+\mu_\gamma= \mu_p + \mu_e\,,\\
    &\text{binding energy}\quad \epsilon_0 := m_p + m_e - m_H = 13.6 eV\,.
\end{align}
Defining the ionization fraction $X_e = n_e/n_b$, from \eqref{eq:saha_equation} we then obtain the Saha equation for the ionization fraction
\begin{align}\label{eq:saha_equation_ionization_fraction}
    \frac{1 - X_e}{X_e^2} = \frac{4 \sqrt{2} \zeta(3)}{\sqrt{\pi}} \left( \frac{m_e T}{2 \pi} \right)^{3/2} \frac{e^{- \epsilon_0/T}}{n_b}\,.
\end{align}

We can understand qualitatively recombination from \eqref{eq:saha_equation_ionization_fraction}.
At high temperatures $T \gg \epsilon_0$ the exponential suppression is negligible and the rhs is very small due to the huge photon-to-baryon ratio, so that $X_e \simeq 1$ and the universe is fully ionized.
As the temperature drops below the hydrogen binding energy $T \lesssim \epsilon_0$ the exponential suppression becomes important and the rhs grows large, so that $X_e \ll 1$ and the universe recombines.
However, due to the huge photon-to-baryon ratio, recombination is delayed down to temperatures $T_{\text{rec}} \sim 0.3 eV  \sim \tfrac{1}{40}\epsilon_0$ since the high-energy tail of the photon distribution is still able to ionize hydrogen atoms.

When the temperature drops further, the reaction rate $\Gamma$ becomes smaller than the Hubble expansion rate $H$ and the Saha approximation breaks down.
To accurately describe recombination in this regime we need to solve the full Boltzmann equation \eqref{eq:boltzmann_eq_recombination}, taking into account all the relevant processes.
With the same manipulations and assumptions as above, we can rewrite the Boltzmann equation in terms of the ionization fraction
\begin{align}\label{eq:boltzmann_eq_ionization_fraction}
    \frac{dX_e}{dt} = - \langle \sigma v \rangle n_b \left( X_e^2 - (1 - X_e) \frac{n_e^{\text{eq}} n_p^{\text{eq}}}{n_H^{\text{eq}} n_\gamma^{\text{eq}}} \right)\,.
\end{align}


%==========================================================
\subsubsection{Optical depth \& visibility function}
%==========================================================