% !TeX root = ../geometry_main.tex
%==========================================================
%=========================================================
\chapter{Pending points}
%=========================================================
%=========================================================


%=====================================================
\section{Conservation laws}
%=====================================================

\begin{itemize}
    \item $\nabla_\mu T^{\mu\nu} = 0$ is a law of math, not of physics. If it where of the form $\nabla_\mu V^\mu$ then it would describe some conservation law and Gauss theorem could be applied, showing the net flux of $V^\mu$ through a closed surface is zero, with the second indices is not true.
    %\item the number of killing is 10, namely 4 boosts, 6=3+3 generalized rotations
    \item furthermore it reflects the fact that we have 16 cohordinates 4x4, while we vary only wrt the symmetric part of the metric so that the antisymmetric bit n(n-1)/2=4(4-1)/2=6 are not dynamical, hence we have 10 d.o.f. and 4 constraints, which are just the 4 equations $\nabla_\mu T^{\mu\nu}=0$.
\end{itemize}

\begin{mytheorem}[static and stationary spacetime, frobeinus conditions]
    
\end{mytheorem}

\begin{mytheorem}[Energy conditions]
\begin{itemize}
    \item weak energy condition
    \item dominant energy condition
    \item strong energy condition
\end{itemize}
    
\end{mytheorem}

%=====================================================
\section{Equation of motion}
%=====================================================

%-------------------------------------------------------
\subsection{Matter}
%-------------------------------------------------------

\begin{itemize}
    \item the true ELE are
    \begin{align}
        \frac{\partial\mathcal{L}}{\partial \psi}-\nabla_\mu\left(\frac{\partial\mathcal{L}}{\partial(\nabla_\mu\psi)}\right)=0
    \end{align}
    where $\psi$ is an arbitrary-spin matter field, not with $\partial_\mu$ but with $\nabla_\mu$ covariant derivative.
\end{itemize}



%-------------------------------------------------------
\subsection{Einstein equation}
%-------------------------------------------------------


\begin{mytheorem}[Sakarov derivation of Einstein equation]
\label{thm:sakarov_einstein}
The Einstein equation can be derived from the Sakharov idea of induced gravity, where the gravitational action is generated by quantum corrections from matter fields.
    
\end{mytheorem}