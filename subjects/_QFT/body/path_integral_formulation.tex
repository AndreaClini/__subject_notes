% !TeX root = ../QFT_main.tex
%=========================================================
%=========================================================
\chapter{Path Integral Formulation}\label{ch:path_integral_formulation}
%========================================================
%=========================================================



%--------------------------------------------------------
%=========================================================
\section{Generating functional}
%=========================================================
%------------------------------------------------------------


\begin{mytheorem}[Quantum theory is to classical theory what Fourier transform is to Legendre transform.\notetag{Polish}]
In formulas
    \begin{equation}
    \text{Quantum theory} \,\,:\,\, \text{classical theory} = \text{Fourier transform}\,\, :\,\, \text{Legendre transform}
\end{equation}
The analogy is made precise in the path integral formulation of QFT via the generating functional $W$ of a field theory
\begin{align}
    e^{iW[J]} &= \int \mathcal{D}\phi\, e^{iS[\phi]} e^{-i\int d^4x\, J(x)\phi(x)}.
\end{align}
In this sense $W[J]$ is the infinite-dimensional Fourier transform of the functional $\phi\mapsto e^{iS[\phi]}$ and $J(x)$ is the conjugate variable to the field $\phi(x)$.
The generating functional is the generator of all connected diagrams, and its knowledge is thus equivalent to the knowledge of the full quantum theory.

The \emph{stationary phase approximation} consists in approximating the path integral by the configuration that contributes the most to the integral, i.e. the one extremizing the action and thus reducing the oscillations of the integrand,
\begin{align}
    &\int \mathcal{D}\phi\, e^{iS[\phi]-\int d^4x\, J(x)\phi(x)} \sim e^{iS[\phi_{cl}]-\int d^4x\, J(x)\phi_{cl}(x)}\quad \text{for $\,\phi_{cl}\,$ solving}\quad \left.\frac{\delta S}{\delta \phi(x)}\right|_{\phi_{cl}} = J(x)\\
    &\quad \Rightarrow \quad W[J] \sim S[\phi_{cl}] - \int d^4x\, J(x)\phi_{cl}(x)\,.
\end{align} 
By definition this approximates $W[J]$ with the Legendre transform of the classical action.

Finally, recalling that the \emph{effective action} $\Gamma[\phi]$ is defined as the Legendre transform of $W[J]$
\begin{align}
    \Gamma[\phi] = W[J] - \int d^4x\, J(x)\phi(x) \quad \text{for $J$ solving}\quad \frac{\delta W}{\delta J(x)} = \phi(x) \quad \xrightarrow[]{\text{stationary ph.}}\quad \Gamma[\phi] \sim S[\phi].
\end{align}
we see that in stationary phase approximation the effective action reduces to the classical action.

To make the analogy clear, consider the 1-dimensional Fourier transform of a function $f(y) = e^{iS(y)}$, where $S(y), g(z)\in\C$ so that there is no loss of generality in the exponentiation,
\begin{align}
    e^{ig(z)}=\int dy\, e^{iS(y)} e^{-izy}.
\end{align}
Then, in stationary phase approximation, we approximate the Fourier transform with the Legendre transform
\begin{align}
    &\int dy\, e^{iS(y)-izy} \sim e^{iS(y_{cl}) - izy_{cl}}\quad \text{for $\,y_{cl}\,$ solving}\quad \left.\frac{dS}{dy}\right|_{y_{cl}} = z \quad \Rightarrow \quad g(z) \sim S(y_{cl}) - zy_{cl}.
\end{align}
%
\end{mytheorem}