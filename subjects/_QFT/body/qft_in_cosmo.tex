% !TeX root = ../QFT_main.tex
%=========================================================
%=========================================================
\chapter{Quantum Field Theory in Curved Spacetime}\label{ch:qft_in_curved_spacetime}
%========================================================
%=========================================================


\section{Initial consideration \todotag{Finish \& remove sec header}}

\begin{mytheorem}[How to formulate QFT in curved spacetime?\todotag{polish}]
%
Canonical quantization is performed in Hamiltonian formalism, which requires a global time coordinate to define equal-time commutation relations, and is thus coordinate-dependent by definition.
We should find a way to identify conjugate variables and commutation relations in a coordinate-independent way.
Alternatively, we can formulate QFT in Lagrangian formalism via path integrals, which is manifestly covariant and does not require singling out a time coordinate.
This is even more powrful since it extends beyond the perturbative regime.
The diagram below summarize the two procedures.

\begin{tikzpicture}[
  font=\small,
  node distance=18mm and 38mm,
  >={Stealth[length=3mm]},
  box/.style={align=left, inner sep=1pt},
  arr/.style={->, line width=1.1pt},
  darr/.style={->, line width=1.1pt, dashed},
  equiv/.style={line width=1.1pt, <->},
  purpletxt/.style={text=purple!70!black},
  greentxt/.style={text=green!45!black},
  orangearr/.style={draw=orange!85!black}
]

% --- Top row (SR) ---
\node[box, purpletxt] (srH) {SR, classical field theory\\Hamiltonian formalism};
\node[box, purpletxt, right=of srH] (srL) {SR, classical field theory\\Lagrangian formalism};

\draw[arr, orangearr] (srH) -- node[above, greentxt, align=center]
  {Legendre transform\\equivalence} (srL);

% --- Middle row (GR, classical) ---
\node[box, purpletxt, below=of srH] (grHc) {GR, classical field theory\\Hamiltonian formalism};
\node[box, purpletxt, below=of srL] (grLc) {GR, classical field theory\\Lagrangian formalism};

\draw[arr, orangearr] (srL) -- node[right, greentxt, align=center]
  {allow curvature \\ \& make covariant} (grLc);

\draw[arr, orangearr] (grLc) -- node[above, greentxt, align=center]
  {Legendre transform\\equivalence} (grHc);

% --- Bottom row (GR, QFT) ---
\node[box, purpletxt, below=of grHc] (grHq) {GR, QFT\\Hamiltonian formalism\\ (Canonical quantization)};
\node[box, purpletxt, below=of grLc] (grLq) {GR, QFT\\Lagrangian formalism\\(Path integral QFT)};

\draw[arr, orangearr] (grHc) -- node[left, greentxt, align=center]
  {as outlined\\already} (grHq);

\draw[arr, orangearr] (grLc) -- node[above, greentxt, align=center]
  {path\\integral} (grLq);

\draw[equiv] (grHq) -- node[below, align=center]
  {\scriptsize Dyson--Schwinger eqs for \\ correlators are the same\\[-1pt]\scriptsize equivalence} (grLq);

\end{tikzpicture}

The latter equivalence is guaranteed since a theory is completely specified by its correlators, which are uniquely determined by the Dyson--Schwinger equations.


\begin{itemize}
  \item We are used to writing down the usual lagrangian or hamilotnian in flat space because we always use inertial coordinates related by global Lorentz transformations.
  If we used other coordinates, e.g. polar coordinates, the expressions for $\mathcal{L}$ and $\mathcal{H}$ would be different (and possibly more complicated) even in flat spacetime, but the physics would be the same.
  \item This open the door to generalizing to curved spacetimes, where we can write down $\mathcal{L}$ and $\mathcal{H}$ in a covariant way using the metric $g_{\mu\nu}$.
  Write an expression that is invariant under change of coordinates in flat spacetime. Then promote to arbitrary spacetimes.
  \item The lagrangian formulation is manifestly covariant and independent of coordinate choice. The Hamiltonian formulation requires fixing a coordinate system and a time slicing.
  Indeed the very expressions of $\mathcal{H}$ and the conjugate field $\pi$ depend on this choice.
  This seems to break covariance, but in facts only hides it.
  Indeed, combining the two hamilton equations for $\phi$ and its coordinate-dependent conjugate field $\pi$ gives back the covariant Euler--Lagrange equations.
  \item This also clarify that \textbf{each hamiltonian $\mathcal{H}$ is the generator of time translations along the chosen time coordinate}.
  Indeed the hamilton equations in the chosen coordinates $(t,\mathbf{x})$ read
  \begin{align}
    &\frac{\partial \phi(t,\mathbf{x})}{\partial t} = \frac{\delta H}{\delta \pi(t,\mathbf{x})}, \quad
    \frac{\partial \pi(t,\mathbf{x})}{\partial t} = -\frac{\delta H}{\delta \phi(t,\mathbf{x})},\\
    &\quad\text{and more generally }\quad \frac{\partial \mathcal{O}}{\partial t} = \{\mathcal{O}, H\}\quad \forall\,\, \text{field}\quad  \mathcal{O}.
  \end{align}
  Or at the quantum level
  \begin{equation}
    \frac{\partial \hat{\mathcal{O}}}{\partial t} = -\frac{i}{\hbar} [\hat{\mathcal{O}}, \hat{H}]\quad \forall \,\,\text{field operator}\quad \hat{\mathcal{O}}.
  \end{equation}
  This tells us that time translations \emph{parametrized by the chosen coordinate $t$} will be given by the operator
  \begin{align}
    &U(t,t_0) = \mathcal{T}\exp\left(-\frac{i}{\hbar}\int_{t_0}^t dt' H(t')\right),
  \end{align}
  If we where to choose another time coordinate $\tau$, we would get another hamiltonian $\tilde{H}$ generating time translations along $\tau$. 
  The two hamiltonians would be different and related so that translations from $t_0=t(\tau_0)$ to $t=t(\tau)$ are the same as translations from $\tau_0$ to $\tau$
  \begin{equation}
    \mathcal{T}\exp\left(-\frac{i}{\hbar}\int_{t(\tau_0)}^{t(\tau)} dt' H(t')\right)
    =
    \mathcal{T}\exp\left(-\frac{i}{\hbar}\int_{\tau_0}^{\tau} d\tau' \tilde{H}(\tau')\right).
  \end{equation}

  \item We remark we should be careful with what we identify with the conjugate field $\pi$. 
  This indeed does depend of the choice of coordinates and the time slicing. So does the form of $\mathcal{H}$. On the other hand, the form of the lagrangian density $\mathcal{L} \sqrt{|g|}$ is formally the same in any coordinates.
  \item The path integral formulation is manifestly covariant and does not require singling out a time coordinate.
  This is even more powrful since it extends beyond the perturbative regime.
  \item Both formulations are equivalent since they give rise to the same Dyson--Schwinger equations for correlators.
\end{itemize}


\end{mytheorem}






\begin{mytheorem}
%
QFT in curved spacetime requires a careful rethinking of our approach.
Indeed canonical quantization is performed in Hamiltonian formalism, which requires a global time coordinate to define conjugate variables and then equal-time commutation relations.
The Hamiltonian formalism is thus coordinate-dependent by definition, and \emph{would seem} doomed to fail in curved spacetime.
We should find a way to identify conjugate variables and commutation relations in a coordinate-independent way.

To develop the intuition, first note that we are used the familiar expression for the lagrangian or hamiltonian in flat spacetime because we always use inertial `rectangular' coordinates related by global Lorentz transformations.
If we used other coordinates, e.g. polar coordinates, the expressions for $\mathcal{L}$ and $\mathcal{H}$ would be different (and possibly more complicated) even in flat spacetime, but the physics would be the same.
This already suggests how to proceed: we should recast the lagrangian in a covariant form, that is formally invariant under change of coordinates in flat spacetime, and then promote it to arbitrary spacetimes by replacing the Minkowski metric $\eta_{\mu\nu}$ with a general metric $g_{\mu\nu}$.

Having obtained a Lagrangian formulation valid in curved spacetime, we can then derive the Hamiltonian formulation by choosing a coordinate system and a time slicing.
This will give us expressions for $\mathcal{H}$ and the conjugate field $\pi$ that depend on this choice.
While this seems to break covariance, in facts it only hides it.
Indeed, regardless of the coordinate choice and the corresponding expressions, combining the two hamilton equations for $\phi$ and its coordinate-dependent conjugate field $\pi$ gives back the covariant Euler--Lagrange equations.

Alternatively, we can formulate QFT directly in Lagrangian formalism via path integrals, which is manifestly covariant and does not require singling out a time coordinate.
This is even more powrful since it extends beyond the perturbative regime.
The diagram below summarize the two procedures.


\end{mytheorem}






\begin{mytheorem}[Hamiltonian as generator of time translations\todotag{Finish}]
%
\end{mytheorem}













%--------------------------------------------------------
%=========================================================
\section{Green functions}
%=========================================================
%------------------------------------------------------------


\begin{mytheorem}[Dyson--Schwinger equations for correlators in curved spacetime\todotag{polish}]
%
A quantum theory is completely specified by its correlators\footnote{In fact, it is even sufficient to only know time-ordered ones.} (expectation values of products of time-evolved field operators).
Correlators are uniquely determined by the Dyson--Schwinger equations, which are an infinite set of coupled differential equations relating all $n$-point functions to each other for all $n\in\mathbb{N}$.
If two theories have all the same Dyson--Schwinger equations for correlators, they are thus completely equivalent.
In particular this proves the equivalence between canonical quantization and path integral quantization in curved spacetime, at least as long as the former approach is applicable.
\end{mytheorem}