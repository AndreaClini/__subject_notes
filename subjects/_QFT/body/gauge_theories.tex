% !TeX root = ../QFT_main.tex
%=========================================================
%=========================================================
\chapter{Gauge Theories}
%========================================================
%=========================================================


\begin{mytheorem}[The emergence of gauge theories.\todotag{Check \& improve.}]
Gauge theories have become a cornerstone of modern theoretical physics, providing a framework for understanding fundamental interactions. The development of gauge theories was driven by the need to reconcile quantum mechanics with special relativity and to describe the behavior of elementary particles.

\begin{itemize}
    \item \textbf{Electromagnetism.} The first gauge theory, formulated by James Clerk Maxwell, describes the electromagnetic force through the exchange of photons, which are massless gauge bosons. The gauge invariance of electromagnetism leads to the conservation of electric charge.
    
    \item \textbf{Weak and Strong Interactions.} The electroweak theory, unifying electromagnetic and weak forces, and quantum chromodynamics (QCD), describing strong interactions, are both based on non-abelian gauge symmetries. These theories introduce additional gauge bosons, such as W and Z bosons for the weak force, and gluons for the strong force.
\end{itemize}
\end{mytheorem}

\begin{mytheorem}[The significance of gauge invariance.\todotag{Do it}]
Gauge invariance is a fundamental principle that dictates the form of the interactions in gauge theories. It ensures that physical predictions remain unchanged under local transformations of the fields. This principle not only leads to the introduction of gauge fields but also imposes constraints on the dynamics of the system.

The implications of gauge theories extend beyond particle physics, influencing areas such as condensed matter physics and cosmology. The concept of spontaneous symmetry breaking, for instance, plays a crucial role in the Higgs mechanism, which endows particles with mass.

In summary, gauge theories provide a unified framework for understanding the fundamental forces of nature, and their development marks a significant milestone in the quest for a deeper understanding of the universe.
\end{mytheorem}


%----------------------------------------------------------
%=========================================================
\section{Abelian Gauge Theories}
%=========================================================
%----------------------------------------------------------



%----------------------------------------------------------
%=========================================================
\section{Non-abelian Gauge Theories}
%=========================================================
%----------------------------------------------------------


%===========================================================
\subsection{BRST symmetry}
%===========================================================