% !TeX root = ../QFT_main.tex
%=========================================================
%=========================================================
\chapter{History \& Principles of Quantum Mechanics and Quantum Field Theory}
%========================================================
%=========================================================


%=========================================================
\section{The rise of Quantum Mechanics} \todotag{To be written}
%=========================================================

%=========================================================
\section{The fall of Quantum Mechanics \& the rise of QFT} \todotag{To be written}
%=========================================================

Many discoveries in the early 20th century showed that Quantum Mechanics (QM) alone was not sufficient to describe all physical phenomena.

%=========================================================
\section{The guiding theorems} \todotag{To be written}
%=========================================================

\begin{theorem}[\textbf{Heisenberg canonical quantization procedure.}]
Take the classical phase space variables \(q_i, p_i\) i.e. the classical DoFs, satisfying the relevant Equations of Motions and  Poisson bracket relations
\[\{q_i, p_j\} = \delta_{ij}, \quad \{q_i, q_j\} = 0, \quad \{p_i, p_j\} = 0.\]
Then, in the quantum theory, promote these variables to operators \(\hat{q}_i, \hat{p}_i\) acting on a Hilbert space \(\mathcal{H}\), satisfying the canonical commutation relations
\[[\hat{q}_i, \hat{p}_j] = i\hbar \delta_{ij}, \quad [\hat{q}_i, \hat{q}_j] = 0, \quad [\hat{p}_i, \hat{p}_j] = 0.\]
These operators are self-adjoint and obey the same equations of motion as their classical counterparts, but now in the operator formalism.
Namely, they are obtained as
\begin{equation}
    \frac{d\hat{O}}{dt} = \frac{i}{\hbar} [\hat{H}, \hat{O}] + \left(\frac{\partial \hat{O}}{\partial t}\right)_{\text{explicit}}.
\end{equation}
In particular, for the fundamental DoFs, we have
\begin{align}
    \frac{d\hat{q}_i}{dt} = \frac{i}{\hbar} [\hat{H}, \hat{q}_i] \quad \frac{d\hat{p}_i}{dt} = \frac{i}{\hbar} [\hat{H}, \hat{p}_i].
\end{align}
\end{theorem}
