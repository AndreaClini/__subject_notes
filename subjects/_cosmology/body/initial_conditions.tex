% !TeX root = ../cosmology_main.tex
%=========================================================
%=========================================================
\chapter{Inflation \& Initial Conditions}
%========================================================
%=========================================================


%------------------------------------------------------
%=========================================================
\section{Problems solved by inflations}
%=========================================================
%-----------------------------------------------------------


Inflation is a period of accelerated expansion in the early universe, characterized by nearly exponential growth of the scale factor with Hubble parameter $H$ approximately constant,
\begin{align}
    H=\frac{\tfrac{da}{dt}}{a}\approx\text{const}, \quad \mathcal{H}=aH=He^{Ht}=-\tfrac{1}{\tau},\quad a(t)\simeq e^{H_*t}= -\tfrac{1}{H\tau}\quad \text{with}\quad t\in\R,\,\,\tau\in(-\infty,0).
\end{align}
Conformal time $\tau$ is negative during inflation approaching zero as $t\to\infty$.
Notably the comoving Hubble radius $d_\mathcal{H}:=\frac{1}{\mathcal{H}}= a^{-1}\frac{1}{H}$ shrinks over time.
Beware the comoving particle horizon $d_\text{hor}$ grows even during inflation, albeit approaching a constant value as $a\to\infty$, while the physical particle horizon grows to infinity in any case as physically expected,
{\small
\begin{align}
    d_\text{hor}(a)=\int_{\tau_{in}}^\tau c\, d\tau = \int_{a_{in}}^a\frac{da}{a^2H}= \frac{1}{H}\left(\frac{1}{a_{in}}-\frac{1}{a}\right)\to \frac{1}{H}\frac{1}{a_{in}},\quad d_\text{hor}^\text{phys}=a\,d_\text{hor}=\frac{1}{H}\left(\frac{a}{a_{in}}-1\right)\to\infty\quad \text{as}\quad a\to\infty.
\end{align}}
Inflation solves several problems of the standard Big Bang cosmology.
\begin{itemize}
    \item The exponential expansion  drives the universe towards flatness, thus solving the flatness problem.
    \item Exponential expansion also dilutes any unwanted relics, such as magnetic monopoles, that may be produced at the high energies of the early universe, thus solving the monopole problem.
    \item The shrining of the Hubble radius during inflation allows regions that are currently causally disconnected to have been in causal contact before inflation, thus solving the horizon problem.
    \item Similarly, quantum fluctuations are strectched to super-horizon scales due to the shrinking Hubble radius, undergoing decoherence and becoming the classical perturbations that seed structure formation in the universe.
    \item This mechanism not only provides a way to generate primordial density perturbations, but also naturally predicts the nearly scale-invariant power spectrum with Gaussian statistics and the adiabatic feature of initial conditions experimentally observed in the CMB and LSS data.
\end{itemize}


%========================================================
\subsection{The horizon problem}
%========================================================

Assuming standard big bang cosmology with a radiation-dominated early universe we find the angle subtended by the particle horizon at the last scattering surface
\begin{align}
    \theta_\text{hor}=\frac{d_\text{hor}(t_\text{ls})}{d_A(t_\text{ls})}= \frac{a_{ls}\int_0^{a_ls}\!\!cd\tau}{a_{ls}\int_{a_{ls}}^{a_0}\!\!cd\tau}=\frac{\int_0^{a_{ls}}\frac{da}{a\mathcal{H}}}{\int_{a_{ls}}^{a_0}\frac{da}{a\mathcal{H}}}\simeq 
    \frac{\tfrac{a_{eq}}{H_0\sqrt{\Omega_{r0}}}+\tfrac{2}{H_0\sqrt{\Omega_{m0}}}\big(a_{ls}^{1/2}-a_{eq}^{1/2}\big)}{\tfrac{2}{H_0\sqrt{\Omega_{m0}}}\big(a_{0}^{1/2}-a_{ls}^{1/2}\big)}\simeq 2^\circ.
\end{align}
This poses a serious problem since the CMB temperature is observed to be isotropic to order $10^{-5}$ across the entire sky.
How could photons be thermalized across $360^\circ \gg 2^\circ$ if they were never in causal contact with each other?

If instead we assume a period of accelerated expansion from $a_i\simeq 0$ to some time $a_{r}$ before radiation era, the above computation is modified to
\begin{align}\label{eq:causal_angle_with_inflation_lss}
    \theta_\text{hor}=...=\frac{\int_0^{a_{ls}}\frac{da}{a\mathcal{H}}}{\int_{a_{ls}}^{a_0}\frac{da}{a\mathcal{H}}}\simeq 
    \frac{\int^{a_{r}}_{a_i}cd\tau+\tfrac{a_{eq}-a_r}{H_0\sqrt{\Omega_{r0}}}+\tfrac{2}{H_0\sqrt{\Omega_{m0}}}\big(a_{ls}^{1/2}-a_{eq}^{1/2}\big)}{\tfrac{2\big(a_{0}^{1/2}-a_{ls}^{1/2}\big)}{H_0\sqrt{\Omega_{m0}}}}.
\end{align}
Recalling $H_\approx H_*=\text{const}$ and assuming inflation lasts long enough $a_r\gg a_i$, we compute
\begin{align}
    \int_{a_i}^{a_r}\frac{da}{a\mathcal{H}}= \frac{1}{a_rH_*}\Big(\frac{a_r}{a_i}-1\Big)\simeq \frac{1}{a_rH_*}\frac{a_r}{a_i} = \frac{1}{a_rH_*}e^{N_e},
\end{align}
where $N_e$ is \emph{defined} as the number of e-folds of inflation.
The first term in \eqref{eq:causal_angle_with_inflation_lss} dominates, and we simply enforce the causal angle at the last scattering surface to be larger than $2\pi=360^\circ$ to get a lower bound on the number of inflation e-folds 
\begin{align}\label{eq:causal_angle_with_inflation_lss_bound}
    \theta_\text{hor}=\simeq 
    \frac{\int^{a_{r}}_{a_i}cd\tau}{\tfrac{2\big(a_{0}^{1/2}-a_{ls}^{1/2}\big)}{H_0\sqrt{\Omega_{m0}}}} \simeq \frac{\frac{1}{a_rH_*}e^{N_e}}{\tfrac{2\big(a_{0}^{1/2}-a_{ls}^{1/2}\big)}{H_0\sqrt{\Omega_{m0}}}}\overset{!}{\geq} 2\pi \quad \Rightarrow \quad e^{N_e}\gtrsim 4\pi\, (a_rH_*)\frac{\big(a_{0}^{1/2}-a_{ls}^{1/2}\big)}{H_0\sqrt{\Omega_{m0}}}.
\end{align}
To get actual numbers we can assume inflation ends roughly at the GUT scale $H_*\simeq 10^{16}\text{GeV}$, and that reheating is efficient enough that radiation dominated era starts immediately afterwards.
We then estimate the reheating temperature $T_{r}$ from the Friedmann equation at the end of inflation
\begin{align}
    H^2\simeq\frac{8\pi G}{3}\rho\simeq \frac{8\pi G}{3}\frac{\pi^2}{30}g_*T^4\quad \Rightarrow \quad T_{\text{r}}\simeq \left(\frac{g_*90}{\pi^3}\right)^{\tfrac14}\left(H_*M_\text{Pl}\right)^{\tfrac12}\simeq 10^{18}\text{GeV}.
\end{align}
Since radiation temperature always scales as $T\propto a^{-1}$, we estimate
\begin{align}
    a_r\simeq a_0\,\frac{T_0}{T_{r}} \simeq \frac{2.7K}{10^{18}\text{GeV}}\sim 10^{-30}.
\end{align}
Plugging numbers into \eqref{eq:causal_angle_with_inflation_lss_bound}, recalling $H_0\simeq 10^{-42}\text{GeV}$, $\Omega_{m0}\simeq 0.3$, $a_{ls}\simeq 10^{-3}$, we find the minimum number of e-folds required
\begin{align}
    e^{N_e}\gtrsim 4\pi\, (a_rH_*)\frac{\big(a_{0}^{1/2}-a_{ls}^{1/2}\big)}{H_0\sqrt{\Omega_{m0}}}\sim 10^{30}\quad \Rightarrow \quad N_e\gtrsim 60-70.
\end{align}


%========================================================
\subsection{The flatness problem}
%========================================================

Consider the density parameters of species relative to the time-dependent critical density $\rho_c(t)=\frac{3H^2(t)}{8\pi G}$,
\begin{align}
    \Omega_r(a)=\frac{\Omega_{r0}}{H^2a^4},\quad \Omega_m(a)=\frac{\Omega_{m0}}{H^2a^3},\quad \Omega_\Lambda(a)=\frac{\Omega_{\Lambda0}}{H^2},\quad \Omega_k(a)=\frac{\Omega_{k0}}{H^2a^2}=-\frac{k}{H^2a^2}.
\end{align}
The currenlty measured value for $\Omega_{k0}=0.0007\pm0.0019$ is consistent with a spatially flat universe.
When a species $x$ dominates we have $H^2\propto a^{-3(1+w_x)}$, and curvature density parameter goes like $\Omega_k(a)\propto a^{1+3w_x}$ over different epochs.
Ignoring vacuum energy only relevant at late times, we find that $\Omega_k(a)$ has always been growing over time up to the tiny value $\Omega_{k0}$ observed today.
Either the universe is exactly flat $\Omega_{k0}\equiv0$, which seems to require some explanation, or it must have started with an extremely small value of $\Omega_k(a)\sim a^{1+3w_x}\Omega_{k0}$ in the early universe so as to match the present value, which similarly requires estreme fine tuning.

If instead a species with $w_x<-1/3$ dominated the early universe, we have accelerated expansion, the Hubble radius shrinks and the curvature density parameter $\Omega_k(a)$ decreases over time, solving the flateness problem.
In the case of approximately de Sitter expansion $H\simeq \text{const}$, i.e. $w_x\sim -1$, we find
\begin{align}
    \Omega_{k}(a)=\frac{\Omega_{k0}}{H^2a^2}\simeq \frac{\Omega_{k0}}{H^2}e^{-2Ht} \to 0\quad \text{as}\quad t\to\infty.
\end{align}



%========================================================
\subsection{The exotic monopole problem}
%========================================================

Grand Unified Theories generically predict the production of topologically stable magnetic monopoles during symmetry-breaking phase transitions in the early universe via the Kibble mechanism. 
Since these objects are massive and behave as non-relativistic matter, their energy density scales as $\rho\propto a^{-3}$ and their energy density would rapidly dominate the universe, leading to a relic abundance vastly exceeding observational bounds. 
In standard Big Bang cosmology this constitutes the monopole problem. 
If instead a period of accelerated expansion occurs before or during monopole production, the exponential growth of the scale factor dilutes their number density as $n_M\propto e^{-3N_e}$, rendering their present abundance negligibly small and thereby solving the monopole problem.



%========================================================
\subsection{Seeding primordial perturbations: scale invariance, Gaussianity, and adiabaticity}
%========================================================

\subsubsection*{Adiabatic initial conditions with single clock inflation}

This assumption, motivated by inflation, ensures that any departure between species arises only from their subsequent dynamics rather than from different initial phases. 
Concretely, adiabatic perturbations enforce that the local state of matter (determined for example by energy density and pressure) at some spacetime point ($\tau,\bar{x}$) of the perturbed universe is the same as in the background universe at some slightly different time $\tau+\delta\tau(\bar x)$, that is
\begin{equation*}
    \delta \rho_{\ell}(\tau, \boldsymbol{x}) \equiv \bar{\rho}_{\ell}(\tau+\delta \tau(\boldsymbol{x}))-\bar{\rho}_{\ell}(\tau)=\dot{\bar{\rho}}_{\ell} \,\delta \tau(\boldsymbol{x}),
\end{equation*}
where the local time shift $\delta\tau(x)$ is the same for all species, and analogously for $\delta P$ and similar quantities.
Then, introducing the fractional density contrast $\delta_{\ell} \equiv \frac{\delta \rho_{\ell}}{\bar{\rho}_{\ell}}$, this implies that  
\begin{equation}\label{eq:adiabatic_ic_relation}
\delta\tau(x)=\frac{\bar{\rho}_\ell}{\dot{\bar\rho}_\ell}\,\delta_{\ell} =\frac{\bar{\rho}_{\ell'}}{\dot{\bar\rho}_{\ell'}}\,\delta_{\ell'} 
\quad \text { for all species } \ell \text { and } \ell'.
\end{equation}
In particular, for species $x,x'$ whose energy density separetely conserved $\nabla_\mu T_{(x)\,0}^\mu=\nabla_\mu T_{(x')\,0}^\mu=0$ we get the relation
\begin{equation}\label{eq:adiabatic_initial_condition_conserved_species_vs_other}
    \frac{\delta_{x}}{1+w_{x}}=\frac{\delta_{x'}}{1+w_{x'}}=\frac{\bar{\rho}_\ell}{\dot{\bar\rho}_\ell}\,\delta_{\ell}\quad \text {for all other species } \ell.
\end{equation}
Crucially, for adiabatic fluctuations the total density perturbation is dominated at any time by the species dominating the background
\begin{equation}\label{eq:total_overdensity_dominated_by_dominant_species}
\delta \rho_{\mathrm{tot}}:=\bar{\rho}_{\mathrm{tot}} \delta_{\mathrm{tot}}=\sum_{\ell} \bar{\rho}_{\ell} \delta_{\ell} \sim \bar{\rho}_{\mathrm{dom}} \delta_{\mathrm{dom}} \quad \text { for adiabatic fluctuations,}
\end{equation}
since by \eqref{eq:adiabatic_initial_condition_conserved_species_vs_other} all the $\delta_{\ell}$ are comparable.


%-----------------------------------------------------
%=============================================================
\section{Decoherence of quantum fluctuations during inflation}\todotag{finish}
%=============================================================
%-------------------------------------------------------

We now explicitly derive the EoM for the inflaton and its perturbations, and show how the quantum fluctuations of the inflaton field are stretched to super-horizon scales during inflation, undergoing decoherence and becoming the classical perturbations that seed structure formation in the universe.



%-----------------------------------------------------
%=============================================================
\section{Initial conditions for the Einstein-Boltzmann system}\todotag{finish}
%=============================================================
%-------------------------------------------------------


We now explicitly derive the initial conditions for the Einstein-Boltzmann system of equations from the conservation of the comoving curvature perturbation $\mathcal{R}$ on super-horizon scales and a suitable analysis of the equations in the superhorizon limit \todotag{finish cf Whinter 16 \& baumann book}