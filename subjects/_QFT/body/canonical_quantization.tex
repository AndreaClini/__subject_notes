% !TeX root = ../QFT_main.tex
%=========================================================
%=========================================================
\chapter{Canonical Quantization}
%========================================================
%=========================================================


%--------------------------------------------------------
%=========================================================
\section{Canonical Quantization in flat spacetime}
%=========================================================
%------------------------------------------------------------



\begin{mytheorem}[Review of classical \& quantum harmonic oscillator]\todotag{Finish}
%
The classical harmonic oscillator with mass $m$ and frequency $\omega$ is described by the Hamiltonian
\begin{equation}
    H = \tfrac{1}{2m}p^2 + \tfrac{m\omega^2}{2} q^2, \quad \{q, p\} = \delta_{ij}.
\end{equation}
where $q$ and $p$ are the canconical conjugate variables, obeying the Poisson brackets.
%
Rewrite the conjugate variables in terms of adimensional ones to match the two prefactors as
\begin{align}
    \tilde{q}=\ell q, \quad \tilde{p}= \frac{\hbar}{\ell} p, \quad \{\tilde{q}, \tilde{p}\} = \hbar \delta_{ij},
     \quad \text{for}\quad \ell= \sqrt{\tfrac{\hbar}{m\omega}}.
\end{align}
The Hamiltonian becomes
\begin{align}
    H &= \hbar \omega \tfrac{1}{2}\left( {\tilde{p}}^2 + {\tilde{q}}^2 \right)
    \\
    &\overset{!}{=}\hbar \omega \left[
        \tfrac{1}{\sqrt{2}}\left( \tilde{p} - i \tilde{q} \right)
        \cdot \tfrac{1}{\sqrt{2}}\left( \tilde{p} + i \tilde{q} \right)
    \right]
\end{align}
where in the second passage we used that $\tilde{p}$ and $\tilde{q}$ commute in the classical theory.
%
Following Heisenberg canonical quantization, promote the classical theory to a quantum one as follows.
\begin{itemize}
    \item Promote the canonical variables to operators acting on a Hilbert space $\mathcal{F}$.
    %
    \item Poisson brackets are replaced by commutators according to the rule $\displaystyle \{A, B\} \mapsto \frac{1}{i\hbar} [\hat{A}, \hat{B}]$.
    %
    \item The equations of motion are left formally unchanged, but now refer to evolving operators.
\end{itemize}
%
The Hamiltonian operator is given by
\begin{equation}
    \hat{H} = \tfrac{1}{2m}{p}^2 + \tfrac{m\omega^2}{2} {q}^2= \hbar \omega \tfrac{1}{2}\left( {\tilde{p}}^2 + {\tilde{q}}^2 \right).
\end{equation}
The Poisson brackets turn into canonical commutation relations
\begin{align}
    [q,p]=i\hbar \delta_{ij} ,
    \qquad
    [\tilde{q}, \tilde{p}] = i \delta_{ij}.
\end{align}
We cannot use the previous factorization since now $\tilde{p}$ and $\tilde{q}$ do not commute anymore.
We have instead
\begin{align}
    \hat{H} &= \hbar \omega \bigg[
        \underbrace{\tfrac{1}{\sqrt{2}}\left( \tilde{p} - i \tilde{q} \right)}_{a}
        \cdot
        \underbrace{\tfrac{1}{\sqrt{2}}\left( \tilde{p} + i \tilde{q} \right)}_{a^{\dagger}}
        + \underbrace{\tfrac{1}{2}[\tilde{p}, i\tilde{q}]}_{=\frac{1}{2} i \hbar}
    \Big] = \hbar \omega \left( a^{\dagger} a + \tfrac{1}{2} \right),
\end{align}
for the creation and annihilationoperators
\begin{align}
    a^\dagger &= \tfrac{1}{\sqrt{2}}\left( \tilde{p} - i \tilde{q} \right) = \tfrac{1}{\sqrt{2\hbar}} \left(\sqrt{m\omega} \,q - \tfrac{i}{\sqrt{m\omega}} p \right), \quad
    a &= \tfrac{1}{\sqrt{2}}\left( \tilde{p} + i \tilde{q} \right) = \tfrac{1}{\sqrt{2\hbar}} \left( \sqrt{m\omega}\, q + \tfrac{i}{\sqrt{m\omega}} p \right).
\end{align}
%
The energy eigenvalues of the quantum harmonic oscillator are
\begin{equation}
    E_n = \hbar \omega \left( n + \tfrac{1}{2} \right), \quad n = 0, 1, 2, \ldots
\end{equation}
%
In the familiar position representation  we have
\begin{align}
    q &= x, \quad p = -i \hbar \partial_x, \qquad 
    \tilde{q}=\xi := \frac{x}{\ell}=\sqrt{\tfrac{m\omega}{\hbar}} x, \quad \tilde{p} = -i \frac{d}{d\xi},\\
    a^\dagger &= \tfrac{1}{\sqrt{2}}\left(\xi-\partial_\xi\right) = \tfrac{1}{\sqrt{2}}\left(\sqrt{\tfrac{m\omega}{\hbar}} x - \sqrt{\tfrac{\hbar}{m\omega}} \partial_x\right), \quad
    \quad a = \tfrac{1}{\sqrt{2}}\left(\xi+\partial_\xi\right)= \tfrac{1}{\sqrt{2}}\left(\sqrt{\tfrac{m\omega}{\hbar}} x + \sqrt{\tfrac{\hbar}{m\omega}} \partial_x\right).
\end{align}

We remark that we could have equivalently startd the quantization procedure directly from the last expression of the \emph{classical} Hamiltonian
\begin{align}
    H&\overset{!}{=}\hbar \omega \left[\tfrac{1}{\sqrt{2}}\left( \tilde{p} - i \tilde{q} \right)
        \cdot \tfrac{1}{\sqrt{2}}\left( \tilde{p} + i \tilde{q} \right)\right] = \hbar \omega \left( a^\dagger a \right).
\end{align}
This corresponds to normal-ordering of the previous \emph{quantum} Hamiltonian, removing the zero-point energy $\hbar \omega/2$.
As long as gravity is ignored, constant energy shifts do not affect the dynamics of the system, and we can safely normal-order the Hamiltonian.
In general relativity, all forms of energy will instead source the energy-momentum tensors and thus the gravitational field, and we cannot simply ignore the zero-point energy anymore.

It is fundamental to understand how the \emph{fixed} frequency $\omega$ of the oscillators determines the likelyhood of finding the system at a given elongation from equilibrium i.e. the amplitude of oscillation.
Indeed, all wave functions are proportional to a Gaussian factor, multiplied by suitable Hermite polynomials,
\begin{align}
    \psi_n(q) = H_n(\xi)\, e^{-\tfrac{\xi^2}{2}} &\propto e^{-\frac{m \omega}{2 \hbar} q^2}.
\end{align}
The crucial point is that the smaller the frequency $\omega$, the wider the Gaussian and in turn the higher the probability of finding the oscillator at large elongations from equilibrium i.e. of observing large amplitudes of oscillation.
This is already familiar from Electrodynamics: it is very unlikely to find high-frequency electromagnetic waves with large amplitudes, whereas low-frequency waves are easily found with large amplitudes.
%
\end{mytheorem}


\begin{mytheorem}[Free Klein-Gordon field as waves on a lake \& Fourier modes as decoupled harmonic oscillators]
\todotag{insert image of lake \& fourier modes}
Under any point of view, we should see the classical Klein-Gordon field as a collection of decoupled harmonic oscillators, one per each Fourier mode.
In this sense, it is completely analogous to the waves on a lake, which can be decomposed into independent Fourier modes, each oscillating with its own frequency.
\end{mytheorem}

\begin{mytheorem}[Probing the Klein-Gordon field with a cork] \todotag{write}
\todotag{Insert image}
How would you probe the waves on a lake? A good idea is to use a cork floating on the water surface.
The cork moves up and down following the local height of the water surface, thus providing a direct measurement of the wave amplitude at that point.
Analogously, to probe the Klein-Gordon field, we can use a point-like detector that measures the field value at a specific spacetime point.
This detector interacts with the field, allowing us to extract information about the field's behavior and dynamics.
The cork would now be an atom or another suitable quantum system that can be acted upon by the field.
\end{mytheorem}



\begin{mytheorem}[Classical theory of the Klein-Gordon field]
%
A real free scalar field $\phi(x)$ obeys the Klein-Gordon equation, 
\begin{equation}
    (\partial_{\mu} \partial^{\mu} + m^2) \phi(x) = 0.
\end{equation}
This is a wave equation with relativistic dispersion relation $E^2 = |\mathbf{p}|^2 + m^2$ for the field excitations.
It is the Euler-Lagrange equation of the (nonunique) Lagrangian density 
\begin{equation}
    \mathcal{L} = \frac{1}{2} \partial_{\mu} \phi \partial^{\mu} \phi - \frac{1}{2} m^2 \phi^2.
\end{equation}
The corresponding conjugate momentum and Hamiltonian density are 
\begin{equation}
    \pi(x) = \frac{\delta \mathcal{L}}{\delta \partial_0 \phi(x)} = \partial_0 \phi(x), 
    \qquad 
    \mathcal{H} = \frac{1}{2} \pi^2 + \frac{1}{2} (\nabla \phi)^2 + \frac{1}{2} m^2 \phi^2.
\end{equation}
In the classical theory, the fields $\phi(x)$ and $\pi(x)$ satisfy the equal-time Poisson brackets 
\begin{align}
    \{\phi(t, \mathbf{x}), \pi(t, \mathbf{y})\} = \delta^{(3)}(\mathbf{x} - \mathbf{y}), \quad
    \{\phi(t, \mathbf{x}), \phi(t, \mathbf{y})\} = 0, \quad
    \{\pi(t, \mathbf{x}), \pi(t, \mathbf{y})\} = 0.
\end{align}
More generally, the equation of motions for any functional $\mathcal{O}[\phi, \pi]$ of the fields is given by the Poisson brcket
\begin{equation}
    \frac{d\mathcal{O}}{dt} = \{\mathcal{O}, H\} + \frac{\partial \mathcal{O}}{\partial t},
    \quad \text{for}\quad
    \{\mathcal{O},H\} = \int d^3 x\, \left( \frac{\delta \mathcal{O}}{\delta \phi(x)} \frac{\delta H}{\delta \pi(x)} - \frac{\delta \mathcal{O}}{\delta \pi(x)} \frac{\delta H}{\delta \phi(x)} \right).
\end{equation}
where the Hamiltonian is $H = \int d^3 x\, \mathcal{H}$. 
In particular, the fundamental fields obey the equations of motion, equivalent to the Klein-Gordon equation,
\begin{align}
    \frac{d\phi(t, \mathbf{x})}{dt} = \{\phi(t, \mathbf{x}), H\} = \pi(t, \mathbf{x}), \quad
    \frac{d\pi(t, \mathbf{x})}{dt} = \{\pi(t, \mathbf{x}), H\} = (\nabla^2 - m^2) \phi(t, \mathbf{x}). 
\end{align}
%
\end{mytheorem}


\begin{mytheorem}[Canonical quantization of the Klein-Gordon field]
%
Following Heisenberg canonical quantization procedure, we promote the classical theory to a quantum theory as follows.
\begin{itemize}
    \item Promote the classical fields $\phi(x)$ and $\pi(x)$ to operators $\hat{\phi}(x)$ and $\hat{\pi}(x)$ acting on a Hilbert space $\mathcal{F}$.
    %
    \item Poisson brackets are replaced by commutators according to the rule $\displaystyle \{A, B\} \mapsto \frac{1}{i\hbar} [\hat{A}, \hat{B}].$
    %
    \item The equations of motion are left formally unchanged, but now refer to evolving operators.
\end{itemize}
%
The equal-time Poisson backe of the field become the equal-time caonical commutation relations 
\begin{align}
    [\hat{\phi}(t, \mathbf{x}), \hat{\pi}(t, \mathbf{y})] &= i \delta^{(3)}(\mathbf{x} - \mathbf{y}), \quad
    [\hat{\phi}(t, \mathbf{x}), \hat{\phi}(t, \mathbf{y})] &= 0, \quad
    [\hat{\pi}(t, \mathbf{x}), \hat{\pi}(t, \mathbf{y})] &= 0.
\end{align}
%
The evolution of any operator $\mathcal{O}[\hat{\phi},\hat{\pi}]$ function of the fundamental fields is
\begin{equation}
    \frac{d\hat{\mathcal{O}}}{dt} = \frac{1}{i \hbar} [\hat{\mathcal{O}}, \hat{H}] + \frac{\partial \hat{\mathcal{O}}}{\partial t}, \quad \text{for}\quad
    \hat{H} = \int d^3x\,\, \hat{\mathcal{H}} = \int d^3 x\, \left( \frac{1}{2} \hat{\pi}^2 + \frac{1}{2} (\nabla \hat{\phi})^2 + \frac{1}{2} m^2 \hat{\phi}^2 \right).
\end{equation}
%
In particular, the fundamental fields obey the Heisenberg equations of motion
\begin{align}
    \begin{cases}\displaystyle
    \frac{d\hat{\phi}(t, \mathbf{x})}{dt} = \tfrac{1}{i \hbar} [\hat{\phi}(t, \mathbf{x}), \hat{H}] = \hat{\pi}(t, \mathbf{x}), \\[7pt]
    \displaystyle
    \frac{d\hat{\pi}(t, \mathbf{x})}{dt} = \tfrac{1}{i \hbar} [\hat{\pi}(t, \mathbf{x}), \hat{H}] = (\nabla^2 - m^2) \hat{\phi}(t, \mathbf{x}),
    \end{cases}
    \quad \Longrightarrow\quad  (\partial_{\mu} \partial^{\mu} + m^2) \hat{\phi}(x) = 0.
\end{align}
which together again imply the field operator $\hat{\phi}(x)$ satisfies the Klein-Gordon equation.
%
\end{mytheorem}


\begin{mytheorem}[Fourier transform \& identifying the uncoupled DoFs of the Klein-Gordon field]
%  
The Fourier transform of the fields are defined as
\begin{equation}
    \phi_k(t) :=\int d^3 x\, e^{-i \mathbf{k} \cdot \mathbf{x}} \phi(t, \mathbf{x}), \qquad \pi_k(t) := \int d^3 x\, e^{-i \mathbf{k} \cdot \mathbf{x}} \pi(t, \mathbf{x}) = \frac{d \phi_k(t)}{dt}.
\end{equation}
Note that self-adjointness in position space implies $\phi_k^{\dagger}(t) = \phi_{-k}(t), \,\pi_k^{\dagger}(t) = \pi_{-k}(t)$ in momentum space.
%
The Klein-Gordon equation simply becomes the equation for uoncoupled harmonic oscillators
\begin{equation}
    \frac{d^2 \phi_k(t)}{dt^2} + (|\mathbf{k}|^2 + m^2) \phi_k(t) = 0, \quad \text{with frequency}\quad \omega_k = \sqrt{|\mathbf{k}|^2 + m^2}.
\end{equation}
Coherently, the Hamiltonian is just a sum of independent harmonic oscillator Hamiltonians
\begin{equation}
    H  = \int d^3x\,\left(\tfrac{1}{2} \pi^2 + \tfrac{1}{2} (\nabla \phi)^2 + \tfrac{1}{2} m^2 \phi^2\right) = \int \frac{d^3 k}{(2\pi)^3} \left( \tfrac{1}{2} \pi_k^\dagger\pi_k + \tfrac{1}{2} (|\mathbf{k}|^2 + m^2) \phi_k^\dagger \phi_{k} \right).
\end{equation}
Finally, the equal time commutation relations in momentum space become
\begin{align}
    [\hat{\phi}_k(t), \hat{\pi}_{k'}(t)] = i (2\pi)^3 \delta^{(3)}(\mathbf{k} + \mathbf{k}'), \quad
    [\hat{\phi}_k(t), \hat{\phi}_{k'}(t)] = 0, \quad
    [\hat{\pi}_k(t), \hat{\pi}_{k'}(t)] = 0.
\end{align}

It seems we managed to diagonalize the Hamiltonian and identify its uncoupled degrees of freedom of the theory, behaving as independent harmonic oscillators labelled by the momentum $\mathbf{k}$.
However something is off track.
\begin{itemize}
    \item The field $\phi_k$ is no more self-adjoint, since $\phi_k^{\dagger} = \phi_{-k}$; whereas we would rather our DoFs be observables.
    \item The delta functions of the CAR is $\delta^{(3)}(\mathbf{k} + \mathbf{k}')$ instead of the expected $\delta^{(3)}(\mathbf{k} - \mathbf{k}')$, suggesting that the conjugate momentum of the mode $\phi_k$ is actually $\pi_{-k}$ instead of $\pi_k$.
    \item We have a continuum of Foruier modes $k$ and Dirac $\delta$ instead of Kronecker ones for the DoFs.
\end{itemize}
These problems are all related.
The last one was expected, since we are considering infinite volume and thus a continuum of modes $k$.
Infrared regularization, like placing the theory in a finite box or sphere, would immediately restore the discrete Fourier modes and Kronecker deltas.
In fact, considering a \emph{finite} box with Dirichlet or Neumann boundary conditions instead of periodic ones would yield a Fourier expansion in real sine and cosine modes and solving also the first issue:
\begin{align}
    \hat{\phi}^{\mathrm{DBC}}_k(t) \simeq  \int_{\text{box}} \!\!d^3 x\, \sin(\mathbf{k} \mathbf{x}) \hat{\phi}(t, \mathbf{x}), \quad
    \hat{\phi}^{\mathrm{NBC}}_k(t) \simeq \int_{\text{box}} \!\!d^3 x\, \cos(\mathbf{k} \mathbf{x}) \hat{\phi}(t, \mathbf{x}).
\end{align} 

In any case, the first two issues have more physical relevance and still related.
We can fix them by explicitly rewriting $\phi_k$ as linear combination of \emph{physical} self-adjoint harmonic oscillators $\{q_k,p_k\}_k$ to highlight the true DoFs of the theory.
Alternatively, we can directly write $\phi_k$ in terms of creation and annihilation operators $a_k$, and then in turn identify the corresponding self-adjoint oscillators.
The two approached are completely equivalent and forces us to understand what to consider \emph{the vacuum} of the theory.
\todotag{Write the two approaches in detail.}
%
\end{mytheorem}



\begin{mytheorem}[Observable DoFs of the Klein-Gordon field in flat spacetime]

\end{mytheorem}


\begin{mytheorem}[Fluctuations of the Kelin-Gordon field in momentum \& position space]
%
Having decomposed the Klein-Gordon field into its constitutent uncoupled harmonic oscillators, we can postulate a certain state for each oscillator and then immediatelycompute the corresponding field fluctuations i.e. the probability of finding a given set of Fourier modes $\{\phi_k\}$ or equivalently field values $\{\phi(x)\}$.
We could consider the vacuum state $|0\rangle$ where all oscillators are in their ground state, excited states $|n_{k_1}\dots n_{k_m}\dots \rangle$ with $n_k$ quanta in the mode $k$, coherent states $|\{\alpha_k\}\rangle$ or arbitrary entangled states.
%
Consider some infrared regularization, so that the Fourier modes are discrete and labelled by some countable index $i$.
Consider the pure tensor state with $n_k$ quanta in the mode $k$, that is where the harmonic oscillator labelled by $k$ is in the $n_k$-th excited state,
\begin{equation}
    |\Psi\rangle = |n_{k_1}\dots n_{k_m}\dots \rangle.
\end{equation}
The wave function of $n_k$-th excited state of the oscillator with mode $\vec{k}$ is
\begin{align}
    \psi_{n_k}(\phi_{k}) &= \langle \phi_{k} | n_k \rangle = \left( \frac{m \omega_k}{\pi \hbar} \right)^{1/4} \frac{1}{\sqrt{2^{n_k} n_k!}} H_{n_k}\left( \sqrt{\frac{m \omega_k}{\hbar}} \phi_{k} \right) e^{-\frac{m \omega_k}{2\hbar} \phi_{k}^2}.
\end{align}
That is, the probability density of finding the Fourier mode $\vec{k}$ of the field with value between $\phi_k$ and $\phi_k + d\phi_k$ is
\begin{align}
    P_{n_k}(\phi_k) d\phi_k &= |\psi_{n_k}(\phi_k)|^2 d\phi_k \\
    &= \left( \frac{m \omega_k}{\pi \hbar} \right)^{1/2} \frac{1}{2^{n_k} n_k!} \left[ H_{n_k}\left( \sqrt{\frac{m \omega_k}{\hbar}} \phi_{k} \right) \right]^2 e^{-\frac{m \omega_k}{\hbar} \phi_{k}^2} \,d\phi_k.
\end{align}
%
The wave \emph{functional} $\Psi$ in momentum space is simply the product of all wave functions,
\begin{equation}
    \Psi\big[\{\phi_k\}_k\big] = \langle \{\phi_k\}_k | \Psi \rangle = \prod_k \psi_{n_k}(\phi_k).
\end{equation}
The probability density of finding a given field configuration $\{\phi_k\}_k$ in \emph{momentum} space is thus
\begin{align}
    P_{\Psi}(\{\phi_k\}_k) \,\,D\phi &= |\langle \{\phi_k\}_k | \Psi \rangle|^2 \,D\phi 
    = \prod_k |\psi_{n_k}(\phi_k)|^2 \,d\phi_k
    \\
    &= \prod_k \left( \frac{m \omega_k}{\pi \hbar} \right)^{1/2} \frac{1}{2^{n_k} n_k!} \left[ H_{n_k}\left( \sqrt{\frac{m \omega_k}{\hbar}} \phi_{k} \right) \right]^2 e^{-\frac{m \omega_k}{\hbar} \phi_{k}^2} \,d\phi_k,
\end{align}
where $D\phi = \prod_xd \phi(x) = \prod_k d\phi_k$ is the functional measure in the functional space of field configurations, equivalently expressed in terms of space values or Fourier modes.
%

The states $|\{\phi_k\}_k \rangle \equiv |\{\phi(x)\}_x\rangle$ form a basis of common eigenstates of all the field operators at equal time in both position $\{\hat{\phi}(t,x)\}_x$ and momentum $\{\hat{\phi}_k(t)\}_k$ space.
Indeed if the state 
\begin{align}
    \hat{\phi}(y)&|\{\phi(x)\}_x\rangle = \phi(y) |\{\phi(x)\}_x\rangle, \quad \forall y, \\[6pt]
    \Rightarrow\,\, &
    \hat{\phi}_k(t)|\{\phi(x)\}_x\rangle = \int d^3 x\, e^{-i \mathbf{k}\mathbf{x}} \hat{\phi}(t, \mathbf{x}) |\{\phi(x)\}_x\rangle = \int d^3 x\, e^{-i \mathbf{k}\mathbf{x}} \phi(x) |\{\phi(x)\}_x\rangle =
    \phi_k(t) |\{\phi(x)\}_x\rangle \quad \forall k.
\end{align}
%

Given such a $\hat{\phi}$-eigenstate $|\{f(x)\}_x\rangle$ with explicit \emph{position space consifguration}, we obtain the value of the wave functional on this state by inserting a resolution of the identity on the Fock space $\mathcal{F}$ with the spectral basis of $\hat{\phi}$ explicitated in terms of Fourier modes
\begin{align}
    \mathbb{I}_{\mathcal{F}}= \int D\phi_k \, |\{\phi_k\}_k \rangle \langle \{\phi_k\}_k |.
\end{align}
We immediately obtain
\begin{align}
    \Psi\big[\{f(x)\}_x\big] &= \langle \{f(x)\}_x | \Psi \rangle 
    = \int D\phi_k \, \langle \{f(x)\}_x | \{\phi_k\}_k \rangle \langle \{\phi_k\}_k | \Psi \rangle
    \\
    &= \int D\phi_k \, \prod_k \delta\left( \phi_k - f_k \right) \cdot \Psi\big[\{\phi_k\}_k\big]
    = \Psi\big[\{f_k\}_k\big], \quad \text{for}\quad f_k = \int d^3 x\, e^{-i \mathbf{k} \cdot \mathbf{x}} f(x).
\end{align}
In more practical terms, we simply obtain the Fourier modes of this field configuration
\begin{align}
    f_k=\int d^3 x\, e^{-i \mathbf{k} \cdot \mathbf{x}} f(x),
\end{align}
and we directly evaluate the wave function on the state $|\{f(x)\}_x\rangle\equiv |\{f_k\}_k\rangle$ itself expressing it via its Fourier space configuration.
%
The probability density of finding the field with configuration between $\{f(x)\}_x$ and $\{f(x)+df(x)\}_x$ is thus
\begin{align}
    P_{\Psi}(\{f(x)\}_x) \,Df &= |\Psi\big[\{f(x)\}_x\big]|^2 \,Df 
    = |\Psi\big[\{f_k\}_k\big]|^2 \,Df
    \\
    &= \prod_k \left( \frac{m \omega_k}{\pi \hbar} \right)^{1/2} \frac{1}{2^{n_k} n_k!} \left[ H_{n_k}\left( \sqrt{\frac{m \omega_k}{\hbar}} f_{k} \right) \right]^2 e^{-\frac{m \omega_k}{\hbar} f_{k}^2} \,Df,
\end{align}
where $Df$ is the functional measure in the function space of field configurations. 
%

We remark that nobody really knows what the function space of field configurations--i.e. of the $\hat{\phi}$-eigenvalue collections--should be, nor how to rigorously define the functional measure $Df$.
A naive guess is to consider the space of square-integrable functions $f(x) \in L^2(\mathbb{R}^3)$, but this is definitely not correct, as the typical field configurations are actually distributions rather than functions.
This is already evident from the vacuum fluctuations of the field, which diverge at every point in space.
Rigorous treatments of constructive QFT manage to define suitable function spaces and measures, but they are quite technical and beyond the scope of these notes.
%
\end{mytheorem}





%----------------------------------------------------------------
%===================================================================
\section{The Schrödinger equation in QFT}
%===================================================================
%----------------------------------------------------------------

The good old Schrödinger equation in Quantum Mechanics reads
\begin{equation}
    i \hbar \frac{\partial}{\partial t} |\psi(t)\rangle = \hat{H} |\psi(t)\rangle.
\end{equation}
In this formulation, it is bound to fail when dealing with systems with variable particle number.

However, as the Heisenberg and Schrödinger formulations of Quantum Mechanics are equivalent, we can still use the latter in Quantum Field Theory and get a corresponding equation.
The space $\mathcal{F}=\bigoplus \mathcal{H}^{\otimes n}$ of states in QFT is now a Fock space, i.e. a direct sum of $n$-particle Hilbert spaces.
Taking $\mathcal{H}\simeq L^2(\mathbb{R}^3)$ for simplicity, the general state in $\mathcal{F}$ reads
\begin{equation}
    |\Psi\rangle = \left( \begin{array}{c}
    \psi_0 \\
    \psi_1(\mathbf{x}_1) \\
    \psi_2(\mathbf{x}_1, \mathbf{x}_2) \\
    \vdots
    \end{array} \right),\qquad \psi_n(\mathbf{x}_1, \ldots, \mathbf{x}_n) \in L^2(\mathbb{R}^{3n}).
\end{equation}

The Schrödinger equation in QFT is virtually the same:
\begin{equation}
    i \hbar \frac{\partial}{\partial t} |\Psi(t)\rangle = \hat{H} |\Psi(t)\rangle.
\end{equation}
Here $\hat{H}=\int d^3x \mathcal{H}$ is the Hamiltonian operator acting on the Fock space $\mathcal{F}$, namely the space-integral of the Hamiltonian density operator $\mathcal{H}$.
For example, for the free Klein-Gordon field
\begin{align}
    \hat{H} &= \int d^3 x\, \left( \frac{1}{2} \hat{\pi}^2 + \frac{1}{2} (\nabla \hat{\phi})^2 + \frac{1}{2} m^2 \hat{\phi}^2 \right).
\end{align}


\begin{mytheorem}[Spectral bases \& maximal set of commuting observables.]
%
In ordinary QM, we can expand the state $|\Psi\rangle$ in terms of the eigenstates of suitable (collections of commuting) self-adjoint operators.
Common choices are the position operator $X$, the momentum $P$, the angular momentum $L^2$ and $L_z$, the intrisinc spin $S^2$ and $S_z$ or suitable combinations like $L+S$, helicity, the Hamiltonian etc

A \emph{maximal set of commuting observables} is a collection of self-adjoint operators $\{A_i\}$ such that
\begin{itemize}
    \item $[A_i, A_j] = 0$ for all $i, j$;
    \item there is no other self-adjoint operator $B$ commuting with all $A_i$.
\end{itemize}
The common eigenstates of a maximal set of commuting observables form a basis for the Hilbert space with the property that no two distinct basis states share the same eigenvalues for all observables in the set.
\end{mytheorem}
\begin{proof}
Suppose there is a collection of eigenvalues $\{\lambda_i\}$ shared by two linearly independent eigenstates $|\psi\rangle$ and $|\phi\rangle$.
Then $S=\text{Span}\{|\psi\rangle, |\phi\rangle\}$ is a two-dimensional subspace invariant under all $A_i$ and we can define a new self-adjoint operator $B$ acting non-trivially only on this subspace and commuting with all $A_i$, contradicting the maximality of the set--e.g. it swaps $|\psi\rangle$ and $|\phi\rangle$ and is the identity on $S^{\perp}$.
\end{proof}


\begin{mytheorem}[Position \& momentum bases in QFT]
%
The same can be done in QFT for suitable self-adjoint operators.
At fixed time commutation relations for the real Klein-Gordon field $\hat{\phi}$ and its conjugate momentum $\hat{\pi}=\delta \mathcal{L}/\delta \partial_0 \phi$ are
\begin{align}
    [\hat{\phi}(t, \mathbf{x}), \hat{\pi}(t, \mathbf{y})] &= i \delta^{(3)}(\mathbf{x} - \mathbf{y}), \\
    [\hat{\phi}(t, \mathbf{x}), \hat{\phi}(t, \mathbf{y})] &= 0, \\
    [\hat{\pi}(t, \mathbf{x}), \hat{\pi}(t, \mathbf{y})] &= 0.
\end{align}
At any fixed time $t$, the collections of operators $\{\hat{\phi}(t, \mathbf{x})\}_{\mathbf{x}\in \mathbb{R}^3}$ and $\{\hat{\pi}(t, \mathbf{x})\}_{\mathbf{x}\in \mathbb{R}^3}$ are thus maximal sets of commuting observable, and we can find corresponding spectral bases $\{|f\rangle\}_{f}$ and $\{|g\rangle\}_{g}$ of the Fock space with eigenvalue collection $\{f(\mathbf{x})\}_{\mathbf{x}\in \mathbb{R}^3}$ and $\{g(\mathbf{x})\}_{\mathbf{x}\in \mathbb{R}^3}$ respectively for suitable functions $f, g:\mathbb{R}^3 \to \mathbb{R}$.
These bases obey
\begin{align}
    \hat{\phi}(t, \mathbf{x}) |f\rangle &= f(\mathbf{x}) |f\rangle, \\
    \hat{\pi}(t, \mathbf{x}) |g\rangle &= g(\mathbf{x}) |g\rangle.
\end{align}
It has to be noted that nobody really knows how to rigorously construct these bases, nor where they actually live--i.e. what kind of mathematical objects these functions $f$ and $g$ are..
To fix ideas, a very naive guess would be that the set of eigenvalue collections $\{f(\mathbf{x})\}_{\mathbf{x}\in \mathbb{R}^3}$ is the set of all square-integrable functions $f \in L^2(\mathbb{R}^3)$, but this cannot be true since $L^2(\mathbb{R}^3)$ is a set of equivalence classes of functions.

In any case, these spectral bases must be understood as the field-theoretic analogoues of the position $\{|x\rangle\}_{x\in \mathbb{R}^n}$ and momentum $\{|p\rangle\}_{p\in \mathbb{R}^n}$ bases in ordinary Quantum Mechanics, since $\hat{\phi}(x)$ and $\hat{\pi}(x)$ are the field-theoretic counterpart of the position $\hat{X}$ and momentum $\hat{P}$ operators. 
Indeed, in an ordinary QM system we have a finite number of DoFs, labelled by $i=1, \ldots, n$, with generalized coordinates $x_i$ and conjugate momenta $p_i$ satisfying equal time canonical commutation relations
\begin{align}
    [\hat{X}_i, \hat{P}_j] &= i \hbar \delta_{ij}, \quad
    [\hat{X}_i, \hat{X}_j] &= 0, \quad
    [\hat{P}_i, \hat{P}_j] &= 0.
\end{align}
In QFT we have an infinite number of DoFs, labelled by the spatial position $\mathbf{x} \in \mathbb{R}^3$, with generalized coordinates $\phi(\mathbf{x})$ and conjugate momenta $\pi(\mathbf{x})$ satisfying the analogous canonical commutation relations above.
\end{mytheorem}
