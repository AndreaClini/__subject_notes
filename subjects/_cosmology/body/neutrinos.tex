% !TeX root = ../cosmology_main.tex
%=========================================================
%=========================================================
\chapter{Neutrinos in Cosmology}
%========================================================
%=========================================================

The treatment of neutrios is almost identical to that of photons, since they can be treated as a relativistic fluid down to very late times.
We still assume they retain local thermal equilibrium with a Fermi-Dirac distribution with temperature constrast $\mathcal{N}$, that is
\begin{align}
    f_\nu(\tau, \vec{x}, \vec{p}) = \frac{1}{e^{(E-\mu)/T_\nu}+1}, \quad T_\nu=\bar{T}_\nu(\tau)(1+\mathcal{N}(\tau, \vec{x}, \hat{p}, E)).
\end{align}
The main differences are the following.
\begin{itemize}
    \item Neutrino temperature is lower than the photon temperature $\bar{T}_\nu\simeq (4/11)^{1/3} \bar{T}_\gamma$, since neutrinos decouple before electron-positron annihilation, which heats up photons but not neutrinos.
    The treatment is however unchanged, since the Boltzmann equation only depends on the temperature contrast $\delta T/T$, not on the absolute value.
    \item Neutrinos are fermions, thus following Fermi-Dirac distribution, while photons are bosons, thus with Bose-Einstein distribution. Again this difference is neglible.
    \item Neutrino have no interactions after decoupling, and the RHS of the Boltzmann equation is thus zero.
    In particular they free-stream and develop non-negligible anisotropic stress (Legendre multipole $\mathcal{N}_2$ and possibly higher ones) already at early times, while photons are driven towards the first two multipoles $\Theta_0$, $\Theta_1$ by Thomson scattering until recombination and develop anisotropic stress only after decoupling.
    Neutrino anisotropic stress has a non-negligible effect on the evolution of metric perturbations, and thus on the CMB power spectrum and other observables.
    \item Neutrino are actually massive, and become non-relativistic at late times, while photons are always relativistic. This affects the background evolution of the universe, and thus the expansion history and the growth of structure at late times.
\end{itemize}



%--------------------------------------------------------
%=========================================================
\section{Geodesic \& Boltzmann Equations for neutrinos}
%=========================================================
%------------------------------------------------------------


The geodesic equations in newtonian gauge \label{eq:geodesic_equation_physical_momentum_newtonian} are unchanged
{\small
\begin{align}\label{eq:geodesic_equations_neutrino_newtonian}
\begin{aligned}
    \frac{d E}{d\ct}&=-\frac{p^2}{E}(\mathcal{H}-\dot\phi)-p\hat{p}^j\partial_j\psi,
    \\
    \frac{d p}{d\ct}&=-p(\mathcal{H}-\dot\phi)-E\hat{p}^j\partial_j\psi,
    \\
    \frac{d\hat{p}^i}{d\ct}&=\big(\hat{p}^i\hat{p}^j-\delta^{ij}\big)\left(\frac{E}{p}\partial_j\psi+\frac{p}{E}\partial_j\phi\right).
\end{aligned}
\end{align}}
Beware taking $E=p$ is a perfect approximation down to times/temperatures $T_\nu \sim 1\, \text{eV} \simeq 1.6\times 10^4 \text{ K}$, but will eventually break down at later times.
Also recall neutrinos' temperature is related to photons by $T_\nu=\big(\tfrac{4}{11}\big)^{1/3} T_\gamma$.
Ineed from neutrino oscillations we know that at leaste one neutrino mass $m_\nu \gtrsim 0.05$ eV, while from cosmological bounds we know that the sum of neutrino masses $\sum m_\nu \lesssim 0.1 \text{eV}$.

The last equation implies that to zeroth order in metric perturbations, neutrinos travel along straight lines, while at first order they are deflected by the gravitational potentials $\psi$ and $\phi$.
The first two equations encode the change in energy and momentum along the trajectory.

Conidering the relativistic limit $E=p$ for simplicity it is convenient to trade partial with total time derivatives $\frac{\partial \phi}{\partial \tau}=\frac{d\phi}{d\tau}-\frac{\partial \phi}{\partial x^j}\frac{dx^j}{d\tau}$ to get, 
{\small
\begin{align}
    \frac{d \log E}{d\tau}&= -\mathcal{H}+\frac{d\phi}{d\tau}-\hat{p}^j(\partial_j\phi+\partial_j\psi).
\end{align}}
To order zero it encode the expected redshift behavior $E=p=\propto\, a^{-1}$.
The perturbed part implies energy is red-blueshifted by the gravitational potentials via the istantaneous and integrated Sachs-Wolf effect.
Since the change in $\hat{p}^j$ along the trajectory is already first order, we can integrate it to get
{\small
\begin{align}
    \log \Big(\frac{E_f}{E_i}\Big) &= \int_{\tau_i}^{\tau_f} d\tau\,\, -\mathcal{H} + \frac{d\phi}{d\tau} - \hat{p}^j(\partial_j\phi+\partial_j\psi) \, d\tau\\
    &= \log \Big(\frac{a_i}{a_f}\Big) + \underbrace{\big(\phi(\tau_f,x_f)-\phi(\tau_i,x_i)\big)}_{\text{instantaneous SW (aka gravitational redshift)}} - \int_{\tau_i}^{\tau_f} \underbrace{d\tau\,\, \hat{p}^j(\partial_j\phi+\partial_j\psi)}_{\text{integrated Sachs-Wolfe effect}}.
\end{align}}
Writing $E = \bar{E}(1+\mathcal{N})$ with $\bar{E}\propto \, a^{-1}$ so that $\bar{E}_ia_i=\bar{E}_fa_f$ to remove the background redshift, we get the expected Sachs-Wolfe formula for the observed temperature anisotropies
{\small
\begin{align}
    \mathcal{N}_f(\hat{p})-\mathcal{N}_i(\hat{p})\simeq \log\left(\frac{1+\mathcal{N}_f(\hat{p})}{1+\mathcal{N}_i(\hat{p})}\right) 
    = \phi(\tau_f,x_f)-\phi(\tau_i,x_i) - \int_{\tau_i}^{\tau_f} d\tau\,\, \hat{p}^j(\partial_j\phi+\partial_j\psi).
\end{align}}
The above formulas are exact in the relativistic limit, but nedd to be adjusted using the geodesic equations \eqref{eq:geodesic_equations_neutrino_newtonian} with $E\neq p$ to get the effect of neutrino masses at late times.

The Boltzmann equation \eqref{eq:Boltzmann_LHS_exact_newton_gauge} for neutrinoes, accounting for masses, is again (second order in metric perturbations, exact otherwise)
{\small
\begin{align}\label{eq:Boltzmann_LHS_exact_newton_gauge_neutrino}
\begin{aligned}
    C[f]&=\frac{d f}{d \lambda}
    =
    %
    P^{0}\left[\frac{\partial f}{\partial \tau}+\frac{\partial f}{\partial x^{i}} \frac{P^{i}}{P^{0}}+\frac{\partial f}{\partial E} \frac{1}{P^{0}} \frac{d E}{d \lambda}+\frac{\partial f}{\partial \hat{p}^{i}} \frac{1}{P^{0}} \frac{d \hat{p}^{i}}{d \lambda}\right]
    \\
    &=\frac{E}{a}\!\left[\frac{\partial f}{\partial \ct}(1-\psi)
    \!+\!\left(\hat{p}^j\partial_jf\right) \frac{p}{E}(1\!+\!\phi)
    -\frac{\partial f}{\partial E}\frac{p^2}{E}\left(\mathcal{H}(1\!-\!\psi)-\dot\phi\right)
    -\frac{\partial f}{\partial \bar{p}^j}\partial_j\psi \,E
    \!+\!\left(\frac{\partial f}{\partial p} \hat{p}^j-\frac{\partial f}{\partial \bar{p}^j}\right)\partial_j\phi\,\frac{p^2}{E}\right].
\end{aligned}
\end{align}}
Neutrinos retain local equilibrium with a Fermi-Dirac distribution down to late times\todotag{Discuss}, simply with a time and momentum dependent temperature, and the chemical potential is negligible\todotag{discuss}
{\small
\begin{equation}\label{eq:local_temp_neutrinos}
    f(\ct, x^i, E, \hat{p}^i) = \frac{1}{\exp\left(\frac{E}{T_\nu}\right) + 1}, \quad T_\nu=\bar{T}_\nu(\tau)(1+\mathcal{N}(\tau, x^i, p, \hat{p}^i))=\bar{T}_\nu+\delta T_\nu, \quad \mu\sim 0.
\end{equation}}
More generally, assuming the full distribution is still a function $f(\nicefrac{E}{T})$ of the ratio $\nicefrac{E}{T}$ to some \emph{local} temperature $T$, we Taylor expand the distribution function and its derivative just like for photons \label{eq:photon_distribution_expansion} and rewrite the Boltzmann equation.
At zero order we get the redshift $T\propto a^{-1}$ of neutrino temperature
{\small
\begin{align}\label{eq:Boltzmann_LHS_neutrino_zero_order}
\begin{aligned}
    &\cancel{\tfrac{E}{a}E\partial_E\bar{f}}\bigg[\tfrac{1}{\bar{T}}\tfrac{d\bar{T}}{d\tau}+\mathcal{H}\bigg]=0 \quad \Rightarrow \quad \bar{T}_\nu\propto \frac{1}{a}.\\[3pt]
\end{aligned}
\end{align}}
At first order, replacing $\frac{\dot{T}}{T}=-\mathcal{H}$, we get
{\small
\begin{align}\label{eq:Boltzmann_LHS_neutrino_first_order_newtonian}
\begin{aligned}
    &-\tfrac{E}{a}E\partial_E\bar{f}\bigg[\frac{\partial\mathcal{N}}{\partial \tau}+\hat{p}^j\frac{\partial\mathcal{N}}{\partial x^j}-\mathcal{H}E\frac{\partial \mathcal{N}}{\partial E}+\hat{p}^j\frac{\partial\psi}{\partial x^j}-\frac{\partial \phi}{\partial\tau}\bigg] = 0.
\end{aligned}
\end{align}}
Note there is indeed no right-hand side, since neutrinos have no sizable interactions after decoupling.
On the other hand, we cannot drop the term $\mathcal{H}E\frac{\partial \mathcal{N}}{\partial E}$ as we did for photons, since neutrinos masses will eventually make $\mathcal{N}$ depend on $E$ at late times.


%=====================================================================
\subsection{Passing to multipoles}
%====================================================================
We now pass to Fourier space and decompose the angular dependence of $\mathcal{N}$ in multipoles to get a hierarchy of coupled ODEs for the multipole moments $\mathcal{N}_\ell$.
Passing to Fourier space in \eqref{eq:Boltzmann_LHS_neutrino_first_order_newtonian}, we get
{\small
\begin{align}\label{eq:Boltzmann_LHS_neutrino_first_order_newtonian_Fourier}
\begin{aligned}
    &\bigg[\frac{\partial\mathcal{N}}{\partial \tau}+i\vec{k}\cdot\hat{p}\,\mathcal{N}-\mathcal{H}E\frac{\partial \mathcal{N}}{\partial E}+i\hat{p}\cdot\vec{k}\,\psi-\frac{\partial \phi}{\partial\tau}\bigg] = 0.
\end{aligned}
\end{align}}
The dependence on $\vec{k}$ only enter via the modulus $k$ and the angle $\mu:=\hat{k}\cdot\hat{p}$, so we can decompose the angular dependence in Legendre polynomials $P_\ell(\mu)$ as
{\small
\begin{align}
    \mathcal{N}(\tau, k, \mu) = \sum_{\ell=0}^\infty \frac{2\ell+1}{i^\ell} \mathcal{N}_\ell(\tau, k) P_\ell(\mu)\,,\quad
    \mathcal{N}_\ell(\tau, k) = \frac{i^\ell}{2}\int_{-1}^1 d\mu\, P_\ell(\mu) \mathcal{N}(\tau, k, \mu).
\end{align}}
The first three multipoles have a simple physical interpretation: up to numerical factors, $\mathcal{N}_0$ is the monopole i.e. the average temperature perturbation; $\mathcal{N}_1$ is the dipole i.e. the bulk velocity of the neutrino fluid; $\mathcal{N}_2$ is the quadrupole i.e. the anisotropic stress of the neutrino fluid.
Higher multipoles $\mathcal{N}_\ell$ with $\ell\geq 3$ encode higher-order angular correlations of the temperature perturbation, and are generated by free-streaming of neutrinos after decoupling.

Recall Legendre multipoles are orthogonal, satisfy a recursion relation and the first 3 are
{\small
\begin{align}
    \int_{-1}^1 \!\!\!\!d\mu\, P_\ell(\mu) P_{\ell'}(\mu) = \tfrac{2}{2\ell+1} \delta_{\ell \ell'},\,\,
    \mu P_\ell(\mu) = \tfrac{\ell+1}{2\ell + 1} P_{\ell+1}(\mu) + \tfrac{\ell}{2\ell + 1} P_{\ell-1}(\mu),\,\,
    P_0(\mu) = 1, \,\, P_1(\mu) = \mu, \,\, P_2(\mu) = \frac{3\mu^2 - 1}{2}.
\end{align}}
Multiplying the equation by $\frac{i^\ell}{2}P_\ell(\mu)$ and integrating over $\mu$, using the orthogonality and recursion relations, and the fact that $\psi$ and $\phi$ are independent of $\mu$, we get
{\small
\begin{align}
    \frac{\partial \mathcal{N}_\ell}{\partial \tau} + k\left(\frac{\ell+1}{2\ell + 1} \mathcal{N}_{\ell+1} - \frac{\ell}{2\ell + 1} \mathcal{N}_{\ell-1}\right) - \mathcal{H}E\frac{\partial \mathcal{N}_\ell}{\partial E} -\tfrac13 k\psi \delta_{\ell 1} - \frac{\partial \phi}{\partial\tau} \delta_{\ell 0} = 0.
\end{align}}
The Boltzmann equation becomes a hierarchy of coupled differential equations for multipole moments $\mathcal{N}_\ell$, which can be solved numerically to get the evolution of neutrino perturbations.
Notably, because of masses we retain the energy terms $E\frac{\partial \mathcal{N}_\ell}{\partial E}$ and we get linear first-order PDEs instead of ODEs as for photons.
We could then Fourier transform the energy dependence and solve the resulting system of ODEs for each value of the Fourier variable, albeit computationally expensive.
Another approach is to average over the energy dependence to get a system of ODEs for the energy-averaged multipoles, which is computationally cheaper but less accurate at late times (cf. \cite{MaBertschinger_CosmologicalPerturbationTheorySynchronousConformalNewtonianGauges_1995} for details).

The hierarchy is anyway infinite and thus needs to be truncated at some $\ell_\text{max}$ just like for photons, \questiontag{Shorten \& refer to $\gamma$ section}albeit with a higher $\ell_\text{max}$ since neutrinos free-stream and develop higher multipoles already at early times, while photons are driven towards the first two multipoles by Thomson scattering until recombination and develop higher multipoles only after decoupling.
A convenient way is to impose some boundary condition at $\ell_\text{max}$, for example ${\mathcal{N}}_{\ell_\text{max}+1} = {\mathcal{N}}_{\ell_\text{max}}$ or ${\mathcal{N}}_{\ell_\text{max}+1} = 0$, which is a good approximation since higher multipoles are generated by free-streaming and thus are suppressed by powers of $k/\mathcal{H}$ at early times when the hierarchy is sourced\todotag{write eq from Whinter}
\begin{align}
    ...
\end{align}

Passing to variable $\log a$ to highlight dominant terms in the super(sub)-ubhorizon regime $k/\mathcal{H}\ll1$ we get
{\small\begin{align}
\begin{aligned}
    \frac{\partial \mathcal{N}_0}{\partial \log a} - E{\frac{\partial {\mathcal{N}}_0}{\partial E}}- \frac{\partial \phi}{\partial\log a}+ \frac{k}{\mathcal{H}} {\mathcal{N}}_1 &= 0,\\
    \frac{\partial {\mathcal{N}}_1}{\partial \log a} -E{\frac{\partial {\mathcal{N}}_1}{\partial E}}+ \tfrac{2}{3} \frac{k}{\mathcal{H}} {\mathcal{N}}_2 -\tfrac13 \frac{k}{\mathcal{H}}{\mathcal{N}}_0- \tfrac{1}{3} \frac{k}{\mathcal{H}} \psi &= 0,\\
    \frac{\partial {\mathcal{N}}_2}{\partial \log a} -E{\frac{\partial {\mathcal{N}}_2}{\partial E}}+ \tfrac{3}{5} \frac{k}{\mathcal{H}} {\mathcal{N}}_3 - \tfrac{2}{5} \frac{k}{\mathcal{H}} {\mathcal{N}}_1 &= 0,\\
    \frac{\partial {\mathcal{N}}_\ell}{\partial \log a} -E{\frac{\partial {\mathcal{N}}_\ell}{\partial E}}+ \frac{k}{\mathcal{H}}\left(\tfrac{\ell+1}{2\ell + 1} {\mathcal{N}}_{\ell+1} - \tfrac{\ell}{2\ell + 1} {\mathcal{N}}_{\ell-1}\right) &= 0, \quad \ell\geq 3.
\end{aligned}
\end{align}}
Note only the first 2 equations for ${\mathcal{N}}_0, {\mathcal{N}}_1$ are sourced (by metric potentials), while the 3rd one describe the evolution of the neutrino anisotropic stress ${\mathcal{N}}_2$ which is nonnegligible (for accuracy below 10\%) on small scales even at early times.

Recalling that ${\mathcal{N}}_0= \frac{\delta T_\nu}{T_\nu}= \tfrac14 \delta_\nu$ as confirmed by $\rho_\nu \propto T_\nu^4$ and ${\mathcal{N}}_1 = -\tfrac13 v_\nu$, we see the first 2 equations are simply the continuity and Euler equations for neutrinos, while the last one is the evolution equation for neutrino \todotag{Check \& finish}anisotropic stress
\begin{align}
    ...
\end{align}



%--------------------------------------------------------------
%=====================================================================
\section{Neutrino masses}
%====================================================================
%---------------------------------------------------------------


%==========================================================
\subsection{Other bounds}
%========================================================


%---------------------------------------------------------
\subsubsection{Lower bound from neutrino oscillations.}
%----------------------------------------------------------
Neutrino oscillation experiments determine \emph{mass-squared differences}
\(\Delta m^2_{21} \equiv m_2^2 - m_1^2\) and \(\Delta m^2_{31} \equiv m_3^2 - m_1^2\)
(and the mixing angles), but they do \emph{not} fix the absolute mass scale.
(i.e.\ they are insensitive to adding a common mass to all eigenstates).
Nevertheless, since the measured splittings are nonzero, at least one neutrino mass eigenvalue must satisfy
\begin{align}
\begin{aligned}
	m_\text{heavy} \;\gtrsim\; \sqrt{|\Delta m^2_{31}|}
	\;\sim\; 0.05 \text{eV},
\end{aligned}
\end{align}
so neutrinos cannot all be massless.
Using the oscillation data and taking the lightest mass to zero gives a robust \emph{minimum} for the sum of masses:
\begin{align}
\begin{aligned}
	\Sigma m_\nu \equiv m_1 + m_2 + m_3 \;\gtrsim\;
	\begin{cases}
		0.058\ \text{eV}, & \text{normal ordering (NO)},\\
		0.098\ \text{eV}, & \text{inverted ordering (IO)}.
	\end{cases}
\end{aligned}
\end{align}




%==========================================================
\subsection{Bounds from cosmology}
%========================================================
%
We can get bound on neutrino masses from cosmology quite easily.
For example, since neutrino eventually become nonrelativistic matter, they contribute to the matter density today and we must have
\begin{align}
    \Omega_\nu =\frac{\rho_\nu}{\rho_{c0}}\simeq \frac{m_\nu n_\nu}{\rho_{c0}} \leq \Omega_m \simeq 0.3
\end{align}
The approximation $\rho_\nu \simeq m_\nu n_\nu$ is valid at late times $T_\nu\ll m_\nu$ and can be made more precise integrating the Fermi-Dirac distribution and expanding in $T_\nu/m_\nu$ to get
\begin{align}
    \rho_\nu = m_\nu n_\nu \left(1 + \frac32 \frac{T_\nu}{m_\nu} + O\Big(\frac{T_\nu^2}{m_\nu^2}\Big)\right).
\end{align}
Since neutrinos decouple around $T_\nu=T_\gamma \simeq 1\, \text{MeV}$, their number density just dilute as $n_\nu \propto a^{-3}$ afterward, and we later get $T_\nu= (4/11)^{1/3} T_\gamma$ due to electron-positron annihilation heating up photons but not neutrinos, so that\footnote{The factor $\tfrac34$ is from Fermi vs Bose statistics of neutrinos and photons, while the factor $\tfrac32$ is from 3 neutrino flavors vs 2 photon polarizations.}
\begin{align}
    n_\nu(a_0) = \tfrac34 \tfrac32\Big(\frac{T_\nu}{T_\gamma}\Big)^3 n_\gamma(a_0) = \tfrac34 \tfrac32 \tfrac{3}{11}n_\gamma(a_0) = \frac{27}{88} \frac{2\zeta(3)}{\pi^2} T_{\gamma 0}^3.
\end{align}
Up to numerical factors of $O(1)$, after plugging in numbers, we get
\begin{align}
     m_\nu \lesssim \frac{\Omega_m \rho_{c0}}{n_{\nu0}} \sim \frac{\Omega_{m0}\frac{3H_0^2}{8\pi G}}{\tfrac{1}{10} T_{\gamma 0}^3} \quad \Rightarrow \quad m_\nu \lesssim 0.5\, \text{eV}.
\end{align}
This is just a rough estimate, but more careful analyses using CMB and LSS data give a bound of $\sum m_\nu \lesssim 0.1$ eV on the sum of neutrino masses, which is already quite close to the lower bound from oscillation experiments.