% !TeX root = ../geometry_main.tex
%==========================================================
%=========================================================
\chapter{Towards General Relativity}
%=========================================================
%=========================================================



%-----------------------------------------------------------
%=======================================================
\section{Conservation laws}\todotag{To be written}
%=======================================================
%------------------------------------------------------------


\begin{mytheorem}[What even are energy and mometum?]
In flat spacetime physics we are used to notions like energy and momentum.
To make sense of similar notions in curved spacetime, we first need to investigate what these quantities really are.
Energy and momentum are the associated to symmetries of spacetime. 
Rigorously, they are the Noether charges corresponding to time-translation and space-translation invariance of the laws of physics.
In flat spacetime, the equations we write down typically respect these symmetries, and hence we have conservation of energy and momentum.

In curved spacetime, even if we write down equations that do not explicitly depend on spacetime position, the curvature of spacetime itself can break these symmetries i.e. practically the metric $g_{\mu\nu}$ may depend on spacetime position.
This means that in general curved spacetimes, we may not have global conservation of energy and momentum and in fact we may not even be able to define them properly.

In any case, we will \emph{ientify} energy and momentum with the Noether charges associated to time-translation and space-translation invariance respectively, whenever these symmetries exist.
In practice, whenever we have a timelike Killing vector field, we can define a conserved energy, and whenever we have spacelike Killing vector fields, we can define conserved momentum in that direction.
\end{mytheorem}


\begin{mytheorem}[Killing vector field]\todotag{Finish}
A Killing vector field $K$ is a vector field that generates an isometry of the metric, i.e. the Lie derivative of the metric along $K$ vanishes:
\begin{align}
    \mathcal{L}_K g = 0.
\end{align}
In components, this is equivalent to
\begin{align}
    \nabla_\mu K_\nu + \nabla_\nu K_\mu = 0.
\end{align}

How many Killing vector fields a spacetime has is a measure of its symmetries.
The maximum number of linearly independent Killing vector fields in a $n$-dimensional spacetime is $n(n+1)/2$, which is achieved by maximally symmetric spacetimes like Minkowski, de Sitter and anti-de Sitter spacetimes.
This number is intuitively clear: first, locally--i.e. in the tangent space--we have $n$ directions of translation symmetry and $n(n-1)/2$ directions of rotation symmetry; second, $n(n+1)/2$ is precisely the number of independent components of the metric tensor $g_{\mu\nu}$, so that exceeding this number would overconstrain the metric.
\end{mytheorem}


\begin{mytheorem}[The variation of a tensor is a tensor.]\todotag{move it}
Indeed the vairation of a tensor $A^\bullet_\bullet$ is simply defined as 
\begin{equation}
    \delta A^\bullet_\bullet(x):= \frac{d A^\bullet_\bullet(x,\epsilon)}{d\epsilon}_{\mid \epsilon=0} \cdot \epsilon =\lim_{h\to0} \frac{A^\bullet_\bullet(x,h)-A^\bullet_\bullet(x,0)}{h} \cdot \epsilon,
\end{equation}
where $\epsilon\to A^\bullet_\bullet(x,\epsilon)$ parametrizes a family of tensors at the same point $x\in M$, so that we can take the usual Newton derivative.
At each $|h|>0$ the ratio at the right-hand side is a tensor, being the difference and multiple of tensors, so that also the limit transforms as a tensor.
\end{mytheorem}


\begin{mytheorem}[Energy-momentum tensor \& conservation laws] \todotag{Finish}
The energy-momentum tensor $T^{\mu\nu}$ is a symmetric rank-2 tensor that encodes the density and flux of energy and momentum in spacetime.
In flat spacetime we can define directly as the Noether current associated to invariance of the action under spacetime translations 
\begin{align}
    x^\mu &\to x^\mu + \epsilon^\mu,\qquad \phi_a(x) \to \phi_a(x) + S^a_b \phi_b(x)- \epsilon^\mu \partial_\mu \phi_a(x).
\end{align}
%
In curved spacetime we do not have such symmetries in general, precisely because the nontrivial spacetime curvature $g_{\mu\nu}(x)$ might obstruct these symmetries.
It is then intuitive that \emph{whatever we might identify with energy and momentum} should be very \emph{sensitive to variations of the metric} $g_{\mu\nu}$ since indeed a trivial metric $g_{\mu\nu}=\eta_{\mu\nu}$ would immediately furnish such conserved quantities.
Namely we define the energy-momentum as the tensorial quantity identified in the variation of the matter action $S_\text{matter}$ for $g_{\mu\nu}\mapsto g_{\mu\nu} + \delta g_{\mu\nu}$\todotag{Fix}
\begin{align}
    \delta S_\text{matter} = \int \mathrm{d}^dx\,\sqrt{-g} \delta g^{\mu\nu} \bigg[\frac{\delta\mathcal{L}_{\text{matter}}}{\delta g^{\mu\nu}}-\bigg] =: -\frac{1}{2} \int \mathrm{d}^4 x \sqrt{-g}\,\ \delta g^{\mu\nu} T_{\mu\nu}.
\end{align}
This quantity is indeed a tensor since $\sqrt{|g|}\, \mathrm{d}^4 x $ is the volume form, $\delta g^{\mu\nu}$ is a tensor and the action is a scalar.
The quantity $T_{\mu\nu}$ must then transform as tensor in order to keep $T_{\mu\nu} \delta g^{\mu\nu}$ a scalar.
\smallskip

The energy-momentum tensor obey the conservation equation
\begin{align}
    \nabla_\mu T^{\mu\nu} = 0,\qquad \nu=0,\ldots, n-1,
\end{align}
which is a consequence of diffeomorphism invariance of the matter action.
It is crucial to stress this should \emph{not} be interpreted as a global conservation law for energy and momentum, since in general curved spacetimes we do not have global energy and momentum definitions.
These $n$ equations are indeed just unavoidabe geometric consistency constraints that follows from demanding the action of the matter field be diffeomorphism invariant.
Note that we have precisely $n$ equations, since all diffeomorphisms are generated by $n$ vector fields in $n$ dimensions.
\end{mytheorem}


\begin{mytheorem}[Conserved quantities along Killing vector fields: point \& continuum distributions]\todotag{Finish}
Consider a particle moving along a geodesic $x^\mu(\lambda)$ with $4$-momentum $p^\mu = m\,\mathrm{d}x^\mu/\mathrm{d}\lambda$.
Note that $\lambda$ has energy-dimension $[E]^{-2}$.
For massive particles (time-like geodesics)we simply take $\lambda=\tau/m$ where $\tau$ is the proper time along the worldline  $\partial_\tau x^\mu \partial_\tau x_\mu=-1$, while for massless particles-i.e. null geodesics $\partial_\lambda x^\mu \partial_\lambda x_\mu=0$--the parameter $\lambda$ is \emph{defined} so that the geodesic equation is obeyed.
If $K^\mu$ is a Killing vector field, the following quantity is conserved along the geodesic
\begin{align}
    Q = K_\mu p^\mu, \qquad \frac{\mathrm{d}Q}{\mathrm{d}\lambda} = 0.
\end{align}
Similarly, for a continuum distribution of matter with stress-energy tensor $T^{\mu\nu}$, the following current is conserved
\begin{align}
    J^\mu = T^{\mu\nu} K_\nu,\qquad \nabla_\mu J^\mu = 0.
\end{align}
\end{mytheorem}
\begin{proof}
For the point like particle
\begin{align}
    \frac{\mathrm{d}}{\mathrm{d}\lambda}(K_\mu p^\mu) &= p^\nu \nabla_\nu (K_\mu p^\mu) = p^\nu (\nabla_\nu K_\mu) p^\mu + K_\mu\, p^\nu (\nabla_\nu p^\mu)
     =\tfrac{1}{2}p^\mu p^\nu \underbrace{(\nabla_\nu K_\mu + \nabla_\mu K_\nu)}_{=0} + K_\mu\, \underbrace{p^\nu \nabla_\nu p^\mu}_{=0}= 0.
\end{align}
For the continuum distribution
\begin{align}
    \nabla_\mu J^\mu &= \nabla_\mu (T^{\mu\nu} K_\nu) = (\nabla_\mu T^{\mu\nu}) K_\nu + T^{\mu\nu} (\nabla_\mu K_\nu) = \underbrace{(\nabla_\mu T^{\mu\nu})}_{=0} K_\nu + \tfrac{1}{2} T^{\mu\nu} \underbrace{(\nabla_\mu K_\nu + \nabla_\nu K_\mu)}_{=0} = 0.
\end{align}
\end{proof}

