% !TeX root = ../geometry_main.tex
%==========================================================
%=========================================================
\chapter{Differential Geometry}
%=========================================================
%=========================================================


In this long chapter, we collect various random topics in differential geometry that I feel like writing down.


%-----------------------------------------------------------------------
%========================================================
\section{Fundamentals} \todotag{To be written}
%=========================================================
%----------------------------------------------------------------------




%-----------------------------------------------------------------------
%========================================================
\section{Sub-bundles, flows \& Frobenius theorem} \todotag{To be written}
%=========================================================
%-----------------------------------------------------------------------


\begin{mytheorem}[Commutation of vector fields, flows \& straightening coordinates]
The offset of the flow of two vector fields $X,Y \in \mathfrak{X}(M)$ reads in a chart
\begin{align}
\begin{aligned}
    \phi_{X}^{\epsilon} \circ \phi_{Y}^{\delta} - \phi_{Y}^{\delta} \circ \phi_{X}^{\epsilon}= \epsilon \delta [X,Y] + \mathcal{O}(\epsilon^2, \delta^2).
\end{aligned}
\end{align}
The following conditions are equivalent for the commutation of flows
\begin{itemize}
    \item The flows commute
        \begin{align}
            \phi_{X_1}^{\epsilon_1}\circ\dots\circ \phi_{X_k}^{\epsilon_k} = \phi_{X_{\sigma_1}}^{\epsilon_{\sigma_1}}\circ\dots\circ \phi_{X_{\sigma_k}}^{\epsilon_{\sigma_k}} \quad \forall \epsilon_i \in \mathbb{R},\,\, \forall \sigma \in \mathcal{S}_k.
        \end{align}
    \item The vector fields commute, i.e. all Lie brackets vanish
        \begin{align}
            [X_i, X_j] = 0 \quad \forall i,j=1\dots k.
        \end{align} 
    \item There exist local coordinates $\{x^1, \dots, x^n\}$ such that
        \begin{align}
            X_i = \partial_{x^i} \quad \forall i=1\dots k.
        \end{align}
\end{itemize}
\end{mytheorem}
\begin{proof}
The first equation is follows from Taylor expanding the flows.
Point 3 $\Rightarrow$ 1, 2 trivially follows from the definitions.
Point 1 $\Rightarrow$ 2 follows from the first equation by taking the limit $\epsilon, \delta \to 0$.
For point 2 $\Rightarrow$ 1 take a chart where $X_1 = \partial_{x^1}$, then $\phi_{X_1}^{\epsilon_1}(p)=p+\hat{e}_1\epsilon_1$ and $[X_1,X_j]=-\partial_{x^1} X_j =0$ implies that the components of $X_j$ do not depend on $x^1$.
Then we verify
\begin{equation}
    \phi_{X_1}^{\epsilon_1}\circ \phi_{X_j}^{\epsilon_j}(p) = \phi_{X_j}^{\epsilon_j}(p) + \epsilon_1\, \hat{e}_1= \phi_{X_j}^{\epsilon_j}(p+ \epsilon_1\, \hat{e}_1) = \phi_{X_j}^{\epsilon_j}\circ \phi_{X_1}^{\epsilon_1}(p).
\end{equation}
The rest follows by induction.
Finally for point 1 $\Rightarrow$ 3 consider the map $ \R^k\times \underbrace{M^{(n)}}_{\simeq \R^n}$ to $ \underbrace{M^{(n)}}_{\simeq \R^n}$ defined as
\begin{equation}
    F(\epsilon_1, \dots, \epsilon_k, p) := \phi_{X_1}^{\epsilon_1}\circ\dots\circ \phi_{X_k}^{\epsilon_k} (p).
\end{equation}
By point 1 the $(k+n)\times n$ jacobian matric of $F$ at any $\bar{\epsilon}\in \R^k$ and any $p\in M$ reads
\begin{equation}
    J_{F}(\bar{\epsilon}=0, p) = \begin{pmatrix}
    X_1(p) & X_2(p) & \cdots & X_k(p) & \mid &\mathbb{I}_n
    \end{pmatrix}.
\end{equation}
By the open mapping theorem, we can locally restrict $F$ to a diffeomorphism $\R^k\times \R^{n-k} \to M^(n)$ which furnishes the desired coordinates.
%
\end{proof}


\begin{mytheorem}[Flows \& Lie brackets of $f$-related vector fields]
%
For any given map $f:M\to N$ between two manifolds and vector fields $X_1,X_2\in \mathfrak{X}(M)$ and $Y_1,Y_2 \in \mathfrak{X}(N)$ that are $f$-related, i.e. such that
\begin{equation}
    f_* X_p = Y_{f(p)} \quad \forall p\in M,
\end{equation}
the flows and Lie brackets are also $f$-related, i.e.
\begin{align}
    f_* &(\phi_{X_i,\epsilon} (p)) = \phi_{Y_i,\epsilon} (f(p)) \quad \forall p\in M,
    \\[6pt]
    f_*& [X_1, X_2]_p = [Y_1, Y_2]_{f(p)} \quad \forall p\in M.
\end{align}
\end{mytheorem}


\begin{mytheorem}[Foliations, distributions \& Frobenius theorem]
%
A disribution is a sub-bundle $S \subseteq TM$ of the tangent bundle, that is a smooth assignment of a vector subspace $S_p \subseteq T_p M$ at each point $p\in M$.
A foliation is a decomposition of the manifold $M$ into a union of disjoint isomorphic connected submanifolds (the leaves) of the same dimension, so that locally the manifold looks like a product $M \simeq L^{(k)} \times N^{(n-k)}$ of a leaf $L^{(k)}$ and a `trunk' manifold $N^{(n-k)}$.
In particular for each point $p\in M$ there is a unique leaf $L_p$ passing through it, and the tangent space to the leaf at $p$ defines a subspace $T_p L_p \subseteq T_p M$.
Simple examples of foliations are
\begin{itemize}
    \item The foliation of $\mathbb{R}^3$ into planes of constant $z$ coordinate, with leaves $L^{(2)} = \{(x,y,z_0) \mid x,y\in \mathbb{R}\}$ for each $z_0\in \mathbb{R}$;
    \item The foliation of $\mathbb{R}^3\setminus \{0\}$ into spheres of constant radius, with leaves $L^{(2)} = \{(x,y,z) \mid x^2+y^2+z^2 = r_0^2\}$ for each $r_0>0$.
\end{itemize}

\textbf{Frobenius theorem}: a distribution $S \subseteq TM$ is integrable-- i.e. arises as the tangent bundle of a foliation-- if and only if it is involutive-- i.e. closed under Lie brackets.
In formulas
\begin{align}
\begin{aligned}
    &S \text{ integrable} \quad \Leftrightarrow \quad S_p = T_pL_p \,\, \forall p\in M\text{ for a foliation } M = \bigsqcup L^{(k)} 
    \quad \Leftrightarrow \quad [X,Y] \in \Gamma(S) \quad \forall X,Y \in \Gamma(S).  
\end{aligned}
\end{align}
%
\end{mytheorem}
\begin{proof}
The proof is constructive and quite long, so we just sketch the main ideas.
\begin{itemize}
    \item[$\Rightarrow$] If $S$ is integrable, then for any two vector fields $X,Y \in \Gamma(S)$-- i.e. tangent to the leaves-- their Lie bracket $[X,Y]$ is also tangent to the leaves, so that $[X,Y] \in \Gamma(S)$.
    \item[$\Leftarrow$] If $S$ is involutive, then for any point $p\in M$ we can find a local basis of vector fields $\{X_1, \dots, X_k\} \subseteq \Gamma(S)$ spanning $S$ in a neighborhood of $p$.
    By involutivity, their Lie brackets vanish $[X_i,X_j]=0$, so that by the previous theorem we can find local coordinates $\{x^1, \dots, x^n\}$ such that $X_i = \partial_{x^i}$ for $i=1\dots k$.
    Then the submanifolds defined by constant values of the last $n-k$ coordinates
    \begin{equation}
        L^{(k)}_{c_{k+1}, \dots, c_n} := \{ (x^1, \dots, x^n) \mid x^{k+1} = c_{k+1}, \dots, x^n = c_n\}
    \end{equation}
    define a foliation whose tangent spaces coincide with $S$.
\end{itemize}
\end{proof}
\todotag{Modify pic}
\begin{figure}[h]
    \centering
    \begin{tikzpicture}[scale=1.0]
        \def\R{2.1}
        \def\r{0.55}
        \pgfmathdeclarefunction{torusX}{2}{\pgfmathparse{(\R + \r*cos(#2)) * cos(#1) + 0.35*\r*sin(#2)}}
        \pgfmathdeclarefunction{torusY}{2}{\pgfmathparse{0.6*(\R + \r*cos(#2)) * sin(#1) + 0.45*\r*sin(#2)}}

        % Silhouette of the torus
        \shade[ball color=gray!25, opacity=0.8] (0,0) ellipse (2.9 and 1.8);
        \fill[white] (0,0) ellipse (1.15 and 0.6);
        \draw[line width=1.1pt, black!65] (0,0) ellipse (2.9 and 1.8);
        \draw[line width=1.1pt, black!45] (0,0) ellipse (1.15 and 0.6);

        % Leaves wrapping around the torus with an irrational slope
        \def\alpha{1.45}
        \foreach \shift/\col in {0/forestgreen!80!black, 70/forestgreen!60!black, 140/forestgreen!60!black}{
            \draw[\col, line width=1pt, samples=240, smooth, domain=0:360, variable=\u]
                plot ({torusX(\u,\alpha*\u+\shift)}, {torusY(\u,\alpha*\u+\shift)});
        }
        % A meridional leaf for contrast
        \draw[blue!70!black, line width=0.9pt, samples=200, smooth, domain=0:360, variable=\u]
            plot ({torusX(\u,\u)}, {torusY(\u,\u)});
    \end{tikzpicture}
    \caption{Sample foliation of a $2$-torus embedded in $\mathbb{R}^3$, with leaves wrapping around both directions.}
    \label{fig:torus-foliation}
\end{figure}


%-----------------------------------------------------------------------
%========================================================
\section{Derivatives on tensors}
%=========================================================
%-----------------------------------------------------------------------

There are very many notions of derivatives on tensors.
In general, a (anti)-derivative of degree $(h,k)$ on tensors is a map from a suitable subset $\mathcal{D}$ of the tensor bundle 
\begin{equation}
    \delta: T^{\bullet}_{\bullet} (M)\subseteq \mathcal{D} \to T^{\bullet + h}_{\bullet + k} (M)
\end{equation}
that is linear and satisfies a Leibniz rule, up to a sign.
That is 
\begin{equation}
    \delta (S \otimes T) = (\delta S) \otimes T + (\pm)^{pq} S \otimes (\delta T),
\end{equation}
We call it \emph{anti}derivative if it changes sign when acting on the two factors of a tensor product.
For example, the exterior derivative $d$ is an antiderivative of degree $(0,1)$ on  differential forms, and the inner derivative $i_X$ along a vector field $X$ is an antiderivative of degree $(0,-1)$ on differential forms. 


\begin{mytheorem}[Uniqueness theorem for derivatives on tensors]
An (anti)-derivative $\delta$ on tensors--i.e. a linear map and obeying Leibniz rule up to a possible minus sign-- is completely determined by its action on functions and vector fields, or equivalently on functions and 1-forms.
\end{mytheorem}



\begin{mytheorem}[The variation of a tensor is a tensor.]\todotag{move it}
Indeed the vairation of a tensor $A^\bullet_\bullet$ is simply defined as 
\begin{equation}
    \delta A^\bullet_\bullet(x):= \frac{d A^\bullet_\bullet(x,\epsilon)}{d\epsilon}_{\mid \epsilon=0} \cdot \epsilon =\lim_{h\to0} \frac{A^\bullet_\bullet(x,h)-A^\bullet_\bullet(x,0)}{h} \cdot \epsilon,
\end{equation}
where $\epsilon\to A^\bullet_\bullet(x,\epsilon)$ parametrizes a family of tensors at the same point $x\in M$, so that we can take the usual Newton derivative.
At each $|h|>0$ the ratio at the right-hand side is a tensor, being the difference and multiple of tensors, so that also the limit transforms as a tensor.
\end{mytheorem}


%========================================================
\subsection{Lie derivative}\todotag{Merge various definitions into bullet list}
\label{sec:lie_derivative}
%=========================================================

\begin{mytheorem}[Lie derivative as commutator of vector fields]
The Lie derivative $\mathcal{L}_X$ along a vector field $X$ is the unique derivative of degree $(0,0)$ on tensors that reduces to the usual directional derivative when acting on functions and to the usual Lie bracket $\mathcal{L}_X Y = [X,Y]$ when acting on vector fields.
It is then extended to all tensors using linearity and the Leibniz rule
\begin{equation}
    \mathcal{L}_X (S \otimes T) = (\mathcal{L}_X S) \otimes T + S \otimes (\mathcal{L}_X T).
\end{equation}
\end{mytheorem}


\begin{mytheorem}[Lie derivative as flow derivative]
The Lie derivative is equivalently defined as the infinitesimal generator of the flow induced by a vector field \(X\).
Namely, given a tensor field \(T\in T^{\bullet}_{\bullet} (M)\), we define the Lie derivative along \(X\) as
\begin{equation}
    \mathcal{L}_X T := \lim_{\epsilon \to 0} \frac{\phi_{X,\epsilon}^* (T) - T}{ \epsilon},
\end{equation}
where \(\phi_{X,\epsilon}: M \to M\) is the flow generated by the vector field \(X\) for parameter distance \(\epsilon\) and \(\phi_{X,\epsilon}^*\) is the pullback map induced by \(\phi_{X,\epsilon}\) on tensors.
\end{mytheorem}


\begin{mytheorem}[Cartan formula \& yet another way to define the Lie derivative]
A fundamental result linking the Lie derivative $\mathcal{L}_X$, the exterior derivative $d$ and the inner derivative $i_X$.
Namely, for any differential form $\omega$, we have
\begin{equation}
    \mathcal{L}_X \omega = i_X d \omega + d (i_X \omega).
\end{equation}
The right-hand side is defined only for differential forms, but the left-hand side makes sense for any tensor.
Since differential forms include functions (0-forms) and differentials (1-forms), we can use this formula to \emph{define} the Lie derivative and then extend it to all tensors using linearity and the Leibniz rule.
\end{mytheorem}

\begin{mytheorem}[The Lie derivative is ignorant of the curvature \& the metric]
As evident from any of its definitions, the Lie derivative does not depend on any notion of connection or metric on the manifold.
It only depends on the smooth-- i.e. differential-- structure of the manifold, being defined in terms of commutators, or flows or exterior and inner derivatives.
It thus cannot give any information on the curvature of the manifold beyond the constrained already encoded in the smooth structure itself.
\end{mytheorem}


\begin{mytheorem}[Lie derivative and pullback commute]
Given a diffeomorphism $\phi: M \to N$ between two manifolds and a vector field $Y\in \mathfrak{X}(N)$, the Lie derivative along $Y$ of any tensor field $T$ on $N$ and the pullback $\phi^*$ commute, i.e.
\begin{equation}
    \phi^* (\mathcal{L}_Y T) = \mathcal{L}_{\phi^* Y} (\phi^* T).
\end{equation}
%
More generally, if $X\in \mathfrak{X}(M)$, $Y\in \mathfrak{X}(N)$, and $S\in T^h_kM$, $T\in T^h_kN$ are $f$-related vector and tensor fields for a map $f:M\to N$, then the Lie derivatives along $X$ and $Y$ are also $f$-related, i.e.
\begin{equation}
    f_*\mathcal{L}_XS_{|p}= \mathcal{L}_Y T_{|f(p)} \quad \forall p\in M.
\end{equation}
\end{mytheorem}



\begin{mytheorem}[Lie derivative in coordinate basis \& replacing partial with covariant derivatives]
In a coordinate basis $\{\partial_\mu\}$, the Lie derivative along a vector field $X = X^{\mu} \partial_\mu$ acts on a $(h,k)$-tensor field $T$ as
\begin{align}
\begin{aligned}
    \mathcal{L}_X T^{\nu_1 \dots \nu_h}_{ \rho_1 \dots \rho_k} 
    &= X^{\mu} \partial_\mu\left( T^{\nu_1 \dots \nu_h}_{ \rho_1 \dots \rho_k} \right)
    - \sum_{i=1}^h \left(\partial_\mu X^{\nu_i}\right) T^{\nu_1 \dots \mu \dots \nu_h}_{ \rho_1 \dots \rho_k} 
    + \sum_{j=1}^k \left(\partial_{\rho_j} X^{\mu}\right) T^{\nu_1 \dots \nu_h}_{\rho_1 \dots \mu \dots \rho_k}.
\end{aligned}  
\end{align}
Note that the above expression is a tensor, even if it contains ordinary partial derivatives.
If we are given a metric $g$ \footnote{or equivalently whenever we have a torsion-less connection for which we can then find normal coordinates where the Christoffel symbols vanish} we can replace partial derivatives with covariant derivatives in normal coordinates, obtaining
\begin{align}
\begin{aligned}
    \mathcal{L}_X T^{\nu_1 \dots \nu_h}_{ \rho_1 \dots \rho_k} 
    &= X^{\mu} \nabla_\mu\left( T^{\nu_1 \dots \nu_h}_{ \rho_1 \dots \rho_k} \right)
    - \sum_{i=1}^h \left(\nabla_\mu X^{\nu_i}\right) T^{\nu_1 \dots \mu \dots \nu_h}_{ \rho_1 \dots \rho_k} 
    + \sum_{j=1}^k \left(\nabla_{\rho_j} X^{\mu}\right) T^{\nu_1 \dots \nu_h}_{\rho_1 \dots \mu \dots \rho_k}.
\end{aligned}
\end{align}
The last expression is tensorial by construction.
Since it matches the previous one in normal coordinates, it is indeed equivalent to it in any coordinate system.
\end{mytheorem}



%========================================================
\subsection{Parallel transport, connection \& covariant derivative}
%=========================================================


Parallel transport and covariant derivative are two faces of the same coin.
Namely, covariant derivative can be defined as the infinitesimal generator of parallel transport along curves.

\begin{mytheorem}[Parallel transport]
A parallel transport along a curve $\gamma: [0,1] \to M$ is a \emph{bijective linear} map 
\begin{equation}
    \tau_{\gamma}: T_{\gamma(0)} M \to T_{\gamma(1)} M
\end{equation}
such that for any two tensor fields
\begin{equation}\label{}
    \tau_{\gamma} (S \otimes T) = \tau_{\gamma} (S) \otimes \tau_{\gamma} (T).
\end{equation}
In order to obtain a well-defined notion of parallel transport, we must consistently define such a map for any curve $\gamma$ connecting any two points of the manifold.
Just like for dervatives on tensors, the parallel transport is completely determined by its action on functions and vector fields, or equivalently on functions and 1-forms.
\end{mytheorem}


\begin{mytheorem}[Connection]
A connection is a \emph{derivative} of degree $(0,1)$ on tensors, i.e. obeys linearity and Leibniz rule,
\begin{equation}
    \nabla: T^{\bullet}_{\bullet} (M) \to T^{\bullet}_{\bullet + 1} (M),
\end{equation}
that reduces to the usual directional derivative when acting on function.
It is completely determined by its action on functions and vector fields, or equivalently on functions and 1-forms.
In particular, one defines the \textbf{Christoffel symbols} $\Gamma^{\rho}_{\mu \nu}$ with respect to a coordinate basis as
\begin{equation}
    \nabla_{\partial_\mu} \partial_\nu = \Gamma^{\rho}_{\mu \nu} \partial_\rho.
\end{equation}
For the interpretation of the Christoffel symbols, see the discussion in \todotag{add ref to section}
By the Leibniz rule, the general action on a $(h,k)$-tensor field $T$ reads
\begin{equation}\label{eq:connection_on_tensor}
    \nabla_{\partial_\mu} T^{\nu_1 \dots \nu_h}_{ \rho_1 \dots \rho_k} 
    = \partial_\mu\left( T^{\nu_1 \dots \nu_h}_{ \rho_1 \dots \rho_k} \right)
    + \sum_{i=1}^h \Gamma^{\nu_i}_{\mu \sigma} \,T^{\nu_1 \dots \sigma \dots \nu_h}_{ \rho_1 \dots \rho_k} 
    - \sum_{j=1}^k \Gamma^{\sigma}_{\mu \rho_j}\, T^{\nu_1 \dots \nu_h}_{\rho_1 \dots \sigma \dots \rho_k}.
\end{equation}
Note that the value of $\nabla_X T$ at a point $p\in M$ depends only on the values of $T$ along any curve whose tangent at $p$ is $X_p$.
\end{mytheorem}

\begin{mytheorem}[Covariant derivative \& connection as infinitesimal generator of parallel transport]
Given a curve $\gamma: [0,1] \to M$ and a connection $\nabla$, the parallel transport $\tau_{\gamma}$ along $\gamma$ generated by $\nabla$ is defined as the unique map such that for any tensor field $T\in T_{\gamma(0)} M$ the curve $t \mapsto T(t) := \tau_{\gamma|_{[0,t]}} (T)$ satisfies the parallel transport equation
\begin{equation}
    0 = \nabla_{\dot{\gamma}(t)} T(t) 
\end{equation}
As usual, it is sufficient to define it on functions and vector fields, or equivalently on functions and 1-forms, and then extend it to all tensors using linearity and the Leibniz rule.
The parallel transport map is well-defined thanks to standard theorems on existence and uniqueness of solutions of ODEs.

If we are given a tensor field $S(t)$ along the curve $\gamma$, we can define the covariant derivative along $\gamma$ as
\begin{equation}
    \frac{D}{dt}S(t) := \lim_{\epsilon \to 0} \frac{\tau_{\gamma|_{[t+\epsilon,t]}}^{-1} (S(t+\epsilon)) - S(t)}{\epsilon} 
\end{equation}
That is we use parallel transport induced by $\nabla$ to compare tensors at different points along the curve.
This is indeed way $\nabla$ is called \emph{connection}: it gives a way to connect and compare `generalied tangent planes' i.e. tensors at different points.
As expected, we obtain back the covariant derivative gives back the original expression in terms of the connection
\begin{equation}
    \frac{D}{dt}S(t) := \lim_{\epsilon \to 0} \frac{\tau_{\gamma|_{[t+\epsilon,t]}}^{-1} (S(t+\epsilon)) - S(t)}{\epsilon}
    \equiv \nabla_{\dot{\gamma}(t)} S(t),
\end{equation}
where we use that \eqref{eq:connection_on_tensor} depends only on the values of $S$ along the curve whose tangent at $p$ is $\dot{\gamma}(t)$.
\end{mytheorem}



%========================================================
\subsection{The total covariant exterior derivative}
%=========================================================

See the sec







%-----------------------------------------------------------------------
%========================================================
\section{Differential forms \& integration} \todotag{To be written}
%=========================================================
%-----------------------------------------------------------------------


\begin{mytheorem}[Indices bookkeeping for differential forms]
The canonical basis for $\Lambda^k(M)$ is written as $dx^{i_1} \wedge \cdots \wedge dx^{i_k}$ with $i_1 < i_2 < \cdots < i_k$.
We will insetad use indices $\sigma_1, \dots, \sigma_k \subseteq \{1, \dots, n\}$ to denote unordered indices.
A $k$-form $\omega \in \Lambda^k (M)$ can be written equivalently as
\begin{align}\label{eq:indices_bookkeeping_differential_forms}
\begin{aligned}
    \omega &= \sum_{i_1 < \cdots < i_k} \omega_{i_1 \dots i_k} dx^{i_1} \wedge \cdots \wedge dx^{i_k}
    \\
    &= \frac{1}{k!} \sum_{\sigma_1, \dots, \sigma_k} \omega_{\sigma_1 \dots \sigma_k} dx^{\sigma_1} \wedge \cdots \wedge dx^{\sigma_k}
    \\
    &= \sum_{\sigma_1, \dots, \sigma_k} \omega_{\sigma_1 \dots \sigma_k}\, dx^{\sigma_1} \otimes \cdots \otimes dx^{\sigma_k},
\end{aligned}
\end{align}
In the first expression we do not sum over repeated indices, which are ordered $i_1 \dots i_k$.
In the second expression we instead sum over the \emph{unordered} indices $\sigma_1, \dots, \sigma_k$, we define $\omega_{\sigma_1 \dots \sigma_k}$ to be totally antisymmetric, and divide by $k!$ to avoid overcounting.
In the third expression we explicitate the wedge product in terms of antisymmetrized tensor product: because $\omega_{\sigma_1 \dots \sigma_k}$ is already totally antisymmetric, and we are still summing over the $\sigma$ indices, the double counting cancel $1/k!$ factor.
\end{mytheorem}



%========================================================
\subsection{Derivatives on differential forms} \todotag{To be written}
%=========================================================

For differential forms (anti)-derivatives of degree $k\in \mathbb{Z}$ obeys the Leibniz rule in the form
\begin{equation}
    \delta: \Lambda^\bullet (M) \to \Lambda^{\bullet+k} (M), \qquad \delta (\alpha \wedge \beta) = (\delta \alpha) \wedge \beta + (\pm1)^{|\alpha|} \alpha \wedge (\delta \beta).
\end{equation}
Note the sign is well-defined since differential forms are graded-commutative, i.e. $\alpha \wedge \beta = (-1)^{|\alpha||\beta|} \beta \wedge \alpha$.

\begin{mytheorem}[Lie derivative on differential forms]
The Lie derivative is a well-behaved derivative of degree $(0,0)$ on differential forms, defined as the restriction of the usual Lie derivative on tensors.
From the Leibniz rule for tensors and the linearity of the antisymmetrization map, it immediately follows that the Lie derivative also obeys the Leibniz rule wrt the wedge product
\begin{equation}
    \mathcal{L}_X(\alpha \wedge \beta) = (\mathcal{L}_X \alpha) \wedge \beta + \alpha \wedge (\mathcal{L}_X \beta).
\end{equation}
\end{mytheorem}


\begin{mytheorem}[Inner derivative]
The inner derivative is a \emph{anti}derivative of degree $(0,-1)$ on differential forms, defined as the contraction along a vector field $X$
\begin{equation}
    i_X: \Lambda^k (M) \to \Lambda^{k-1} (M)\quad  \mid\quad  i_X \omega := \omega (X, \cdot, \cdots, \cdot) 
\end{equation}
In a basis of $\Lambda^{k-1}(M)$ i.e. for \emph{ordered} indices $i_1< \dots <i_{k-1}$, but summing over \emph{any} value $\sigma\in \{1\dots n\}$, we have
\begin{equation}\label{eq:interior_derivative_basis_expr}
   (i_X \omega) = X^{\sigma}\, \omega_{\sigma i_1 \dots i_{k-1}} \, dx^{i_1} \wedge \cdots \wedge dx^{i_{k-1}}
\end{equation} 
where again $\omega_{\sigma_0 \dots \sigma_k}$ is totally antisymmetric.
Indeed, using the convention \eqref{eq:indices_bookkeeping_differential_forms}, we have
\begin{align}
\begin{aligned}
    i_X(\omega) &= \sum_{\sigma_0 \dots \sigma_{k-1}} \omega_{\sigma_0 \dots \sigma_{k-1}}\, dx^{\sigma_0}\otimes \cdots \otimes dx^{\sigma_{k-1}}\, (X, \cdots)
    \\
    &= \sum_{\sigma_0 \dots \sigma_{k-1}} X^{\sigma_0} \, \omega_{\sigma_0 \sigma_1 \dots \sigma_{k-1}}dx^{\sigma_1}\otimes \cdots \otimes dx^{\sigma_{k-1}}
    \\
    &= \sum_{\sigma_0 \in\{1\dots n\},\,\,i_1 < \cdots < i_{k-1}} X^{\sigma_0} \, \omega_{\sigma_0 i_1 \dots i_{k-1}}\,\,dx^{i_1}\wedge\cdots \wedge dx^{i_{k-1}}.
\end{aligned}
\end{align}
\end{mytheorem}


\begin{mytheorem}[Exterior derivative]
The exterior derivative is an \emph{anti}derivative of degree $(0,1)$ on differential forms, defined as the unique map
\begin{equation}
    d: \Lambda^k (M) \to \Lambda^{k+1} (M)
\end{equation}
that reduces to the usual differential when acting on functions and satisfies $d^2 = 0$.
Under this conditions, one proves the usual formula in a coordinate basis
\begin{equation}\label{eq:exterior_deriv_basis_expr}
    d \Big(\omega_{i_1 \dots i_k} dx^{i_1} \wedge \cdots \wedge dx^{i_k}\Big) = \partial_\mu \omega_{i_1 \dots i_k} \, dx^{\mu} \wedge dx^{i_1} \wedge \cdots \wedge dx^{i_k}.
\end{equation}
\end{mytheorem}


\begin{mytheorem}[Replacing partial with covariant derivatives]
The exterior derivative on a form $\omega=\omega_{i_1 \dots i_k} dx^{i_1} \wedge \cdots \wedge dx^{i_k}$ can equivalently be written in terms of covariant derivatives associated to a torsion-less connection,
\begin{align}
    d\omega &= \partial_\mu \omega_{i_1 \dots i_k} \, dx^{\mu} \wedge dx^{i_1} \wedge \cdots \wedge dx^{i_k}
    \equiv
    \nabla_\mu \omega_{i_1 \dots i_k} \, dx^{\mu} \wedge dx^{i_1} \wedge \cdots \wedge dx^{i_k}.
\end{align}
Indeed both expressions define tensors, the latter by definition of covariant derivative and the former by construction.
Since they coincide in normal coordinates where the Christoffel symbols vanish, they coincide in any coordinate system.
%
\end{mytheorem}


%========================================================
\subsection{Integration} \todotag{To be written}
%=========================================================

\begin{mytheorem}[Integration of differential forms]
Given an oriented submanifold $S^{(k)} \subseteq M^{(n)}$ of dimension $k$, we can integrate $k$-forms $\omega \in \Lambda^k (M)$ over $S$.
Namely, given a parametrization $\phi: U \subseteq \mathbb{R}^k \to S \subseteq M$, we define
\begin{equation}
    \int_S \omega := \int_U \phi^* (\omega).
\end{equation}
Note that $\phi^* (\omega) \in \Lambda^k (U)$ is a $k$-form on $\mathbb{R}^k$, which can be written as $f(x) dx^1 \wedge \cdots \wedge dx^k$ for some function $f: U \to \mathbb{R}$, and then integrated as usual.
The orientation of $S$ is necessary to have a well-defined integral, since changing the orientation changes the sign of the integral, but orientability of $M$ is not necessary.
\end{mytheorem}

\begin{mytheorem}[Stokes theorem]
This is simply the fundamental theorem of calculus for differential forms.
Namely, for any \emph{oriented} manifold $M^{(n)}$ with boundary $\partial M$ and any differential form $\omega \in \Lambda^{n-1} (M)$, we have
\begin{equation}
    \int_M d\omega = \int_{\partial M} \omega.
\end{equation}
In particular, Stokes theorem formalizes a fundamental relation between the purely algebraic property $d^2 = 0$ and the purely geometric property $\partial ^2 M =0$ i.e. the `boundary of a boundary is empty'.
\end{mytheorem}


%-----------------------------------------------------------------------------
\subsubsection{Rediscovering familiar operations}
%-----------------------------------------------------------------------------

\begin{mytheorem}[Divergence \& Gauss theorem.]
Given a volume form $\Omega$ and a vector field $X$ on a manifold $M^{(n)}$, define the \textbf{divergence} as the $n$-form
\begin{equation}
    \mathrm{div}_\Omega (X) := \mathcal{L}_X \Omega = d (i_X \Omega) + i_X \graycancel{d\Omega} = d (i_X \Omega).
\end{equation}
Note that this definition is independent of the notion of metric once we are given a volume form $\Omega$.
We can compute the divergence using the expressions \eqref{eq:interior_derivative_basis_expr}-\eqref{eq:exterior_deriv_basis_expr} for $i_x$ and $d$, but it is much easier to use the Leibniz rule for $\mathcal{L}$.
Writing $\Omega = a(x) dx^1 \wedge \cdots \wedge dx^n$ for some nowhere-vanishing function $a$
\begin{align}
    \mathrm{div}_\Omega (X) 
    &= \big(\mathcal{L}_X a\big)\,\,dx^1 \wedge \cdots \wedge dx^n + a(x) \sum_{\mu=1}^n dx^1 \wedge \cdots \wedge \mathcal{L}_X(dx^\mu) \wedge \cdots \wedge dx^n
    \\&= X(a) \, dx^1 \wedge \cdots \wedge dx^n + a(x) \sum_{\mu=1}^n dx^1 \wedge \cdots \wedge dX^{\mu} \wedge \cdots \wedge dx^n
    \\&= \left( X(a) + a(x) \sum_{\mu=1}^n \partial_{\mu} X^{\mu} \right) dx^1 \wedge \cdots \wedge dx^n 
    \\&= \left( \frac{1}{a(x)} \partial_{\mu} \big(a(x) X^{\mu}\big) \right) \, \Omega.
\end{align}
Note that we recover the usual formula for divergence in $\mathbb{R}^n$ when $a(x) = 1$.
\smallskip

More generally, when $\Omega = \sqrt{|g|} dx^1 \wedge \cdots \wedge dx^n$ is the \textbf{metric volume form}, we have
\begin{equation}
    \mathrm{div}_g (X) = \left( \frac{1}{\sqrt{|g|}} \partial_{\mu} \Big(\sqrt{|g|} X^{\mu}\Big) \right) \Omega_g = \left( \nabla_{\mu} X^{\mu} \right) \Omega_g.
\end{equation}
The last equality follows from the definition $\nabla_{\mu}X^\mu= \partial_\mu X^\mu + \Gamma_{\mu \nu}^\mu X^\nu$ and the well-known identity $\Gamma_{\mu \nu}^\mu = \frac{1}{\sqrt{|g|}} \partial_\nu \sqrt{|g|}$ proven in \eqref{eq:metric_christoffel_contraction_identity}.
\smallskip

\textbf{Gauss theorem} is immediately proved using Cartan formula and Stokes theorem.
Let $M$ be an oriented manifold with boundary $\partial M$ and $X$ a vector field.
Then
\begin{equation}
    \int_M \mathrm{div}_\Omega (X) = \int_M \mathcal{L}_X \Omega = \int_M d (i_X \Omega) + i_X \graycancel{d\Omega} = \int_{\partial M} i_X \Omega.
\end{equation}
In the case $\Omega$ is the volume form induced by a metric $g$, we can rewrite the right-hand side and get
\begin{equation}
    \int_M \mathrm{div}_g (X) = \int_{\partial M} i_X \Omega_g = \int_{\partial M} g(X, N)\,\, \tilde{\Omega}_g,
\end{equation}
where $N$ is the outward-pointing unit normal vector field on $\partial M$ and $\tilde{\Omega}$ is the volume form induced by the metric on $\partial M$.
It is indeed immediate to check that $i_X \Omega_g = g(X,N) \tilde{\Omega}_g$.

In turn, we have the notion of \textbf{integration by parts}.
First note that for any \emph{arbitrary} volume form
\begin{equation}
    df\wedge i_x \Omega = \left(\partial_\mu f \,dx^\mu\right)\wedge \left(X^{\nu_1} \Omega_{\nu_1 \dots \nu_n} dx^{\nu_2} \wedge \cdots \wedge dx^{\nu_n}\right) = \partial_\mu f X^{\mu}\,\, \Omega = df(X) \,\Omega.
\end{equation}
Then we have 
\begin{equation}
    \int_{\partial M} f\,\, i_X \Omega = \int_M d(f\, i_X \Omega) = \int_M df \wedge i_X \Omega + f\, d i_X\Omega = \int_M df(X) \,\Omega + \int_M f\, \mathrm{div}_\Omega (X).
\end{equation}
In case $\Omega = \sqrt{|g|} dx^1 \wedge \cdots \wedge dx^n$ is the metric volume form this reads
\begin{align}
\begin{aligned}
    \int_{M} f\, \underbrace{\nabla_\mu X^{\mu} \, \Omega_g}_{\mathrm{div}_g(X)}
    &= \int_{\partial M} f\,\, i_X\Omega_g\, - \int_M \underbrace{df(X)}_{= X^{\mu} \partial_\mu f} \, \Omega_g
    = \int_{\partial M} f\,\, g(X,N)\, \tilde{\Omega}_g - \int_M \underbrace{X^{\mu} \partial_\mu f}_{=g(X,\nabla f)}\, \Omega_g,
\end{aligned}
\end{align}
where $\tilde{\Omega}_g$ is the volume form induced by the metric on $\partial M$ and $N$ is the outward-pointing unit normal vector field on $\partial M$.
\end{mytheorem}


\begin{mytheorem}[Curl \& 3D Stokes theorem.]
Given a \emph{Riemannian} manifold $(M,g)$ of dimension $3$ and a vector field $X$, define the 1-from $g(X,\cdot)$, that is lower an index $X_\mu=g_{\mu \nu} X^{\nu}$.
Define the curl as the 2-from obained by taking the exterior derivative
\begin{equation}
    \mathrm{curl} (X) := d (g(X,\cdot)) = d \Big(X_{\mu} \wedge dx^{\mu}\Big).
\end{equation}
Note that in $\mathbb{R}^3$ with the standard Euclidean metric, this definition coincides with the usual notion of curl.
\smallskip

The general Stokes theorem immediately gives the familiar version in $\mathbb{R}^3$:
\begin{equation}
    \int_S \mathrm{curl} (X) =\int_S d (g(X,\cdot)) = \int_{\partial S} g(X,\cdot),
\end{equation}
for any oriented surface $S$ with boundary $\partial S$.
\end{mytheorem}


\begin{mytheorem}[Complex integration \& Gauss-Green Theorem.]
Consider the complex plane as a Riemannian manifold $(\mathbb{R}^2, g)$ with the standard Euclidean metric.
Given a complex homolorphic function $f: \mathbb{C} \to \mathbb{C}$, both the real and imaginary parts \todotag{Finish this}
\end{mytheorem}



%========================================================
\subsection{Hodge dual \& friends} 
%=========================================================

\begin{mytheorem}[Hodge dual]
%
Given a pseudoriemannian manifold \((M^{(n)}, g)\), write the metric volume form as
\begin{equation}
    \Omega = \sqrt{|g|}dx^1 \wedge \cdots \wedge dx^n 
    = \sqrt{|g|} \epsilon_{\alpha_1 \dots \alpha_n} dx^{\alpha_1} \otimes \cdots \otimes dx^{\alpha_n}
    = \frac{1}{n!} \sqrt{|g|} \epsilon_{\alpha_1 \dots \alpha_n} dx^{\alpha_1} \wedge \cdots \wedge dx^{\alpha_n},
\end{equation}
where the first expression is \emph{not} summed over indices, while the latter two are, and $\epsilon_{\alpha_1 \dots \alpha_n}$ is the Levi-Civita totally antisymmetric \emph{symbol} $\epsilon_{1 2 \dots n} = +1$, not the tensor.

The Hodge dual is the map
\begin{align}
    \star: \Lambda^\bullet (M) &\to \Lambda^{n-\bullet} (M) \quad \mid \quad \theta \mapsto \Omega_g(\theta^{\#}, \cdots),
\end{align}
where \(\theta^{\#}\) is the tensor of rank $(\bullet,0)$ obtained by raising all indices of $\theta$ with the metric.
That is, transform a $k$-form into a $(k,0)$-tensor and then feed it to the first $k$ slots of the volume form to get a $(n-k)$ form.
%
Explicitly, with the index convention \eqref{eq:indices_bookkeeping_differential_forms} for {\small $j_1<\cdots<j_k$} ordered and {\small $\mu_1\cdots\mu_k$} arbitrary, the form $\theta$ is
{\small
\begin{align}
    \theta &= \theta_{j_1 \dots j_k} dx^{j_1} \wedge \cdots \wedge dx^{j_k} = \theta_{\mu_1 \dots \mu_k} dx^{\mu_1} \otimes \cdots \otimes dx^{\mu_k}\quad \Rightarrow \quad \theta^{\#} = g^{\mu_1 \alpha_1} \cdots g^{\mu_k \alpha_k} \theta_{\mu_1 \dots \mu_k} \partial_{\alpha_1}\otimes \cdots \otimes \partial_{\alpha_k}.
\end{align}
}
The Hodge dual of the \(k\)-form \(\theta\) is then 
\begin{align}
    \star \theta &:= \Omega_g(\theta^{\#}, \cdots)
    = \sqrt{|g|}\epsilon_{\beta_1\dots \beta_n} dx^{\beta_1} \otimes \cdots \otimes dx^{\beta_n} \left( g^{\mu_1 \alpha_1} \cdots g^{\mu_k \alpha_k} \theta_{\mu_1 \dots \mu_k} \partial_{\alpha_1}\otimes \partial_{\alpha_k}, \cdots \right)
    \\
    &= \sqrt{|g|}\epsilon_{\alpha_1\dots \alpha_k\beta_{k+1}\dots \beta_n} g^{\mu_1 \alpha_1} \cdots g^{\mu_k \alpha_k} \theta_{\mu_1 \dots \mu_k} dx^{\beta_{k+1}} \otimes \cdots \otimes dx^{\beta_n}
    \\
    &= \frac{1}{(n-k)!} \sqrt{|g|}\epsilon_{\alpha_1\dots \alpha_k\beta_{k+1}\dots \beta_n} \theta^{\alpha_1 \dots \alpha_k} dx^{\beta_{k+1}} \wedge \cdots \wedge dx^{\beta_n}.
\end{align}

\noindent
\todotag{Check signs etc}
The Hodge dual obeys the following important properties.
\begin{itemize}
    %
    \item \textbf{Idempotent up to a sign.} For a metric {\small \(g=(+1\dots+1,-1\dots-1)\)} of signature {\small \((n-s,s)\)}, and any form \(\alpha\) we have
    \begin{align}
        \star \star \alpha = (-1)^{|\alpha|(n-|\alpha|) + s} \alpha
    \end{align}
    %
    \item \textbf{Symmetry wrt $\wedge$ product.} For any two \(k\)-forms \(\alpha, \beta\) of the same degree
    \begin{equation}
        \alpha \wedge \star \beta = \beta \wedge \star \alpha = \langle \alpha, \beta \rangle_g \, \cdot \Omega_g,\quad \text{where}\quad \langle \alpha, \beta \rangle_g = g^{\mu_1 \nu_1} \cdots g^{\mu_k \nu_k} \alpha_{\mu_1 \dots \mu_k} \beta_{\nu_1 \dots \nu_k}
    \end{equation}
    is the pointwise scalar product induced by the metric \(g\) on $(0,k)$-tensors, times the volume form.
    % 
    \item \textbf{Star of a wedge (degree additivity).}
    For any two form $\alpha\in\Omega^p$ and $\beta\in\Omega^q$
    \begin{align}
        \star(\alpha\wedge\beta)
        =\iota_{\beta^\sharp}(\star\alpha)
        =(-1)^{pq}\,\iota_{\alpha^\sharp}(\star\beta).
    \end{align}
    The inner derivative $i_{\beta^{\#}}$ of a $j$-form with respect to the  $(\ell,0)$-tensor $\beta^\sharp$ means contracting the first $\ell$ slots of the form with the tensor, and gives zero if $\ell>j$. That is when $p+q>n$.
    %
\end{itemize}

\end{mytheorem}


\begin{mytheorem}[The Hodge scalar product on differential forms.\todotag{Finish this}]
Given two \(k\)-forms \(\alpha, \beta\) on a pseudoriemannian manifold \((M,g)\), the Hodge scalar product is
\begin{equation}
    \langle \alpha, \beta \rangle_{\,\mathrm{Hodge}} := \int_M \alpha \wedge \star \beta
    = \int_M \beta \wedge \star \alpha 
    = \int_M \underbrace{\alpha_{\mu_1 \dots \mu_k}\, \beta_{\nu_1 \dots \nu_k} \,g^{\mu_1\nu_1} \dots g^{\mu_k \nu_k}}_{=\langle \alpha, \beta \rangle_g} \,\,\, \Omega_g\,.
\end{equation}
It is indeed \emph{real} and \emph{symmetric} thanks to the Hodge dual properties, and coincides with the metric contraction of the two forms integrated over the manifold via the volume form.
In particular, when \(\alpha = \beta\)
\begin{equation}
    \langle \alpha, \alpha \rangle_{\mathrm{Hodge}} = \int_M \alpha \wedge \star \alpha = \int_M ||\alpha||^2 \,\Omega,
\end{equation}
where \(||\alpha||^2\) is the pointwise norm of \(\alpha\) induced by the metric \(g\).
Note that the signature of this scalar product on $\Lambda^k(M)$ depends on both the degree \(k\) of the forms and the signature of the metric \(g\).
For example, on the subspace of 1-froms $\Lambda^1 (M)$, the Hodge product has the same signature $(n-s,s)$ as the metric \(g\) itself.
\end{mytheorem}


\begin{mytheorem}[The codifferential. \todotag{Finish}]
%
The codifferential \(\delta\) is defined as the adjoint of the exterior derivative \(d\) with respect to the Hodge scalar product\footnote{Beware that $\delta$ is sometimes defined as \emph{minus} the adjoint of $d$ to make the D'Alambertian $\Box =\delta d+d\delta$ positive definite rather than negative in the Riemannian case. It is just a convention.}
\begin{align}
    \langle \delta \alpha, \beta \rangle_{\mathrm{Hodge}} &:= \langle \alpha, d \beta \rangle_{\mathrm{Hodge}} \quad \Rightarrow \quad \delta := (-1)^{\bullet} \star^{-1}d\,\star = (-1)^{n(\bullet-1)+s+1} \star d \,\star\,\,.
\end{align}
Indeed for any  $k$-form \(\alpha\) and $k\!-\!1$ form \(\beta\)
\begin{align}
    \langle \alpha, d\beta \rangle_{\mathrm{H}} 
    &= \langle d \beta, \alpha \rangle_{\mathrm{H}}
    = \int_M d \beta \wedge \star \alpha
    = \graycancel{\int_M d(\beta \wedge \star \alpha)} - \int_M (-1)^{|\beta|} \beta \wedge d\star \alpha
    \\
    &=
    \underbrace{(-1)^{|\beta|+1+(|\alpha|-1)(n-|\alpha|+1)+s}}_{=(-1)^{n(|\alpha|-1)+s}} \int_M  \beta \wedge (\star\star d\star \alpha)
    \equiv
    (-1)^{n(|\alpha|-1)+s+1} \big\langle\, \star d\,\star \alpha\,, \beta \big\rangle_{\mathrm{H}}.  
\end{align}
In a chart we have\todotag{Check formula}
\begin{align}
    (\delta \alpha)_{\mu_1 \dots \mu_{k-1}} 
    &= - \nabla^{\nu} \alpha_{\nu \mu_1 \dots \mu_{k-1}} 
    = - \frac{1}{\sqrt{|g|}} \partial_{\nu} \Big(\sqrt{|g|} g^{\nu \rho} \alpha_{\rho \mu_1 \dots \mu_{k-1}}\Big).
\end{align}
The codifferential is a linear map of degree \((0,-1)\)
\begin{equation}
    \delta: \Lambda^\bullet (M) \to \Lambda^{\bullet-1} (M),
\end{equation}
but it fails the Leibniz rule i.e. it does not respect the wedge product, so it is not an (anti)-derivative.


\smallskip

\noindent
It obeys the fundamental properties.
\begin{itemize}
    \item \textbf{Nilpotency} \(\delta^2 = 0\), from nilpotency of $d$ and indempotency of $\star$ $\Rightarrow$ homology!
    \item \textbf{Poincaré lemma.} Co-closed implies locally co-exact: \(\delta \alpha = 0 \Rightarrow \alpha = \delta \beta\).
    \item \textbf{Adjointness} \(\langle \delta \alpha, \beta \rangle_{\mathrm{Hodge}} = \langle \alpha, d \beta \rangle_{\mathrm{Hodge}}\).
    \item \textbf{Integration by parts} $\displaystyle \int_M d\alpha \wedge \star \beta = \int_M \alpha \wedge \star \delta \beta \quad $ for any \(k\) and $k\!-\!1$ form \(\alpha,\beta\).

    \item \textbf{Relations with star} {$\quad \displaystyle\star \delta = (-1)^{\bullet} d \star \quad \text{and} \quad \delta \star = (-1)^{\bullet +1} \star d.$}
\end{itemize}

\end{mytheorem}


\begin{mytheorem}[Example: Maxwell equations in terms of differential forms \& Hodge dual]
In any dimension $n$, define the electromagnetic field strength 2-form \(F\) as
\begin{equation}
    F := \frac{1}{2} F_{\mu \nu} dx^{\mu} \wedge dx^{\nu}.
\end{equation}
In $4$-dimension the electric and magnetic fields are encoded in the components of $F$ as
\begin{equation}
    F_{\mu \nu} = \qquad F_{\mu\nu} = \begin{pmatrix}
    0 & -E_1 & -E_2 & -E_3 \\
    E_1 & 0 & B_3 & -B_2 \\
    E_2 & -B_3 & 0 & B_1 \\
    E_3 & B_2 & -B_1 & 0
    \end{pmatrix}.
\end{equation}
%
Given a current $n$-vector $J^\mu$, define the current $1$-form lowering the indices with the metric
\begin{equation}
    J := J_{\mu} dx^{\mu}.
\end{equation}
One also defines the current $(n-1)$--form as its Hodge dual
{\small
\begin{align}
    \star J &:= \Omega_g(J^\mu \partial_\mu,\cdots)= \sqrt{|g|}dx^0\wedge \cdots \wedge dx^{n-1} (J^\mu \partial_\mu,\cdots) 
    = \sqrt{|g|}\,\epsilon_{\alpha_0\dots \alpha_{n-1}} dx^{\alpha_0} \otimes \cdots \otimes dx^{\alpha_{n-1}} (J^\mu \partial_\mu,\cdots)
    \\
    &=  \sqrt{|g|}\,\epsilon_{\alpha_0\dots \alpha_{n-1}} J^{\alpha_0}\, dx^{\alpha_1} \otimes \cdots \otimes dx^{\alpha_{n-1}} 
    = \sqrt{|g|}\,\frac{1}{(n-1)!}\epsilon_{\alpha_0\dots \alpha_{n-1}} J^{\alpha_0}\, dx^{\alpha_1} \wedge \cdots \wedge dx^{\alpha_{n-1}}.
\end{align}
}
%
The Maxwell equations read
\begin{equation}
    dF = 0, \qquad \delta F = J \quad \text{or equivalently} \quad d \star F = \star J.
\end{equation}
The first equation encodes the homogeneous Maxwell equations (Faraday law \& no magnetic monopoles), while the second encodes the inhomogeneous ones (Gauss law \& Ampere-Maxwell law).
The first equation \(dF=0\) implies the 2-form is closed and thus locally exact, i.e. there exists a 1-form $A$ identified with the electromagnetic potential such that
\begin{equation}
    F = dA,\qquad A = A_{\mu} dx^{\mu}.
\end{equation}
Moreover, it automatically implies invariance under gauge transformations $A_\mu\mapsto A_\mu + \partial_\mu a$.
\end{mytheorem}




\begin{mytheorem}[Laplace-de Rham-Hodge-D'Alambert operator\todotag{Finish}]
%
The Hodge--de Rham operator \(\Box\) is defined on all, possibly fiber valued, differential forms as
\begin{equation}
    \Box: \Lambda^\bullet(M) \otimes E_F \to \Lambda^\bullet(M) \otimes E_F, \qquad \Box := d \delta + \delta d \equiv (d + \delta)^2.
\end{equation}
It preserves the degree of forms, but it does \emph{not} satisfy the Leibniz rule in any easy sense.
\smallskip

It satisfies the following properties.
\begin{itemize}
    \item \textbf{Commutations.} $d \,\Box = \Box\, d,\quad \delta\, \Box = \Box\, \delta,\quad \star\, \Box = \Box\, \star$.
    \item \textbf{Self-adjoint wrt Hodge product.} \(\langle \Box \alpha, \beta \rangle_{\mathrm{Hodge}} = \langle \alpha, \Box \beta \rangle_{\mathrm{Hodge}}\).
    In the Riemannian case the Hodge product is positive definite, and $\Box$ is thus diagonalizable by the spectral theorem.
    It has nonpositive spectrum $\text{Spec}(\Box)\subseteq (-\infty,0]$ (cf. below).
\end{itemize}
\smallskip

Depending on the signature of the metric \(g\), it generalizes either the Laplacian (Riemannian case), the D'Alambertian (Lorentzian case) or anything in between (general pseudoriemannian case). 
In the Riemannian case it is an elliptic operator (positive semidefinite), while in the Lorentzian case it is hyperbolic.
This is easily seen in flat Euclidean or Minkowski space, passing to Fourier space
\begin{equation}
    p_E^2 = p_1^2 + p_2^2 + p_3^2 + \cdots \geq 0,\quad p_M^2 = p_0^2 - p_1^2 - p_2^2 - \cdots.
\end{equation}
\smallskip

In\todotag{Check formula} a chart, it acts on a $k$-form $\alpha$ as
\begin{align}
    (\Box \alpha)_{\mu_1 \dots \mu_k} 
    =& - \nabla^{\nu} \nabla_{\nu} \alpha_{\mu_1 \dots \mu_k} + \sum_{i=1}^k (-1)^i R^{\rho}_{\ \mu_i} \alpha_{\mu_1 \dots \mu_{i-1} \rho \mu_{i+1} \dots \mu_k} 
    \\
    &  - 2 \sum_{1\leq i < j \leq k} (-1)^{i+j} R^{\rho}_{\,\mu_i \mu_j}{}^{\sigma} \alpha_{\mu_1 \dots \mu_{i-1} \rho \mu_{i+1} \dots \mu_{j-1} \sigma \mu_{j+1} \dots \mu_k},
\end{align}

In particular, on 0-forms i.e. functions it reduces to the usual Laplace-D'Alambert operator
\begin{equation}
    \Box f =  \nabla^{\nu} \nabla_{\nu} f.
\end{equation}
In fact, the Hodge operator could also  be defined as the \textbf{unique operator of degree $0$ on differential forms that reduces to the usual Laplacian/D'Alambert operator on functions and commutes with both \(d\) and \(\delta\).}
\medskip

The Hodge operator $\Box$ can also be understood as the \textbf{square of the Dirac operator} acting on differential forms, defined as
\begin{equation}
    \slashed{D} := d + \delta.
\end{equation}
Indeed, from nilpotency of $\delta$ and $d$ we have $\slashed{D}^2 = (d + \delta)^2 = d^2 + d \delta + \delta d + \delta^2 = d \delta + \delta d = \Box$.

Finally, its action on 0-forms can be understood as the operator yielding the Euler-Lagrange equations for the action of a free scalar field \(\phi\).
Indeed we have 
\begin{equation}
    S[\phi] =  \int_M\!\! \mathrm{d}^n x\, \sqrt{|g|}\,\tfrac{1}{2} \partial_\mu \phi \partial^\mu \phi\
    = \int_M \tfrac{1}{2}  ||d\phi||^2 \Omega_g = \int_M \tfrac{1}{2} d\phi \wedge \star d\phi 
    = \tfrac{1}{2}\langle d\phi, d\phi \rangle_{\mathrm{Hodge}}.
\end{equation}
Varying with respect to \(\phi\mapsto \phi +\varepsilon\), one finds
\begin{equation}
    \delta S[\phi] = \int_M d \varepsilon \wedge \star d\phi = \int_M \graycancel{d\left(\varepsilon \star d\phi\right)} - \varepsilon\,\, d \star d\phi \quad \Rightarrow \quad d \star d\,\,\phi  = 0.
\end{equation}
Equivalently, more abstractly,
\begin{equation}
    \delta S[\phi] = \delta \left(\tfrac{1}{2}\langle d\phi, d\phi \rangle_{\mathrm{Hodge}}\right)
    = \langle d \phi, d \varepsilon \rangle_{\mathrm{Hodge}}
    = \langle \delta d \phi, \varepsilon \rangle_{\mathrm{Hodge}} \quad \Rightarrow \quad \delta\, d\, \phi = 0.
\end{equation}
Since $\delta \simeq \star d\,\star$ and $\delta \phi=0$ for 0-forms, both expressions ultimately gives equation of motion  $\Box\, \phi = 0$.
%
\end{mytheorem}



%========================================================
\subsection{Spectral geometry} 
%=========================================================

\begin{mytheorem}[Introduction: can you hear the shape of a drum?]
%
Can we detect and describe (pseudo-)Riemannian structures \(\mathcal{E}\) directly, without reference to a particular metric \(g\)?
Possibly yes, using \textbf{spectral geometry}.
The idea is that the vibration spectrum \(\{\lambda_n\}\) of a manifold (the eigenvalues of a Laplace-type operator) depends only on \(\mathcal{E}\) and not on the choice of coordinates.
The classical question of spectral geometry, posed by Hermann Weyl in 1911, is:
%
\begin{center}
\emph{Does the spectrum \(\{\lambda_n\}\) encode all information about the shape, i.e. the structure \(\mathcal{E}\)?}
\end{center}
%
The expectation is that the spectra \(\mathrm{spec}(\Delta_p)\) of the laplacian on fiber valued p-forms for different $p$ and fibers contain different information about $M$, possibly enough to reconstruct its shape completely.
%
\end{mytheorem}


\vspace{-0.3mm}
\begin{wrapfigure}[9]{r}{0.34\textwidth}
  \centering
  \vspace{-\intextsep}
  \includegraphics[width=\linewidth]{images/seismic_wave.png}
  %\caption{Shear vs. pressure seismic wave}
  \label{fig:seismic_waves}
\end{wrapfigure}

\noindent
For example, it is known that scalar (a.k.a. pressure, longitudinal) and vector (ak.a. shear, transverse) seismic waves travel and reflect off materials differently, e.g. shear waves propagate slower and do not travel through liquids since liquids do not support shear stresses by definition. Their spectra thus contain different information about the Earth's interior structure. This is indeed how seismologists study the Earth's structure and discovered, for example, that its core is liquid in the outer part and solid in the inner part.
\bigskip


\begin{mytheorem}[Types of waves on \(M\)]
%
On a given manifold we can consider fiber valued p-forms \(w(t)\) with time evolution governed by various differential equations.
%
\begin{enumerate}
\item \textbf{Schrödinger equation }
\(
 i\hbar\,\partial_t w(x,t)
 \;=\; -\frac{\hbar^2}{2m}\,\Delta_p\,w(x,t).
\)
\item \textbf{Heat equation }
\(
 \partial_t w(x,t)
 \;=\; \,\Delta_p\,w(x,t).
\)
\item \textbf{Klein-Gordon/wave equation}
\(
 \,\partial_t^2 w(x,t)
 \;=\; \beta^2\,\Delta_p\,w(x,t),\,
\)
where \(\beta\) is the propagation speed of the given type of wave, e.g. the speed of sound $\beta=c_s$ for pressure waves or the speed of shear waves $ \beta=c_{sh}$.
%
\end{enumerate}
%
Suppose we find an eigenform \(\tilde{\omega}(x)\) of \(\Delta_p\) with eigenvalue \(\lambda\):
\[
 \Delta_p\,\tilde{\omega}(x) \;=\; \lambda\,\tilde{\omega}(x).
\]
Since \(\Delta_p\) is self-adjoint wrt the Hodge inner product \((w,v)=\int_M w\wedge *v\), such eigenforms exist and form an orthonormal basis if $M$ is Riemannian, compact and without boundary (for simplicity).
%
Each evolution equations above is then solved by separation of variables as 
\begin{align*}
 \text{Schrödinger:}&\quad
 w(x,t) \;=\; e^{\tfrac{i\hbar\lambda}{2m}\,t}\,\tilde{\omega}(x),\\[0.3em]
 \text{Heat:}&\quad
 w(x,t) \;=\; e^{-\,2\lambda\,t}\,\tilde{\omega}(x),\\[0.3em]
 \text{Klein–Gordon:}&\quad
 w(x,t) \;=\; e^{i\beta\lambda\,t}\,\tilde{\omega}(x).
\end{align*}
%
It follows that the spectrum \(\mathrm{spec}(\Delta_p)\) is the \emph{overtone spectrum} of p-form type waves on the manifold \(M\).
%
\end{mytheorem}


\begin{mytheorem}[Practical mathematical aspects of spectral geometry]
%
Several practical aspects must be considered regarding the rigorous mathematical formulation of spectral geometry.
%
\begin{itemize}
\item \textbf{(Non)-compact riemannian manifolds} (theorem).
If \((M,g)\) is a compact Riemannian manifold without boundary (it has then finite volume), for each degree $p$ the spectrum \(\mathrm{spec}(\Delta_p)\) of the Laplace-de Rham operator acting on $T^\star_\bullet M$-valued $p$-forms is discrete, with finite degeneracies and without accumulation points.

On the other hand, if the manifold \(M\) has infinite volume (non-compact) then the spectrum of \(\Delta\) must develop a continuous part\footnote{Besides a possible still discrete point-spectrum.}, and we cannot hope to recover information from it since it is the same for all such manifolds, i.e. `all infinite volume drums sound the same'.

\textbf{In practice} one models an arbitrarily large part of the universe by a compact Riemannian manifold \((M,g)\).  This allows us to describe, for example, 3-dimensional space at a fixed time.
%
\item \textbf{Pseudo-Riemannian manifolds}(open problem).
The spectral geometry of \emph{pseudo}-Riemannian manifolds is still very little developed.
The main difficulty is that the Hodge inner product has indefinite signature, so that Laplace-type operators on such manifolds are typically not elliptic and the spectral theorem does not apply.

\textbf{In practice} one typically models spacetime as a compact Riemannian manifold \((M,g)\) after Wick rotation and then tries to recover the Lorentzian structure by analytic continuation.
%
\item \textbf{Boundaries.} The spectral geometry of manifolds is indeed studied.
One needs to impose suitable boundary conditions on the eigenforms of \(\Delta_p\) at \(\partial M\), for example Dirichlet or Neumann conditions.

\textbf{In practice} one typically models space(time) as a compact Riemannian manifold without boundary, or with boundary conditions that effectively make the boundary invisible to the waves considered like periodic BCs.
%
\end{itemize}

\end{mytheorem}

\begin{mytheorem}[Can you hear the shape of a drum? In general no, but often yes.]
%
In general, even in the simplest case of compact Riemannian manifolds without boundary, the spectra of $\Delta$ on \textbf{standard} $\R$\textbf{valued} $p$-forms do \textbf{not contain enough information to uniquely identify the geometric structure}.
Indeed we shall see below that there is relatively little independent information contained in the spectra of $p$-forms because of the many symmetries of $\Delta$ relating the spectra for different $p$.
%
There are indeed examples of pairs \((M,g)\) and \((M',g')\) isospectral for \(\Delta\) on all degrees \(p\), but not isometric.
Nevertheless, all known \textbf{counterexamples have limited physical significance}.
They involve manifolds that are locally, but not globally, isometric (e.g.  one is a $k$-fold covering of the other); manifolds that are $\Delta_p$ isospectral only for some $p$; or manifolds occurring in discrete copies (e.g. mirror images).
%

The situation is different if one considers the spectra of Laplacians acting on more general tensor fields.
For example, rank 2 \textbf{tensor fields} (in particular, metric perturbations) \textbf{contain new information} about the Riemannian structure wrt the sole $p$-forms.
Indeed, a rank-2 tensor can be decomposed into scalar, vector and tensor parts, each with its own spectrum.
While the spectra of scalar and vector parts are related to those of 0- and 1-forms, the tensor part contains genuinely new information not present in the spectra of any $\R$-valued $p$-form.
%
\end{mytheorem}



\begin{mytheorem}[Hodge decomposition: exact, closed and harmonic forms \& spectra of $\Delta$]
%
For simplicity we consider a compact Riemannian manifold $(M,g)$ without boundary or with suitable boundary conditions like DBc or PBc.
\emph{Partial} extensions to pseudo-riemannian manifolds or noncompact ones are possible.

The spaces of differential p-forms admit the \textbf{Hodge decomposition} into subspaces orthogonal wrt the Hodge inner product
\begin{align}
    \Lambda_p \;=\;d\Lambda_{p-1}\;\oplus\;\delta\Lambda_{p+1}\;\oplus\;\Lambda_p^0.
\end{align}
The decomposition holds for arbitrary fiber-valued form since the operators \(d\), \(\delta\) and \(\Delta\) act only on the base manifold, so that 
 \begin{equation}
    \Lambda_p \otimes E_F = \left(d\Lambda_{p-1}\;\oplus\;\delta\Lambda_{p+1}\;\oplus\;\Lambda_p^0\right)\otimes E_F = \left(d\Lambda_{p-1}\otimes E_F\right)\;\oplus\;\left(\delta\Lambda_{p+1}\otimes E_F\right)\;\oplus\;\left(\Lambda_p^0\otimes E_F\right).
\end{equation}
Each subspace is invariant under the action of the Laplace-de Rham operator \(\Delta_p\) because $\Lambda_p^0:=\ker(\Delta_p)$ and the commutation relations \([\Delta,d]=0\) and \([\Delta,\delta]=0\) and $\Lambda_p^0:=\ker(\Delta_p)$.
In particular $\Delta_p$ is invertible on the subspaces of exact and co-exact forms.
Any eigenform of $\Delta$ lies entirely in one of these subspaces and its spectrum decomposes into
\begin{equation}
    \mathrm{spec}(\Delta_p) \;=\; \Big(\,\mathrm{spec}(\Delta_p|_{d\Lambda_{p-1}}) \;\cup\; \mathrm{spec}(\Delta_p|_{\delta\Lambda_{p+1}})\,\Big) \sqcup\; \underbrace{\mathrm{spec}(\Delta_p|_{\Lambda_p^0})}_{\equiv\{0\}}.
\end{equation}
The decomposition is \textbf{irredubile} i.e. there are no further invariant subspaces of \(\Delta_p\).
\end{mytheorem}
\begin{proof}
\todotag{Insert}
\end{proof}

\begin{mytheorem}[Hodge decomposition \& de Rham cohomology]
Each deRham cohomology class contains exactly one harmonic representative, because of the following.
\begin{itemize}
    \item Each harmonic form is closed and co-closed, but neither exact nor co-exact
    \begin{align}
        \Delta \omega = 0 \quad \Longrightarrow \quad d\omega = 0, \quad \delta \omega = 0.
    \end{align}
    \item Every exact (resp. co-exact) form is \emph{not} co-closed (resp. closed)
    \begin{align}
        \alpha = d\beta\,\Rightarrow\, \delta \alpha \neq 0, \qquad \eta = \delta \theta\,\Rightarrow\, d\eta \neq 0.
    \end{align}
\end{itemize} 
The vector space of harmonic p-forms is thus linearly isomorphic to the p-th deRham cohomology group of the manifold
\begin{align}
    \Lambda_p^0 \cong H^p_{dR}(M) := \frac{\ker(d:\Lambda_p\to\Lambda_{p+1})}{\mathrm{im}(d:\Lambda_{p-1}\to\Lambda_p)}\,,\qquad B_p=\dim(H^p_{dR}(M)) = \dim(\Lambda_p^0).
\end{align}
The space $\Lambda_p^0$ is thus a topological invariant of the manifold, independent of the metric \(g\).
This is non-trivial: even though $d$ depends only on the differentiable structure of $M$, the codifferential $\delta$ and hence the Laplacian $\Delta$ does depend on the metric $g$.
\end{mytheorem}
\begin{proof}
\todotag{Insert}
\end{proof}

\begin{mytheorem}[Symmetries of $\Delta$ spectra]
%
The above `(co-)exatcness vs (co-)closedness' properties and the commutations relations $[\Delta,*]=[\Delta,d]=[\Delta,\delta]=0$ of the Laplace operator imply the following isomorphisms and symmetries of $\Delta$ spectra.
\begin{itemize}
    \item \textbf{Hodge duality.} The Hodge map
    \begin{align}
        \star:\underbrace{d\Lambda_{p-1}\oplus \delta\Lambda_{p+1} \oplus \Lambda_p^0}_{\Lambda_p} \longrightarrow \underbrace{d\Lambda_{n-p-1}\oplus \delta\Lambda_{n-p+1} \oplus \Lambda_{n-p}^0}_{\Lambda_{n-p}}
    \end{align}
    is an isomorphism interchaning the decomposition and preserving the spectra of $\Delta$ as 
    \begin{align}
        &\star: d\Lambda_{p-1} \;\longrightarrow\; \delta\Lambda_{n-p+1},\quad \mathrm{spec}(\Delta_{p}|_{d\Lambda_{p-1}}) = \mathrm{spec}(\Delta_{n-p}|_{\delta\Lambda_{n-p+1}}),\\[6pt]
        &\star: \delta\Lambda_{p+1} \;\longrightarrow\; d\Lambda_{n-p-1},\quad \mathrm{spec}(\Delta_{p}|_{\delta\Lambda_{p+1}}) = \mathrm{spec}(\Delta_{n-p}|_{d\Lambda_{n-p-1}}),\\[6pt]
        &\star: \Lambda_p^0 \;\longrightarrow\; \Lambda_{n-p}^0, \qquad \qquad\mathrm{spec}(\Delta_p|_{\Lambda_p^0}) = \mathrm{spec}(\Delta_{n-p}|_{\Lambda_{n-p}^0})\equiv \{0\}.
    \end{align}
    %
    \item \textbf{(Co-)Exact duality.} The following maps are isomorphisms, respect spectra of $\Delta$ 
    \begin{align}
        &d: \delta\Lambda_{p+1} \;\longrightarrow\; d\Lambda_{p},\quad
        \delta: d\Lambda_{p} \;\longrightarrow\; \delta\Lambda_{p+1}, \quad \mathrm{spec}(\Delta_{p+1}|_{d\Lambda_{p}}) = \mathrm{spec}(\Delta_p|_{\delta\Lambda_{p+1}})\\[7pt]
        &\delta\circ d_{|\delta\Lambda_{p+1}} = \Delta_{p|\delta\Lambda_{p+1}},\qquad d\circ \delta_{|d\Lambda_p} = \Delta_{p+1|d\Lambda_p},
    \end{align}
    and the compositions are the restrictions of $\Delta$ to the respective subspaces. 
\end{itemize}
%
This is suggestively shown in the following diagram, where equal colors indicate isomorphic subspaces with identical spectra of $\Delta$,

\begin{samepage}
\begin{align}
  \begin{array}{l}
    {\vdots}\\[0pt]
    \Lambda_{p-1} = {\color{lightblue}d\Lambda_{p-2}}\;\oplus\;{\tikzmarknode{delp}{\color{blue}\delta\Lambda_p}}\;\oplus\;{\color{forestgreen}\Lambda_{p-1}^0}\\[12pt]
    \Lambda_p     = {\tikzmarknode{dp}{\color{blue}d\Lambda_{p-1}}}\;\oplus\;{\tikzmarknode{delp1}{\color{violet}\delta\Lambda_{p+1}}}\;\oplus\;{\tikzmarknode{hp}{\color{yellow}\Lambda_p^0}}\\[8pt]
    \Lambda_{p+1} = {\color{violet}d\Lambda_p}\;\oplus\;{\color{red}\delta\Lambda_{p+2}}\;\oplus\;\Lambda_{p+1}^0\\[0pt]
    {\vdots} \\[0pt]
    \Lambda_{n-p} = {\tikzmarknode{dnp}{\color{violet}d\Lambda_{n-p-1}}}\;\oplus\;{\tikzmarknode{delnp}{\color{blue}\delta\Lambda_{n-p+1}}}\;\oplus\;{\tikzmarknode{hnp}{\color{yellow}\Lambda_{n-p}^0}}\\[6pt]
    \Lambda_{n-p+1} = {\color{blue}d\Lambda_{n-p}}\;\oplus\;{\color{lightblue}\delta\Lambda_{n-p+2}}\;\oplus\;{\color{forestgreen}\Lambda_{n-p+1}^0}\\[0pt]
    {\vdots}
  \end{array}.
\end{align}
%
\begin{tikzpicture}[remember picture,overlay,>=Stealth,shorten >=2pt,shorten <=2pt]
  %\draw[blue,<->] (delp.south) to[bend left=15] node[left=2pt] {\scriptsize $d\!-\!\delta$} (dp.north);
  %\draw[blue,<->] (delp.south) to[bend left=13]node[left=2pt,yshift=4pt] {\scriptsize $d$ - $\delta$}(dp.north);
  \draw[blue,<->] (delp.south) to[bend left=4] node[pos=0.0,left=2pt,yshift=-8pt] {\scriptsize $\delta$} node[pos=0.7,left=2pt,yshift=3pt] {\scriptsize $d$}(dp.north);
  %\draw[yellow,<->]      (hp.south east) -- node[right=2pt] {$\star$} (hnp.north east);
  %\draw[yellow,<->] (hp.south east) to[bend left=15] node[right=2pt] {$\star$} (hnp.north east);
  \draw[yellow,<->] (hp.south east) to[bend left=15] node[right=2pt] {$\star$} ($(hnp.north east)+(-1.6em,0)$);
  %\draw[violet,<->]      (delp1.south east) to[bend left=15] node[right=2pt] {$\star$} (dnp.north east);
  %\draw[blue,<->]        (dp.south east) to[bend left=15] node[right=2pt] {$\star$} (delnp.north east);
  \draw[violet,<->] (delp1.south west) to[bend right=22] node[near end,right=-1pt,yshift=-2pt] {$\star$} ($(dnp.north east)+(-2.5em,0)$);
\draw[blue,<->]   ($(dp.south east)+(-1.9em,0)$)    to[bend left=19] node[near end,right=2pt,yshift=-3pt] {$\star$} ($(delnp.north west)+(0,0)$);
\end{tikzpicture}
\end{samepage}

\noindent
This pattern reveals there is relatively little independent information contained in the spectra of $p$-forms because of the many symmetries of $\Delta$.
%
For example, when \(\dim(M)=3\) 
\begin{align}
\begin{aligned}
  \Lambda_0 \;=\; 0\;\;&\oplus\;{\color{blue}\delta\Lambda_1}\;\oplus\;{\color{forestgreen}\Lambda_0^0}\\
  \Lambda_1 \;=\; {\color{blue}d\Lambda_0}\;&\oplus\;{\color{red}\delta\Lambda_2}\;\oplus\;{\color{green}\Lambda_1^0}\\
  \Lambda_2 \;=\; {\color{red}d\Lambda_1}\;&\oplus\;{\color{blue}\delta\Lambda_3}\;\oplus\;{\color{green}\Lambda_2^0}\\
  \Lambda_3 \;=\; {\color{blue}d\Lambda_2}\;&\oplus\; 0\; \oplus\;{\color{forestgreen}\Lambda_3^0}\,,\\
  .
\end{aligned}
\qquad \text{and}\qquad
\begin{aligned}
  \Lambda_0 \;=\; 0\;\;&\oplus\;{\color{blue}\delta\Lambda_1}\;\oplus\;{\color{forestgreen}\Lambda_0^0}\\
  \Lambda_1 \;=\; {\color{blue}d\Lambda_0}\;&\oplus\;{\color{red}\delta\Lambda_2}\;\oplus\;{\color{green}\Lambda_1^0}\\
  \Lambda_2 \;=\; {\color{red}d\Lambda_1}\;&\oplus\;{\color{red}\delta\Lambda_3}\;\oplus\;{\color{yellow}\Lambda_2^0}\\
  \Lambda_3 \;=\; {\color{red}d\Lambda_2}\;&\oplus\;{\color{blue}\delta\Lambda_4}\;\oplus\;{\color{green}\Lambda_3^0}\\
  \Lambda_4 \;=\; {\color{blue}d\Lambda_3}\;&\oplus\; 0\; \oplus\;{\color{forestgreen}\Lambda_4^0}\,.
\end{aligned}
\end{align}
%
In 3D all the information is contained in the spectra of 0-forms and 1-forms, that is scalar and vector waves.
The same is essentially true in 4D, since harmonic forms are homotopy invariants independent of the metric.
All the information about the curvature of $(M,g)$ is contained in the spectra of $\Lambda_0$ and $\Lambda_1$, while the genuinely new piece of information $\Lambda_2^0$ is only topological and insensitive to the metric $g$.
%
\end{mytheorem}
\begin{proof}
\todotag{Insert}
\end{proof}


\begin{mytheorem}[Perturbative spectral geometry \& Lie exponentiation of infinitesimal changes.]
%    
Geometry is highly nonlinear.
In full generality it is impossible to predict how a finite modification to the shape of a manifold (e.g. growing a bump) will change its spectra.
A possible approach is thus to consider infinitesimal perturbations to a known pseudo-riemannian structure $(M,g)$ and study the corresponding infinitesimal changes in the spectra of its Laplace-type operators.
The infinitesimal changes can then be iterated to reconstuct the full shape of the manifold and corresponding spectra.
This is analogous to the way Lie groups are constructed by exponentiating their Lie algebras. 


Assume both the manifold and its spectra are given: a compact Riemannian manifold \((M,g)\) without boundary and the spectra \(\{\lambda_n^{(p)}\}_n\) of Laplacian operators  \(\Delta\) on $p$-forms and more genrally on tensor fields.
%
In particular, it is possible to define an elliptic self-adjoint Laplacian acting on $T_2(M)$ and find a corresponding orthonormal basis \(\{b_n(x)\}\) with eigenvalues \(\{\lambda_n^{(p)}\}\) satisfying
\begin{equation}
 \Delta\, b_n(x) \;=\; \lambda_n\, b_n(x).
\end{equation}

Now consider an infinitesimal (linear) perturbation to the shape $(M,g)$ and the corresponding infinitesimal (linear) change in  spectra, where $\bullet$ denote suitable tensor types,
\begin{equation}
    g\mapsto g+h\qquad \Rightarrow\qquad \{\lambda_n^{\bullet}\}\,\mapsto\,
 \{\lambda_n^{\bullet}+\mu_n^{\bullet}\}.
\end{equation}
%
The metric perturbation \(h\in T_2(M)\) can be expanded in the above basis as
\begin{equation}
 h \;=\; \sum_{n=1}^\infty h_n\, b_n(x).
\end{equation}
Schematically one can think of adding a bump described by the coefficients \(\{h_n\}_{n=1}^\infty\) to the metric \(g\) and correspondingly shifting the spectra by \(\{\mu_n^{(\bullet)}\}_{n=1}^\infty\).
% 
Concretely, we thus obtain a \emph{linear map}
\begin{equation}
    S\colon \{h_n\}\;\longrightarrow\;\{\mu_n^\bullet\}\quad \mid \quad h_n \mapsto \mu_n^\bullet = S_{mn}\,h_m.
\end{equation}
Sine the manifold is compact, the indices are indeed countable discrete, so that $S$ is an infinite square matrix.
If we further introduce some UV regularization, allowing only eigenvectors and eigenvalues up to some cut-off scale, then we have an (equal) finite number $N$ of parameters \(\{h_n\}_{n=1}^N\) and spectral shifts \(\{\mu_n^{(m)}\}_{n=1}^N\), and $S$ is a finite \(N\times N\) square matrix.
If \(\det S\neq 0\), as it is the case for \emph{general} matrices, then \(S\) is invertible and one can recover \(h\) from the spectral shift.
That is, we can infer the infinitesimal change in the shape from the infinitesimal change in the sound of the (originally known) drum.
One should then iterate these infinitesimal perturbations.

\noindent
\textbf{Further remarks.}
\begin{itemize}
\item Not every perturbation \(h\) actually changes the shape.  If \(h=\mathcal{L}_X g\) for some vector field \(X\) then \(g\mapsto g+h\) is merely the infinitesimal change of chart belonging to the flow generated by \(X\).
\item Symmetric covariant 2-tensors such as \(h\) admit a canonical decomposition similar to the Hodge decomposition.  Accordingly, the Laplacian \(\Delta\) has three distinct spectra on the space \(T_2(M)\) of such tensors.
\end{itemize}

\noindent
\textbf{References.}
This is the subject of ongoing research by Achim Kempf at PI and others.
For more details see, for example, Kempf's talk at the Perimeter Institute: \url{http://pirsa.org/15090062}.
The idea of infinitesimal spectral geometry arises from his paper on how spacetime could be simultaneously continuous and discrete, in the same way that information can.

\end{mytheorem}




%========================================================
\subsection{Closed vs exact forms \& de Rham cohomology} \todotag{To be written}
%=========================================================


\begin{mytheorem}[Closed \& exact forms]
A differential form \(\omega \in \Lambda^k (M)\) is called \textbf{closed} if \(d\omega = 0\).
It is called \textbf{exact} if there exists a \((k-1)\)-form \(\alpha\) such that \(\omega = d\alpha\).
Because \(d^2 = 0\), every exact form is closed.
The converse is not true in general, as we will see when discussing de Rham cohomology.
\end{mytheorem}


\begin{theorem}[Poincaré lemma]
Let \(M\) be a smooth manifold and let \(U \subseteq M\) be a contractible open subset.
Then, every closed differential form \(\omega \in \Lambda^k (U)\) is exact, i.e. there exists a \((k-1)\)-form \(\alpha\) such that \(\omega = d\alpha\).
This theorem has extensions to more general \emph{homotopically trivial} subsets.
\end{theorem}


%-----------------------------------------------------------------------
%======================================================================
\section{Torsion \& equivalence principle}\todotag{To be written}
%========================================================================
%-------------------------------------------------------------------------


\begin{mytheorem}[Torsion measures the nonclosure of parallelograms]\todotag{Insert pic.}
Consider a manifold \(M\) with an affine connection \(\nabla\).
Let \(X,Y\) be two vector fields on \(M\).
Construct the parallelogram starting from a point \(p \in M\) by first following the flow of \(X\) for a parameter distance \(\epsilon\), then the flow of \(Y\) for a parameter distance \(\epsilon\), then the flow of \(-X\) for a parameter distance \(\epsilon\), and finally the flow of \(-Y\) for a parameter distance \(\epsilon\).
Then, the endpoint of this parallelogram differs from the starting point \(p\) by a vector given, at leading order in \(\epsilon\), by
\begin{equation}
    \Delta p = \epsilon^2 T(X,Y) + \mathcal{O}(\epsilon^3),
\end{equation}
where \(T(X,Y)\) is the torsion tensor evaluated on \(X\) and \(Y\).
\end{mytheorem}


\begin{mytheorem}[Why torsion is a tensor?]
The fact that torsion is a tensor, even though the Christoffel symbols are not, follows directly from its geometric definition as measuring the nonclosure of parallelograms.
Alternatively, we could view it from the `physicitsts definition of tensors' and its antisymmetric nature.
Under a change of frame the Christoffel symbols transform as
\begin{equation}
    \Gamma^{\rho}_{\mu \nu} = \frac{\partial x^{\rho}}{\partial x^{\prime \sigma}} \frac{\partial x^{\prime \alpha}}{\partial x^{\mu}} \frac{\partial x^{\prime \beta}}{\partial x^{\nu}} \Gamma^{\prime \sigma}_{\alpha \beta} + \frac{\partial x^{\rho}}{\partial x^{\prime \sigma}} \frac{\partial^2 x^{\prime \sigma}}{\partial x^{\mu} \partial x^{\nu}}.
\end{equation}
The nontensorial part of the transformation is symmetric in the lower indices \(\mu, \nu\).
Therefore by splitting
\begin{equation}
    \Gamma^{\rho}_{\mu \nu} = \Gamma^{\rho}_{(\mu \nu)} + \Gamma^{\rho}_{[\mu \nu]},
\end{equation}
we see that indeed $T_{\mu,\nu}^{\quad \rho} = 2 \Gamma^{\rho}_{[\mu \nu]}$ transforms as a tensor.
\end{mytheorem}


\begin{mytheorem}[Torsion \& the equivalence principle]
The equivalence principle is \emph{completely equivalent} to the vanishing of torsion.
Indeed, the equivalence principle states that at any point there exists a \emph{local freely falling frame}, i.e. a frame where the Christoffel symbols vanish at that point.
Physically, we demand the geodesic equation reduce to a straight line
\begin{align}
    \nabla_{\dot{x}}\dot{x} = \frac{d^2 x^{\sigma}}{d \lambda^2} + \Gamma^{\sigma}_{\mu \nu} \frac{d x^{\mu}}{d \lambda} \frac{d x^{\nu}}{d \lambda} = 0
    \quad \xrightarrow[\text{local free fall at $p$}]{} \quad
    \frac{d^2 x^{\sigma}}{d \lambda^2}_{\mid p} = 0.
\end{align} 
Imposing this holds for any choice of initial velocity \(\frac{d x^{\mu}}{d \lambda}_{\mid p}\) implies the symmetric part $\Gamma_{(\mu\nu)}^\sigma=0$ at \(p\).

Note that this seems to leave room for torsion--i.e. for nonvanishing antisymmetric part $\Gamma_{[\mu\nu]}^\sigma$.
However, the equivalence principle actually demands \emph{any} mechanical experiment reduce to the special relativistic one in such local freely falling frames.
In particular, it requires that in absence of forces the angular momentum $L$ of spinning bodies stays constant, i.e. it is parallely transported via the connection.
That is 
\begin{align}
    \nabla_{\dot{x}}L = \frac{d L^\sigma}{d \lambda} + \Gamma^{\sigma}_{\mu \nu} \frac{d x^\mu}{d \lambda} L^\nu = 0
    \quad \xrightarrow[\text{local free fall at $p$}]{} \quad
    \frac{d L^\sigma}{d \lambda}_{\mid p} = 0.
\end{align}
Since both vectors $L=L^\sigma \partial_\sigma$ and $x=x^\sigma \partial_\sigma$ are arbitrary at \(p\), all of $\Gamma_{\mu\nu}^\sigma$ must vanish at \(p\).

In fact, as we mentioned, the vanishing ot torsion not only is needed, but is completely \emph{equivalent} to the existence of such local freely falling frames as we next show.
\end{mytheorem}


\begin{theorem}
Let \(M\) be a manifold with an affine connection \(\nabla\).
Then, the torsion of \(\nabla\) vanishes if and only if for any point \(p \in M\) there exists a local coordinate system \(\{x^{\mu}\}\) around \(p\) such that the connection coefficients vanish \(\Gamma^{\rho}_{\mu \nu} (p) = 0\) at $p$.
\end{theorem}
\begin{proof}
The torsion is a tensor, so if it vanishes in one frame it vanishes in all frames and is thus zero.
Conversely, Riemann normal coordinates--i.e. coordinates obtained via the exponential map at \(p\)--always exist and by construction they are such that the symmetric part $\Gamma^{\rho}_{(\mu \nu)}(p)=0$ vanishes at this point.
Indeed, in such cohordinates stratight lines $\lambda\mapsto \exp(\lambda v)$ are mapped to geodesics through $p$ in $M$ 
\begin{equation}
    \graycancel{\frac{d^2}{d \lambda^2} (\lambda v^\sigma)}+ \Gamma^{\sigma}_{\mu \nu} v^\mu v^\nu = 0.
\end{equation}
Since $v\in T_pM$ is arbitrary, this implies $\Gamma^{\sigma}_{(\mu \nu)}(p)=0$.
\end{proof}


\begin{mytheorem}[Quantum gravity \& allowing for nonzero torsion]
If we want to quantize gravity, it is wise and natural to allow for some nonzero torsion.
Indeed, in quantum mechanics all quantities are subject to quantum fluctuations.
Therefore, if the metric is quantum fluctuating, it might well be the case that it explores also configurations with nonzero torsion.

In fact, we have several examples where quantum fuctuations violate classical constraints.
An example is quantum tunnelling, where  particles can cross classically forbidden regions.
Another one is the nonvanishing of correlators between spacelike separated operators in quantum field theory, violating classical causality.
In this second case, the `rules' of classical general relativity are restored by the fact that these acausal correlators are exponentially suppressed at scales larger than the Compton wavelength of the particles involved.
Furthermore, the analogous contributions from particles and antiparticles cancel each other in the actually observable correlator.
\end{mytheorem}


\begin{mytheorem}[The Levi-Civita torsion-fixed metric connection.]
Given a pseudoriemannian manifold \((M,g)\) and a $(1,2)$-tensor $T$ antisymmetric in the lower indices, there exists a unique connection \(\nabla\) that is metric-compatible and has torsion \(T\), i.e. 
\begin{equation}
    \nabla g = 0,  \qquad T_\nabla(X,Y):=\nabla_XY-\nabla_YX-[X,Y]\equiv T(X,Y).
\end{equation}
The (anti)-symmetric parts $\Gamma_{\mu\nu}^\rho=\Gamma^\rho_{(\mu\nu)}+\Gamma^\rho_{[\mu\nu]}$ of the Christoffel symbols are uniquely determined by 
\begin{equation}\label{eq:levi_civita_christoffel_symbols}
    \Gamma^\rho_{(\mu\nu)}= \frac{1}{2} g^{\rho \sigma} \left( \partial_{\mu} g_{\nu \sigma} + \partial_{\nu} g_{\mu \sigma} - \partial_{\sigma} g_{\mu \nu} \right), \qquad 
    \Gamma^\rho_{[\mu\nu]}= T_{\mu \nu}^{\rho}.
\end{equation}
%
The \textbf{contraction of the (symmetrized) Christoffel symbols} of the Levi-Civita connection satisfy the identity
\begin{equation}\label{eq:metric_christoffel_contraction_identity}
    \Gamma^{\mu}_{(\mu \nu)} = \frac{1}{\sqrt{|g|}} \partial_{\nu} \sqrt{|g|}.
\end{equation}   
\end{mytheorem}
\begin{proof}
From the Levi-Civita formula \eqref{eq:levi_civita_christoffel_symbols} we have
\begin{equation}
    \Gamma^{\mu}_{(\mu \nu)} = \frac{1}{2} g^{\mu \sigma} \left( \partial_{\mu} g_{\nu \sigma} + \partial_{\nu} g_{\mu \sigma} - \partial_{\sigma} g_{\mu \nu} \right)
    = \frac{1}{2} g^{\mu \sigma} \partial_{\nu} g_{\mu \sigma} + \underbrace{\frac{1}{2} g^{\mu \sigma} \left( \partial_{\mu} g_{\nu \sigma} - \partial_{\sigma} g_{\mu \nu} \right)}_{=0},
\end{equation}
where two terms cancel due to the symmetry of \(g^{\mu \sigma}\) and dummy indices.
Differentiating the determinant of the metric, 
\begin{align}
\begin{aligned}
    \partial_{\nu} \sqrt{|g|} &= \frac{1}{2 \sqrt{|g|}} \text{sign}(g) \det(g_{\mu \nu}) \, \text{Tr}\big(g^{\alpha \beta} \partial_{\nu} g_{\beta \gamma}\big)
    = \frac{1}{2} \sqrt{|g|}\,\, g^{\alpha \beta} \partial_{\nu} g_{\beta \alpha}
    = \sqrt{|g|} \,\,\Gamma^{\mu}_{(\mu \nu)}.
\end{aligned}
\end{align}
\end{proof}


\begin{mytheorem}[Interpretation of the Christoffel symbols]
In view of what discssed above, we can interpret the Christoffel symbols and their derivatives as follows.
\begin{itemize}
    \item The antisymmetric part of the Christoffel symbols-- the torsion--measures the presence of fifth forces fields that violate the equivalence principle i.e. beyond standard gravity.
    Indeed, we showed that nonzero torsion is equivalent to the impossibility of finding a local freely falling frame.
    This antisymmetric part can \emph{not} be absorbed into (the symmetric part of) the energy-momentum tensor, since this must indeed match the \emph{symmetric} Einstein tensor.
    The presence of torsion thus genuinely corresponds to `extra forces' beyond standard gravity.
    %
    \item The symmetric part of the Christoffel symbols corresponds to `standard pseudoforces', like Coriolis force, that can be eliminated by going to a freely falling frame.
    Notably, the choice of what to attribute to gravity and what to inertial forces is \emph{conventional} and depends on the choice of connection.
    Indeed, the difference of two connections $\nabla,\tilde{\nabla}$ is indeed a $(1,2)$-tensor since Christoffel symbols transform as
    \begin{align}
        \Gamma^{\rho}_{\mu \nu}- \tilde{\Gamma}^{\rho}_{\mu \nu} = \frac{\partial x^{\rho}}{\partial x^{\prime \sigma}} \frac{\partial x^{\prime \alpha}}{\partial x^{\mu}} \frac{\partial x^{\prime \beta}}{\partial x^{\nu}} \left(\Gamma^{\prime \sigma}_{\alpha \beta}-\tilde{\Gamma}^{\prime \sigma}_{\alpha \beta}\right) + \graycancel{ \frac{\partial x^{\rho}}{\partial x^{\prime \sigma}} \frac{\partial^2 x^{\prime \sigma}}{\partial x^{\mu} \partial x^{\nu}}}.
    \end{align}
    If both of them are symmetric in the lower indices, their difference is also symmetric and can thus be interpreted as a field contributing to the energy-momentum tensor at the right
    %
    \item Finally, the derivatives of the Christoffel symbols measure curvature, that is tidal forces that cannot be eliminated even when going to a freely falling frame.
    Indeed, a freely falling frame is `exactly freely falling' only at a single point, but as soon as we depart from it infinitesimal tidal forces start to act.
    Curvature--i.e. essentially the Riemann tensor $R\sim \partial \Gamma$--therefore measures how much nearby freely falling frames deviate from each other.
\end{itemize}
\end{mytheorem}


%-----------------------------------------------------------------------
%======================================================================
\section{Curvature}\todotag{To be written}
%========================================================================
%---------------------------------------------------------------------------


%======================================================================
\subsection{What is curvature \& how to quantize it?}\todotag{To be written}
%========================================================================


\begin{mytheorem}[How to measure curvature? How to quantize gravity?]
%
What is curvature? How do we even measure it? Intuitively, curvature is how much a space deviates from being flat.
Here are some ways to quantify this. 
In order to quantize gravity--i.e. curvature--it will be useful to have several notions of curvature at hand, since we do not know yet which one will be the most useful in a quantum theory of gravity and what object will have to be quantized.
\begin{enumerate}
\item \textbf{Violation of angle sum laws.}  For a triangle with angles \(\alpha,\beta,\gamma\) one has
\[
\alpha + \beta + \gamma \neq 180^\circ.
\]
Such a deficit angle can be used to encode the shape. This notion of curvature thus ultimately boils down to \emph{measuring angles and possibly quantize them}. (This idea is ideed used in some approaches to quantum gravity.)
%
\item \textbf{Violation of Pythagoras' law.}  For a triangle with side lengths \(a,b,c\) one has
\[
a^2 + b^2 \neq c^2.
\]
In this case we \emph{encode the shape through metric distances}--i.e. via a Riemannian metric \(g\) on \(M\)--and we might \emph{quantize the metric} $g$ itself (as in canonical quantum gravity) or some functionals of it (as in loop quantum gravity).
%
\item \textbf{Tensorial definiton: Riemann curvature tensor or Gaussian curvature.} This is an umbrella term for the most common notion of curvature, where this is encoded in suitable tensors.
The most common is the Riemann tensor $R$, but we might equivalently define it via other \emph{equivalent} tensorial means like the \textbf{sectional curvature}.
We would then quantize a suitable expression of these?
%
\item \textbf{Nontrivial parallel transport around loops.}  A vector transported around a closed loop generally does not return to itself.  One can therefore encode the shape through an affine connection \(\Gamma\) on \(M\), which we might then quantize somehow.
%
\item \textbf{Green functions of quantum fields.} Namely, n-point correlators of quantum fields do know about the geometry of the underlying manifold.
For example, the 2-point function of a free scalar field theory on a curved manifold \((M,g)\) satisfies
\begin{equation}
    (\Box_x + m^2) G(x,y) = - \frac{1}{\sqrt{|g|}} \delta^{(n)} (x-y),
\end{equation}
where \(\Box\) is the D'Alambert operator associated to the metric \(g\).
From the knowledge of the Green function \(G(x,y)\) for all pairs of points \((x,y) \in M \times M\), one might try to reconstruct the metric \(g\) and thus the curvature of the manifold. 
In doing so, quantization conditions on $g$ might emerge naturally from the quantum nature of the field theory itself.
%
\item \textbf{Spectral geometry.} The idea is that the spectrum of differential operators defined on a manifold \((M,g)\) encodes the geometry of the manifold itself.
For example, the eigenvalues of the Laplace-D'Alambert operator \(\Box\) depend on the metric \(g\) and thus on the curvature of \((M,g)\).
One might then try to quantize gravity by quantizing the spectrum of such operators. (cf. relevant section) 
\end{enumerate}
%
Crucially, the above \textbf{local descriptions carry redundant information}  (excluding Green functions and spectral geometry).
Indeed two pseudo-Riemannian manifolds \((M,g)\) and \((M,g')\) should be considered equivalent— they describe the same spacetime— if there exists an isometric diffeomorphism
\[
e:(M,g)\longrightarrow (M,g'), \quad \text{s.t.}\quad g = e^*(g').
\]
In other words \(e\) is merely a change of chart, the two manifolds are just different coordinate descriptions of the \emph{same} manifold and hence have the same \emph{shape}.
This redundancy makes it hard to identify the true degrees of freedom that should be quantized.
This is the gauge-fixing problem already well-known from other gauge theories like QED.
%
\end{mytheorem}



\begin{mytheorem}[Pseudo-Riemannian structures \& fixing the gauge.]
%
If \((M,g)\) and \((M,g')\) are related by an isometry then they are merely different coordinate descriptions of the same manifold and hence have the same `shape'
%
A \textbf{pseudo-Riemannian structure} \(\mathcal{E}\) is an equivalence class of pseudo-Riemannian manifolds which can be mapped into each other via metric-preserving diffeomorphisms i.e. coordinate changes.
Hence space(time) should be modelled as a (pseudo-)Riemannian structure \(\mathcal{E}\), i.e. an equivalence class of pairs \((M,g)\).

One would like to \textbf{fix the gauge}, i.e. identify exactly one representative \((M,g)\) for each class \(\mathcal{E}\).
%
\textbf{Problem.}  These equivalence classes are hard to handle because the absence or existence of an isometry relating two metrics \(g\) and \(g'\) is difficult to check.
There is no known general algorithm to decide whether two metrics are isometric or not.
Instead one typically relies on (incomplete) sets of invariants, like curvature scalars, holonomies, spectra of differential operators, K-theory classes, etc.

This issue is particularly evident in the \textbf{path integral formulation of gravity}.
Indeed we would like to integrate over all possible \emph{inequivalent} structures \(\mathcal{E}\)
\begin{equation}
    \int_{\mathrm{geom. structs.}} e^{i S[\mathcal{E}]} \;D\mathcal{E},
\end{equation}
for a suitable isometry-invariant action $S[M,g]\equiv S[\mathcal{E}]$, like the Einstein-Hilbert action.
%
What we initially have is however an integral over all metrics or connections or other objects uniquely identifying a riemannian structure, i.e.
\begin{equation}
    \int_{\mathrm{metrics}} \!\!Dg\, e^{i S[g]} 
    \quad \text{or} \quad
    \int_{\mathrm{connections}}\!\!D\Gamma\,  e^{i S[\Gamma]}\quad \text{etc.}
\end{equation}
%
We must then introduce a gauge-fixing functional \(\delta(\mathcal{E})\) so that from each equivalence class of metrics, or other objects uniquely identifying a pseudo-Riemannian structure, only one representative contributes to the path integral.
%
A viable option is to perform the integration directly over \textbf{non-redundant invariants} that uniquely identify a pseudo-Riemannian structure, like spectra of suitable differential operators on tensor valued forms or K-theory classes, so that no gauge-fixing is needed.
%
\end{mytheorem}






%=======================================================
\subsection{The Riemann curvature tensor}
%=======================================================

The Riemann curvature tensor is defined as the operator measuring the noncommutativity of covariant derivatives.

\begin{mytheorem}[Ricci identity: the Riemann tensor measures noncommutativity of parallel transport in two directions]\todotag{finish}
If we neglect torsion the path obtained by first following the flow of a vector field \(X\) for infinitesimal distance \(\epsilon\), then the flow of a vector field \(Y\) for infinitesimal distance \(\epsilon\), then the flow of \(-X\) for infinitesimal distance \(\epsilon\), and finally the flow of \(-Y\) for infinitesimal distance \(\epsilon\), closes to leading order in \(\epsilon\).
In presence of torsion, the parallelogram does not close, but the Riemann tensor retains its meaning as measuring the noncommutativity of covariant derivatives.
\end{mytheorem}

\begin{mytheorem}[The Riemann tensor is just a rotation matrix in disguise\todotag{Finish}]
For the sake of the interpretation, assume torsion vanishes, everything holds even if not.
For fixed $X,Y\in \mathfrak{X}(M)$, the Riemann curvature tensor \(R(X,Y)\) is a linear map
\begin{equation}
    R(X,Y): T_p M \to T_p M \quad \mid Z\mapsto R(X,Y)Z,
\end{equation}
that returns the vector obtained by parallel transporting \(Z\) infinitesimally first in the $X$ direction and then in the $Y$ direction, and subtracting the vector obtained by doing the same but first in the $Y$ direction and then in the $X$ direction.
This is nothing but an infinitesimal rotation of the vector \(Z\) around the infinitesimal parallelogram spanned by \(X\) and \(Y\).
The same interpretation holds in presence of torsion, but the parallelogram does not close anymore and we have to correct for that.
In this sense, at fixed $X$ and $Y$ the Riemann tensor is just a rotation matrix in disguise.
%
\end{mytheorem}




\begin{mytheorem}[Symmetries of the Riemann tensor\todotag{Finish}]
The Riemann curvature tensor \(R\) has the following symmetries.
\begin{itemize}
    \item \textbf{Antisymmetry in last two indices} (no need metric, even if torsion is present):
    \begin{equation}
        R(X,Y)(\cdot) = - R(Y,X)(\cdot) \quad \Longrightarrow \quad
        R^\alpha_{\,\beta,\, \mu\nu} = - R^\alpha_{\,\beta,\, \nu\mu}.
    \end{equation}
    \item \textbf{Antisymmetry in first two lowered indices} (if metric compatible, \emph{even} if torsion is present):
    \begin{equation}
        g(R(X,Y)Z,W) = - g(R(X,Y)W,Z) \quad \Longrightarrow \quad
        R_{\alpha \beta,\, \mu\nu} = - R_{\beta \alpha,\, \mu\nu}.
    \end{equation}
    \item \textbf{Symmetry under exchange of pairs of indices} (if metric compatible) \todotag{not sure if torsion is present}
    \begin{equation}
        R_{\alpha \beta,\, \mu\nu} = R_{\mu \nu,\, \alpha \beta}.
    \end{equation}
    \item \textbf{First Bianchi identity}
    \begin{align}
        R(X,Y)Z& + R(Y,Z)X + R(Z,X)Y = \nabla_X T(Y,Z) + \nabla_Y T(Z,X) + \nabla_Z T(X,Y) \\
        & + T(T(X,Y),Z) + T(T(Y,Z),X) + T(T(Z,X),Y).
    \end{align}
    In components, this reads
    \begin{equation}       
        R^\alpha_{\,\beta,\, \mu\nu} + R^\alpha_{\,\mu,\, \nu\beta} + R^\alpha_{\,\nu,\, \beta\mu} = \dots
    \end{equation}
    \item \textbf{Second Bianchi identity}
    \begin{equation}
        \nabla_X R(Y,Z) + \nabla_Y R(Z,X) + \nabla_Z R(X,Y) = \dots
    \end{equation}
    In components, this reads
    \begin{equation}
        \nabla_\lambda R^\alpha_{\,\beta,\, \mu\nu} + \nabla_\mu R^\alpha_{\,\beta,\, \nu\lambda} + \nabla_\nu R^\alpha_{\,\beta,\, \lambda\mu} = \dots
    \end{equation}
\end{itemize}
    
\end{mytheorem}
\begin{proof}
The first identity just follows from the definition of the Riemann tensor 
\begin{align}
    R(X,Y) &= \nabla_X \nabla_Y - \nabla_Y \nabla_X - \nabla_{[X,Y]} \\
    & = - \left( \nabla_Y \nabla_X - \nabla_X \nabla_Y - \nabla_{[Y,X]} \right) = - R(Y,X).
\end{align}
The second follow from metric compatibility
\begin{align}
    \nabla g=0 &\Rightarrow X \big( g(Z,W) \big) = \graycancel{(\nabla_X g)(Z,W)}+ g(\nabla_X Z, W) + g(Z, \nabla_X W).
\end{align}
Then 
\begin{align}
    g(R(X,Y)Z,W) &= g(\nabla_X \nabla_Y Z, W) - g(\nabla_Y \nabla_X Z, W) - g(\nabla_{[X,Y]} Z, W)
    \\
    & = X \big( g(\nabla_Y Z, W) \big) - Y \big( g(\nabla_X Z, W) \big) - [X,Y]( g(Z, W)) \\
    & - g(\nabla_Y Z, \nabla_X W) + g(\nabla_X Z, \nabla_Y W) + g(Z, \nabla_{[X,Y]} W) \\
    & = \dots
\end{align}

\end{proof}





\begin{mytheorem}[Sectional curvature]\todotag{Finish.}
Sectional curvature is the simplest and most familiar notion of curvature one can define on a Riemannian manifold \((M,g)\).
Given a point \(p \in M\) and a 2-dimensional subspace \(\sigma \subseteq T_p M\) of the tangent space at \(p\), the sectional curvature \(K(\sigma)\) is defined in various equivalent ways.
\begin{itemize}
    \item \textbf{Gauss curvature.}  It is defined as the Gaussian curvature of the surface obtained by exponentiating \(\sigma\) via the exponential map at \(p\).
    Namely, given the surface $S = \{\exp_p (v) \,|\, v \in \sigma \}$, we set
    \begin{equation}
        K(\sigma) := K_{Gauss} (S).
    \end{equation}
    \item \textbf{Riemann tensor.}  We define
    \begin{equation}
        K(\sigma) := \frac{R(u,v,v,u)}{||u||^2 ||v||^2 - (g(u,v))^2},
    \end{equation}
    where \(\{u,v\}\) is any basis of \(\sigma\).
\end{itemize}
\end{mytheorem}






%-----------------------------------------------------------
%=======================================================
\section{Tetrad formalism \& tensor-valued forms}\todotag{Reformat whole section}
%=======================================================
%------------------------------------------------------------

The tetrad formalism amounts to allowing arbitrary frames for the tangent $\{e_i(x)\}_i$ and cotangent  $\{\theta^i(x)\}_i$ bundles, not necessarily induced by a coordinate system.
It is the natural generalization to the framework needed for arbitrary fiber bundles $E_F\to M$.
The only typical requirement is that the two frame be dual to each other i.e. $\theta^i (e_j) = \delta^i_j$.
In the literature, tetrads are also called \emph{vierbeins} (4-legs) in 4 dimensions or more generally \emph{vielbeins} (n-legs) in n dimensions.

To fix the notation, we reserve Greek letters $\mu,\nu,\rho\dots$ for coordinate indices and Latin letters $a,b,i,j,\dots$ for tetrad indices.
We use $\{\phi^a, f_a\}_a$ for tetrads of a generic fiber bundle $E_F\to M$, and $\{\theta^i,e_i\}_i$ for specific tetrads of the tangent bundle $TM\to M$.

\begin{mytheorem}[Avantages \& disadvantages of the tetrad formalism \answerinline{Ok}]
The main advantage of the tetrad formalism is that coefficients of tensors in arbitrary frames transform as scalars under coordinate transformations.
That is, their coordinates with respect to the tetrad bases do not change under change of coordinates, simply because we change the cohordinates $x^\mu\to \tilde{x}^\mu$ but the tetrad bases $\{e_a(x)\}_a$, $\{\theta^a(x)\}_a$ remain the same.
The main disadvantage is that expressions might get longer or more cumbersome, since we cannot use simplifications coming from coordinate bases e.g. the fact that $d (dx^\mu) = 0$ or $[\partial_\mu, \partial_\nu] = 0$.

In any case, this has several advtanges.
For example, we can choose a tetrad that is orthonormal wrt a given metric $g$ i.e. we have Minkowski (or Euclidean) metric at every point
\begin{equation}
    g = \eta_{ij} \theta^i \otimes \theta^j,\quad \eta_{ij}=\text{diag}(\pm1\dots \pm1)\,.
\end{equation}
This is particularly useful for spinors.
By definition, in any chosen frame, the $\gamma$ matrices are required to respect the Clifford algebra
\begin{equation}
    \{\gamma^\mu, \gamma^\nu\} = 2 g^{\mu\nu} \mathbb{I}.
\end{equation}
In such an orthonormal frame, this reduces to $\{\gamma^\mu, \gamma^\nu\} = 2 \eta^{\mu\nu} \mathbb{I}$, and we can thus take the $\gamma^\mu$ constantly equal to the usual Clifford algebra in flat space all over the manifold.
\end{mytheorem}



\begin{mytheorem}[Fiber valued differential forms \answerinline{Ok}]
Fiber valued differential forms are just usual $p$-forms taking values in a fiber space $F$.
That is, given a fibre bundle $E_F\to M$ with fibre $F$, a fiber valued $p$-form is a \emph{section} of the product bundle
\begin{equation}
     E_F \otimes \Lambda^p (M) \to M.
\end{equation}
Given a local frame $\{f_a\}_a$ for the fiber bundle $E_F \to M$, any fiber valued $p$-form $\alpha \in \Lambda^p (M) \otimes E_F$ can be locally written as
\begin{equation}
    \alpha =  f_a \otimes \alpha^a,
\end{equation}
where $\alpha^a \in \Lambda^p (M)$ are ordinary differential $p$-forms.
\smallskip

We define a wedge product between any two fiber valued forms already at the level of linear spaces at each point $p\in M$
\begin{align}
    &E_F \otimes \Lambda^p (M)_{\mid p} \times E_G \otimes \Lambda^q (M)_{\mid p} \to (E_F \otimes E_G) \otimes \Lambda^{p+q} (M)_{\mid p} \quad \text{s.t.}\\[5pt]
    &\alpha,\,\beta \mapsto \alpha \wedge \beta := (f_a \otimes \alpha^a) \wedge (g_b \otimes \beta^b) 
    := (f_a \otimes g_b) \otimes (\alpha^a \wedge \beta^b) \in (E_F \otimes E_G) \otimes \Lambda^{p+q} (M).
\end{align}
The wedge prodcut naturally extends to sections of the fiber bundles i.e. fiber-valued differential forms.
\smallskip 

Notably, when $E_F = T^h_kM$ is the bundle of $(h,k)$-tensors we have tensor-valued differential forms, and this notion generalizes all the tensorial structures seen so far.
\begin{itemize}
    \item Ordinary differential $p$-form are just fiber valued $p$-form with trivial fiber $F = \mathbb{R}$.
    \item An $(h,k)$-tensor field is just a fiber valued $0$-form with fiber $F = T^h_k M$.
\end{itemize}
%
In fact, since $\Lambda^*(M)\subset T^*_*(M)$ the splitting in `differntial form part' and `tensor valued image' is somewhat arbitrary.
The most natural convention is to take all the antisymmetric part as the differential form part and the rest as the tensor valued image, as in the following examples.
\begin{itemize}
    \item The torsion $T_{\mu\nu}^{\, \rho}$, antisymmetric in the $\mu\nu$ indices, is a $(1,0)$-tensor valued 2-form i.e. a fiber valued 2-form with fiber $F = T^1_0 M$.
    \item The Riemann curvature $R_{\,\rho, \mu \nu}^{\sigma}$, antisymmetric in the $\mu\nu$ indices, is a $(1,1)$-tensor valued 2-form i.e. a fiber valued 2-form with fiber $F = T^1_1 M$.
\end{itemize}
\end{mytheorem}



\begin{mytheorem}[Exterior covariant derivative\answerinline{OK}]
%
Given a fiber bundle $E_F \to M$ with fiber $F$ and a connection $\nabla$ on this bundle, we define the exterior covariant derivative $d_\nabla$ acting on fiber valued $p$-forms as the unique map
\begin{equation}
    d_\nabla: E_F \otimes \Lambda^\bullet (M)  \to E_F \otimes \Lambda^{\bullet +1} (M)
\end{equation}
thta is linear, satisfies the Leibniz rule 
\begin{equation}
    d_\nabla(\alpha \wedge \beta) = (d_\nabla \alpha) \wedge \beta + (-1)^p \alpha \wedge (d_\nabla \beta),\qquad \alpha \in E_F \otimes \Lambda^p (M),\, \beta \in E_G \otimes \Lambda^\bullet (M),
\end{equation}
and reduces to the usual covariant derivative $\nabla$ when acting on fiber valued $0$-forms and the usual exterior derivative $d$ when acting on ordinary differential forms i.e. fiber valued forms with trivial fiber $F = \mathbb{R}$.

\textbf{Beware  the exterior covariant derivative does not square to zero} in general, i.e. $d_\nabla^2 \neq 0$ due to the nontrivial connection 1-forms $\omega^a_b$.
\smallskip

Concretely, given a local frame $\{f_a\}_a$ for the fiber bundle $E_F \to M$, any fiber valued $p$-form $\alpha \in \Lambda^p (M) \otimes E_F$ can be locally written as $\alpha = f_a\otimes \alpha^a$.
The condition that $d_\nabla$ reduces to $\nabla$ on fiber valued $0$-forms implies and to the usual exterior derivative on forms with trivial fiber $F = \mathbb{R}$ then implies that the exterior covariant derivative acts as
\begin{equation}
    d_\nabla \alpha = d_\nabla ( f_a\otimes \alpha^a)= \nabla f_a \wedge \alpha^a + f_a \otimes d \alpha^a  
    = f_a\otimes\left( \omega^a_{b} \wedge \alpha^b + d \alpha^a \right),
\end{equation}
where note that $f_a$ is a fiber valued $0$-form, and $\nabla f_a$ is a fiber valued $1$-form $T^1_0M\ni X \mapsto \nabla_X f_a \in F$, and $\omega^a_{b}$ are the connection 1-forms of the connection $\nabla$ in the chosen frame $\{f_a\}_a$.
\smallskip
\begin{proof}
The Leibniz rule is then obeyed since
\begin{align}
    d_\nabla(\alpha \wedge \beta) 
    &= \nabla (f_a\otimes g_b) \wedge (\alpha^a \wedge \beta^b) + (f_a\otimes g_b) \otimes d(\alpha^a \wedge \beta^b) \\
    &= \left( \nabla f_a \otimes g_b + f_a \otimes \nabla g_b \right) \wedge (\alpha^a \wedge \beta^b) +
    (f_a\otimes g_b) \otimes \left( d \alpha^a \wedge \beta^b + (-1)^p \alpha^a \wedge d \beta^b \right)
    \\
    &= f_a \otimes g_b \otimes \omega^a_{(F)\,c} \wedge \alpha^c \wedge \beta^b + f_a \otimes g_b \otimes \omega^b_{(G)\,d} \wedge \alpha^a \wedge \beta^d \\
    &\quad + f_a \otimes g_b \otimes d \alpha^a \wedge \beta^b + (-1)^p f_a \otimes g_b \otimes \alpha^a \wedge d \beta^b
    \\
    &= f_a\otimes\left( \omega^a_{(F)\,c} \wedge \alpha^c + d \alpha^a \right) \wedge (g_b \otimes \beta^b) + (-1)^p (f_a \otimes \alpha^a) \wedge \left( g_d \otimes \left( \omega^b_{(G)\,d} \wedge \beta^d + d \beta^b \right) \right)\\
    &=(d_\nabla \alpha) \wedge \beta + (-1)^p \alpha \wedge (d_\nabla \beta)\,.
\end{align}
\end{proof}
\end{mytheorem}


\begin{mytheorem}[Connections on arbitrary fiber bundles induced from tangent bundle connection \answerinline{OK}]
The general idea is that we start with a connection $\nabla$ on the tangent bundle $TM\to M$.
Consider tetrads $\{e_i\}_i$ on the tangent bundle $T^1_0M\to M$ transforming under some (representation of a) structure group $G\subseteq GL(n,\R)$.
The connection form on the tangent bundle $\omega_{TM}$ must then take values in the representation of the Lie algebra $\mathfrak{g}\curvearrowright T^1_0M$ induced by the action of $G \curvearrowright T^1_0M$, with concrete matrix elements defined in the chosen tetrad $\{e_i\}_i$ as
\begin{align}
    \nabla_\cdot e_j &=: \omega_{TM}(\cdot) e_j =: e_i \otimes \omega^i_{\, j}(\cdot)\,.
\end{align}
We extend the connection $\nabla$ on the cotangent bundle $T^0_1M$ via the ajoint representation of $\mathfrak{g}$, and to all tensor bundles $T^h_k M$ via the tensor product of (adjoint) representations.

Given an arbitrary fiber bundle $E_F\to M$ with the same structure group $G$ as the tangent bundle i.e. fiber $F$ transforming under some representation $G\curvearrowright F$ with corresponding Lie algebra representation $\mathfrak{g}\curvearrowright F$, we can naturally induce a connection on $E_F\to M$.
Indeed, regardless of the specific representation, the connection 1-form on the tangent bundle can be seen as taking values in the \emph{abstract} Lie algebra
\begin{align}
    \omega: TM &\to \mathfrak{g} \,\, \mid T_pM \ni X \mapsto \omega(X) \in \mathfrak{g}.
\end{align}
We then simply define the connection 1-form $\omega_{F}$ on $E_F\to M$ as the 1-form taking values in the above representation of $\mathfrak{g}\curvearrowright F$.
In a given local frame $\{f_a\}_a$ for $E_F\to M$, the explicit matrix elements are defined as
\begin{align}
    \nabla_\cdot f_b &=: \omega_{F}(\cdot) f_b =: f_a \otimes \omega^a_{F\, b}(\cdot)\,.
\end{align}
In particular, the covariant derivative of a section $\psi = \psi^a f_a$ reads
\begin{equation}
    \nabla \psi = f_a \otimes \big(d\psi^a + \omega_{F\, b}^a \, \psi^b\big)\,.
\end{equation}
In particular we remark that, as abstract elements of $\mathfrak{g}$, the connection 1-forms on $E_F\to M$ and on $TM\to M$ coincide i.e. $[\omega_{F}(\cdot)]\equiv [ \omega_{TM}(\cdot)] \equiv \omega(\cdot)\in \mathfrak{g}$.
\end{mytheorem}



\begin{mytheorem}[Connection 1-form\todotag{Redo accounting for general $F$}]
%
Given an affine connection $\nabla$ on a fibre bundle $E_F \to M$ with fiber $F$ and structure group $G$, define the $\mathfrak{g}$-valued connection 1-forms $\omega_F$ as the 1-form that in a given local frame $\{f_i\}_i$ for the fiber bundle $E_F \to M$ satisfy
\begin{equation}
     \nabla_X f_j=:\omega_F(X) f_j =: f_i\otimes \omega^i_{F\, j}(X)\,\quad \forall X \in \mathfrak{X}(M).
\end{equation}
In particular, the covariant derivative of a section $\psi = \psi^i f_i$ is the fiber valued 1-form
\begin{equation}
    \nabla \psi = f_i \otimes (d\psi^i + \omega^i_{F\, j}\,\, \psi^j)\,.
\end{equation}
Under a change of local frame $\{\tilde{f}_a,\tilde{\phi}^a\}_a$ with $\tilde{f}_a= A_a^i f_i$ for some $A: U \subseteq M \to G$, the connection 1-form transforms as
\begin{equation}\label{eq:transformation_law_connection_1_form}
    \tilde{\omega}_a^b= (A^{-1})_i^b \,\, d A_a^i + (A^{-1})_i^b \,\, \omega^i_{j} \,\, A_a^j.
\end{equation}
This is nothing else that the abstract tetrad form of the well-known transformation of Christoffel symbols under change of coordinates. (cf. connections on the tangent bundle below).

\todotag{Fix from here}To confirm the connection 1-form does indeed take values in the Lie algebra $\mathfrak{g}$ of the structure group $G$, note that $\nabla$-parallel transport in $E_F\to M$ along a given curve $\gamma: [0,1] \to M$ yields linear isomorphisms (also metric preserving if $\nabla$ is a metric connection)
\begin{align}
    P_\gamma(t): E_{F,\,\gamma(0)} &\to E_{F,\,\gamma(t)} \mid e_a \to M(t)_a^b e_b\quad \text{for some }M(t)\in G \subseteq GL(F).
\end{align}  
The covariant derivative then reurns the infinitesimal change, that is 
\begin{align}
    P_\gamma(\epsilon) f = f + \epsilon \frac{d}{d t} \Big|_{t=0} P_\gamma(t) f + O(\epsilon^2) = f + \epsilon \nabla_{\dot{\gamma}(0)} f + O(\epsilon^2)
\end{align}
On the other hand, since $P_\gamma(t)$ is a linear isomorphism we have $M(t) = \exp(Lt)$ for some $L \in \mathfrak{g}$ so that
\begin{align}
    P_\gamma(\epsilon) f = M(\epsilon)^a_b f^b e_a = \left( \delta^a_b + \epsilon \frac{d}{d t} \Big|_{t=0} M(t)^a_b + O(\epsilon^2) \right) f^b e_a = f + \epsilon L f + O(\epsilon^2).
\end{align}
Comparing the two expressions we get
\begin{align}
    \nabla_{\dot{\gamma}(0)}  = \left( \frac{d}{d t} \Big|_{t=0} M(t) \right) = L \in \mathfrak{g}.
\end{align}

In turn, we understand that the transformation law \eqref{eq:transformation_law_connection_1_form} of the connection 1-from is nothing else than the expected transformation at the Lie algebra level induced by the corresponding \emph{local} transformation of the represenation space $F$.
Indeed consdier two local frames $\{f_a\}$ and $\{\tilde{f}_a = A_a^{\, b} f_b\}$ related by a local transformation $A: U \subseteq M \to G$ taking this matrix form in the chosen bases.
Parallel transport along a curve $\tau \mapsto \gamma(\tau)$ yields a parameter dependent transformation
\begin{align}
    G \ni M(t) = \exp\left( - \int_0^t \omega_{(F)}(\dot{\gamma}(\tau)) d\tau \right) : {E_F}_{\gamma(0)} \longrightarrow {E_F}_{\gamma(t)}\,.
\end{align}
The matrix elements of the transfromation in the two local frames are related by
\begin{align}
    \tilde{M}(t) &= A(\gamma(t)) M(t) A^{-1}(\gamma(0)) 
     = A(\gamma(t)) \exp\left( - \int_0^t \omega_{(F)}(\dot{\gamma}(\tau)) d\tau \right) A^{-1}(\gamma(0)) \\[6pt]
    & \Rightarrow\quad \tilde{\omega}_{F}(\dot{\gamma}(0)) \equiv \frac{d}{dt}\tilde{M}(t)_{\mid t=0} = dA(\dot{\gamma}(0)) A^{-1}(\gamma(0)) + A(\gamma(0)) \, \omega_{(F)}(\dot{\gamma}(0)) \, A^{-1}(\gamma(0)).
\end{align}
This is precisely the transformation law \eqref{eq:transformation_law_connection_1_form} evaluated on the vector $\dot{\gamma}(0)\in T_{\gamma(0)}M$.


Just like the other forms, the connection 1-form can be seen as a fiber valued differential form.
The fiber is the Lie algebra of the structure group of the frame bundle, i.e. $\mathfrak{gl}(n,\R)$ for a general connectin, and $\mathfrak{so}(p,q)$ for an orthonormal frame of metric connection with signature $(p,q)$.
Indeed, for an orthonormal tetrad $\{e_\mu\}_\mu$, metric compatibility $\nabla g=0$ implies
\begin{align}
    0 = \nabla_\bullet \eta_{\mu\nu} &= \nabla_\bullet g(e_\mu, e_\nu) = g(\nabla_\bullet e_\mu, e_\nu) + g(e_\mu, \nabla_\bullet e_\nu) = g(\omega_\mu^\rho e_\rho, e_\nu) + g(e_\mu, \omega_\nu^\sigma e_\sigma) = \omega_{\mu \nu} + \omega_{\nu \mu}.
\end{align}
Lowering the indices of the connection form with the metric we thus get an antisymmetric form $\omega_{\mu\nu} = - \omega_{\nu\mu}$, which is the hallmark of the Lie algebra $\mathfrak{so}(p,q)$.
If $G\subseteq GL(T_pM) \equiv GL(n,\R)$ is the subgroup preserving the metric $g_p$ at point $p\in M$, the connection 1-form then takes values in the Lie algebra $\mathfrak{g}$ of $G$.
Namely, for any curve $t\mapsto A(t)\sim\mathbb{I}+t\omega \in G\subseteq GL(T_pM)$
\begin{align}
    g_{\rho\sigma}A^\rho_\mu A^\sigma_\nu = g_{\mu\nu}
    &\Rightarrow 0 = \frac{d}{dt} \Big|_{t=0} \left( g_{\rho\sigma}A^\rho_\mu(t) A^\sigma_\nu(t) \right) = g_{\rho\sigma} \left( \frac{d}{dt} \Big|_{t=0} A^\rho_\mu(t) \right) \delta^\sigma_\nu + g_{\rho\sigma} \delta^\rho_\mu \left( \frac{d}{dt} \Big|_{t=0} A^\sigma_\nu(t) \right) \\
    &\Rightarrow g_{\rho\nu} \omega^\rho_\mu + g_{\mu\sigma} \omega^\sigma_\nu = 0.
\end{align}
%
This was indeed expected: the connection ultimately yields the infinitesimal change of a vector under parallel transport, which preserves angles and lengths by definition.
The change of a vector can thus only be an infinitesimal `generalized rotation' in the tangent space, i.e. an element of the Lie algebra of the orthogonal group of $g_p$, which is $SO(p,q)$ in a suitable basis.
That is, for $\epsilon \mapsto X(\epsilon)$ a parallelly transported vector along a curve $\epsilon \mapsto \gamma(\epsilon)$,
\begin{align}
    X(\epsilon) &= X(0) + \epsilon\, \nabla_{\dot{\gamma}(0)}\, X + O(\epsilon^2) \\
    &= X(0) + \epsilon\,\, \omega(\dot{\gamma}(0))\, X(0) + O(\epsilon^2) \\
    & = \left( \mathbb{I} + \epsilon \omega(\dot{\gamma}(0)) \right) X(0) + O(\epsilon^2).
\end{align}

This tells us how the connection $\nabla$ should act on a general fiber bundle $E_F\to M$ with structure group $G$: via the representation of the Lie algebra $\mathfrak{g}$ on the fiber $F$.
Namely, if $E_F \to M$ is associated to the frame bundle via a representation $\rho: G \to GL(F)$, the connection 1-form on $E_F \to M$ will be given by
\begin{equation}
    \omega^a_{\, b} = \rho_*(\omega^i_{\, j}),
\end{equation}
where $\rho_*: \mathfrak{g} \to \mathfrak{gl}(F)$ is the induced Lie algebra representation.
For example, for the tensor bundle $T^h_kM \to M$ the representation is the tensor product of $h$ copies of the given representation on $T_pM$ and $k$ copies of the dual representation of $G$.
%
For spinors, it is the spin representation of the spin group $\text{Spin}(p,q)$, that is the double cover of $SO(p,q)$.
If $\omega_{\mu\nu}$ is are the connection 1-forms of the connection on $TM$ in a given tetrad frame, and 
$\{\gamma^a\}_a$ are the gamma matrices in a given representation with basis $s_1\dots s_m$ of the spinor space, the spin representation $\rho_{spin}: \mathfrak{so}(p,q) \to \mathfrak{gl}(\C^m)$ is given by
\begin{equation}
    \omega^{\text{sp.}\, b}_{~a}:=\rho_{spin}(\omega_{\mu\nu})^b_{\,a} := \frac{1}{4} \omega_{\mu\nu} \,\{\gamma^\mu, \gamma^\nu\}^b_{\,a}.
\end{equation}
The connection 1-form act on basis spinors as 
\begin{equation}
    \nabla s_a = \omega^{\text{sp.}\, b}_{~ a}{} s_b = \tfrac{1}{4}\omega_{\mu\nu}\,[\gamma^\mu, \gamma^\nu]^b_{\,a} s_b,
\end{equation}
and on general spinor fields $\psi = \psi^a s_a$ as
\begin{equation}
    \nabla \psi = d \psi^a \otimes s_a + \psi^a \, \nabla s_a = \left( d \psi^b + \tfrac{1}{4} \omega_{\mu\nu}\, [\gamma^\mu, \gamma^\nu]^b_{\,a} \, \psi^a \right) \otimes s_b,\,\,\quad \text{i.e.} \quad
    \nabla_\mu \psi^b = \partial_\mu \psi^b + \tfrac{1}{4} \omega_{\mu\nu}\, [\gamma^\mu, \gamma^\nu]^b_{\,a} \, \psi^a.
\end{equation}
%
\end{mytheorem}


\begin{mytheorem}[Metric compatibility in arbitrary fiber bundle\answerinline{OK}]
%
Given a fiber bundle $E_F \to M$ with fiber $F$ and structure group $G$, endowed with a metric 
$g\in\Gamma(\otimes^2 E_F^*)$ and a connection $\nabla$ on $E_F$ with $\mathfrak{g}$-valued connection form $\omega$, we say that the connection is metric compatible if 
\begin{align}
    \nabla g \equiv 0\quad \text{i.e.}\quad d \left( g(s,t) \right) = g(\nabla s, t) + g(s, \nabla t)\quad \text{$\forall$ sections }s,t \in \Gamma(E_F).
\end{align}
In a tetrad $\{f_a\}_a$ for $E_F \to M$ with connection forms $\omega_{\,a}^{b}$ and metric $g_{ab} := g(f_a, f_b)$, the condition reads
\begin{equation}
    d g_{ab} = \omega_{\,a}^{c}\, g_{cb} + \omega_{\,b}^{c} \,g_{ac}.
\end{equation}
If the tetrad is orthonormal i.e. $g_{ab} = (\pm1\dots\pm1)$ constant, the condition reduces to \emph{antisymmetry} of the connection 1-forms in this tetrad, upon lowering an index with the metric,
\begin{equation}
    \omega_{ab} = - \omega_{ba}\quad \text{ where }\quad \omega_{ab} := g_{ac} \, \omega^c_{~b}\,,
\end{equation}
which is the hallmark of the Lie algebra $\omega \in \mathfrak{so}(p,q)$ where $(p,q)$ is the signature of the metric.
%
\end{mytheorem}


\begin{mytheorem}[The example: connection on spinors\todotag{Redo}]
%

\end{mytheorem}


\begin{mytheorem}[Curvature 2-form \answerinline{OK}]
Given a connection $\nabla$ on a vector bundle $E_F\to M$ with structure group $G$, and a tetrad frame $\{\phi^a,f_a\}_a$, we define the $G$-valued curvature 2-form $\Omega$ with $\R$-valued matrix curvature 2-forms $\Omega^b_{\, a}$ in the chosen tetrad as
\begin{equation}
    R(X,Y) (f_a) = \Omega(X,Y) (f_a) = f_b \otimes \Omega^b_{\, a}(X,Y), \quad X,Y \in \mathfrak{X}(M)\,.
\end{equation}
Note that indeed $\Omega$ is $G$-valued since it yields a \emph{linear} transformation\todotag{Make sure it is not $\mathfrak{g}$ valued} 
\[R(X,Y) =\nabla_X \nabla_Y - \nabla_Y \nabla_X - \nabla_{[X,Y]}: E_{F\vert p} \to E_{F\vert p}.\]
On a general element $\tilde{f}=\tilde{f}^a f_a$ we thus have
\begin{equation}
    R(X,Y) (\tilde{f}) = \tilde{f}^a R(X,Y) (f_a) = \tilde{f}^a f_b \otimes \Omega^b_{\, a}(X,Y).
\end{equation}
Given also a tetrad frame $\{\theta^\mu, e_\nu\}_\nu$ on $TM\to M$, we can explicitate all indices and write
\begin{align}
    R^b_{a,\,\mu\nu}= \Omega^b_a(e_\mu,e_\nu) \theta^\mu \wedge \theta^\nu.
\end{align}
\end{mytheorem}


\begin{mytheorem}[Cartan structure equations\answerinline{OK}]
%
Consider a connection $\nabla$ on a fiber bundle $E_F \to M$ with fiber $F$ and structure group $G$.
Given a tetrad frame $\{\phi^a,f_a\}_a$, consider the $(1,0)$ tensor valued 1-form $\phi :=  f_a \otimes \phi^a$.
The connection 1-form and curvature 2-form satisfy the Cartan structure equations, expressed concisely in terms of the exterior covariant derivative $d_\nabla$ as
\begin{align}
    T &:= d_\nabla \phi = f_a \otimes \left(d \phi^a + \omega^a_{\, b} \wedge \phi^b \right) 
    \quad \text{i.e.} \quad T^a = d \phi^a + \omega^a_{\, b} \wedge \phi^b,
    \\
    \Omega_{\, b} &= d_\nabla \omega_{\, b} = f_a \otimes \left(d \omega^a_{\, b} + \omega^a_{\,c} \wedge \omega^c_{\, b}\right) 
    \quad\text{i.e.} \quad \Omega^a_{\, b} = d \omega^a_{\, b} + \omega^a_{\, c} \wedge \omega^c_{\, b}.
\end{align}
The \textbf{Cartan equations are essentially definitions} of the torsion $T$ and curvature 2-form in terms of the connection 1-forms and the tetrad basis.
In fact, for \textbf{general fiber bundles} the first line is the \textbf{only definition of torsion}, while it coincides with the usual torsion tensor in case $E_F = TM$ is the tangent bundle.
In particular, in absence of torsion $T=0$ the first Cartan structure equation reduces to
\begin{equation}
    d \phi^a + \omega^a_{\, b} \wedge \phi^b = 0.
\end{equation}
\end{mytheorem}
\vspace{-3mm}
\begin{proof}
From the definition of the exterior covariant derivative we have indeed
\begin{align}
    d_\nabla \phi &= d_\nabla (f_a \otimes \phi^a) = f_a \otimes d \phi^a  +  \nabla f_a \wedge \phi^a = f_a\otimes (d \phi^a + \omega^a_{\, b} \wedge \phi^b) ,
    \\
    d_\nabla \omega_{\, b} &= d_\nabla (f_a \otimes \omega^a_{\, b} ) = f_a\otimes d \omega^a_{\, b}  + \nabla f_a \wedge \omega^a_{\, b} = f_a\otimes (d \omega^a_{\, b} + \omega^a_{\, c} \wedge \omega^c_{\, b}).
\end{align}
For a general fibre bundle case, the first equation is just a definition of the torsion $T$ as the exterior covariant derivative of the tetrad 1-form $\phi$, and we need only to prove the second expression does indeed coincide with the usual curvature operator.
Using linearity of the 2-form, we prove it on a coordinate induced basis $\partial_\mu,\,\partial_\nu$ so that 
\begin{align}
    \Omega(\partial_\mu,\,\partial_\nu)(f_b)&:= \nabla_\mu \nabla_\nu f_b - \nabla_\nu \nabla_\mu f_b - \graycancel{\nabla_{[\partial_\mu,\,\partial_\nu]} f_b}
    \\
    &= \nabla_\mu \left( f_a \otimes \omega^a_{\, b}(\partial_\nu) \right) - \nabla_\nu \left( f_a \otimes \omega^a_{\, b}(\partial_\mu) \right)
    \\
    &= f_a \otimes \left( \partial_\mu \omega^a_{\, b}(\partial_\nu) + \omega^a_{\, c}(\partial_\mu) \, \omega^c_{\, b}(\partial_\nu) - \partial_\nu \omega^a_{\, b}(\partial_\mu) - \omega^a_{\, c}(\partial_\nu) \, \omega^c_{\, b}(\partial_\mu) \right)
    \\
    &\equiv f_a \otimes \left( d \omega^a_{\, b} + \omega^a_{\, c} \wedge \omega^c_{\, b} \right)(\partial_\mu,\,\partial_\nu) =: \left( d_\nabla \omega_{\, b} \right)(\partial_\mu,\,\partial_\nu).
\end{align}
%
\end{proof}


\begin{mytheorem}[Bianchi identities in the tetrad formalism\answerinline{ok}]
%
The Bianchi identities in a tetrad $\{f_i,\,\phi^i\}_i$ are concisely expressed via the exterior covariant derivative $d_\nabla$ as
\begin{align}
    d_\nabla T &= \Omega_{\, j} \wedge \phi^j\qquad \text{i.e.}\quad d_\nabla T^i = \Omega^i_{\, j} \wedge \phi^j,
    \\
    d \Omega^i_{\, j} &= 0\qquad \text{i.e.}\quad d_\nabla \Omega_{\, j} = f_i \otimes \Big(\omega_k^i\wedge \Omega^k_j + \underbrace{d \Omega_{\, j}^i}_{=0}\Big) = f_i \otimes\omega_k^i\wedge \Omega^k_j\,.
\end{align}
Beware the 2nd identity says the exterior derivative of the $\R$-valued 2-forms $\Omega^i_{\, j}$ vanish, \emph{not} that the exterior covariant derivative of the $(1,0)$-tensor valued 2-form $\Omega_{\, j}$ vanishes!
The \textbf{Bianchi identities} essentially express \textbf{symmetry contraints} that follow \textbf{from the very definition} of torsion and curvature.
In particular, in absence of torsion $T=0$ the first Bianchi identity reduces to
\begin{equation}
    \Omega^i_{\, j} \wedge \phi^j = 0.
\end{equation}
\end{mytheorem}
\vspace{-3mm}
\begin{proof}
The first identity follows from direct computation of the exterior covariant derivative of the torsion 2-form and the second Cartan equation above
\begin{align}
    d_\nabla T &= d_\nabla (d_\nabla \phi) = d_\nabla \left[f_i\otimes \left(d \phi^i + \omega^i_{\, j} \wedge \phi^j \right)\right]
    \\
    &=
    f_i\otimes \left(\graycancel{d^2 \phi^i} + d\omega^i_{\, j} \wedge \phi^j \bluecancel{- \omega^i_{\, j} \wedge d\phi^j}  \right)
    + f_i \otimes \omega^i_k\wedge \left(\bluecancel{d \phi^k} + \omega^k_{\, j} \wedge \phi^j \right) 
    \\
    &=
    f_i \otimes \left( d\omega^i_{\, j} + \omega^i_{\, k} \wedge \omega^k_{\, j} \right) \wedge \phi^j
    \equiv \Omega_{\, j} \wedge \phi^j.
\end{align}
For the second equation, renaming dummy indices, expressing $\Omega^i_{\, j}$ in terms of $\omega$ and using it commutes under the wedge product since it is a 2-form, we have
\begin{align}
    d \Omega_{\,j}^i &= d \left( d \omega^i_{\, j} + \omega^i_{\, k} \wedge \omega^k_{\, j} \right) = \graycancel{d^2 \omega^i_{\, j}} + d\omega^i_{\, k} \wedge \omega^k_{\, j} - \omega^i_{\, k} \wedge d\omega^k_{\, j}
    \\
    &= \big(\Omega^i_k-\omega^i_\ell\wedge \omega^\ell_k\big)\wedge \omega^k_{\, j} - \omega^i_{\, k} \wedge \big(\Omega_j^k-\omega^k_\ell\wedge \omega^\ell_j\big)
    \\
    & = \underbrace{\Omega^i_k \wedge \omega^k_{\, j} - \omega^i_{\, \ell}\wedge \Omega^\ell_j}_{=0}
    \,\,\,\,\underbrace{-\omega^i_\ell\wedge \omega^\ell_k\wedge \omega^k_{\, j}
    + \omega^i_{\, k} \wedge \omega^k_\ell\wedge \omega^\ell_j}_{=0}
    = 0.
\end{align}
The rest follows from the definition of the exterior covariant derivative.
\end{proof}






%===============================================================
\subsection{Tetrad formalism in the tangent bundle}
%===============================================================


\begin{mytheorem}[Structure constants of a tetrad frame\answerinline{OK}]
%
Given a tetrad $\{\theta^i,e_i\}_i$ for the tangent bundle, the structure constants $C^i_{jk}$ are defined as the coefficients of the commutator expansion
\begin{equation}
    [e_j, e_k] = C^i_{jk} e_i \,\, \quad \text{antisymmetric in $j,k$}.
\end{equation}
By construction, the structure constants are antisymmetric in the lower indices $C^i_{jk} = - C^i_{kj}$.

They can be equivalently defined as the coefficients of the exterior derivative expansion
\begin{equation}
    d \theta^i = - \frac{1}{2} C^i_{jk} \theta^j \wedge \theta^k = - C^i_{j k} \,\, \theta^j \otimes \theta^k \quad \text{(sum over repeated indices)}.
\end{equation}

The structure constants measure the nonclosure of infinitesimal parallelograms constructed using the tetrad vector fields $\{e_i\}_i$.
The \textbf{structure constants vanish} identically $C_{jk}^i \equiv 0$ if and only if the tetrad is \emph{holonomic} i.e. \textbf{induced by coordinates}  $\theta^\mu=dx^\mu,\,e_\nu=\frac{\partial}{\partial x^\nu}$

Beware the structure constants make sense only for the tangent bundle $TM$ since only there we have a natural notion of commutator of vector fields, and in fact arbitrary fiber-valued forms are expressed as linear combinations of ordinary $\R$-valued forms times fiber basis elements.
\end{mytheorem}
\begin{proof}
Fix any coordinate system $\{x^\mu\}_\mu$ and let $\theta^i = \theta^i_\mu dx^\mu$, $e_j = e_j^\nu \partial_\nu$.
By duality of the tetrad we have
\begin{align}
    0= \partial_\mu \delta^i_j &= \partial_\mu \big(\theta^i (e_j) \big) = \partial_\mu\big(\theta^i_\nu e_j^\nu \big) = (\partial_\mu \theta^i_\nu) e_j^\nu + \theta^i_\nu (\partial_\mu e_j^\nu)
    \quad \Rightarrow\quad (\partial_\mu \theta^i_\nu) e_j^\nu =- \theta^i_\nu (\partial_\mu e_j^\nu)\,.
\end{align}
By definition of exterior derivative we have
\begin{align}
    d \theta^i &= (\partial_\mu \theta^i_\nu) dx^\mu \wedge dx^\nu = \partial_\mu \theta^i_\nu \Big(dx^\mu\otimes dx^\nu- dx^\nu \otimes dx^\mu \Big)\,.
\end{align}
Evaluating on the tetrad and using the previous relation we get
\begin{align}
    d\theta^i (e_j, e_k) 
    = \partial_\mu \theta^i_\nu \Big( e_j^\mu e_k^\nu - e_j^\nu e_k^\mu \Big) 
    = - \theta^i_\nu \Big( (\partial_\mu e_j^\nu) e_k^\mu - (\partial_\mu e_k^\nu) e_j^\mu \Big) 
     =: - \theta^i \big( [e_j, e_k]\big) = - C^i_{jk}\,.
\end{align}
That is to say
\begin{align}
    d\theta^i = - \frac{1}{2} C^i_{jk} \theta^j \wedge \theta^k\, \Leftrightarrow [e_j, e_k] = C^i_{jk} e_i\,.
\end{align}
Finally the fact that $C_{ij}^k \equiv 0$ if and only if $e_\mu\equiv \partial_\mu$ for some coordinate system $\{x^\mu\}_\mu$ follows from Frobenius theorem.
%
\end{proof}



\begin{mytheorem}[Christoffel symbols in a tetrad frame of the tangent bundle\answerinline{OK}]
Given a tetrad $\{\theta^i,\, e_i\}_i$ on the tangent bundle $TM\to M$ with affine connection \(\nabla\), define the Christoffel symbols \(\Gamma^i_{jk}\) by
\begin{equation}
    \nabla_{e_j} e_k =: \Gamma^i_{jk} e_i\,,\qquad\text{that is}\qquad \omega^i_{\, j}(\cdot) = \Gamma^i_{ jk}\,\, \theta^k(\cdot).
\end{equation}
Under a change of tetrad \(\{\tilde{\theta}^a,\, \tilde{e}_a\}_a\) related to the previous one by \(\tilde{e}_a = A_a^{\, j} e_j\), the Christoffel symbols transform as
\begin{equation}
    \tilde{\Gamma}^c_{ab} = (A^{-1})^c_k \left( \Gamma^k_{ij} A_a^{\, i} A_b^{\, j} + A_a^{\, i}\,\,\partial_{e_i} A_b^{\, k}  \right).
\end{equation}
%
For a metric connection $\nabla g\equiv 0$ with arbitrary torsion $T$, the \textbf{Koszul formula} is
\begin{align}
\begin{aligned}
    2\,g(\nabla_X Y, Z) ={}& X\!\left(g(Y,Z)\right) + Y\!\left(g(Z,X)\right) - Z\!\left(g(X,Y)\right)\\
    &\, + g([X,Y],Z) - g([Y,Z],X) - g([X,Z],Y) \\
    & \,+ g\!\left(T(X,Y),Z\right) - g\!\left(T(Y,Z),X\right) - g\!\left(T(X,Z),Y\right).
\end{aligned}
\end{align}
We then express the Christoffel symbols in terms of metric $g_{ij} := g(e_i, e_j)$, structure constants \(C^i_{\, jk}:=\theta^k([e_i,e_j])\) and torsion components \(T^i_{\, jk}:=\theta^i(T(e_j,e_k))\) in a given tetrad as
\begin{align}
    \Gamma_{ij}^k=&\tfrac{1}{2} g^{k\ell} \Big( \partial_{e_i}\,g_{j \ell} + \partial_{e_j}\,g_{i \ell} - \partial_{e_\ell}\,g_{i j}\Big)\\
    & + \tfrac{1}{2} g^{k\ell} \Big( g_{m \ell} C^m_{ij} - g_{m j} C^m_{i \ell} - g_{m i} C^m_{j \ell} \Big) \\
    & +\tfrac{1}{2} g^{k\ell} \Big( g_{m \ell} T^m_{ij} -g_{m i} T^m_{j \ell} - g_{m j} T^m_{i \ell}  \Big).
\end{align}
If $C_{ij}^k\equiv 0$ we recover usual expression for Christoffel symbols in a coordinate basis\footnote{In fact we have $C_{ij}^k\equiv 0$ iff there exist some coordinates $x^\mu$ s.t. $e_\mu=\partial_\mu$.}, split into symmetric and antisymmetric part in the lower indices 
\begin{equation}
    \Gamma_{(ij)}^k=\tfrac{1}{2} g^{k\ell} \Big( \partial_{e_i}\,g_{j \ell} + \partial_{e_j}\,g_{i \ell} - \partial_{e_\ell}\,g_{i j}\Big) - \tfrac{1}{2} g^{k\ell} \Big( g_{m i} T^m_{j \ell} + g_{m j} T^m_{i \ell}  \Big)\,,
    \quad \Gamma_{[ij]}^k = \tfrac{1}{2} T^k_{ij}\,.
\end{equation}
In the absence of torsion the second line vanishes, but $\Gamma_{ij}^k$ is not yet symmetric in general in the lower indices due to the possibly nonzero structure constants $C_{ij}^k=[e_i,e_j]^k$.
Finally in an orthonormal frame $g_{ij} = \eta_{ij}$  and the first line $\propto \, \partial g$ vanishes.
\end{mytheorem}
\begin{proof}
By metric compatibility $\nabla g=0$, that is $0=X(g(Y,Z)) - g(\nabla_X Y,Z) - g(Y, \nabla_X Z)$.
By definition of torsion $\nabla_X Y = \nabla_Y X + [X,Y] + T(X,Y)$.
Permuting the vector fields $X,Y,Z$ cyclically and using these two identities we get the Koszul formula.
Setting \(X=e_j,\, Y=e_k,\, Z=e_i\), using the definition of the Christoffel symbols and structure constants, and then raising indices with the metric inverse gives the final expression.
\end{proof}


\begin{mytheorem}[Torsion 2-form in a tetrad frame on the tangent bundle\answerinline{OK}]
%
Beware torsion makes sense only for the tangent bundle since only there we have a natural notion of commutator of vector fields $[X,Y]$.
Indeed torsion measures the nonclosure of the parallelogram obtained by following the flow of the first vector field, then the flow of the parallel transport of the second vector field along the first, and then viceversa, an operation which is only defined on the manifold.

Given a tetrad frame \(\{\theta^i,\, e_i\}_i\) on the tangent bundle \(TM\to M\) with structure constants \(C^i_{jk}\) and a connection \(\nabla\) with Christoffel symbols \(\Gamma^i_{jk}\), the torsion 2-forms in this tetrad read
\begin{equation}
    T^i = \tfrac{1}{2} T^i_{jk} \,\, \theta^j \wedge \theta^k,
    \qquad \text{where}\qquad
    T^i_{jk} = \Gamma^i_{jk} - \Gamma^i_{kj} - C^i_{jk}.
\end{equation}
In a coordinate induced frame $C_{ij}^k\equiv0$, and we recover the usual expression for the torsion components.
%
\end{mytheorem}


\begin{mytheorem}[Riemann curvature tensor in a tetrad frame on the tangent bundle\answerinline{OK}]
%
Given a tetrad frame \(\{\theta^i,\, e_i\}_i\) with structure constants \(C^i_{jk}\) and a connection $\nabla$ on the tangent bundle $TM$ with Christoffel symbols \(\Gamma^i_{jk}\), the Riemann tensor components in this tetrad read
\begin{equation}
    R^\ell_{k,\,ij} = \partial_{e_i} \Gamma^\ell_{j k} - \partial_{e_j} \Gamma^\ell_{i k} + \Gamma^\ell_{i m} \Gamma^m_{j k} - \Gamma^\ell_{j m} \Gamma^m_{i k} - C^m_{ij} \Gamma^\ell_{m k}.
\end{equation}
In a coordinate induced frame $C_{ij}^m\equiv0$ and we recover the usual expression for the Riemann tensor components.
In particular, the curvature 2-forms can be expressed in terms of the Riemann tensor components as
\begin{equation}
    \Omega^\ell_{\, k} = \tfrac{1}{2} R^\ell_{k,\,ij} \,\, \theta^i \wedge \theta^j.
\end{equation}
%
\end{mytheorem}



\begin{mytheorem}[Cartan structure equations\answerinline{OK}]
%
Consider a connection $\nabla$ on the tangent bundle $TM \to M$.
Given a tetrad frame $\{\theta^i,e_j\}_i$, consider the $(1,0)$ tensor valued 1-form $\theta :=  e_i \otimes \theta^i$.
The connection 1-form, and the torsion and curvature 2-forms satisfy the Cartan structure equations, expressed concisely in terms of the exterior covariant derivative $d_\nabla$ as
\begin{align}
    T &= d_\nabla \theta = e_a \otimes \left(d \theta^a + \omega^a_{\, b} \wedge \theta^b \right) 
    \quad \text{i.e.} \quad T^a = d \theta^a + \omega^a_{\, b} \wedge \theta^b,
    \\
    \Omega_{\, b} &= d_\nabla \omega_{\, b} = e_a \otimes \left(d \omega^a_{\, b} + \omega^a_{\,c} \wedge \omega^c_{\, b}\right) 
    \quad\text{i.e.} \quad \Omega^a_{\, b} = d \omega^a_{\, b} + \omega^a_{\, c} \wedge \omega^c_{\, b}.
\end{align}
The \textbf{Cartan equations are essentially definitions} of the torsion and curvature 2-forms in terms of the connection 1-forms and the tetrad basis.
In the case of the tangent bundle, the first line does indeed coincides with the \textbf{usual definition of torsion} tensor.
In particular, in \textbf{absence of torsion $T=0$ the first Cartan equation} reduces to
\begin{equation}
    d \theta^a + \omega^a_{\, b} \wedge \theta^b = 0.
\end{equation}
\end{mytheorem}
\begin{proof}
The second equalities just follow from the definition of the exterior covariant derivative and are already shown in the case of general fibre bundles above.
We need to show the expressions coincide with the old notions of torsion and curvature tensors.
Since the identity are tensorial, we can prove them by direct computation using a coordinate frame $\{\theta^\mu=dx^\mu,\, e_\nu=\partial_{x^\nu}\}_\nu$, the known expressions $T\sim \Gamma - \Gamma^T$ and $R\sim \partial \Gamma + \Gamma \Gamma$ for the torsion and Riemann tensor, and the relations
\begin{align}
    \theta^\rho = dx^\rho,\quad d\theta^\rho = 0,\quad
    \omega^\rho_{\, \nu} = \Gamma^\rho_{\mu \nu} dx^\mu,
    \quad
    d \omega^\rho_{\, \nu} = \partial_\sigma \Gamma^\rho_{\mu \nu} dx^\sigma \wedge dx^\mu,
    \quad
    \omega^\rho_{\, \sigma} \wedge \omega^\sigma_{\, \nu} = \Gamma^\rho_{\mu \sigma} \Gamma^\sigma_{\lambda \nu} dx^\mu \wedge dx^\lambda.
\end{align}
%
\end{proof}



\begin{mytheorem}[Bianchi identities on the tangent bundle \& recovering coordinate formalism\answerinline{OK}]
%
The Bianchi identities on the tangent bundle are unchanged from a general fiber bundle and so is the proof.
In a tetrad $\{e_a,\theta^a\}_a$ of $TM\to M$ they read
\begin{align}
    d_\nabla T &= \Omega_{\, b} \wedge \theta^b\qquad \text{i.e.}\quad d_\nabla T^a = \Omega^a_{\, b} \wedge \theta^b,
    \\
    d \Omega^a_{\, b} &= 0\qquad \text{i.e.}\quad d_\nabla \Omega_{\, b} = e_a \otimes \Big(\omega_c^a\wedge \Omega^c_b + \underbrace{d \Omega_{\, b}^a}_{=0}\Big) = e_a \otimes\omega_c^a\wedge \Omega^c_b\,.
\end{align}
Beware the 2nd identity says the exterior derivative of the $\R$-valued 2-forms $\Omega^i_{\, j}$ vanish, \emph{not} that the exterior covariant derivative of the $(1,0)$-tensor valued 2-form $\Omega_{\, j}$ vanishes!
The \textbf{Bianchi identities} essentially express \textbf{symmetry contraints} that follow \textbf{from the very definition} of torsion and curvature.
In particular, \textbf{in absence of torsion} $T=0$ the \textbf{first Bianchi identity becomes}
\begin{equation}
    \Omega^a_{\, b} \wedge \theta^b = 0.
\end{equation}
\end{mytheorem}







\begin{mytheorem}[Einstein tensor in a tetrad frame of the tangent bundle\todotag{Check and finish}]
%
Here we also discuss the tensor (cf. Kempf 15)
\begin{equation}
    H_{\mu\nu\rho}=\dots
\end{equation}
\end{mytheorem}


\begin{mytheorem}[EM tensor varying wrt tetrad fields\todotag{Check and finish}]
%
Since the tetrad formalism is completely equivalent to the metric formalism, we can express the EM tensor as the variation of the matter action wrt the tetrad fields $\{\theta^i\}_i$ instead of the metric $g$. \todotag{Maybe move it to the GR section?}
    
\end{mytheorem}



\bigskip
{
\color{red}
From Kempf GR L10 and then L15 missing:
\begin{itemize}
    \item connection $\mathfrak{g}$-valued 1-form [ANY FIBER]
    \item change of cohordinates $dA A^{-1}+ A\omega A^{-1}$ for connection 1-form [ANY FIBER]
    \item curvature $T^1_1F$-valued 2-form [ANY FIBER]
    \item torsion $T^1_0M$-valued 2-form [ONLY TM]
    \item Cartan structure equations
    \item Bianchi identities
    \item metric compatibility in arbitrary fiber bundle [ANY FIBER WITH METRIC]
    \item Christoffel symbols for metric connection on tangent bundle via structure constants [ONLY TM]
    \item Riemann curvature tensor components in tetrad frame [ONLY TM]
\end{itemize}
}
\bigskip

\noindent
\textbf{INDEX OF SECTION}
\begin{enumerate}
    \item general intro Ok
    \item advantages of the tetrad formalism Ok
    \item fiber valued differential forms Ok
    \item total exterior covariant derivative Ok
    \item inducing connections on associated fiber bundles Ok
    \item connection 1-form on arbitrary fiber bundle  {\color{red}MISSING (swap with above point when done)}
    \item metric compatibility in arbitrary fiber bundle OK
    \item example: connection on spinors {\color{red}MISSING}
    \item curvature 2-form on arbitrary fiber bundle {\color{red} CHECK NEEDED}
    \item Cartan structure equations OK
    \item Bianchi identities OK
\end{enumerate}

\noindent
\textbf{INDEX OF TANGENT BUNDLE}
\begin{enumerate}
    \item structure constants of a tetrad frame Ok
    \item Christoffel symbols in a tetrad frame of the tangent bundle OK
    \item torsion 2-form in a tetrad frame on the tangent bundle Ok
    \item Riemann curvature tensor in a tetrad frame on the tangent bundle OK
    \item Cartan equations on the tangent bundle \& recovering coordinate formalism OK
    \item Bianchi identities on the tangent bundle OK
    \item Einstein tensor in a tetrad frame of the tangent bundle {\color{red} MISSING}
    \item EM tensor in general 1. varying wrt the metric \& 2. varying wrt tetrad fields  {\color{red}MISSING}
\end{enumerate}