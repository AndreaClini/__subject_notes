% !TeX root = ../QFT_main.tex
%=========================================================
%=========================================================
\chapter{History \& Principles of Quantum Mechanics}
%========================================================
%=========================================================


%---------------------------------------------------------
%=========================================================
\section{The rise and fall of Quantum Mechanics} 
%=========================================================
%----------------------------------------------------------


\begin{mytheorem}[The rise of Quantume Mechanics.\questiontag{Check \& improve.}]
At the turn of the 20th century, several experimental observations challenged the classical understanding of physics and led to the development of Quantum Mechanics.
Some of the key discoveries are the following.
\begin{itemize}
    \item \textbf{Blackbody Radiation.} Max Planck's solution to the ultraviolet catastrophe by introducing the concept of quantized energy levels.
    \item \textbf{Photoelectric Effect.} Albert Einstein's explanation of the photoelectric effect, proposing that light consists of discrete packets of energy called photons.
    \item \textbf{Atomic Spectra \& stability of matter.} Niels Bohr's model of the hydrogen atom, introducing quantized orbits for electrons, solved the problem of radiation by accelerated charges that would predict electrons to spiral down into the nucleus.
    Ultimately, the solution was found in the uncertainty principle: as $\Delta x \Delta p \gtrsim \hbar$, as the electron gets closer to the nucleus, its momentum uncertainty increases, preventing it from collapsing into the nucleus.
    \item \textbf{Wave-Particle Duality \& double-slit experiment.} Louis de Broglie's hypothesis that particles exhibit wave-like properties, leading to the development of wave mechanics by Erwin Schrödinger.
    \item \textbf{Uncertainty Principle.} Heisenberg's formulation of the uncertainty principle, which states that certain pairs of physical properties cannot be simultaneously known to arbitrary precision.
\end{itemize}
\end{mytheorem}


\begin{mytheorem}[The fall of Quantum Mechanics. \questiontag{Check \& improve}]
Despite its successes, Quantum Mechanics soon after faced several challenges that progressively highlighted its limitations, up to making it untenable, and paved the way for the development of Quantum Field Theory.
%
\begin{itemize}
    \item \textbf{Relativistic Incompatibility \& negative energies.} Quantum Mechanics, as originally formulated, was not compatible with the principles of special relativity.
    Attempts to create a relativistic version of Quantum Mechanics, such as the Klein-Gordon and Dirac equations, led to difficulties in interpreting negative energy solutions and the need for a new framework.
    \item \textbf{Particle Creation and Annihilation.} Quantum Mechanics treated particles as fixed entities, unable to account for processes where particles are created or destroyed, which are common in high-energy physics.
    This limitation necessitated a theory that could handle variable particle numbers.
    \item \textbf{Infinities and Renormalization.} Early attempts to apply Quantum Mechanics to electromagnetic interactions resulted in infinities in calculated quantities, such as the electron's self-energy.
    The development of renormalization techniques was required to make sense of these infinities, but it indicated that a more fundamental theory was needed.
    \item \textbf{Gauge Invariance and Interactions.} Quantum Mechanics struggled to incorporate gauge invariance, a fundamental symmetry underlying the interactions of fundamental forces.
    The need for a framework that could naturally include gauge fields and their interactions led to the development of Quantum Field Theory.
\end{itemize}
\end{mytheorem}


\begin{mytheorem}[The price of noncommutativity.\notetag{Check}]
If we demand that observables are represented by noncommuting operators, we have to pay a price: such observable must be operators acting on some vector space $\mathcal{H}$, i.e. they must be matrices, since standard numbers commute.
Furthermore, taking the trace of the canonical commutation relations we have
\begin{align}
    [\hat{q}_i, \hat{p}_j] = i\hbar\delta_{ij} \Longrightarrow 0=\text{Tr}([\hat{q}_i, \hat{p}_j]) = i\hbar \,\text{Tr}(\delta_{ij}) = i \hbar \cdot \dim(\mathcal{H}).
\end{align}
The only way to resolve this absurdity is to have $\dim(\mathcal{H}) = \infty$ so that the trace is not well-defined and the above equality is meaningless.
The take home is that such operators must act on an \emph{infinite}-dimensional Hilbert space.
Note the problem would get even worse if we had a continuum of DoFs, since then we would have to deal with $\text{Tr}\delta(x-y)$ in the above equality, which is even more ill-defined and infinite.

Incidentally, note that this is not necessarily the case for all quantum systems: for instance, spin systems have a finite number of DoFs and can be described by finite-dimensional Hilbert spaces.
The primary example is the spin-$\frac{1}{2}$ system, whose Hilbert space is just $\mathbb{C}^2$ with Pauli matrices as operators obeying $[\sigma_i, \sigma_j] = 2i \epsilon_{ijk} \sigma_k$.
\end{mytheorem}


\begin{mytheorem}[The idea of second quantization.\todotag{finish}]
At some point people started wondering: in quantum mechanics everything is subject to fluctuations, so why not the very number of particles themselves? That is to say, why not to promote the wave function to an opertor itself?
\end{mytheorem}


\begin{mytheorem}[How to make the Schr\"odinger equation relativistic? Klein-Gordon is \emph{always} the answer. \questiontag{Check \& improve}]
%
Several attempts were made to formulate a relativistic version of the Schr\"odinger equation.
The Schr\"odinger equation is first-order in time and second-order in space derivatives, and enforces a non-relativistic energy-momentum relation
\begin{equation}
    -i\partial_t\psi = -\frac{\hbar^2}{2m} \nabla^2 \psi\quad \Longrightarrow\quad E = \frac{p^2}{2m}.
\end{equation}
It was clear that in order to be relativistic, the equation must treat space and time on an equal footing.
Several approaches were attempted, ultimately depending on the spin of the particle considered.
\begin{itemize}
    \item \textbf{Klein-Gordon Equation.} For spin-0 particles, the Klein-Gordon equation was proposed:
    \begin{equation}
        (\partial_{\mu} \partial^{\mu} + m^2) \phi(x) = 0\,.
    \end{equation}
    This is second-order in both time and space derivatives and respects the relativistic energy-momentum relation \(E^2 = p^2 + m^2\).
    However, it faced challenges in interpreting negative energy solutions and the associated probability density.
    \item \textbf{Dirac Equation.} For spin-$\frac{1}{2}$ particles, Dirac proposed a `square root' of the Klein-Gordon equation that would be first-order in both time and space derivatives:
    \begin{equation}
        (i\gamma^{\mu} \partial_{\mu} - m) \psi(x) = 0.
    \end{equation}
    In doing so, it was necessary to introduce $\gamma$-matrices to yield a Lorentz vector to be contracted with $\partial_{\mu}\psi$ to make the equation Lorentz invariant.
    The Dirac equation can indeed be derived by seeking a first-order equation whose square reproduces the Klein-Gordon equation:
    \begin{align}
        ia^\mu\partial_\mu\psi + b \psi = 0 &\Longrightarrow (ia^\mu\partial_\mu + b)^*(ia^\nu\partial_\nu + b)\psi = 0 \\
        &\Longrightarrow (a^\mu a^\nu \partial_\mu \partial_\nu + i(b a^\mu-a^\mu b)\partial_\mu + b^2)\psi = 0.
    \end{align}
    Imposing this matches the Klein-Gordon equation requires
    \begin{align}
        \{a^\mu, a^\nu\} = 2\eta^{\mu\nu}, \quad
        a^\mu b - b a^\mu = 0, \quad
        b^2= m^2\quad \Rightarrow \quad a^\mu\equiv \gamma^\mu, \quad b \equiv m\,.
    \end{align}
    These relations cannot be satisfied by ordinary numbers, leading to the introduction of the $\gamma$-matrices.
    The Dirac equation successfully incorporated spin and predicted the existence of antimatter, but similarly introduced complexities related to negative energy states.
    %
    \item For spin-1 particles, the Proca and Maxwell equation was developed--or better borrowed from classical electrodynamics--to describe massive and massless vector bosons respectively:
    \begin{equation}
        \partial_{\mu} F^{\mu\nu} + m^2 A^{\nu} = 0.
    \end{equation}
\end{itemize}
All these equations ultimately obey the Klein-Gordon equation, as it should be since it encodes the relativistic energy-momentum relation \(E^2 = p^2 + m^2\) that all particles must satisfy.
\end{mytheorem}


\begin{mytheorem}[The Casimir effect. \todotag{TBC}]
%
The Casimir effect is a physical force which arises e.g. from the zero-point energy of the \emph{quantized} electromagnetic field in the presence of conducting boundaries and pushes them closer together.
It arises since the zero-point energy of vacuum is reduced as the plates get closer thus diminishing the allowed modes of the electromagnetic field between them.
There is an extermely close analogy with the attractive force exerted between two ships travelling parallel to each other in the sea.
Indeed, waves with arbitrary wavelength can exist outside the ships and hit the rightmost ship from the right (resp. the leftmost one from the left) pushing them closer together.
On the other hand, only waves with countable discrete wavelengths $\sim L/n$ can fit between the ships and push them further apart, where $L$ is the distance between them.
There is thus a net attractive force between the ships since the pressure from outside is larger than the one from inside.

\begin{wrapfigure}[11]{r}{0.45\textwidth} % 16 = lines to wrap; adjust as needed
    \centering
    \vspace{-0.5\baselineskip}            % optional: nudge up
    \includegraphics[width=0.4\textwidth]{images/casimir_effect.jpeg}
    \caption{Illustration of the Casimir effect between two parallel conducting plates.}
    \label{fig:casimir_effect}
\end{wrapfigure}
In the context of QFT, consider the electromagnetic field confined between two perfectly-conducting parallel plates separated by a distance $a$.
Boundary conditions discretize the momentum perpendicular to the plates,
\begin{align}
    k_z = \frac{n\pi}{a},\quad n\in\Z\,,\qquad
    \omega_{n,\mathbf{k}_\perp} = c\sqrt{|\mathbf{k}_\perp|^2 + \Big(\tfrac{n\pi}{a}\Big)^2}.
\end{align}
The zero-point energy per unit area is formally divergent and given by
\begin{equation}
    \frac{E_0(a)}{A} = \sum_{n\in\Z}\int \frac{d^2 k_\perp}{(2\pi)^2}\, \tfrac{1}{2}\hbar\omega_{n,\mathbf{k}_\perp},
\end{equation}
The physically meaningful quantity is the \emph{difference} with empty space (no plates)
\begin{equation}
    \frac{E_{\text{Cas}}(a)}{A} := \frac{E_0(a)}{A} - \frac{E_0(\infty)}{A}.
\end{equation}
Using analytic regularization (e.g. zeta-function with $\zeta(-3)=\tfrac{1}{120}$), one obtains a finite energy per unit area, yielding an attractive force between the plates
{\small
\begin{equation}
    \frac{E_{\text{Cas}}(a)}{A} = -\,\frac{\pi^2\,\hbar c}{720\,a^3} \quad \Rightarrow\quad 
     \frac{F}{A} = -\,\frac{d}{da}\!\left(\frac{E_{\text{Cas}}}{A}\right) = -\,\frac{\pi^2\,\hbar c}{240\,a^4} < 0.
\end{equation}
}
%
This force has been measured with high precision and exemplifies how vacuum fluctuations plus boundary conditions produce macroscopic effects.
Crucially, it is a direct proof that vacuum fluctuations do exist at least down to length scales of order $a$, typically microns or less in experiments, placing a lower bound on the UV cutoff of Qunatum Field Theories.
In particular, including zero-point energies only down to the micron scale, much bigger than the Planck scale, already yields a vacuum energy density $\sim 10^{60}$ times larger than the observed cosmological constant, highlighting the severity of the cosmological constant problem.
%
\end{mytheorem}








%-------------------------------------------------------------
%=========================================================
\section{The guiding theorems} 
%=========================================================
%--------------------------------------------------------------


\begin{mytheorem}[Heisenberg canonical quantization procedure.\todotag{Expand \& mimick what done in sec 2}]
Take the classical phase-space variables \(q_i, p_i\), that is the classical DoFs, satisfying the relevant Equations of Motions and  Poisson bracket relations (possibly passing to the continuum limit $\delta_{ij} \to \delta(x-y)$)
\[\{q_i, p_j\} = \delta_{ij}, \quad \{q_i, q_j\} = 0, \quad \{p_i, p_j\} = 0.\]
Then, in the quantum theory, promote these variables to operators \(\hat{q}_i, \hat{p}_i\) acting on a Hilbert space \(\mathcal{H}\), satisfying the canonical commutation relations (again possibly in the continuum limit)
\[[\hat{q}_i, \hat{p}_j] = i\hbar \delta_{ij}, \quad [\hat{q}_i, \hat{q}_j] = 0, \quad [\hat{p}_i, \hat{p}_j] = 0.\]
These \emph{operators are self-adjoint} and obey the \emph{same equations of motion as the classical counterparts}, but now in the operator formalism.
Namely, they are obtained as
\begin{equation}
    \frac{d\hat{O}}{dt} = \frac{1}{i\hbar} [\hat{O},\hat{H}] + \Big(\frac{\partial \hat{O}}{\partial t}\Big)_{\text{explicit}}.
\end{equation}
In particular, for the fundamental DoFs we have
\begin{align}
    i\hbar\frac{d\hat{q}_i}{dt} = [\hat{q}_i, \hat{H}] \quad i\hbar\frac{d\hat{p}_i}{dt} = [\hat{p}_i, \hat{H}].
\end{align}
\end{mytheorem}


\begin{mytheorem}[Heisenberg vs Schr\"odinger picture.\notetag{Check}]
There are two equivalent ways to describe the time evolution of a system in Quantum Mechanics.
They are completely equivalent and related by a unitary transformation.
\begin{itemize}
    \item In the \emph{Heisenberg picture}, operators evolve in time according to the Heisenberg equation of motion, while state vectors remain constant.
    \item In the \emph{Schr\"odinger picture}, state vectors evolve in time according to the Schr\"odinger equation, while operators remain constant.
\end{itemize}
In the Schr\"odinger picture states evolves in time as
\begin{equation}
    i\hbar \frac{d}{dt} |\psi(t)\rangle = \hat{H} |\psi(t)\rangle.
\end{equation}
The Schr\"odinger picture is often more convenient for practical calculations, especially in time-dependent problems.

On the other hand, the Heisenberg picture is particularly useful in Quantum Field Theory and in the study of symmetries and conservation laws.
More importantly, it makes the connection with classical mechanics transparent via the canonical quantization procedure: the EoM are left unchanged, classical DoFs are promoted to operators, and Poisson brackets become commutators.
In particular, it makes it easier to see how classical observables correspond to quantum operators and to recover the classical limit of quantum systems when $\hbar \to 0$.
\end{mytheorem}


\begin{mytheorem}[Ehrenfest theorem.\todotag{Check \& finish}]
The Ehrenfest theorem establishes a connection between the quantum mechanical expectation values of observables and their classical counterparts.
It states that the time evolution of the expectation value of an operator \(\hat{O}\) in a quantum state \(|\psi(t)\rangle\) is the same as the classical equation of motion for the corresponding observable \(O\), provided that the operator \(\hat{O}\) does not explicitly depend on time.
Mathematically, the Ehrenfest theorem is expressed as
\begin{equation}
    \frac{d}{dt} \langle \hat{O} \rangle = \frac{1}{i\hbar} \langle [\hat{O}, \hat{H}] \rangle + \Big\langle \Big(\frac{\partial \hat{O}}{\partial t}\Big)_{\text{explicit}} \Big\rangle,
\end{equation}
where \(\langle \hat{O} \rangle = \langle \psi(t) | \hat{O} | \psi(t) \rangle\) is the expectation value of the operator \(\hat{O}\) in the state \(|\psi(t)\rangle\), and \(\hat{H}\) is the Hamiltonian operator of the system.
In particular, applying the Ehrenfest theorem to the position and momentum operators \(\hat{q}_i\) and \(\hat{p}_i\), we obtain
\begin{align}
    \frac{d}{dt} \langle \hat{q}_i \rangle = \frac{1}{m} \langle \hat{p}_i \rangle, \quad
    \frac{d}{dt} \langle \hat{p}_i \rangle = - \langle \frac{\partial V}{\partial q_i} \rangle,
\end{align}
which resemble the classical equations of motion for a particle in a potential \(V(q)\).
The Ehrenfest theorem thus provides a bridge between quantum mechanics and classical mechanics, showing that the average behavior of quantum systems follows classical laws under certain conditions.
%
\end{mytheorem}


\begin{mytheorem}[Stone-von Neumann theorem.\todotag{Finish \& Check (cf. Kempf QFT 13)}]
%
By the postulate, the Hilbert space of quantum mechanics are required to be separable.
In particular any two such space are unitarily isomorphic
\begin{align}
    U::\mathcal{H}_1 \to \mathcal{H}_2, \quad U^\dagger U = \mathbb{I}_{\mathcal{H}_1}, \quad U U^\dagger = \mathbb{I}_{\mathcal{H}_2}.
\end{align}  
The Stone-von Neumann theorem states that, if the number of degrees of freedom is finite, then up to unitary isomorphisms, there is a unique irreducible representation of the canonical commutation relations (CCR) for the DoFs of the system.
That is, denoting the DoFs of the system as $\{q^a_i, p^a_i\}_{i=1}^N$ for $a=1,2$, any two irreducible representations of the CCR 
\begin{align}
    [\hat{q}^a_i, \hat{p}^a_j] = i\hbar \delta_{ij}, \quad [\hat{q}^a_i, \hat{q}^a_j] = 0, \quad [\hat{p}^a_i, \hat{p}^a_j] = 0,
\end{align}
acting on separable Hilbert spaces $\mathcal{H}_a$, are unitarily isomorphic.
This means that the unitary operator above \(U: \mathcal{H}_1 \to \mathcal{H}_2\) can be taken so that
\begin{align}
    U \hat{q}^1_i U^\dagger = \hat{q}^2_i, \quad U \hat{p}^1_i U^\dagger = \hat{p}^2_i.
\end{align}
This theorem ensures that the physical predictions of Quantum Mechanics are independent of the specific representation chosen for the operators, as long as the number of degrees of freedom is finite.

The theorem \emph{does not hold} when the number of degrees of freedom is infinite, as in Quantum Field Theory.
In typical cases, we can impose IR regularization, making the DoFs countable discrete, and UV regularization, making the DoFs finite, thus recovering the theorem.
This regularizatin procedure might however fail in particular circumstances, leading to unitarily inequivalent representations of the CCR.
For example
\begin{itemize}
    \item if a (weird) theory does not enjoy a proper separation of scales, so that low energy physics is sensitive to the UV completion and the physical predictions depend on the UV regularization scheme;
    \item in the context of phase transitions, where different phases of the system can be described by unitarily inequivalent representations of the CCR, reflecting the distinct physical properties of each phase. Indeed, by the Lee-Yang theorem, phase transitions can only occur in the thermodynamic limit where the space is \emph{truly} infinite and thus the number of DoFs is uncountable infinite.
\end{itemize}
%
\end{mytheorem}


%-------------------------------------------------------------
%=========================================================
\section{Further principles and general ideas (miscellaneous)} 
%=========================================================
%-------------------------------------------------------------


\begin{mytheorem}[Spontaneous symmetry breaking \& anomalies.]
In a nutshell, spontaneous symmetry breaking occurs when the underlying laws (Lagrangian or Hamiltonian) of a system possess a certain symmetry, but the ground state (vacuum state) of the system does not exhibit that symmetry.
This phenomenon has profound implications in various areas of physics, including condensed matter physics and particle physics.

Anomalies, on the other hand, refer to situations where a symmetry that is present at the classical level of a theory is broken already when the theory is quantized--i.e. it is not even a symmetry of the quantized lagrangian/hamiltonian.
Anomalies can have significant consequences, such as the violation of conservation laws associated with the broken symmetry.
\end{mytheorem}
