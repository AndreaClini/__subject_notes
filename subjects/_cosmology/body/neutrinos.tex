% !TeX root = ../cosmology_main.tex
%=========================================================
%=========================================================
\chapter{Neutrinos in Cosmology}
%========================================================
%=========================================================

The treatment of neutrios is almost identical to that of photons, since they can be treated as a relativistic fluid down to very late times.
We still assume they retain local thermal equilibrium with a Fermi-Dirac distribution with temperature constrast $\mathcal{N}$, that is
\begin{align}
    f_\nu(\tau, \vec{x}, \vec{p}) = \frac{1}{e^{(E-\mu)/T_\nu}+1}, \quad T_\nu=\bar{T}_\nu(\tau)(1+\mathcal{N}(\tau, \vec{x}, \hat{p}, E)).
\end{align}
The main differences are the following.
\begin{itemize}
    \item Neutrino temperature is lower than the photon temperature $\bar{T}_\nu\simeq (4/11)^{1/3} \bar{T}_\gamma$, since neutrinos decouple before electron-positron annihilation, which heats up photons but not neutrinos.
    The treatment is however unchanged, since the Boltzmann equation only depends on the temperature contrast $\delta T/T$, not on the absolute value.
    \item Neutrinos are fermions, thus following Fermi-Dirac distribution, while photons are bosons, thus with Bose-Einstein distribution. Again this difference is neglible.
    \item Neutrino have no interactions after decoupling, and the RHS of the Boltzmann equation is thus zero.
    In particular they free-stream and develop non-negligible anisotropic stress (Legendre multipole $\mathcal{N}_2$ and possibly higher ones) already at early times, while photons are driven towards the first two multipoles $\Theta_0$, $\Theta_1$ by Thomson scattering until recombination and develop anisotropic stress only after decoupling.
    Neutrino anisotropic stress has a non-negligible effect on the evolution of metric perturbations, and thus on the CMB power spectrum and other observables.
    \item Neutrino are actually massive, and become non-relativistic at late times, while photons are always relativistic. This affects the background evolution of the universe, and thus the expansion history and the growth of structure at late times.
\end{itemize}


%--------------------------------------------------------
%=========================================================
\section{Geodesic \& Boltzmann Equations for neutrinos}
%=========================================================
%------------------------------------------------------------


The geodesic equations \label{eq:geodesic_equation_physical_momentum_newtonian} are unchanged, again with $E=p$ to good approximation down to very late times,
{\small
\begin{align}\label{eq:geodesic_equations_neutrino_newtonian}
\begin{aligned}
    \frac{d E}{d\ct}&=-\frac{p^2}{E}(\mathcal{H}-\dot\phi)-p\hat{p}^j\partial_j\psi,
    \\
    \frac{d p}{d\ct}&=-p(\mathcal{H}-\dot\phi)-E\hat{p}^j\partial_j\psi,
    \\
    \frac{d\hat{p}^i}{d\ct}&=\big(\hat{p}^i\hat{p}^j-\delta^{ij}\big)\left(\frac{E}{p}\partial_j\psi+\frac{p}{E}\partial_j\phi\right).
\end{aligned}
\end{align}}
The last equation implies that to zeroth order in metric perturbations, neutrinos travel along straight lines, while at first order they are deflected by the gravitational potentials $\psi$ and $\phi$.
The first two equations encode the change in energy and momentum along the trajectory.
It is convenient to rewrite it trading partial with total time derivatives $\frac{\partial \phi}{\partial \tau}=\frac{d\phi}{d\tau}-\frac{\partial \phi}{\partial x^j}\frac{dx^j}{d\tau}$ to get, in the relativistic limit $E=p$,
{\small
\begin{align}
    \frac{d \log E}{d\tau}&= -\mathcal{H}+\frac{d\phi}{d\tau}-\hat{p}^j(\partial_j\phi+\partial_j\psi).
\end{align}}
To order zero it encode the expected redshift behavior $E=p=\propto\, a^{-1}$.
The perturbed part implies energy is red-blueshifted by the gravitational potentials via the istantaneous and integrated Sachs-Wolf effect.
Since the change in $\hat{p}^j$ along the trajectory is already first order, we can integrate it to get
{\small
\begin{align}
    \log \Big(\frac{E_f}{E_i}\Big) &= \int_{\tau_i}^{\tau_f} d\tau\,\, -\mathcal{H} + \frac{d\phi}{d\tau} - \hat{p}^j(\partial_j\phi+\partial_j\psi) \, d\tau\\
    &= \log \Big(\frac{a_i}{a_f}\Big) + \underbrace{\big(\phi(\tau_f,x_f)-\phi(\tau_i,x_i)\big)}_{\text{instantaneous SW (aka gravitational redshift)}} - \int_{\tau_i}^{\tau_f} \underbrace{d\tau\,\, \hat{p}^j(\partial_j\phi+\partial_j\psi)}_{\text{integrated Sachs-Wolfe effect}}.
\end{align}}
Writing $E = \bar{E}(1+\mathcal{N})$ with $\bar{E}\propto \, a^{-1}$ so that $\bar{E}_ia_i=\bar{E}_fa_f$ to remove the background redshift, we get the expected Sachs-Wolfe formula for the observed temperature anisotropies
{\small
\begin{align}
    \mathcal{N}_f(\hat{p})-\mathcal{N}_i(\hat{p})\simeq \log\left(\frac{1+\mathcal{N}_f(\hat{p})}{1+\mathcal{N}_i(\hat{p})}\right) 
    = \phi(\tau_f,x_f)-\phi(\tau_i,x_i) - \int_{\tau_i}^{\tau_f} d\tau\,\, \hat{p}^j(\partial_j\phi+\partial_j\psi).
\end{align}}


The Boltzmann equation \eqref{eq:Boltzmann_LHS_exact_newton_gauge} for neutrinoes is again (second order in metric perturbations, exact otherwise)
{\small
\begin{align}\label{eq:Boltzmann_LHS_exact_newton_gauge_neutrino}
\begin{aligned}
    C[f]&=\frac{d f}{d \lambda}
    =
    %
    P^{0}\left[\frac{\partial f}{\partial \tau}+\frac{\partial f}{\partial x^{i}} \frac{P^{i}}{P^{0}}+\frac{\partial f}{\partial E} \frac{1}{P^{0}} \frac{d E}{d \lambda}+\frac{\partial f}{\partial \hat{p}^{i}} \frac{1}{P^{0}} \frac{d \hat{p}^{i}}{d \lambda}\right]
    \\
    &=\frac{E}{a}\!\left[\frac{\partial f}{\partial \ct}(1-\psi)
    \!+\!\left(\hat{p}^j\partial_jf\right) \frac{p}{E}(1\!+\!\phi)
    -\frac{\partial f}{\partial E}\frac{p^2}{E}\left(\mathcal{H}(1\!-\!\psi)-\dot\phi\right)
    -\frac{\partial f}{\partial \bar{p}^j}\partial_j\psi \,E
    \!+\!\left(\frac{\partial f}{\partial p} \hat{p}^j-\frac{\partial f}{\partial \bar{p}^j}\right)\partial_j\phi\,\frac{p^2}{E}\right].
\end{aligned}
\end{align}}

Neutrinos retain local equilibrium with a Fermi-Dirac distribution down to late times\todotag{Discuss}, simply with a time and momentum dependent temperature, 
{\small
\begin{equation}\label{eq:local_temp_neutrinos}
    f(\ct, x^i, E, \hat{p}^i) = \frac{1}{\exp\left(\frac{E}{T_\nu}\right) + 1}, \quad T_\nu=\bar{T}_\nu(\tau)(1+\mathcal{N}(\tau, x^i, p, \hat{p}^i))=\bar{T}_\nu+\delta T_\nu.
\end{equation}}
More generally, assuming the full distribution is still a function $f(\nicefrac{E}{T})$ of the ratio $\nicefrac{E}{T}$ to some \emph{local} temperature $T$, we Taylor expand the distribution function and its derivative just like for photons \label{eq:photon_distribution_expansion} and rewrite the Boltzmann equation.
At zero order we get the redshift $T\propto a^{-1}$ of neutrino temperature
{\small
\begin{align}\label{eq:Boltzmann_LHS_neutrino_zero_order}
\begin{aligned}
    &\cancel{\tfrac{E}{a}E\partial_E\bar{f}}\bigg[\tfrac{1}{\bar{T}}\tfrac{d\bar{T}}{d\tau}+\mathcal{H}\bigg]=0 \quad \Rightarrow \quad \bar{T}_\nu\propto \frac{1}{a}.\\[3pt]
\end{aligned}
\end{align}}
At first order, replacing $\frac{\dot{T}}{T}=-\mathcal{H}$, we get
{\small
\begin{align}\label{eq:Boltzmann_LHS_neutrino_first_order_newtonian}
\begin{aligned}
    &-\tfrac{E}{a}E\partial_E\bar{f}\bigg[\frac{\partial\mathcal{N}}{\partial \tau}+\hat{p}^j\frac{\partial\mathcal{N}}{\partial x^j}-\mathcal{H}E\cancel{\frac{\partial \mathcal{N}}{\partial E}}+\hat{p}^j\frac{\partial\psi}{\partial x^j}-\frac{\partial \phi}{\partial\tau}\bigg] = 0.
\end{aligned}
\end{align}}
Note there is indeed no right-hand side, since neutrinos have no sizable interactions after decoupling.




%=====================================================================
\subsection*{Passing to multipoles}
%====================================================================
We can now pass to Fourier space and decompose the angular dependence of $\mathcal{N}$ in multipoles to get a hierarchy of coupled ODEs for the multipole moments $\mathcal{N}_\ell$.
Passing to Fourier space in \eqref{eq:Boltzmann_LHS_neutrino_first_order_newtonian}, we get\todotag{Fix RHS}
{\small
\begin{align}\label{eq:Boltzmann_LHS_neutrino_first_order_newtonian_Fourier}
\begin{aligned}
    &\bigg[\frac{\partial\mathcal{N}}{\partial \tau}+i\vec{k}\cdot\hat{p}\,\mathcal{N}-\mathcal{H}E\cancel{\frac{\partial \mathcal{N}}{\partial E}}+i\hat{p}\cdot\vec{k}\,\psi-\frac{\partial \phi}{\partial\tau}\bigg] = 0.
\end{aligned}
\end{align}}
The dependence on $\vec{k}$ only enter via the modulus $k$ and the angle $\mu:=\hat{k}\cdot\hat{p}$, so we can decompose the angular dependence in Legendre polynomials $P_\ell(\mu)$ as
{\small
\begin{align}
    \mathcal{N}(\tau, k, \mu) = \sum_{\ell=0}^\infty (-i)^\ell (2\ell+1) \mathcal{N}_\ell(\tau, k) P_\ell(\mu).
\end{align}}
The multipole moments $\mathcal{N}_\ell$ are given by the inverse transform
{\small
\begin{align}
    \mathcal{N}_\ell(\tau, k) = \frac{i^\ell}{2}\int_{-1}^1 d\mu\, P_\ell(\mu) \mathcal{N}(\tau, k, \mu).
\end{align}}
The first three multipoles have a simple physical interpretation: up to numerical factors, $\mathcal{N}_0$ is the monopole i.e. the average temperature perturbation; $\mathcal{N}_1$ is the dipole i.e. the bulk velocity of the neutrino fluid; $\mathcal{N}_2$ is the quadrupole i.e. the anisotropic stress of the neutrino fluid.
Higher multipoles $\mathcal{N}_\ell$ with $\ell\geq 3$ encode higher-order angular correlations of the temperature perturbation, and are generated by free-streaming of neutrinos after decoupling.
The Boltzmann equation then becomes a hierarchy of coupled ODEs for the multipole moments $\mathcal{N}_\ell$, which can be solved numerically to get\todotag{Write eq}
{\small
\begin{align}
    ...
\end{align}}