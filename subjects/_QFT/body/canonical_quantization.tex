% !TeX root = ../QFT_main.tex
%=========================================================
%=========================================================
\chapter{Canonical Quantization}
%========================================================
%=========================================================


%--------------------------------------------------------
%=========================================================
\section{Canonical Quantization in flat spacetime}
%=========================================================
%------------------------------------------------------------



\begin{mytheorem}[Review of classical \& quantum harmonic oscillator]\todotag{Finish}
The classical harmonic oscillator with mass $m$ and frequency $\omega$ is described by the Hamiltonian
\begin{equation}
    H = \frac{p^2}{2m} + \frac{1}{2} m \omega^2 q^2,
\end{equation}
where $q$ and $p$ are the canonical position and momentum variables, satisfying the Poisson bracket relation
\begin{equation}
    \{q, p\} = 1.
\end{equation}
Upon quantization, the canonical variables are promoted to operators $\hat{q}$ and $\hat{p}$ acting on a Hilbert space $\mathcal{H}$, satisfying the commutation relation
\begin{equation}
    [\hat{q}, \hat{p}] = i \hbar.
\end{equation}
The Hamiltonian operator is given by
\begin{equation}
    \hat{H} = \frac{\hat{p}^2}{2m} + \frac{1}{2} m \omega^2 \hat{q}^2.
\end{equation}
The energy eigenvalues of the quantum harmonic oscillator are
\begin{equation}
    E_n = \hbar \omega \left( n + \frac{1}{2} \right), \quad n = 0, 1, 2, \ldots
\end{equation}
It is crucial to understand how the frequency of the oscillators determines the likelyhood of finding the system at a give elongation--i.e. amplitude--from equilibrium.
Indeed, all wave functions are proportional to a Gaussian factor, multiplied by suitable Hermite polynomials,
\begin{align}
    \psi_n(q) &\propto e^{-\frac{m \omega}{2 \hbar} q^2}.
\end{align}
The crucial point is that the smaller the frequency $\omega$, the wider the Gaussian, and thus the larger the likelyhood of finding the oscillator at large elongations from equilibrium.
This is something already familiar from Electrodynamics: it is very unlikely to find high-frequency electromagnetic waves with large amplitudes, whereas low-frequency waves can easily have large amplitudes.
\end{mytheorem}


\begin{mytheorem}[Free Klein-Gordon field as waves on a lake \& Fourier modes as decoupled harmonic oscillators]
\todotag{insert image of lake \& fourier modes}
Under any point of view, we should see the classical Klein-Gordon field as a collection of decoupled harmonic oscillators, one per each Fourier mode.
In this sense, it is completely analogous to the waves on a lake, which can be decomposed into independent Fourier modes, each oscillating with its own frequency.
\end{mytheorem}

\begin{mytheorem}[Probing the Klein-Gordon field with a cork] \todotag{write}
\todotag{Insert image}
How would you probe the waves on a lake? A good idea is to use a cork floating on the water surface.
The cork moves up and down following the local height of the water surface, thus providing a direct measurement of the wave amplitude at that point.
Analogously, to probe the Klein-Gordon field, we can use a point-like detector that measures the field value at a specific spacetime point.
This detector interacts with the field, allowing us to extract information about the field's behavior and dynamics.
The cork would now be an atom or another suitable quantum system that can be acted upon by the field.
\end{mytheorem}



\begin{mytheorem}[Classical theory of the Klein-Gordon field]
%
A real scalar field $\phi(x)$ obeys the Klein-Gordon equation
\begin{equation}
    (\partial_{\mu} \partial^{\mu} + m^2) \phi(x) = 0.
\end{equation}
The (nonunique) Lagrangian density yielding the Klein-Gordon equation as Euler-Lagrange equation is
\begin{equation}
    \mathcal{L} = \frac{1}{2} \partial_{\mu} \phi \partial^{\mu} \phi - \frac{1}{2} m^2 \phi^2.
\end{equation}
The corresponding Hamiltonian density is
\begin{equation}
    \mathcal{H} = \frac{1}{2} \pi^2 + \frac{1}{2} (\nabla \phi)^2 + \frac{1}{2} m^2 \phi^2,
\end{equation}
for the conjugate momentum field
\begin{align}
    \pi(x) &= \frac{\delta \mathcal{L}}{\delta \partial_0 \phi(x)} = \partial_0 \phi(x).
\end{align}
In the classicual field theory, the fields $\phi(x)$ and $\pi(x)$ satisfy the equal-time Poisson bracket relations
\begin{align}
    \{\phi(t, \mathbf{x}), \pi(t, \mathbf{y})\} &= \delta^{(3)}(\mathbf{x} - \mathbf{y}), \\
    \{\phi(t, \mathbf{x}), \phi(t, \mathbf{y})\} &= 0, \\
    \{\pi(t, \mathbf{x}), \pi(t, \mathbf{y})\} &= 0.
\end{align}
The equation of motions for any functional $\mathcal{O}[\phi, \pi]$ of the fields is given by
\begin{equation}
    \frac{d\mathcal{O}}{dt} = \{\mathcal{O}, H\} + \frac{\partial \mathcal{O}}{\partial t},
\end{equation}
where the Hamiltonian is
\begin{equation}
    H = \int d^3 x\, \mathcal{H} = \int d^3 x\, \left( \frac{1}{2} \pi^2 + \frac{1}{2} (\nabla \phi)^2 + \frac{1}{2} m^2 \phi^2 \right).
\end{equation}
\end{mytheorem}

\begin{mytheorem}[Canonical quantization of the Klein-Gordon field]
%
Following Heisenberg canonical quantization procedure, we promote the classical fields $\phi(x)$ and $\pi(x)$ to operators $\hat{\phi}(x)$ and $\hat{\pi}(x)$ acting on a Hilbert space $\mathcal{H}$.
%
The equal time Poisson brackets are replaced by commutators accordin to the rule 
\begin{equation}
    \{A, B\} \mapsto \frac{1}{i\hbar} [\hat{A}, \hat{B}].
\end{equation}
yielding the equal time cannonical commutation relations
\begin{align}
    [\hat{\phi}(t, \mathbf{x}), \hat{\pi}(t, \mathbf{y})] &= i \delta^{(3)}(\mathbf{x} - \mathbf{y}), \\
    [\hat{\phi}(t, \mathbf{x}), \hat{\phi}(t, \mathbf{y})] &= 0, \\
    [\hat{\pi}(t, \mathbf{x}), \hat{\pi}(t, \mathbf{y})] &= 0.
\end{align}

The equation of motions are formally left unchanged, with the understanding that now refer to evolving operators.
With the replacement of Poisson brackets with commutators, the evolution of any operator $\mathcal{O}$ is given by
\begin{equation}
    \frac{d\hat{\mathcal{O}}}{dt} = \frac{1}{i \hbar} [\hat{\mathcal{O}}, \hat{H}] + \frac{\partial \hat{\mathcal{O}}}{\partial t},
\end{equation}
where the Hamiltonian operator is
\begin{equation}
    \hat{H} = \int d^3 x\, \left( \frac{1}{2} \hat{\pi}^2 + \frac{1}{2} (\nabla \hat{\phi})^2 + \frac{1}{2} m^2 \hat{\phi}^2 \right).
\end{equation}

In particular, the fundamental fields obey the Heisenberg equations of motion
\begin{align}
    \frac{d\hat{\phi}(t, \mathbf{x})}{dt} &= \frac{1}{i \hbar} [\hat{\phi}(t, \mathbf{x}), \hat{H}] = \hat{\pi}(t, \mathbf{x}), \\
    \frac{d\hat{\pi}(t, \mathbf{x})}{dt} &= \frac{1}{i \hbar} [\hat{\pi}(t, \mathbf{x}), \hat{H}] = (\nabla^2 - m^2) \hat{\phi}(t, \mathbf{x}),
\end{align}
which together imply that the field operator $\hat{\phi}(x)$ satisfies the Klein-Gordon equation
\begin{equation}
    (\partial_{\mu} \partial^{\mu} + m^2) \hat{\phi}(x) = 0.
\end{equation}
\end{mytheorem}


\begin{mytheorem}[Fourier transform \& identifying the uncoupled DoFs of the Klein-Gordon field]
%  
The Fourier transform of the fields are defined as
\begin{equation}
    \phi_k(t) :=\int d^3 x\, e^{-i \mathbf{k} \cdot \mathbf{x}} \phi(t, \mathbf{x}).
\end{equation}
Coeherently, the conjugate momentum in momentum space is
\begin{equation}
    \pi_k(t) := \int d^3 x\, e^{-i \mathbf{k} \cdot \mathbf{x}} \pi(t, \mathbf{x}) = \frac{d \phi_k(t)}{dt}.
\end{equation}
The Klein-Gordon equation simply becomes the equation for uoncoupled harmonic oscillator with frequency $\omega_k = \sqrt{|\mathbf{k}|^2 + m^2}$
\begin{equation}
    \frac{d^2 \phi_k(t)}{dt^2} + (|\mathbf{k}|^2 + m^2) \phi_k(t) = 0,
\end{equation}
Coherently, the Hamiltonian reads
\begin{equation}
    \mathcal{H} = \int d^3x\,\left(\frac{1}{2} \pi^2 + \frac{1}{2} (\nabla \phi)^2 + \frac{1}{2} m^2 \phi^2\right) = \int \frac{d^3 k}{(2\pi)^3} \left( \frac{1}{2} \pi_k^\dagger\pi_k + \frac{1}{2} (|\mathbf{k}|^2 + m^2) \phi_k^\dagger \phi_{k} \right).
\end{equation}
Finally, the equal time commutation relations in momentum space become
\begin{align}
    [\hat{\phi}_k(t), \hat{\pi}_{k'}(t)] &= i (2\pi)^3 \delta^{(3)}(\mathbf{k} + \mathbf{k}'), \\
    [\hat{\phi}_k(t), \hat{\phi}_{k'}(t)] &= 0, \\
    [\hat{\pi}_k(t), \hat{\pi}_{k'}(t)] &= 0.
\end{align}

It seems we managed to diagonalize the theory and identify its uncoupled degrees of freedom, behaving as independent harmonic oscillators labelled by the momentum $\mathbf{k}$, but something is off track.
\begin{itemize}
    \item The field $\phi_k$ is no more self-adjoint, since $\phi_k^{\dagger} = \phi_{-k}$; whereas we would like our DoFs to be observables;
    \item The delta functions of the CAR is $\delta^{(3)}(\mathbf{k} + \mathbf{k}')$ instead of the expected $\delta^{(3)}(\mathbf{k} - \mathbf{k}')$, suggesting that the conjugate momentum of the mode $\phi_k$ is actually $\pi_{-k}$ instead of $\pi_k$.
    \item We have Dirac deltas instead of Kronecker deltas for our DoFs.
\end{itemize}
The last problem is expected since we are considering infinite volume and thus a continuum of modes $k$.
An infrared regularization, like placing the theory in a finite box or sphere would immediately restore the disctreness of the finite space and the discreteness of the modes and the Kronecker deltas.
The first two issues are much more serious and deeply related.

We can fix this by explicitly rewriting $\phi_k$ as linear combination of \emph{physical} self-adjoint harmonic oscillators $\{q_k,p_k\}_k$ to highlight the true DoFs of the theory.
Alternatively, and equivalently, we can directly write $\phi_k$ in terms of creation and annihilation operators, and then possibly identify the corresponding self-adjoint oscillators.
\todotag{Write the two approaches in detail.}
    
\end{mytheorem}



%----------------------------------------------------------------
%===================================================================
\section{The Schrödinger equation in QFT}
%===================================================================
%----------------------------------------------------------------

The good old Schrödinger equation in Quantum Mechanics reads
\begin{equation}
    i \hbar \frac{\partial}{\partial t} |\psi(t)\rangle = \hat{H} |\psi(t)\rangle.
\end{equation}
In this formulation, it is bound to fail when dealing with systems with variable particle number.

However, as the Heisenberg and Schrödinger formulations of Quantum Mechanics are equivalent, we can still use the latter in Quantum Field Theory and get a corresponding equation.
The space $\mathcal{F}=\bigoplus \mathcal{H}^{\otimes n}$ of states in QFT is now a Fock space, i.e. a direct sum of $n$-particle Hilbert spaces.
Taking $\mathcal{H}\simeq L^2(\mathbb{R}^3)$ for simplicity, the general state in $\mathcal{F}$ reads
\begin{equation}
    |\Psi\rangle = \left( \begin{array}{c}
    \psi_0 \\
    \psi_1(\mathbf{x}_1) \\
    \psi_2(\mathbf{x}_1, \mathbf{x}_2) \\
    \vdots
    \end{array} \right),\qquad \psi_n(\mathbf{x}_1, \ldots, \mathbf{x}_n) \in L^2(\mathbb{R}^{3n}).
\end{equation}

The Schrödinger equation in QFT is virtually the same:
\begin{equation}
    i \hbar \frac{\partial}{\partial t} |\Psi(t)\rangle = \hat{H} |\Psi(t)\rangle.
\end{equation}
Here $\hat{H}=\int d^3x \mathcal{H}$ is the Hamiltonian operator acting on the Fock space $\mathcal{F}$, namely the space-integral of the Hamiltonian density operator $\mathcal{H}$.
For example, for the free Klein-Gordon field
\begin{align}
    \hat{H} &= \int d^3 x\, \left( \frac{1}{2} \hat{\pi}^2 + \frac{1}{2} (\nabla \hat{\phi})^2 + \frac{1}{2} m^2 \hat{\phi}^2 \right).
\end{align}


\begin{mytheorem}[Spectral bases \& maximal set of commuting observables.]
%
In ordinary QM, we can expand the state $|\Psi\rangle$ in terms of the eigenstates of suitable (collections of commuting) self-adjoint operators.
Common choices are the position operator $X$, the momentum $P$, the angular momentum $L^2$ and $L_z$, the intrisinc spin $S^2$ and $S_z$ or suitable combinations like $L+S$, helicity, the Hamiltonian etc

A \emph{maximal set of commuting observables} is a collection of self-adjoint operators $\{A_i\}$ such that
\begin{itemize}
    \item $[A_i, A_j] = 0$ for all $i, j$;
    \item there is no other self-adjoint operator $B$ commuting with all $A_i$.
\end{itemize}
The common eigenstates of a maximal set of commuting observables form a basis for the Hilbert space with the property that no two distinct basis states share the same eigenvalues for all observables in the set.
\end{mytheorem}
\begin{proof}
Suppose there is a collection of eigenvalues $\{\lambda_i\}$ shared by two linearly independent eigenstates $|\psi\rangle$ and $|\phi\rangle$.
Then $S=\text{Span}\{|\psi\rangle, |\phi\rangle\}$ is a two-dimensional subspace invariant under all $A_i$ and we can define a new self-adjoint operator $B$ acting non-trivially only on this subspace and commuting with all $A_i$, contradicting the maximality of the set--e.g. it swaps $|\psi\rangle$ and $|\phi\rangle$ and is the identity on $S^{\perp}$.
\end{proof}


\begin{mytheorem}[Position \& momentum bases in QFT]
%
The same can be done in QFT for suitable self-adjoint operators.
At fixed time commutation relations for the real Klein-Gordon field $\hat{\phi}$ and its conjugate momentum $\hat{\pi}=\delta \mathcal{L}/\delta \partial_0 \phi$ are
\begin{align}
    [\hat{\phi}(t, \mathbf{x}), \hat{\pi}(t, \mathbf{y})] &= i \delta^{(3)}(\mathbf{x} - \mathbf{y}), \\
    [\hat{\phi}(t, \mathbf{x}), \hat{\phi}(t, \mathbf{y})] &= 0, \\
    [\hat{\pi}(t, \mathbf{x}), \hat{\pi}(t, \mathbf{y})] &= 0.
\end{align}
At any fixed time $t$, the collections of operators $\{\hat{\phi}(t, \mathbf{x})\}_{\mathbf{x}\in \mathbb{R}^3}$ and $\{\hat{\pi}(t, \mathbf{x})\}_{\mathbf{x}\in \mathbb{R}^3}$ are thus maximal sets of commuting observable, and we can find corresponding spectral bases $\{|f\rangle\}_{f}$ and $\{|g\rangle\}_{g}$ of the Fock space with eigenvalue collection $\{f(\mathbf{x})\}_{\mathbf{x}\in \mathbb{R}^3}$ and $\{g(\mathbf{x})\}_{\mathbf{x}\in \mathbb{R}^3}$ respectively for suitable functions $f, g:\mathbb{R}^3 \to \mathbb{R}$.
These bases obey
\begin{align}
    \hat{\phi}(t, \mathbf{x}) |f\rangle &= f(\mathbf{x}) |f\rangle, \\
    \hat{\pi}(t, \mathbf{x}) |g\rangle &= g(\mathbf{x}) |g\rangle.
\end{align}
It has to be noted that nobody really knows how to rigorously construct these bases, nor where they actually live--i.e. what kind of mathematical objects these functions $f$ and $g$ are..
To fix ideas, a very naive guess would be that the set of eigenvalue collections $\{f(\mathbf{x})\}_{\mathbf{x}\in \mathbb{R}^3}$ is the set of all square-integrable functions $f \in L^2(\mathbb{R}^3)$, but this cannot be true since $L^2(\mathbb{R}^3)$ is a set of equivalence classes of functions.

In any case, these spectral bases must be understood as the field-theoretic analogoues of the position $\{|x\rangle\}_{x\in \mathbb{R}^n}$ and momentum $\{|p\rangle\}_{p\in \mathbb{R}^n}$ bases in ordinary Quantum Mechanics, since $\hat{\phi}(x)$ and $\hat{\pi}(x)$ are the field-theoretic counterpart of the position $\hat{X}$ and momentum $\hat{P}$ operators. 
Indeed, in an ordinary QM system we have a finite number of DoFs, labelled by $i=1, \ldots, n$, with generalized coordinates $x_i$ and conjugate momenta $p_i$ satisfying equal time canonical commutation relations
\begin{align}
    [\hat{X}_i, \hat{P}_j] &= i \hbar \delta_{ij}, \quad
    [\hat{X}_i, \hat{X}_j] &= 0, \quad
    [\hat{P}_i, \hat{P}_j] &= 0.
\end{align}
In QFT we have an infinite number of DoFs, labelled by the spatial position $\mathbf{x} \in \mathbb{R}^3$, with generalized coordinates $\phi(\mathbf{x})$ and conjugate momenta $\pi(\mathbf{x})$ satisfying the analogous canonical commutation relations above.
\end{mytheorem}
