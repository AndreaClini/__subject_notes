% !TeX root = ../cosmology_main.tex
%=========================================================
%=========================================================
\chapter{Cosmic Microwave Background}\label{ch:cmb}
%========================================================
%=========================================================



%--------------------------------------------------------
%=========================================================
\section{Geodesic \& Boltzmann Equations for photons}
%=========================================================
%------------------------------------------------------------

We report the newtonian-gauge geodesic equations \eqref{eq:geodesic_equation_covariant_momentum_newtonian}-\eqref{eq:geodesic_equation_physical_momentum_newtonian} and Boltzmann equation \eqref{eq:Boltzmann_LHS_exact_newton_gauge}, which simplify for photons since $E=p$.
We have 
\begin{align}\label{eq:geodesic_equations_photon_newtonian}
\begin{aligned}
    \frac{d P^0}{d\lambda}
    &= -\frac{E^2}{a^2}\left(\mathcal{H}(1-2\psi)+\dot\psi\right)-2\frac{Ep}{a^2}\hat{p}^j\partial_j\psi-\frac{p^2}{a^2}\left(\mathcal{H}(1-2\psi)+\dot\phi\right),
    \\
    \frac{dP^k}{d\lambda}&=-\frac{E^2}{a^2}\partial_k\psi-\frac{Ep}{a^2}\hat{p}^k\,2\left(\mathcal{H}(1-\psi+\phi)-\dot\phi\right)-\frac{p^2}{a^2}\left(\partial_k\phi-2\hat{p}^k\hat{p}^j\partial_j\phi\right).
    \\
    \frac{d E}{d\ct}&=-\frac{p^2}{E}(\mathcal{H}-\dot\phi)-\hat{p}^j\partial_j\psi,
    \\
    \frac{d p}{d\ct}&=-p(\mathcal{H}-\dot\phi)-\frac{E}{p}\hat{p}^j\partial_j\psi,
    \\
    \frac{d\hat{p}^i}{d\ct}&=\big(\hat{p}^i\hat{p}^j-\delta^{ij}\big)\left(\frac{E}{p}\partial_j\psi+\frac{p}{E}\partial_j\phi\right).
\end{aligned}
\end{align}
Similarly, the Boltzmann equation for photons becomes
{\small
\begin{align}\label{eq:Boltzmann_LHS_exact_newton_gauge_photon}
\begin{aligned}
    C[f]=\frac{d f}{d \lambda}
    %
    &=
    %
    P^{0}\left[\frac{\partial f}{\partial \tau}+\frac{\partial f}{\partial x^{i}} \frac{P^{i}}{P^{0}}+\frac{\partial f}{\partial E} \frac{1}{P^{0}} \frac{d E}{d \lambda}+\frac{\partial f}{\partial \hat{p}^{i}} \frac{1}{P^{0}} \frac{d \hat{p}^{i}}{d \lambda}\right]
    %
    \\&=
    %
    \frac{E}{a}\!\left[\frac{\partial f}{\partial \ct}(1-\psi)
    \!+\!\left(\hat{p}^j\partial_jf\right) \frac{p}{E}(1\!+\!\phi)
    -\frac{\partial f}{\partial E}\frac{p^2}{E}\left(\mathcal{H}(1\!-\!\psi)-\dot\phi\right)
    -\frac{\partial f}{\partial \bar{p}^j}\partial_j\psi \,E
    \!+\!\left(\frac{\partial f}{\partial p} \hat{p}^j-\frac{\partial f}{\partial \bar{p}^j}\right)\partial_j\phi\,\frac{p^2}{E}\right].
\end{aligned}
\end{align}}

Photons retain local equilibrium with a blackbody distribution at all time\todotag{Discuss}, simply with a time and momentum dependent temperature, 
{\small
\begin{equation}\label{eq:local_temp_photons}
    f(\ct, x^i, E, \hat{p}^i) = \frac{1}{\exp\left(\frac{E}{T(\ct, x^i)}\right) - 1}, \quad T=\bar{T}(\tau)(1+\Theta(\tau, x^i, p, \hat{p}^i))=\bar{T}+\delta T.
\end{equation}}
More generally, assuming the full distribution is still a function $f(\nicefrac{E}{T})$ of the ratio $\nicefrac{E}{T}$ to some \emph{local} temperature $T$, we Taylor expand around the background $\bar{f}(\tau,E)=\bar{f}(\nicefrac{E}{\bar{T}(\tau)})$ to get
{\small
\begin{align}
    f&= \bar{f}+\frac{\partial f}{\partial T}\delta T = \bar{f}-\frac{\partial \bar{f}}{\partial E}E\Theta = \bar{f}-\frac{d \bar{f}}{d (\nicefrac{E}{\bar{T}})} \frac{E}{\bar{T}}\Theta,\qquad\qquad \text{with}\quad \frac{d \bar{f}}{d (\nicefrac{E}{\bar{T}})}=\frac{\partial \bar{f}}{\partial E}_{|\tau}\bar{T}=-\frac{\partial \bar{f}}{\partial \bar{T}}_{|E}\frac{\bar{T}^2}{E}.
\end{align}}
In turn we compute the various derivatives appearing in the LHS of Boltzmann equation 
{\small
\begin{align}\label{eq:photon_distribution_expansion}
\begin{aligned}
    \frac{\partial f}{\partial \tau}&= -\frac{d\bar{f}}{d(\nicefrac{E}{\bar{T}})}\frac{E}{\bar{T}^2}\frac{d\bar{T}}{d\tau}+ \frac{d^2\bar{f}}{d(\nicefrac{E}{\bar{T}})^2} \frac{E^2}{\bar{T}^3}\frac{d\bar{T}}{d\tau}\Theta+\frac{d \bar{f}}{d(\nicefrac{E}{\bar{T}})}\frac{E}{\bar{T}^2}\frac{d\bar{T}}{d\tau}\Theta-\frac{d \bar{f}}{d (\nicefrac{E}{\bar{T}})} \frac{E}{\bar{T}}\frac{\partial \Theta}{\partial \tau}\\
    &= \frac{\partial \bar{f}}{\partial E}\frac{E}{\bar{T}}\frac{d\bar{T}}{d\tau}+\frac{\partial^2 \bar{f}}{\partial E^2}\frac{E^2}{\bar{T}}\frac{d\bar{T}}{d\tau}\Theta+\frac{\partial \bar{f}}{\partial E}\frac{E}{\bar{T}}\frac{d\bar{T}}{d\tau}\Theta-\frac{\partial \bar{f}}{\partial E}E\frac{\partial \Theta}{\partial \tau},
    \\
    \frac{\partial f}{\partial E}&=\frac{\partial \bar{f}}{\partial E}-\frac{\partial^2 \bar{f}}{\partial E^2}E\Theta-\frac{\partial \bar{f}}{\partial E}\Theta-\frac{\partial \bar{f}}{\partial E}E\frac{\partial \Theta}{\partial E},\\
    \frac{\partial f}{\partial \hat{p}^i}&=-E\frac{\partial \bar{f}}{\partial E}\frac{\partial \Theta}{\partial \hat{p}^i},\\
    \frac{\partial f}{\partial x^i}&=-E\frac{\partial \bar{f}}{\partial E}\frac{\partial \Theta}{\partial x^i}.
\end{aligned}
\end{align}}
Plugging \eqref{eq:photon_distribution_expansion} into the LHS of Boltzmann equation \eqref{eq:Boltzmann_LHS_exact_newton_gauge_photon} at zero order we get the redshift $T\propto a^{-1}$ of photon temperature
{\small
\begin{align}\label{eq:Boltzmann_LHS_photon_zero_order}
\begin{aligned}
    &\cancel{\tfrac{E}{a}E\partial_E\bar{f}}\bigg[\tfrac{1}{\bar{T}}\tfrac{d\bar{T}}{d\tau}+\mathcal{H}\bigg]=0 \quad \Rightarrow \quad \bar{T}\propto \frac{1}{a}.\\[3pt]
\end{aligned}
\end{align}}
At first order, replacing $\frac{\dot{T}}{T}=-\mathcal{H}$, we get
{\small
\begin{align}\label{eq:Boltzmann_LHS_photon_first_order_newtonian}
\begin{aligned}
    &-\tfrac{E}{a}E\partial_E\bar{f}\bigg[\frac{\partial\Theta}{\partial \tau}+\hat{p}^j\frac{\partial\Theta}{\partial x^j}-\mathcal{H}E\cancel{\frac{\partial \Theta}{\partial E}}+\hat{p}^j\frac{\partial\psi}{\partial x^j}-\frac{\partial \phi}{\partial\tau}\bigg] = C[f].
\end{aligned}
\end{align}}
At late times collisions with baryons are essentially low-energy Compton scatterings.
To good approximation, these scatterings do not change the energy of photons but only the direction of their momentum, and we can thus assume $\Theta=\Theta(\tau, x^i, \hat{p}^i)$, i.e. the temperature perturbation is independent of the photon energy.

We need to compute the collision term $C[f]$ for Compton scatterings, which we do in the next section.




%--------------------------------------------------------
%=========================================================
\section{The photon-baryon fluid}
%=========================================================
%------------------------------------------------------------

The relevant interactions for the photon-baryon fluid are
\begin{align}
    e^-+p \leftrightarrow e^-+p, \quad \text{(Coulomb scattering)}\\
    e^-+\gamma \leftrightarrow e^-+\gamma, \quad \text{(Thomson scattering:=low-energy Compton scattering)}\\
    p+\gamma \leftrightarrow p+\gamma, \quad \text{(Thomson scattering:=low-energy Compton scattering)}
\end{align}
Coulomb scattering is extremely efficient at keeping electrons and protons tightly coupled at \emph{all times}, allowing to treat them as a single baryon fluid with common bulk quantities to extreme accuracy
\begin{align}
    \delta_e=\delta_p=\delta_b, \quad \vec{u}_e=\vec{u}_p=\vec{u}_b.
\end{align}
The second keeps photons tightly coupled to electrons, and thus to protons, making sense of the photon-baryon fluid picture at early times.
The last is negligible since the Thompson cross-section $\sigma_T \propto m^{-2}$ and thus for protons is suppressed by a factor $(m_e/m_p)^2$.


Thompson scattering does not change number density, but does exchange momentum between photons and baryons, and thus appears as a source term in the Euler equations of the two fluids, with opposite signs to ensure total momentum conservation.
That is, the conservation equations read
\begin{align}
    \nabla_\mu T^\mu_{(\gamma)\,0}=\nabla_\mu T^\mu_{(b)\,0}=0\quad \text{while}\quad \nabla_\mu T^\mu_{(\gamma)\,j}=-\nabla_\mu T^\mu_{(b)\,j}=:Q_j\neq 0.
\end{align}
Regardless of the interaction, the RHS for the Euler equations is written in terms of the velocity $v_j=\frac{1}{(\rho+p)}T^0_{\;j}$.
Therefore if we know the RHS of either the photon or baryon Euler equation, we can immediately get the RHS of the other by
{\small
\begin{align}
    \frac{\partial v_\gamma^j}{\partial \tau}+\ldots = \frac{1}{(\rho_\gamma+p_\gamma)} Q_j=:X\quad\Rightarrow\quad \frac{\partial v_b^j}{\partial \tau}+\ldots = -\frac{1}{(\rho_b+p_b)} Q_j = -\frac{(\rho_\gamma+p_\gamma)}{(\rho_b+p_b)} \frac{1}{(\rho_\gamma+p_\gamma)} Q_j = -\frac{4}{3} \frac{\rho_\gamma}{\rho_b} X.
\end{align}}







%--------------------------------------------------------
%=========================================================
\section{CMB Lensing}
%=========================================================
%------------------------------------------------------------