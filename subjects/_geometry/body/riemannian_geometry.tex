% !TeX root = ../geometry_main.tex
%==========================================================
%=========================================================
\chapter{pseudo-Riemannian Geometry}
%=========================================================
%=========================================================




%-----------------------------------------------------------------------
%========================================================
\section{Constant curvature spaces} \todotag{To be written}
%=========================================================
%-----------------------------------------------------------------------


\begin{mytheorem}[Einstein manifolds\todotag{write}]

\end{mytheorem}


\begin{mytheorem}[Kulkarni--Nomizu tensor in constant curvature spaces\todotag{finish}]
In a (pseudo-)Riemannian manifold of constant sectional curvature \(K\), the Riemann curvature tensor can be expressed as
\[
    R_{abcd} = K (g_{ac} g_{bd} - g_{ad} g_{bc}).
\]
This is proved using the properties of the Kulkarni--Nomizu product and the fact that sectional curvature uniquely determines the Riemann tensor.
%
\end{mytheorem}

%-----------------------------------------------------------------------
%========================================================
\section{Isometric immersions} \todotag{To be written}
%=========================================================
%-----------------------------------------------------------------------